% language=uk

\environment luatex-style

\startcomponent luatex-math

\startchapter[reference=math,title={Math}]

\startsection[title={Traditional alongside \OPENTYPE}]

\topicindex {math}

The handling of mathematics in \LUATEX\ differs quite a bit from how \TEX82 (and
therefore \PDFTEX) handles math. First, \LUATEX\ adds primitives and extends some
others so that \UNICODE\ input can be used easily. Second, all of \TEX82's
internal special values (for example for operator spacing) have been made
accessible and changeable via control sequences. Third, there are extensions that
make it easier to use \OPENTYPE\ math fonts. And finally, there are some
extensions that have been proposed or considered in the past that are now added
to the engine.

\stopsection

\startsection[title={Unicode math characters}]

\topicindex {math+\UNICODE}
\topicindex {\UNICODE+math}

Character handling is now extended up to the full \UNICODE\ range (the \type {\U}
prefix), which is compatible with \XETEX.

The math primitives from \TEX\ are kept as they are, except for the ones that
convert from input to math commands: \type {mathcode}, and \type {delcode}. These
two now allow for a 21-bit character argument on the left hand side of the equals
sign.

Some of the new \LUATEX\ primitives read more than one separate value. This is
shown in the tables below by a plus sign.

The input for such primitives would look like this:

\starttyping
\def\overbrace{\Umathaccent 0 1 "23DE }
\stoptyping

The altered \TEX82 primitives are:

\starttabulate[|l|l|r|c|l|r|]
\DB primitive       \BC min \BC max    \BC \kern 2em \BC min \BC max    \NC \NR
\TB
\NC \prm {mathcode} \NC 0   \NC 10FFFF \NC =         \NC 0   \NC 8000   \NC \NR
\NC \prm {delcode}  \NC 0   \NC 10FFFF \NC =         \NC 0   \NC FFFFFF \NC \NR
\LL
\stoptabulate

The unaltered ones are:

\starttabulate[|l|l|r|]
\DB primitive          \BC min \BC max     \NC \NR
\TB
\NC \prm {mathchardef} \NC 0   \NC    8000 \NC \NR
\NC \prm {mathchar}    \NC 0   \NC    7FFF \NC \NR
\NC \prm {mathaccent}  \NC 0   \NC    7FFF \NC \NR
\NC \prm {delimiter}   \NC 0   \NC 7FFFFFF \NC \NR
\NC \prm {radical}     \NC 0   \NC 7FFFFFF \NC \NR
\LL
\stoptabulate

For practical reasons \prm {mathchardef} will silently accept values larger
that \type {0x8000} and interpret it as \lpr {Umathcharnumdef}. This is needed
to satisfy older macro packages.

The following new primitives are compatible with \XETEX:

% somewhat fuzzy:

\starttabulate[|l|l|r|c|l|r|]
\DB primitive                             \BC min       \BC max         \BC \kern 2em \BC min       \BC max         \NC \NR
\TB
\NC \lpr {Umathchardef}                   \NC 0+0+0     \NC 7+FF+10FFFF \NC           \NC           \NC             \NC \NR
\NC \lpr {Umathcharnumdef}\rlap{\high{5}} \NC -80000000 \NC    7FFFFFFF \NC           \NC           \NC             \NC \NR
\NC \lpr {Umathcode}                      \NC 0         \NC      10FFFF \NC =         \NC 0+0+0     \NC 7+FF+10FFFF \NC \NR
\NC \lpr {Udelcode}                       \NC 0         \NC      10FFFF \NC =         \NC 0+0       \NC   FF+10FFFF \NC \NR
\NC \lpr {Umathchar}                      \NC 0+0+0     \NC 7+FF+10FFFF \NC           \NC           \NC             \NC \NR
\NC \lpr {Umathaccent}                    \NC 0+0+0     \NC 7+FF+10FFFF \NC           \NC           \NC             \NC \NR
\NC \lpr {Udelimiter}                     \NC 0+0+0     \NC 7+FF+10FFFF \NC           \NC           \NC             \NC \NR
\NC \lpr {Uradical}                       \NC 0+0       \NC   FF+10FFFF \NC           \NC           \NC             \NC \NR
\NC \lpr {Umathcharnum}                   \NC -80000000 \NC    7FFFFFFF \NC           \NC           \NC             \NC \NR
\NC \lpr {Umathcodenum}                   \NC 0         \NC      10FFFF \NC =         \NC -80000000 \NC    7FFFFFFF \NC \NR
\NC \lpr {Udelcodenum}                    \NC 0         \NC      10FFFF \NC =         \NC -80000000 \NC    7FFFFFFF \NC \NR
\LL
\stoptabulate

Specifications typically look like:

\starttyping
\Umathchardef\xx="1"0"456
\Umathcode   123="1"0"789
\stoptyping

The new primitives that deal with delimiter|-|style objects do not set up a
\quote {large family}. Selecting a suitable size for display purposes is expected
to be dealt with by the font via the \lpr {Umathoperatorsize} parameter.

For some of these primitives, all information is packed into a single signed
integer. For the first two (\lpr {Umathcharnum} and \lpr {Umathcodenum}), the
lowest 21 bits are the character code, the 3 bits above that represent the math
class, and the family data is kept in the topmost bits. This means that the values
for math families 128--255 are actually negative. For \lpr {Udelcodenum} there
is no math class. The math family information is stored in the bits directly on
top of the character code. Using these three commands is not as natural as using
the two- and three|-|value commands, so unless you know exactly what you are
doing and absolutely require the speedup resulting from the faster input
scanning, it is better to use the verbose commands instead.

The \lpr {Umathaccent} command accepts optional keywords to control various
details regarding math accents. See \in {section} [mathacc] below for details.

There are more new primitives and all of these will be explained in following
sections:

\starttabulate[|l|l|]
\DB primitive                \BC value range (in hex) \NC \NR
\TB
\NC \lpr {Uroot}           \NC 0 + 0--FF + 10FFFF   \NC \NR
\NC \lpr {Uoverdelimiter}  \NC 0 + 0--FF + 10FFFF   \NC \NR
\NC \lpr {Uunderdelimiter} \NC 0 + 0--FF + 10FFFF   \NC \NR
\NC \lpr {Udelimiterover}  \NC 0 + 0--FF + 10FFFF   \NC \NR
\NC \lpr {Udelimiterunder} \NC 0 + 0--FF + 10FFFF   \NC \NR
\LL
\stoptabulate

\stopsection

\startsection[title={Math styles}]

\subsection{\lpr {mathstyle}}

\topicindex {math+styles}

It is possible to discover the math style that will be used for a formula in an
expandable fashion (while the math list is still being read). To make this
possible, \LUATEX\ adds the new primitive: \lpr {mathstyle}. This is a \quote
{convert command} like e.g. \prm {romannumeral}: its value can only be read,
not set.

The returned value is between 0 and 7 (in math mode), or $-1$ (all other modes).
For easy testing, the eight math style commands have been altered so that they can
be used as numeric values, so you can write code like this:

\starttyping
\ifnum\mathstyle=\textstyle
    \message{normal text style}
\else \ifnum\mathstyle=\crampedtextstyle
    \message{cramped text style}
\fi \fi
\stoptyping

Sometimes you won't get what you expect so a bit of explanation might help to
understand what happens. When math is parsed and expanded it gets turned into a
linked list. In a second pass the formula will be build. This has to do with the
fact that in order to determine the automatically chosen sizes (in for instance
fractions) following content can influence preceding sizes. A side effect of this
is for instance that one cannot change the definition of a font family (and
thereby reusing numbers) because the number that got used is stored and used in
the second pass (so changing \type {\fam 12} mid|-|formula spoils over to
preceding use of that family).

The style switching primitives like \prm {textstyle} are turned into nodes so the
styles set there are frozen. The \prm {mathchoice} primitive results in four
lists being constructed of which one is used in the second pass. The fact that
some automatic styles are not yet known also means that the \lpr {mathstyle}
primitive expands to the current style which can of course be different from the
one really used. It's a snapshot of the first pass state. As a consequence in the
following example you get a style number (first pass) typeset that can actually
differ from the used style (second pass). In the case of a math choice used
ungrouped, the chosen style is used after the choice too, unless you group.

\startbuffer[1]
    [a:\mathstyle]\quad
    \bgroup
    \mathchoice
        {\bf \scriptstyle       (x:d :\mathstyle)}
        {\bf \scriptscriptstyle (x:t :\mathstyle)}
        {\bf \scriptscriptstyle (x:s :\mathstyle)}
        {\bf \scriptscriptstyle (x:ss:\mathstyle)}
    \egroup
    \quad[b:\mathstyle]\quad
    \mathchoice
        {\bf \scriptstyle       (y:d :\mathstyle)}
        {\bf \scriptscriptstyle (y:t :\mathstyle)}
        {\bf \scriptscriptstyle (y:s :\mathstyle)}
        {\bf \scriptscriptstyle (y:ss:\mathstyle)}
    \quad[c:\mathstyle]\quad
    \bgroup
    \mathchoice
        {\bf \scriptstyle       (z:d :\mathstyle)}
        {\bf \scriptscriptstyle (z:t :\mathstyle)}
        {\bf \scriptscriptstyle (z:s :\mathstyle)}
        {\bf \scriptscriptstyle (z:ss:\mathstyle)}
    \egroup
    \quad[d:\mathstyle]
\stopbuffer

\startbuffer[2]
    [a:\mathstyle]\quad
    \begingroup
    \mathchoice
        {\bf \scriptstyle       (x:d :\mathstyle)}
        {\bf \scriptscriptstyle (x:t :\mathstyle)}
        {\bf \scriptscriptstyle (x:s :\mathstyle)}
        {\bf \scriptscriptstyle (x:ss:\mathstyle)}
    \endgroup
    \quad[b:\mathstyle]\quad
    \mathchoice
        {\bf \scriptstyle       (y:d :\mathstyle)}
        {\bf \scriptscriptstyle (y:t :\mathstyle)}
        {\bf \scriptscriptstyle (y:s :\mathstyle)}
        {\bf \scriptscriptstyle (y:ss:\mathstyle)}
    \quad[c:\mathstyle]\quad
    \begingroup
    \mathchoice
        {\bf \scriptstyle       (z:d :\mathstyle)}
        {\bf \scriptscriptstyle (z:t :\mathstyle)}
        {\bf \scriptscriptstyle (z:s :\mathstyle)}
        {\bf \scriptscriptstyle (z:ss:\mathstyle)}
    \endgroup
    \quad[d:\mathstyle]
\stopbuffer

\typebuffer[1]

% \typebuffer[2]

This gives:

\blank $\displaystyle \getbuffer[1]$ \blank
\blank $\textstyle    \getbuffer[1]$ \blank

Using \prm {begingroup} \unknown\ \prm {endgroup} instead gives:

\blank $\displaystyle \getbuffer[2]$ \blank
\blank $\textstyle    \getbuffer[2]$ \blank

This might look wrong but it's just a side effect of \lpr {mathstyle} expanding
to the current (first pass) style and the number being injected in the list that
gets converted in the second pass. It all makes sense and it illustrates the
importance of grouping. In fact, the math choice style being effective afterwards
has advantages. It would be hard to get it otherwise.

\subsection{\lpr {Ustack}}

\topicindex {math+stacks}

There are a few math commands in \TEX\ where the style that will be used is not
known straight from the start. These commands (\prm {over}, \prm {atop},
\prm {overwithdelims}, \prm {atopwithdelims}) would therefore normally return
wrong values for \lpr {mathstyle}. To fix this, \LUATEX\ introduces a special
prefix command: \lpr {Ustack}:

\starttyping
$\Ustack {a \over b}$
\stoptyping

The \lpr {Ustack} command will scan the next brace and start a new math group
with the correct (numerator) math style.

\subsection{Cramped math styles}

\topicindex {math+styles}
\topicindex {math+spacing}
\topicindex {math+cramped}

\LUATEX\ has four new primitives to set the cramped math styles directly:

\starttyping
\crampeddisplaystyle
\crampedtextstyle
\crampedscriptstyle
\crampedscriptscriptstyle
\stoptyping

These additional commands are not all that valuable on their own, but they come
in handy as arguments to the math parameter settings that will be added shortly.

In Eijkhouts \quotation {\TEX\ by Topic} the rules for handling styles in scripts
are described as follows:

\startitemize
\startitem
    In any style superscripts and subscripts are taken from the next smaller style.
    Exception: in display style they are in script style.
\stopitem
\startitem
    Subscripts are always in the cramped variant of the style; superscripts are only
    cramped if the original style was cramped.
\stopitem
\startitem
    In an \type {..\over..} formula in any style the numerator and denominator are
    taken from the next smaller style.
\stopitem
\startitem
    The denominator is always in cramped style; the numerator is only in cramped
    style if the original style was cramped.
\stopitem
\startitem
    Formulas under a \type {\sqrt} or \prm {overline} are in cramped style.
\stopitem
\stopitemize

In \LUATEX\ one can set the styles in more detail which means that you sometimes
have to set both normal and cramped styles to get the effect you want. (Even) if
we force styles in the script using \prm {scriptstyle} and \lpr
{crampedscriptstyle} we get this:

\startbuffer[demo]
\starttabulate
\DB style         \BC example \NC \NR
\TB
\NC default       \NC $b_{x=xx}^{x=xx}$ \NC \NR
\NC script        \NC $b_{\scriptstyle x=xx}^{\scriptstyle x=xx}$ \NC \NR
\NC crampedscript \NC $b_{\crampedscriptstyle x=xx}^{\crampedscriptstyle x=xx}$ \NC \NR
\LL
\stoptabulate
\stopbuffer

\getbuffer[demo]

Now we set the following parameters

\startbuffer[setup]
\Umathordrelspacing\scriptstyle=30mu
\Umathordordspacing\scriptstyle=30mu
\stopbuffer

\typebuffer[setup]

This gives a different result:

\start\getbuffer[setup,demo]\stop

But, as this is not what is expected (visually) we should say:

\startbuffer[setup]
\Umathordrelspacing\scriptstyle=30mu
\Umathordordspacing\scriptstyle=30mu
\Umathordrelspacing\crampedscriptstyle=30mu
\Umathordordspacing\crampedscriptstyle=30mu
\stopbuffer

\typebuffer[setup]

Now we get:

\start\getbuffer[setup,demo]\stop

\stopsection

\startsection[title={Math parameter settings}]

\subsection {Many new \lpr {Umath*} primitives}

\topicindex {math+parameters}

In \LUATEX, the font dimension parameters that \TEX\ used in math typesetting are
now accessible via primitive commands. In fact, refactoring of the math engine
has resulted in many more parameters than were not accessible before.

\starttabulate
\DB primitive name                   \BC description \NC \NR
\TB
\NC \lpr {Umathquad}               \NC the width of 18 mu's \NC \NR
\NC \lpr {Umathaxis}               \NC height of the vertical center axis of
                                         the math formula above the baseline \NC \NR
\NC \lpr {Umathoperatorsize}       \NC minimum size of large operators in display mode \NC \NR
\NC \lpr {Umathoverbarkern}        \NC vertical clearance above the rule \NC \NR
\NC \lpr {Umathoverbarrule}        \NC the width of the rule \NC \NR
\NC \lpr {Umathoverbarvgap}        \NC vertical clearance below the rule \NC \NR
\NC \lpr {Umathunderbarkern}       \NC vertical clearance below the rule \NC \NR
\NC \lpr {Umathunderbarrule}       \NC the width of the rule \NC \NR
\NC \lpr {Umathunderbarvgap}       \NC vertical clearance above the rule \NC \NR
\NC \lpr {Umathradicalkern}        \NC vertical clearance above the rule \NC \NR
\NC \lpr {Umathradicalrule}        \NC the width of the rule \NC \NR
\NC \lpr {Umathradicalvgap}        \NC vertical clearance below the rule \NC \NR
\NC \lpr {Umathradicaldegreebefore}\NC the forward kern that takes place before placement of
                                       the radical degree \NC \NR
\NC \lpr {Umathradicaldegreeafter} \NC the backward kern that takes place after placement of
                                       the radical degree \NC \NR
\NC \lpr {Umathradicaldegreeraise} \NC this is the percentage of the total height and depth of
                                       the radical sign that the degree is raised by; it is
                                       expressed in \type {percents}, so 60\% is expressed as the
                                       integer $60$ \NC \NR
\NC \lpr {Umathstackvgap}          \NC vertical clearance between the two
                                       elements in a \prm {atop} stack \NC \NR
\NC \lpr {Umathstacknumup}         \NC numerator shift upward in \prm {atop} stack \NC \NR
\NC \lpr {Umathstackdenomdown}     \NC denominator shift downward in \prm {atop} stack \NC \NR
\NC \lpr {Umathfractionrule}       \NC the width of the rule in a \prm {over} \NC \NR
\NC \lpr {Umathfractionnumvgap}    \NC vertical clearance between the numerator and the rule \NC \NR
\NC \lpr {Umathfractionnumup}      \NC numerator shift upward in \prm {over} \NC \NR
\NC \lpr {Umathfractiondenomvgap}  \NC vertical clearance between the denominator and the rule \NC \NR
\NC \lpr {Umathfractiondenomdown}  \NC denominator shift downward in \prm {over} \NC \NR
\NC \lpr {Umathfractiondelsize}    \NC minimum delimiter size for \type {\...withdelims} \NC \NR
\NC \lpr {Umathlimitabovevgap}     \NC vertical clearance for limits above operators \NC \NR
\NC \lpr {Umathlimitabovebgap}     \NC vertical baseline clearance for limits above operators \NC \NR
\NC \lpr {Umathlimitabovekern}     \NC space reserved at the top of the limit \NC \NR
\NC \lpr {Umathlimitbelowvgap}     \NC vertical clearance for limits below operators \NC \NR
\NC \lpr {Umathlimitbelowbgap}     \NC vertical baseline clearance for limits below operators \NC \NR
\NC \lpr {Umathlimitbelowkern}     \NC space reserved at the bottom of the limit \NC \NR
\NC \lpr {Umathoverdelimitervgap}  \NC vertical clearance for limits above delimiters \NC \NR
\NC \lpr {Umathoverdelimiterbgap}  \NC vertical baseline clearance for limits above delimiters \NC \NR
\NC \lpr {Umathunderdelimitervgap} \NC vertical clearance for limits below delimiters \NC \NR
\NC \lpr {Umathunderdelimiterbgap} \NC vertical baseline clearance for limits below delimiters \NC \NR
\NC \lpr {Umathsubshiftdrop}       \NC subscript drop for boxes and subformulas \NC \NR
\NC \lpr {Umathsubshiftdown}       \NC subscript drop for characters \NC \NR
\NC \lpr {Umathsupshiftdrop}       \NC superscript drop (raise, actually) for boxes and subformulas \NC \NR
\NC \lpr {Umathsupshiftup}         \NC superscript raise for characters \NC \NR
\NC \lpr {Umathsubsupshiftdown}    \NC subscript drop in the presence of a superscript \NC \NR
\NC \lpr {Umathsubtopmax}          \NC the top of standalone subscripts cannot be higher than this
                                       above the baseline \NC \NR
\NC \lpr {Umathsupbottommin}       \NC the bottom of standalone superscripts cannot be less than
                                       this above the baseline \NC \NR
\NC \lpr {Umathsupsubbottommax}    \NC the bottom of the superscript of a combined super- and subscript
                                       be at least as high as this above the baseline \NC \NR
\NC \lpr {Umathsubsupvgap}         \NC vertical clearance between super- and subscript \NC \NR
\NC \lpr {Umathspaceafterscript}   \NC additional space added after a super- or subscript \NC \NR
\NC \lpr {Umathconnectoroverlapmin}\NC minimum overlap between parts in an extensible recipe \NC \NR
\LL
\stoptabulate

Each of the parameters in this section can be set by a command like this:

\starttyping
\Umathquad\displaystyle=1em
\stoptyping

they obey grouping, and you can use \type {\the\Umathquad\displaystyle} if
needed.

\subsection{Font|-|based math parameters}

\topicindex {math+parameters}

While it is nice to have these math parameters available for tweaking, it would
be tedious to have to set each of them by hand. For this reason, \LUATEX\
initializes a bunch of these parameters whenever you assign a font identifier to
a math family based on either the traditional math font dimensions in the font
(for assignments to math family~2 and~3 using \TFM|-|based fonts like \type
{cmsy} and \type {cmex}), or based on the named values in a potential \type
{MathConstants} table when the font is loaded via Lua. If there is a \type
{MathConstants} table, this takes precedence over font dimensions, and in that
case no attention is paid to which family is being assigned to: the \type
{MathConstants} tables in the last assigned family sets all parameters.

In the table below, the one|-|letter style abbreviations and symbolic tfm font
dimension names match those used in the \TeX book. Assignments to \prm
{textfont} set the values for the cramped and uncramped display and text styles,
\prm {scriptfont} sets the script styles, and \prm {scriptscriptfont} sets the
scriptscript styles, so we have eight parameters for three font sizes. In the
\TFM\ case, assignments only happen in family~2 and family~3 (and of course only
for the parameters for which there are font dimensions).

Besides the parameters below, \LUATEX\ also looks at the \quote {space} font
dimension parameter. For math fonts, this should be set to zero.

\def\MathLine#1#2#3#4#5%
  {\TB
   \NC \llap{\high{\tx #2\enspace}}\ttbf \string #1 \NC \tt #5 \NC \NR
   \NC \tx #3 \NC \tt #4 \NC \NR}

\starttabulate[|l|l|]
\DB variable / style \BC tfm / opentype \NC \NR
\MathLine{\Umathaxis}               {}   {}                     {AxisHeight}                              {axis_height}
\MathLine{\Umathoperatorsize}       {6}  {D, D'}                {DisplayOperatorMinHeight}                {\emdash}
\MathLine{\Umathfractiondelsize}    {9}  {D, D'}                {FractionDelimiterDisplayStyleSize}       {delim1}
\MathLine{\Umathfractiondelsize}    {9}  {T, T', S, S', SS, SS'}{FractionDelimiterSize}                   {delim2}
\MathLine{\Umathfractiondenomdown}  {}   {D, D'}                {FractionDenominatorDisplayStyleShiftDown}{denom1}
\MathLine{\Umathfractiondenomdown}  {}   {T, T', S, S', SS, SS'}{FractionDenominatorShiftDown}            {denom2}
\MathLine{\Umathfractiondenomvgap}  {}   {D, D'}                {FractionDenominatorDisplayStyleGapMin}   {3*default_rule_thickness}
\MathLine{\Umathfractiondenomvgap}  {}   {T, T', S, S', SS, SS'}{FractionDenominatorGapMin}               {default_rule_thickness}
\MathLine{\Umathfractionnumup}      {}   {D, D'}                {FractionNumeratorDisplayStyleShiftUp}    {num1}
\MathLine{\Umathfractionnumup}      {}   {T, T', S, S', SS, SS'}{FractionNumeratorShiftUp}                {num2}
\MathLine{\Umathfractionnumvgap}    {}   {D, D'}                {FractionNumeratorDisplayStyleGapMin}     {3*default_rule_thickness}
\MathLine{\Umathfractionnumvgap}    {}   {T, T', S, S', SS, SS'}{FractionNumeratorGapMin}                 {default_rule_thickness}
\MathLine{\Umathfractionrule}       {}   {}                     {FractionRuleThickness}                   {default_rule_thickness}
\MathLine{\Umathskewedfractionhgap} {}   {}                     {SkewedFractionHorizontalGap}             {math_quad/2}
\MathLine{\Umathskewedfractionvgap} {}   {}                     {SkewedFractionVerticalGap}               {math_x_height}
\MathLine{\Umathlimitabovebgap}     {}   {}                     {UpperLimitBaselineRiseMin}               {big_op_spacing3}
\MathLine{\Umathlimitabovekern}     {1}  {}                     {0}                                       {big_op_spacing5}
\MathLine{\Umathlimitabovevgap}     {}   {}                     {UpperLimitGapMin}                        {big_op_spacing1}
\MathLine{\Umathlimitbelowbgap}     {}   {}                     {LowerLimitBaselineDropMin}               {big_op_spacing4}
\MathLine{\Umathlimitbelowkern}     {1}  {}                     {0}                                       {big_op_spacing5}
\MathLine{\Umathlimitbelowvgap}     {}   {}                     {LowerLimitGapMin}                        {big_op_spacing2}
\MathLine{\Umathoverdelimitervgap}  {}   {}                     {StretchStackGapBelowMin}                 {big_op_spacing1}
\MathLine{\Umathoverdelimiterbgap}  {}   {}                     {StretchStackTopShiftUp}                  {big_op_spacing3}
\MathLine{\Umathunderdelimitervgap} {}   {}                     {StretchStackGapAboveMin}                 {big_op_spacing2}
\MathLine{\Umathunderdelimiterbgap} {}   {}                     {StretchStackBottomShiftDown}             {big_op_spacing4}
\MathLine{\Umathoverbarkern}        {}   {}                     {OverbarExtraAscender}                    {default_rule_thickness}
\MathLine{\Umathoverbarrule}        {}   {}                     {OverbarRuleThickness}                    {default_rule_thickness}
\MathLine{\Umathoverbarvgap}        {}   {}                     {OverbarVerticalGap}                      {3*default_rule_thickness}
\MathLine{\Umathquad}               {1}  {}                     {<font_size(f)>}                          {math_quad}
\MathLine{\Umathradicalkern}        {}   {}                     {RadicalExtraAscender}                    {default_rule_thickness}
\MathLine{\Umathradicalrule}        {2}  {}                     {RadicalRuleThickness}                    {<not set>}
\MathLine{\Umathradicalvgap}        {3}  {D, D'}                {RadicalDisplayStyleVerticalGap}          {default_rule_thickness+abs(math_x_height)/4}
\MathLine{\Umathradicalvgap}        {3}  {T, T', S, S', SS, SS'}{RadicalVerticalGap}                      {default_rule_thickness+abs(default_rule_thickness)/4}
\MathLine{\Umathradicaldegreebefore}{2}  {}                     {RadicalKernBeforeDegree}                 {<not set>}
\MathLine{\Umathradicaldegreeafter} {2}  {}                     {RadicalKernAfterDegree}                  {<not set>}
\MathLine{\Umathradicaldegreeraise} {2,7}{}                     {RadicalDegreeBottomRaisePercent}         {<not set>}
\MathLine{\Umathspaceafterscript}   {4}  {}                     {SpaceAfterScript}                        {script_space}
\MathLine{\Umathstackdenomdown}     {}   {D, D'}                {StackBottomDisplayStyleShiftDown}        {denom1}
\MathLine{\Umathstackdenomdown}     {}   {T, T', S, S', SS, SS'}{StackBottomShiftDown}                    {denom2}
\MathLine{\Umathstacknumup}         {}   {D, D'}                {StackTopDisplayStyleShiftUp}             {num1}
\MathLine{\Umathstacknumup}         {}   {T, T', S, S', SS, SS'}{StackTopShiftUp}                         {num3}
\MathLine{\Umathstackvgap}          {}   {D, D'}                {StackDisplayStyleGapMin}                 {7*default_rule_thickness}
\MathLine{\Umathstackvgap}          {}   {T, T', S, S', SS, SS'}{StackGapMin}                             {3*default_rule_thickness}
\MathLine{\Umathsubshiftdown}       {}   {}                     {SubscriptShiftDown}                      {sub1}
\MathLine{\Umathsubshiftdrop}       {}   {}                     {SubscriptBaselineDropMin}                {sub_drop}
\MathLine{\Umathsubsupshiftdown}    {8}  {}                     {SubscriptShiftDownWithSuperscript}       {\emdash}
\MathLine{\Umathsubtopmax}          {}   {}                     {SubscriptTopMax}                         {abs(math_x_height*4)/5}
\MathLine{\Umathsubsupvgap}         {}   {}                     {SubSuperscriptGapMin}                    {4*default_rule_thickness}
\MathLine{\Umathsupbottommin}       {}   {}                     {SuperscriptBottomMin}                    {abs(math_x_height/4)}
\MathLine{\Umathsupshiftdrop}       {}   {}                     {SuperscriptBaselineDropMax}              {sup_drop}
\MathLine{\Umathsupshiftup}         {}   {D}                    {SuperscriptShiftUp}                      {sup1}
\MathLine{\Umathsupshiftup}         {}   {T, S, SS,}            {SuperscriptShiftUp}                      {sup2}
\MathLine{\Umathsupshiftup}         {}   {D', T', S', SS'}      {SuperscriptShiftUpCramped}               {sup3}
\MathLine{\Umathsupsubbottommax}    {}   {}                     {SuperscriptBottomMaxWithSubscript}       {abs(math_x_height*4)/5}
\MathLine{\Umathunderbarkern}       {}   {}                     {UnderbarExtraDescender}                  {default_rule_thickness}
\MathLine{\Umathunderbarrule}       {}   {}                     {UnderbarRuleThickness}                   {default_rule_thickness}
\MathLine{\Umathunderbarvgap}       {}   {}                     {UnderbarVerticalGap}                     {3*default_rule_thickness}
\MathLine{\Umathconnectoroverlapmin}{5}  {}                     {MinConnectorOverlap}                     {0}
\LL
\stoptabulate

Note 1: \OPENTYPE\ fonts set \lpr {Umathlimitabovekern} and \lpr
{Umathlimitbelowkern} to zero and set \lpr {Umathquad} to the font size of the
used font, because these are not supported in the \type {MATH} table,

Note 2: Traditional \TFM\ fonts do not set \lpr {Umathradicalrule} because
\TEX82\ uses the height of the radical instead. When this parameter is indeed not
set when \LUATEX\ has to typeset a radical, a backward compatibility mode will
kick in that assumes that an oldstyle \TEX\ font is used. Also, they do not set
\lpr {Umathradicaldegreebefore}, \lpr {Umathradicaldegreeafter}, and \lpr
{Umathradicaldegreeraise}. These are then automatically initialized to
$5/18$quad, $-10/18$quad, and 60.

Note 3: If \TFM\ fonts are used, then the \lpr {Umathradicalvgap} is not set
until the first time \LUATEX\ has to typeset a formula because this needs
parameters from both family~2 and family~3. This provides a partial backward
compatibility with \TEX82, but that compatibility is only partial: once the \lpr
{Umathradicalvgap} is set, it will not be recalculated any more.

Note 4: When \TFM\ fonts are used a similar situation arises with respect to \lpr
{Umathspaceafterscript}: it is not set until the first time \LUATEX\ has to
typeset a formula. This provides some backward compatibility with \TEX82. But
once the \lpr {Umathspaceafterscript} is set, \prm {scriptspace} will never be
looked at again.

Note 5: Traditional \TFM\ fonts set \lpr {Umathconnectoroverlapmin} to zero
because \TEX82\ always stacks extensibles without any overlap.

Note 6: The \lpr {Umathoperatorsize} is only used in \prm {displaystyle}, and is
only set in \OPENTYPE\ fonts. In \TFM\ font mode, it is artificially set to one
scaled point more than the initial attempt's size, so that always the \quote
{first next} will be tried, just like in \TEX82.

Note 7: The \lpr {Umathradicaldegreeraise} is a special case because it is the
only parameter that is expressed in a percentage instead of a number of scaled
points.

Note 8: \type {SubscriptShiftDownWithSuperscript} does not actually exist in the
\quote {standard} \OPENTYPE\ math font Cambria, but it is useful enough to be
added.

Note 9: \type {FractionDelimiterDisplayStyleSize} and \type
{FractionDelimiterSize} do not actually exist in the \quote {standard} \OPENTYPE\
math font Cambria, but were useful enough to be added.

\stopsection

\startsection[title={Math spacing}]

\subsection{Inline surrounding space}

\topicindex {math+spacing}

Inline math is surrounded by (optional) \prm {mathsurround} spacing but that is a fixed
dimension. There is now an additional parameter \lpr {mathsurroundskip}. When set to a
non|-|zero value (or zero with some stretch or shrink) this parameter will replace
\prm {mathsurround}. By using an additional parameter instead of changing the nature
of \prm {mathsurround}, we can remain compatible. In the meantime a bit more
control has been added via \lpr {mathsurroundmode}. This directive can take 6 values
with zero being the default behaviour.

\start

\def\OneLiner#1#2%
  {\NC \type{#1}
   \NC \dontleavehmode\inframed[align=normal,offset=0pt,frame=off]{\mathsurroundmode#1\relax\hsize 100pt   x$x$x}
   \NC \dontleavehmode\inframed[align=normal,offset=0pt,frame=off]{\mathsurroundmode#1\relax\hsize 100pt x $x$ x}
   \NC #2
   \NC \NR}

\startbuffer
\mathsurround    10pt
\mathsurroundskip20pt
\stopbuffer

\typebuffer \getbuffer

\starttabulate[|c|c|c|pl|]
\DB mode \BC x\$x\$x \BC x \$x\$ x \BC effect \NC \NR
\TB
\OneLiner{0}{obey \prm {mathsurround} when \lpr {mathsurroundskip} is 0pt}
\OneLiner{1}{only add skip to the left}
\OneLiner{2}{only add skip to the right}
\OneLiner{3}{add skip to the left and right}
\OneLiner{4}{ignore the skip setting, obey \prm {mathsurround}}
\OneLiner{5}{disable all spacing around math}
\OneLiner{6}{only apply \lpr {mathsurroundskip} when also spacing}
\OneLiner{7}{only apply \lpr {mathsurroundskip} when no spacing}
\LL
\stoptabulate

\stop

Method six omits the surround glue when there is (x)spacing glue present while
method seven does the opposite, the glue is only applied when there is (x)space
glue present too. Anything more fancy, like checking the begining or end of a
paragraph (or edges of a box) would not be robust anyway. If you want that you
can write a callback that runs over a list and analyzes a paragraph. Actually, in
that case you could also inject glue (or set the properties of a math node)
explicitly. So, these modes are in practice mostly useful for special purposes
and experiments (they originate in a tracker item). Keep in mind that this glue
is part of the math node and not always treated as normal glue: it travels with
the begin and end math nodes. Also, method 6 and 7 will zero the skip related
fields in a node when applicable in the first occasion that checks them
(linebreaking or packaging).

\subsection{Pairwise spacing}

\topicindex {math+spacing}

Besides the parameters mentioned in the previous sections, there are also 64 new
primitives to control the math spacing table (as explained in Chapter~18 of the
\TEX book). The primitive names are a simple matter of combining two math atom
types, but for completeness' sake, here is the whole list:

\starttwocolumns
\startlines
\lpr {Umathordordspacing}
\lpr {Umathordopspacing}
\lpr {Umathordbinspacing}
\lpr {Umathordrelspacing}
\lpr {Umathordopenspacing}
\lpr {Umathordclosespacing}
\lpr {Umathordpunctspacing}
\lpr {Umathordinnerspacing}
\lpr {Umathopordspacing}
\lpr {Umathopopspacing}
\lpr {Umathopbinspacing}
\lpr {Umathoprelspacing}
\lpr {Umathopopenspacing}
\lpr {Umathopclosespacing}
\lpr {Umathoppunctspacing}
\lpr {Umathopinnerspacing}
\lpr {Umathbinordspacing}
\lpr {Umathbinopspacing}
\lpr {Umathbinbinspacing}
\lpr {Umathbinrelspacing}
\lpr {Umathbinopenspacing}
\lpr {Umathbinclosespacing}
\lpr {Umathbinpunctspacing}
\lpr {Umathbininnerspacing}
\lpr {Umathrelordspacing}
\lpr {Umathrelopspacing}
\lpr {Umathrelbinspacing}
\lpr {Umathrelrelspacing}
\lpr {Umathrelopenspacing}
\lpr {Umathrelclosespacing}
\lpr {Umathrelpunctspacing}
\lpr {Umathrelinnerspacing}
\lpr {Umathopenordspacing}
\lpr {Umathopenopspacing}
\lpr {Umathopenbinspacing}
\lpr {Umathopenrelspacing}
\lpr {Umathopenopenspacing}
\lpr {Umathopenclosespacing}
\lpr {Umathopenpunctspacing}
\lpr {Umathopeninnerspacing}
\lpr {Umathcloseordspacing}
\lpr {Umathcloseopspacing}
\lpr {Umathclosebinspacing}
\lpr {Umathcloserelspacing}
\lpr {Umathcloseopenspacing}
\lpr {Umathcloseclosespacing}
\lpr {Umathclosepunctspacing}
\lpr {Umathcloseinnerspacing}
\lpr {Umathpunctordspacing}
\lpr {Umathpunctopspacing}
\lpr {Umathpunctbinspacing}
\lpr {Umathpunctrelspacing}
\lpr {Umathpunctopenspacing}
\lpr {Umathpunctclosespacing}
\lpr {Umathpunctpunctspacing}
\lpr {Umathpunctinnerspacing}
\lpr {Umathinnerordspacing}
\lpr {Umathinneropspacing}
\lpr {Umathinnerbinspacing}
\lpr {Umathinnerrelspacing}
\lpr {Umathinneropenspacing}
\lpr {Umathinnerclosespacing}
\lpr {Umathinnerpunctspacing}
\lpr {Umathinnerinnerspacing}
\stoplines
\stoptwocolumns

These parameters are of type \prm {muskip}, so setting a parameter can be done
like this:

\starttyping
\Umathopordspacing\displaystyle=4mu plus 2mu
\stoptyping

They are all initialized by \type {initex} to the values mentioned in the table
in Chapter~18 of the \TEX book.

Note 1: for ease of use as well as for backward compatibility, \prm {thinmuskip},
\prm {medmuskip} and \prm {thickmuskip} are treated specially. In their case a
pointer to the corresponding internal parameter is saved, not the actual \prm
{muskip} value. This means that any later changes to one of these three
parameters will be taken into account.

Note 2: Careful readers will realise that there are also primitives for the items
marked \type {*} in the \TEX book. These will not actually be used as those
combinations of atoms cannot actually happen, but it seemed better not to break
orthogonality. They are initialized to zero.

\subsection{Skips around display math}

\topicindex {math+spacing}

The injection of \prm {abovedisplayskip} and \prm {belowdisplayskip} is not
symmetrical. An above one is always inserted, also when zero, but the below is
only inserted when larger than zero. Especially the latter makes it sometimes hard
to fully control spacing. Therefore \LUATEX\ comes with a new directive: \lpr
{mathdisplayskipmode}. The following values apply:

\starttabulate[|c|l|]
\DB value  \BC meaning \NC \NR
\TB
\NC 0 \NC normal \TEX\ behaviour \NC \NR
\NC 1 \NC always (same as 0) \NC \NR
\NC 2 \NC only when not zero \NC \NR
\NC 3 \NC never, not even when not zero \NC \NR
\LL
\stoptabulate

\subsection {Nolimit correction}

\topicindex {math+limits}

There are two extra math parameters \lpr {Umathnolimitsupfactor} and \lpr
{Umathnolimitsubfactor} that were added to provide some control over how limits
are spaced (for example the position of super and subscripts after integral
operators). They relate to an extra parameter \lpr {mathnolimitsmode}. The half
corrections are what happens when scripts are placed above and below. The
problem with italic corrections is that officially that correction italic is used
for above|/|below placement while advanced kerns are used for placement at the
right end. The question is: how often is this implemented, and if so, do the
kerns assume correction too. Anyway, with this parameter one can control it.

\starttabulate[|l|ck1|ck1|ck1|ck1|ck1|ck1|]
    \NC
        \NC \mathnolimitsmode0    $\displaystyle\int\nolimits^0_1$
        \NC \mathnolimitsmode1    $\displaystyle\int\nolimits^0_1$
        \NC \mathnolimitsmode2    $\displaystyle\int\nolimits^0_1$
        \NC \mathnolimitsmode3    $\displaystyle\int\nolimits^0_1$
        \NC \mathnolimitsmode4    $\displaystyle\int\nolimits^0_1$
        \NC \mathnolimitsmode8000 $\displaystyle\int\nolimits^0_1$
    \NC \NR
    \TB
    \BC mode
        \NC \tttf 0
        \NC \tttf 1
        \NC \tttf 2
        \NC \tttf 3
        \NC \tttf 4
        \NC \tttf 8000
    \NC \NR
    \BC superscript
        \NC 0
        \NC font
        \NC 0
        \NC 0
        \NC +ic/2
        \NC 0
    \NC \NR
    \BC subscript
        \NC -ic
        \NC font
        \NC 0
        \NC -ic/2
        \NC -ic/2
        \NC 8000ic/1000
    \NC \NR
\stoptabulate

When the mode is set to one, the math parameters are used. This way a macro
package writer can decide what looks best. Given the current state of fonts in
\CONTEXT\ we currently use mode 1 with factor 0 for the superscript and 750 for
the subscripts. Positive values are used for both parameters but the subscript
shifts to the left. A \lpr {mathnolimitsmode} larger that 15 is considered to
be a factor for the subscript correction. This feature can be handy when
experimenting.

\subsection {Math italic mess}

\topicindex {math+italics}

The \lpr {mathitalicsmode} parameter can be set to~1 to force italic correction
before noads that represent some more complex structure (read: everything
that is not an ord, bin, rel, open, close, punct or inner). We show a Cambria
example.

\starttexdefinition Whatever #1
    \NC \type{\mathitalicsmode = #1}
    \NC \mathitalicsmode#1\ruledhbox{$\left|T^1\right|$}
    \NC \mathitalicsmode#1\ruledhbox{$\left|T\right|$}
    \NC \mathitalicsmode#1\ruledhbox{$T+1$}
    \NC \mathitalicsmode#1\ruledhbox{$T{1\over2}$}
    \NC \mathitalicsmode#1\ruledhbox{$T\sqrt{1}$}
    \NC \NR
\stoptexdefinition

\start
    \switchtobodyfont[cambria]
    \starttabulate[|c|c|c|c|c|c|]
        \Whatever{0}%
        \Whatever{1}%
    \stoptabulate
\stop

This kind of parameters relate to the fact that italic correction in \OPENTYPE\
math is bound to fuzzy rules. So, control is the solution.

\subsection {Script and kerning}

\topicindex {math+kerning}
\topicindex {math+scripts}

If you want to typeset text in math macro packages often provide something \type
{\text} which obeys the script sizes. As the definition can be anything there is
a good chance that the kerning doesn't come out well when used in a script. Given
that the first glyph ends up in a \prm {hbox} we have some control over this.
And, as a bonus we also added control over the normal sublist kerning. The \lpr
{mathscriptboxmode} parameter defaults to~1.

\starttabulate[|c|l|]
\DB value     \BC meaning \NC \NR
\TB
\NC \type {0} \NC forget about kerning \NC \NR
\NC \type {1} \NC kern math sub lists with a valid glyph \NC \NR
\NC \type {2} \NC also kern math sub boxes that have a valid glyph \NC \NR
\NC \type {2} \NC only kern math sub boxes with a boundary node present\NC \NR
\LL
\stoptabulate

Here we show some examples. Of course this doesn't solve all our problems, if
only because some fonts have characters with bounding boxes that compensate for
italics, while other fonts can lack kerns.

\startbuffer[1]
    $T_{\tf fluff}$
\stopbuffer

\startbuffer[2]
    $T_{\text{fluff}}$
\stopbuffer

\startbuffer[3]
    $T_{\text{\boundary1 fluff}}$
\stopbuffer

\unexpanded\def\Show#1#2#3%
  {\doifelsenothing{#3}
     {\small\tx\typeinlinebuffer[#1]}
     {\doifelse{#3}{-}
        {\small\bf\tt mode #2}
        {\switchtobodyfont[#3]\showfontkerns\showglyphs\mathscriptboxmode#2\relax\inlinebuffer[#1]}}}

\starttabulate[|lBT|c|c|c|c|c|]
    \NC          \NC \Show{1}{0}{}         \NC\Show{1}{1}{}         \NC \Show{2}{1}{}         \NC \Show{2}{2}{}         \NC \Show{3}{3}{}         \NC \NR
    \NC          \NC \Show{1}{0}{-}        \NC\Show{1}{1}{-}        \NC \Show{2}{1}{-}        \NC \Show{2}{2}{-}        \NC \Show{3}{3}{-}        \NC \NR
    \NC modern   \NC \Show{1}{0}{modern}   \NC\Show{1}{1}{modern}   \NC \Show{2}{1}{modern}   \NC \Show{2}{2}{modern}   \NC \Show{3}{3}{modern}   \NC \NR
    \NC lucidaot \NC \Show{1}{0}{lucidaot} \NC\Show{1}{1}{lucidaot} \NC \Show{2}{1}{lucidaot} \NC \Show{2}{2}{lucidaot} \NC \Show{3}{3}{lucidaot} \NC \NR
    \NC pagella  \NC \Show{1}{0}{pagella}  \NC\Show{1}{1}{pagella}  \NC \Show{2}{1}{pagella}  \NC \Show{2}{2}{pagella}  \NC \Show{3}{3}{pagella}  \NC \NR
    \NC cambria  \NC \Show{1}{0}{cambria}  \NC\Show{1}{1}{cambria}  \NC \Show{2}{1}{cambria}  \NC \Show{2}{2}{cambria}  \NC \Show{3}{3}{cambria}  \NC \NR
    \NC dejavu   \NC \Show{1}{0}{dejavu}   \NC\Show{1}{1}{dejavu}   \NC \Show{2}{1}{dejavu}   \NC \Show{2}{2}{dejavu}   \NC \Show{3}{3}{dejavu}   \NC \NR
\stoptabulate

Kerning between a character subscript is controlled by \lpr {mathscriptcharmode}
which also defaults to~1.

Here is another example. Internally we tag kerns as italic kerns or font kerns
where font kerns result from the staircase kern tables. In 2018 fonts like Latin
Modern and Pagella rely on cheats with the boundingbox, Cambria uses staircase
kerns and Lucida a mixture. Depending on how fonts evolve we might add some more
control over what one can turn on and off.

\def\MathSample#1#2#3%
  {\NC
   #1 \NC
   #2 \NC
   \showglyphdata \switchtobodyfont[#2,17.3pt]$#3T_{f}$         \NC
   \showglyphdata \switchtobodyfont[#2,17.3pt]$#3\gamma_{e}$    \NC
   \showglyphdata \switchtobodyfont[#2,17.3pt]$#3\gamma_{ee}$   \NC
   \showglyphdata \switchtobodyfont[#2,17.3pt]$#3T_{\tf fluff}$ \NC
   \NR}

\starttabulate[|Tl|Tl|l|l|l|l|]
    \FL
    \MathSample{normal}{modern}  {\mr}
    \MathSample{}      {pagella} {\mr}
    \MathSample{}      {cambria} {\mr}
    \MathSample{}      {lucidaot}{\mr}
    \ML
    \MathSample{bold}  {modern}  {\mb}
    \MathSample{}      {pagella} {\mb}
    \MathSample{}      {cambria} {\mb}
    \MathSample{}      {lucidaot}{\mb}
    \LL
\stoptabulate

\subsection{Fixed scripts}

We have three parameters that are used for this fixed anchoring:

\starttabulate[|c|l|]
\DB parameter \BC register \NC \NR
\NC $d$ \NC \lpr {Umathsubshiftdown}    \NC \NR
\NC $u$ \NC \lpr {Umathsupshiftup}      \NC \NR
\NC $s$ \NC \lpr {Umathsubsupshiftdown} \NC \NR
\LL
\stoptabulate

When we set \lpr {mathscriptsmode} to a value other than zero these are used
for calculating fixed positions. This is something that is needed for instance
for chemistry. You can manipulate the mentioned variables to achieve different
effects.

\def\SampleMath#1%
  {$\mathscriptsmode#1\mathupright CH_2 + CH^+_2 + CH^2_2$}

\starttabulate[|c|c|c|p|]
\DB mode \BC down          \BC up            \BC example        \NC \NR
\TB
\NC 0    \NC dynamic       \NC dynamic       \NC \SampleMath{0} \NC \NR
\NC 1    \NC $d$           \NC $u$           \NC \SampleMath{1} \NC \NR
\NC 2    \NC $s$           \NC $u$           \NC \SampleMath{2} \NC \NR
\NC 3    \NC $s$           \NC $u + s - d$   \NC \SampleMath{3} \NC \NR
\NC 4    \NC $d + (s-d)/2$ \NC $u + (s-d)/2$ \NC \SampleMath{4} \NC \NR
\NC 5    \NC $d$           \NC $u + s - d$   \NC \SampleMath{5} \NC \NR
\LL
\stoptabulate

The value of this parameter obeys grouping but applies to the whole current
formula.

% if needed we can put the value in stylenodes but maybe more should go there

\subsection{Penalties: \lpr {mathpenaltiesmode}}

\topicindex {math+penalties}

Only in inline math penalties will be added in a math list. You can force
penalties (also in display math) by setting:

\starttyping
\mathpenaltiesmode = 1
\stoptyping

This primnitive is not really needed in \LUATEX\ because you can use the callback
\cbk {mlist_to_hlist} to force penalties by just calling the regular routine
with forced penalties. However, as part of opening up and control this primitive
makes sense. As a bonus we also provide two extra penalties:

\starttyping
\prebinoppenalty = -100 % example value
\prerelpenalty   =  900 % example value
\stoptyping

They default to inifinite which signals that they don't need to be inserted. When
set they are injected before a binop or rel noad. This is an experimental feature.

\subsection{Equation spacing: \lpr {matheqnogapstep}}

By default \TEX\ will add one quad between the equation and the number. This is
hard coded. A new primitive can control this:

\startsyntax
\matheqnogapstep = 1000
\stopsyntax

Because a math quad from the math text font is used instead of a dimension, we
use a step to control the size. A value of zero will suppress the gap. The step
is divided by 1000 which is the usual way to mimmick floating point factors in
\TEX.

\stopsection

\startsection[title={Math constructs}]

\subsection {Unscaled fences}

\topicindex {math+fences}

The \lpr {mathdelimitersmode} primitive is experimental and deals with the
following (potential) problems. Three bits can be set. The first bit prevents an
unwanted shift when the fence symbol is not scaled (a cambria side effect). The
second bit forces italic correction between a preceding character ordinal and the
fenced subformula, while the third bit turns that subformula into an ordinary so
that the same spacing applies as with unfenced variants. Here we show Cambria
(with \lpr {mathitalicsmode} enabled).

\starttexdefinition Whatever #1
    \NC \type{\mathdelimitersmode = #1}
    \NC \mathitalicsmode1\mathdelimitersmode#1\ruledhbox{\showglyphs\showfontkerns\showfontitalics$f(x)$}
    \NC \mathitalicsmode1\mathdelimitersmode#1\ruledhbox{\showglyphs\showfontkerns\showfontitalics$f\left(x\right)$}
    \NC \NR
\stoptexdefinition

\start
    \switchtobodyfont[cambria]
    \starttabulate[|l|l|l|]
        \Whatever{0}\Whatever{1}\Whatever{2}\Whatever{3}%
        \Whatever{4}\Whatever{5}\Whatever{6}\Whatever{7}%
    \stoptabulate
\stop

So, when set to 7 fenced subformulas with unscaled delimiters come out the same
as unfenced ones. This can be handy for cases where one is forced to use \prm
{left} and \prm {right} always because of unpredictable content. As said, it's an
experimental feature (which somehow fits in the exceptional way fences are dealt
with in the engine). The full list of flags is given in the next table:

\starttabulate[|c|l|]
\DB value  \BC meaning \NC \NR
\TB
\NC \type{"01} \NC don't apply the usual shift \NC \NR
\NC \type{"02} \NC apply italic correction when possible \NC \NR
\NC \type{"04} \NC force an ordinary subformula \NC \NR
\NC \type{"08} \NC no shift when a base character \NC \NR
\NC \type{"10} \NC only shift when an extensible \NC \NR
\LL
\stoptabulate

The effect can depend on the font (and for Cambria one can use for instance \type {"16}).

\subsection[mathacc]{Accent handling}

\topicindex {math+accents}

\LUATEX\ supports both top accents and bottom accents in math mode, and math
accents stretch automatically (if this is supported by the font the accent comes
from, of course). Bottom and combined accents as well as fixed-width math accents
are controlled by optional keywords following \lpr {Umathaccent}.

The keyword \type {bottom} after \lpr {Umathaccent} signals that a bottom accent
is needed, and the keyword \type {both} signals that both a top and a bottom
accent are needed (in this case two accents need to be specified, of course).

Then the set of three integers defining the accent is read. This set of integers
can be prefixed by the \type {fixed} keyword to indicate that a non-stretching
variant is requested (in case of both accents, this step is repeated).

A simple example:

\starttyping
\Umathaccent both fixed 0 0 "20D7 fixed 0 0 "20D7 {example}
\stoptyping

If a math top accent has to be placed and the accentee is a character and has a
non-zero \type {top_accent} value, then this value will be used to place the
accent instead of the \prm {skewchar} kern used by \TEX82.

The \type {top_accent} value represents a vertical line somewhere in the
accentee. The accent will be shifted horizontally such that its own \type
{top_accent} line coincides with the one from the accentee. If the \type
{top_accent} value of the accent is zero, then half the width of the accent
followed by its italic correction is used instead.

The vertical placement of a top accent depends on the \type {x_height} of the
font of the accentee (as explained in the \TEX book), but if a value turns out
to be zero and the font had a \type {MathConstants} table, then \type
{AccentBaseHeight} is used instead.

The vertical placement of a bottom accent is straight below the accentee, no
correction takes place.

Possible locations are \type {top}, \type {bottom}, \type {both} and \type
{center}. When no location is given \type {top} is assumed. An additional
parameter \nod {fraction} can be specified followed by a number; a value of for
instance 1200 means that the criterium is 1.2 times the width of the nucleus. The
fraction only applies to the stepwise selected shapes and is mostly meant for the
\type {overlay} location. It also works for the other locations but then it
concerns the width.

\subsection{Radical extensions}

\topicindex {math+radicals}

The new primitive \lpr {Uroot} allows the construction of a radical noad
including a degree field. Its syntax is an extension of \lpr {Uradical}:

\starttyping
\Uradical <fam integer> <char integer> <radicand>
\Uroot    <fam integer> <char integer> <degree> <radicand>
\stoptyping

The placement of the degree is controlled by the math parameters \lpr
{Umathradicaldegreebefore}, \lpr {Umathradicaldegreeafter}, and \lpr
{Umathradicaldegreeraise}. The degree will be typeset in \prm
{scriptscriptstyle}.

\subsection{Super- and subscripts}

The character fields in a \LUA|-|loaded \OPENTYPE\ math font can have a \quote
{mathkern} table. The format of this table is the same as the \quote {mathkern}
table that is returned by the \type {fontloader} library, except that all height
and kern values have to be specified in actual scaled points.

When a super- or subscript has to be placed next to a math item, \LUATEX\ checks
whether the super- or subscript and the nucleus are both simple character items.
If they are, and if the fonts of both character items are \OPENTYPE\ fonts (as
opposed to legacy \TEX\ fonts), then \LUATEX\ will use the \OPENTYPE\ math
algorithm for deciding on the horizontal placement of the super- or subscript.

This works as follows:

\startitemize
    \startitem
        The vertical position of the script is calculated.
    \stopitem
    \startitem
        The default horizontal position is flat next to the base character.
    \stopitem
    \startitem
        For superscripts, the italic correction of the base character is added.
    \stopitem
    \startitem
        For a superscript, two vertical values are calculated: the bottom of the
        script (after shifting up), and the top of the base. For a subscript, the two
        values are the top of the (shifted down) script, and the bottom of the base.
    \stopitem
    \startitem
        For each of these two locations:
        \startitemize
            \startitem
                find the math kern value at this height for the base (for a subscript
                placement, this is the bottom_right corner, for a superscript
                placement the top_right corner)
            \stopitem
            \startitem
                find the math kern value at this height for the script (for a
                subscript placement, this is the top_left corner, for a superscript
                placement the bottom_left corner)
            \stopitem
            \startitem
                add the found values together to get a preliminary result.
            \stopitem
        \stopitemize
    \stopitem
    \startitem
        The horizontal kern to be applied is the smallest of the two results from
        previous step.
    \stopitem
\stopitemize

The math kern value at a specific height is the kern value that is specified by the
next higher height and kern pair, or the highest one in the character (if there is no
value high enough in the character), or simply zero (if the character has no math kern
pairs at all).

\subsection{Scripts on extensibles}

\topicindex {math+scripts}
\topicindex {math+delimiters}
\topicindex {math+extensibles}

The primitives \lpr {Uunderdelimiter} and \lpr {Uoverdelimiter} allow the
placement of a subscript or superscript on an automatically extensible item and
\lpr {Udelimiterunder} and \lpr {Udelimiterover} allow the placement of an
automatically extensible item as a subscript or superscript on a nucleus. The
input:

% these produce radical noads .. in fact the code base has the numbers wrong for
% quite a while, so no one seems to use this

\startbuffer
$\Uoverdelimiter  0 "2194 {\hbox{\strut  overdelimiter}}$
$\Uunderdelimiter 0 "2194 {\hbox{\strut underdelimiter}}$
$\Udelimiterover  0 "2194 {\hbox{\strut  delimiterover}}$
$\Udelimiterunder 0 "2194 {\hbox{\strut delimiterunder}}$
\stopbuffer

\typebuffer will render this:

\blank \startnarrower \getbuffer \stopnarrower \blank

The vertical placements are controlled by \lpr {Umathunderdelimiterbgap}, \lpr
{Umathunderdelimitervgap}, \lpr {Umathoverdelimiterbgap}, and \lpr
{Umathoverdelimitervgap} in a similar way as limit placements on large operators.
The superscript in \lpr {Uoverdelimiter} is typeset in a suitable scripted style,
the subscript in \lpr {Uunderdelimiter} is cramped as well.

These primitives accepts an option \type {width} specification. When used the
also optional keywords \type {left}, \type {middle} and \type {right} will
determine what happens when a requested size can't be met (which can happen when
we step to successive larger variants).

An extra primitive \lpr {Uhextensible} is available that can be used like this:

\startbuffer
$\Uhextensible width 10cm 0 "2194$
\stopbuffer

\typebuffer This will render this:

\blank \startnarrower \getbuffer \stopnarrower \blank

Here you can also pass options, like:

\startbuffer
$\Uhextensible width 1pt middle 0 "2194$
\stopbuffer

\typebuffer This gives:

\blank \startnarrower \getbuffer \stopnarrower \blank

\LUATEX\ internally uses a structure that supports \OPENTYPE\ \quote
{MathVariants} as well as \TFM\ \quote {extensible recipes}. In most cases where
font metrics are involved we have a different code path for traditional fonts end
\OPENTYPE\ fonts.

Sometimes you might want to act upon the size of a delimiter, something that is
not really possible because of the fact that they are calculated {\em after} most
has been typeset already. In the following example the all|-|zero specification
is the trigger to make a fake box with the last delimiter dimensions and shift.
It's an ugly hack but its relative simple and not intrusive implementation has no
side effects. Any other heuristic solution would not satisfy possible demands
anyway. Here is a rather low level example:

\startbuffer
\startformula
\Uleft  \Udelimiter 5 0 "222B
\frac{\frac{a}{b}}{\frac{c}{d}}
\Uright \Udelimiter 5 0 "222B
\kern-2\fontcharwd\textfont0 "222B
\mathlimop{\Uvextensible \Udelimiter 0 0 0}_1^2 x
\stopformula
\stopbuffer

\typebuffer

The last line, by passing zero values, results in a fake operator that has the
dimensions of the previous delimiter. We can then backtrack over the (presumed)
width and the two numbers become limit operators. As said, it's not pretty but it
works.

\subsection{Fractions}

\topicindex {math+fractions}

The \prm {abovewithdelims} command accepts a keyword \type {exact}. When issued
the extra space relative to the rule thickness is not added. One can of course
use the \type {\Umathfraction..gap} commands to influence the spacing. Also the
rule is still positioned around the math axis.

\starttyping
$$ { {a} \abovewithdelims() exact 4pt {b} }$$
\stoptyping

The math parameter table contains some parameters that specify a horizontal and
vertical gap for skewed fractions. Of course some guessing is needed in order to
implement something that uses them. And so we now provide a primitive similar to the
other fraction related ones but with a few options so that one can influence the
rendering. Of course a user can also mess around a bit with the parameters
\lpr {Umathskewedfractionhgap} and \lpr {Umathskewedfractionvgap}.

The syntax used here is:

\starttyping
{ {1} \Uskewed / <options> {2} }
{ {1} \Uskewedwithdelims / () <options> {2} }
\stoptyping

where the options can be \type {noaxis} and \type {exact}. By default we add half
the axis to the shifts and by default we zero the width of the middle character.
For Latin Modern the result looks as follows:

\def\ShowA#1#2#3{$x + { {#1} \Uskewed           /    #3 {#2} } + x$}
\def\ShowB#1#2#3{$x + { {#1} \Uskewedwithdelims / () #3 {#2} } + x$}

\start
    \switchtobodyfont[modern]
    \starttabulate[||||||]
        \NC \NC
            \ShowA{a}{b}{} \NC
            \ShowA{1}{2}{} \NC
            \ShowB{a}{b}{} \NC
            \ShowB{1}{2}{} \NC
        \NR
        \NC \type{exact} \NC
            \ShowA{a}{b}{exact} \NC
            \ShowA{1}{2}{exact} \NC
            \ShowB{a}{b}{exact} \NC
            \ShowB{1}{2}{exact} \NC
        \NR
        \NC \type{noaxis} \NC
            \ShowA{a}{b}{noaxis} \NC
            \ShowA{1}{2}{noaxis} \NC
            \ShowB{a}{b}{noaxis} \NC
            \ShowB{1}{2}{noaxis} \NC
        \NR
        \NC \type{exact noaxis} \NC
            \ShowA{a}{b}{exact noaxis} \NC
            \ShowA{1}{2}{exact noaxis} \NC
            \ShowB{a}{b}{exact noaxis} \NC
            \ShowB{1}{2}{exact noaxis} \NC
        \NR
    \stoptabulate
\stop

The keyword \type {norule} will hide the rule with the above variants while
keeping the rule related spacing.

\subsection {Delimiters: \type{\Uleft}, \prm {Umiddle} and \prm {Uright}}

\topicindex {math+delimiters}

Normally you will force delimiters to certain sizes by putting an empty box or
rule next to it. The resulting delimiter will either be a character from the
stepwise size range or an extensible. The latter can be quite differently
positioned than the characters as it depends on the fit as well as the fact if
the used characters in the font have depth or height. Commands like (plain \TEX
s) \type {\big} need use this feature. In \LUATEX\ we provide a bit more control
by three variants that support optional parameters \type {height}, \type {depth}
and \type {axis}. The following example uses this:

\startbuffer
\Uleft   height 30pt depth 10pt      \Udelimiter "0 "0 "000028
\quad x\quad
\Umiddle height 40pt depth 15pt      \Udelimiter "0 "0 "002016
\quad x\quad
\Uright  height 30pt depth 10pt      \Udelimiter "0 "0 "000029
\quad \quad \quad
\Uleft   height 30pt depth 10pt axis \Udelimiter "0 "0 "000028
\quad x\quad
\Umiddle height 40pt depth 15pt axis \Udelimiter "0 "0 "002016
\quad x\quad
\Uright  height 30pt depth 10pt axis \Udelimiter "0 "0 "000029
\stopbuffer

\typebuffer

\startlinecorrection
\ruledhbox{\mathematics{\getbuffer}}
\stoplinecorrection

The keyword \type {exact} can be used as directive that the real dimensions
should be applied when the criteria can't be met which can happen when we're
still stepping through the successively larger variants. When no dimensions are
given the \type {noaxis} command can be used to prevent shifting over the axis.

You can influence the final class with the keyword \type {class} which will
influence the spacing. The numbers are the same as for character classes.

\stopsection

\startsection[title={Extracting values}]

\subsection{Codes}

\topicindex {math+codes}

You can extract the components of a math character. Say that we have defined:

\starttyping
\Umathcode 1 2 3 4
\stoptyping

then

\starttyping
[\Umathcharclass1] [\Umathcharfam1] [\Umathcharslot1]
\stoptyping

will return:

\starttyping
[2] [3] [4]
\stoptyping

These commands are provides as convenience. Before they come available you could
do the following:

\starttyping
\def\Umathcharclass{\directlua{tex.print(tex.getmathcode(token.scan_int())[1])}}
\def\Umathcharfam  {\directlua{tex.print(tex.getmathcode(token.scan_int())[2])}}
\def\Umathcharslot {\directlua{tex.print(tex.getmathcode(token.scan_int())[3])}}
\stoptyping

\subsection {Last lines}

\topicindex {math+last line}

There is a new primitive to control the overshoot in the calculation of the
previous line in mid|-|paragraph display math. The default value is 2 times
the em width of the current font:

\starttyping
\predisplaygapfactor=2000
\stoptyping

If you want to have the length of the last line independent of math i.e.\ you don't
want to revert to a hack where you insert a fake display math formula in order to
get the length of the last line, the following will often work too:

\starttyping
\def\lastlinelength{\dimexpr
    \directlua {tex.sprint (
        (nodes.dimensions(node.tail(tex.lists.page_head).list))
    )}sp
\relax}
\stoptyping

\stopsection

\startsection[title={Math mode}]

\subsection {Verbose versions of single|-|character math commands}

\topicindex {math+styles}

\LUATEX\ defines six new primitives that have the same function as
\type {^}, \type {_}, \type {$}, and \type {$$}:

\starttabulate[|l|l|]
\DB primitive                  \BC explanation \NC \NR
\TB
\NC \lpr {Usuperscript}      \NC duplicates the functionality of \type {^} \NC \NR
\NC \lpr {Usubscript}        \NC duplicates the functionality of \type {_} \NC \NR
\NC \lpr {Ustartmath}        \NC duplicates the functionality of \type {$}, % $
                                   when used in non-math mode. \NC \NR
\NC \lpr {Ustopmath}         \NC duplicates the functionality of \type {$}, % $
                                   when used in inline math mode. \NC \NR
\NC \lpr {Ustartdisplaymath} \NC duplicates the functionality of \type {$$}, % $$
                                   when used in non-math mode. \NC \NR
\NC \lpr {Ustopdisplaymath}  \NC duplicates the functionality of \type {$$}, % $$
                                   when used in display math mode. \NC \NR
\LL
\stoptabulate

The \lpr {Ustopmath} and \lpr {Ustopdisplaymath} primitives check if the current
math mode is the correct one (inline vs.\ displayed), but you can freely intermix
the four mathon|/|mathoff commands with explicit dollar sign(s).

\subsection{Script commands \lpr {Unosuperscript} and \lpr {Unosubscript}}

\topicindex {math+styles}
\topicindex {math+scripts}

These two commands result in super- and subscripts but with the current style (at the
time of rendering). So,

\startbuffer[script]
$
    x\Usuperscript  {1}\Usubscript  {2} =
    x\Unosuperscript{1}\Unosubscript{2} =
    x\Usuperscript  {1}\Unosubscript{2} =
    x\Unosuperscript{1}\Usubscript  {2}
$
\stopbuffer

\typebuffer[script]

results in \inlinebuffer[script].

\subsection{Allowed math commands in non|-|math modes}

\topicindex {math+text}
\topicindex {text+math}

The commands \prm {mathchar}, and \lpr {Umathchar} and control sequences that are
the result of \prm {mathchardef} or \lpr {Umathchardef} are also acceptable in
the horizontal and vertical modes. In those cases, the \prm {textfont} from the
requested math family is used.

% \startsection[title={Math todo}]
%
% The following items are still todo.
%
% \startitemize
% \startitem
%     Pre-scripts.
% \stopitem
% \startitem
%     Multi-story stacks.
% \stopitem
% \startitem
%     Flattened accents for high characters (maybe).
% \stopitem
% \startitem
%     Better control over the spacing around displays and handling of equation numbers.
% \stopitem
% \startitem
%     Support for multi|-|line displays using \MATHML\ style alignment points.
% \stopitem
% \stopitemize
%
% \stopsection

\stopsection

\startsection[title={Goodies}]

\subsection {Flattening: \lpr {mathflattenmode}}

\topicindex {math+flattening}

The \TEX\ math engine collapses \type {ord} noads without sub- and superscripts
and a character as nucleus. and which has the side effect that in \OPENTYPE\ mode
italic corrections are applied (given that they are enabled).

\startbuffer[sample]
\switchtobodyfont[modern]
$V \mathbin{\mathbin{v}} V$\par
$V \mathord{\mathord{v}} V$\par
\stopbuffer

\typebuffer[sample]

This renders as:

\blank \start \mathflattenmode\plusone \getbuffer[sample] \stop \blank

When we set \lpr {mathflattenmode} to 31 we get:

\blank \start \mathflattenmode\numexpr1+2+4+8+16\relax \getbuffer[sample] \stop \blank

When you see no difference, then the font probably has the proper character
dimensions and no italic correction is needed. For Latin Modern (at least till
2018) there was a visual difference. In that respect this parameter is not always
needed unless of course you want efficient math lists anyway.

You can influence flattening by adding the appropriate number to the value of the
mode parameter. The default value is~1.

\starttabulate[|Tc|c|]
\DB mode \BC class \NC \NR
\TB
\NC  1   \NC ord   \NC \NR
\NC  2   \NC bin   \NC \NR
\NC  4   \NC rel   \NC \NR
\NC  8   \NC punct \NC \NR
\NC 16   \NC inner \NC \NR
\LL
\stoptabulate

\subsection {Less Tracing}

\topicindex {math+tracing}

Because there are quite some math related parameters and values, it is possible
to limit tracing. Only when \type {tracingassigns} and|/|or \type
{tracingrestores} are set to~2 or more they will be traced.

\subsection {Math options with \lpr {mathoption}}

The logic in the math engine is rather complex and there are often no universal
solutions (read: what works out well for one font, fails for another). Therefore
some variations in the implementation are driven by parameters (modes). In
addition there is a new primitive \lpr {mathoption} which will be used for
testing. Don't rely on any option to be there in a production version as they are
meant for development.

This option was introduced for testing purposes when the math engine got split
code paths and it forces the engine to treat new fonts as old ones with respect
to italic correction etc. There are no guarantees given with respect to the final
result and unexpected side effects are not seen as bugs as they relate to font
properties. Ther eis currently only one option:

\startbuffer
\mathoption old 1
\stopbuffer

The \type {oldmath} boolean flag in the \LUA\ font table is the official way to
force old treatment as it's bound to fonts. Like with all options we may
temporarily introduce with this command this feature is not meant for production.

% % obsolete:
%
% \subsubsection {\type {\mathoption noitaliccompensation}}
%
% This option compensates placement for characters with a built|-|in italic
% correction.
%
% \startbuffer
% {\showboxes\int}\quad
% {\showboxes\int_{|}^{|}}\quad
% {\showboxes\int\limits_{|}^{|}}
% \stopbuffer
%
% \typebuffer
%
% Gives (with computer modern that has such italics):
%
% \startlinecorrection[blank]
%     \switchtobodyfont[modern]
%     \startcombination[nx=2,ny=2,distance=5em]
%         {\mathoption noitaliccompensation 0\relax \mathematics{\getbuffer}}
%             {\nohyphens\type{0:inline}}
%         {\mathoption noitaliccompensation 0\relax \mathematics{\displaymath\getbuffer}}
%             {\nohyphens\type{0:display}}
%         {\mathoption noitaliccompensation 1\relax \mathematics{\getbuffer}}
%             {\nohyphens\type{1:inline}}
%         {\mathoption noitaliccompensation 1\relax \mathematics{\displaymath\getbuffer}}
%             {\nohyphens\type{1:display}}
%     \stopcombination
% \stoplinecorrection

% % obsolete:
%
% \subsubsection {\type {\mathoption nocharitalic}}
%
% When two characters follow each other italic correction can interfere. The
% following example shows what this option does:
%
% \startbuffer
% \catcode"1D443=11
% \catcode"1D444=11
% \catcode"1D445=11
% P( PP PQR
% \stopbuffer
%
% \typebuffer
%
% Gives (with computer modern that has such italics):
%
% \startlinecorrection[blank]
%     \switchtobodyfont[modern]
%     \startcombination[nx=2,ny=2,distance=5em]
%         {\mathoption nocharitalic 0\relax \mathematics{\getbuffer}}
%             {\nohyphens\type{0:inline}}
%         {\mathoption nocharitalic 0\relax \mathematics{\displaymath\getbuffer}}
%             {\nohyphens\type{0:display}}
%         {\mathoption nocharitalic 1\relax \mathematics{\getbuffer}}
%             {\nohyphens\type{1:inline}}
%         {\mathoption nocharitalic 1\relax \mathematics{\displaymath\getbuffer}}
%             {\nohyphens\type{1:display}}
%     \stopcombination
% \stoplinecorrection

% % obsolete:
%
% \subsubsection {\type {\mathoption useoldfractionscaling}}
%
% This option has been introduced as solution for tracker item 604 for fuzzy cases
% around either or not present fraction related settings for new fonts.

\stopsection

\stopchapter

\stopcomponent
