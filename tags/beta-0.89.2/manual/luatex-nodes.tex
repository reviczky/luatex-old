\environment luatex-style
\environment luatex-logos

\startcomponent luatex-nodes

\startchapter[reference=nodes,title={Nodes}]

\section{\LUA\ node representation}

\TEX's nodes are represented in \LUA\ as userdata object with a variable set of
fields. In the following syntax tables, such the type of such a userdata object
is represented as \syntax {<node>}.

The current return value of \type {node.types()} is:
\startluacode
    for id, name in table.sortedhash(node.types()) do
        context.type(name)
        context(" (%s), ",id)
    end
    context.removeunwantedspaces()
    context.removepunctuation()
\stopluacode
. % period

The \type {\lastnodetype} primitive is \ETEX\ compliant. The valid range is still
$[-1,15]$ and glyph nodes (formerly known as char nodes) have number~0 while
ligature nodes are mapped to~7. That way macro packages can use the same symbolic
names as in traditional \ETEX. Keep in mind that these \ETEX\ node numbers are
different from the real internal ones and that there are more \ETEX\ node types
than~15.

You can ask for a list of fields with the \type {node.fields} (which takes an id)
and for valid subtypes with \type {node.subtypes} (which takes a string because
eventually we might support more used enumerations) .

\subsection{Auxiliary items}

A few node|-|typed userdata objects do not occur in the \quote {normal} list of
nodes, but can be pointed to from within that list. They are not quite the same
as regular nodes, but it is easier for the library routines to treat them as if
they were.

\subsubsection{attribute_list and attribute items}

The newly introduced attribute registers are non|-|trivial, because the value
that is attached to a node is essentially a sparse array of key|-|value pairs.

It is generally easiest to deal with attribute lists and attributes by using the
dedicated functions in the \type {node} library, but for completeness, here is
the low|-|level interface.

An \type {attribute_list} item is used as a head pointer for a list of attribute
items. It has only one user-visible field:

\starttabulate[|lT|l|p|]
\NC \ssbf field \NC \bf type \NC \bf explanation \NC \NR
\NC next \NC \syntax{<node>} \NC
    pointer to the first attribute
\NC \NR
\stoptabulate

A normal node's attribute field will point to an item of type \type
{attribute_list}, and the \type {next} field in that item will point to the first
defined \quote {attribute} item, whose \type {next} will point to the second
\quote {attribute} item, etc.

Valid fields in \type {attribute} items:

\starttabulate[|lT|l|p|]
\NC \ssbf field \NC \bf type        \NC \bf explanation \NC \NR
\NC next        \NC \syntax{<node>} \NC pointer to the next attribute \NC \NR
\NC number      \NC number          \NC the attribute type id \NC \NR
\NC value       \NC number          \NC the attribute value \NC \NR
\stoptabulate

As mentioned it's better to use the official helpers rather than edit these
fields directly. For instance the \type {prev} field is used for other purposes
and there is no double linked list.

\subsection{Main text nodes}

These are the nodes that comprise actual typesetting commands.

A few fields are present in all nodes regardless of their type, these are:

\starttabulate[|lT|l|p|]
\NC \ssbf field \NC \bf type        \NC \bf explanation \NC \NR
\NC next        \NC \syntax{<node>} \NC the next node in a list, or nil \NC \NR
\NC id          \NC number          \NC the node's type (\type {id}) number \NC \NR
\NC subtype     \NC number          \NC the node \type {subtype} identifier \NC \NR
\stoptabulate

The \type {subtype} is sometimes just a stub entry. Not all nodes actually use
the \type {subtype}, but this way you can be sure that all nodes accept it as a
valid field name, and that is often handy in node list traversal. In the
following tables \type {next} and \type {id} are not explicitly mentioned.

Besides these three fields, almost all nodes also have an \type {attr} field, and
there is a also a field called \type {prev}. That last field is always present,
but only initialized on explicit request: when the function \type {node.slide()}
is called, it will set up the \type {prev} fields to be a backwards pointer in
the argument node list.

\subsubsection{hlist nodes}

Valid fields: \showfields{hlist}\crlf
Id: \showid{hlist}

\starttabulate[|lT|l|p|]
\NC \ssbf field \NC \bf type        \NC \bf explanation \NC \NR
\NC subtype     \NC number          \NC \type {0} = unknown origin,
                                        \type {1} = created by linebreaking,
                                        \type {2} = explicit box command,
                                        \type {3} = paragraph indentation box,
                                        \type {4} = alignment column or row,
                                        \type {5} = alignment cell
                                        \type {6} = equation
                                        \type {7} = equation number \NC \NR
\NC attr        \NC \syntax{<node>} \NC The head of the associated attribute
                                        list \NC \NR
\NC width       \NC number          \NC \NC \NR
\NC height      \NC number          \NC \NC \NR
\NC depth       \NC number          \NC \NC \NR
\NC shift       \NC number          \NC a displacement perpendicular to the
                                        character progression direction \NC \NR
\NC glue_order  \NC number          \NC a number in the range $[0,4]$, indicating
                                        the glue order \NC \NR
\NC glue_set    \NC number          \NC the calculated glue ratio \NC \NR
\NC glue_sign   \NC number          \NC \type {0} = normal,
                                        \type {1} = stretching,
                                        \type {2} = shrinking \NC \NR
\NC head        \NC \syntax{<node>} \NC the first node of the body of this
                                        list \NC \NR
\NC dir         \NC string          \NC the direction of this box,
                                        see~\in[dirnodes] \NC \NR
\stoptabulate

A warning: never assign a node list to the \type {head} field unless you are sure
its internal link structure is correct, otherwise an error may result.

Note: the new field name \type {head} was introduced in 0.65 to replace the old
name \type {list}. Use of the name \type {list} is now deprecated, but it will
stay available until at least version 0.80.

\subsubsection{vlist nodes}

Valid fields: As for hlist, except that \quote {shift} is a displacement
perpendicular to the line progression direction, and \quote {subtype} only has
subtypes~0, 4, and~5.

\subsubsection{rule nodes}

\subsubsubsection{normal rules}

Valid fields: \showfields{rule}\crlf
Id: \showid{rule}

We have three subtypes. Subtype~0 is just a normal rule, a rectangle
filled with ink. Subtype~1 is a reusable box, while subtype_2 is an
image.

\starttabulate[|lT|l|p|]
\NC \ssbf field \NC \bf type        \NC \bf explanation \NC \NR
\NC subtype     \NC number          \NC 0 upto 3 \NC \NR
\NC attr        \NC \syntax{<node>} \NC \NC \NR
\NC width       \NC number          \NC the width of the rule; the special value
                                        $-1073741824$ is used for \quote
                                        {running} glue dimensions \NC \NR
\NC height      \NC number          \NC the height of the rule (can be
                                        negative) \NC \NR
\NC depth       \NC number          \NC the depth of the rule (can be
                                        negative) \NC \NR
\NC dir         \NC string          \NC the direction of this rule,
                                        see~\in[dirnodes] \NC \NR
\NC index       \NC number          \NC an optional index that can be referred
                                        to (only for subtypes 1 and~2 and
                                        backend specific). \NC \NR
\stoptabulate

The subtypes 1 and~2 replace the xform and ximage whatsits and in nodelists they
behave like rules of subtype_0 when it comes to dimensions. Subtype~3 only has
dimensions.

\subsubsection{ins nodes}

Valid fields: \showfields{ins}\crlf
Id: \showid{ins}

\starttabulate[|lT|l|p|]
\NC \ssbf field \NC \bf type        \NC \bf explanation \NC \NR
\NC subtype     \NC number          \NC the insertion class \NC \NR
\NC attr        \NC \syntax{<node>} \NC \NC \NR
\NC cost        \NC number          \NC the penalty associated with this
                                        insert \NC \NR
\NC height      \NC number          \NC \NC \NR
\NC depth       \NC number          \NC \NC \NR
\NC head/list   \NC \syntax{<node>} \NC the first node of the body of this
                                        insert \NC \NR
\NC spec        \NC \syntax{<node>} \NC a pointer to the \type {\splittopskip}
                                        glue spec \NC \NR
\stoptabulate

A warning: never assign a node list to the \type {head} field unless you are sure
its internal link structure is correct, otherwise an error may be result. You can use
\type {list} instead (often in functions you want to use local variable swith similar
names and both names are equally sensible).

\subsubsection{mark nodes}

Valid fields: \showfields{mark}\crlf
Id: \showid{mark}

\starttabulate[|lT|l|p|]
\NC \ssbf field \NC \bf type        \NC \bf explanation \NC \NR
\NC subtype     \NC number          \NC unused \NC \NR
\NC attr        \NC \syntax{<node>} \NC \NC \NR
\NC class       \NC number          \NC the mark class \NC \NR
\NC mark        \NC table           \NC a table representing a token list \NC \NR
\stoptabulate

\subsubsection{adjust nodes}

Valid fields: \showfields{adjust}\crlf
Id: \showid{adjust}

\starttabulate[|lT|l|p|]
\NC \ssbf field \NC \bf type        \NC \bf explanation \NC \NR
\NC subtype     \NC number          \NC \type {0} = normal,
                                        \type {1} = \quote{pre} \NC \NR
\NC attr        \NC \syntax{<node>} \NC \NC \NR
\NC head/list   \NC \syntax{<node>} \NC adjusted material \NC \NR
\stoptabulate

A warning: never assign a node list to the \type {head} field unless you are sure
its internal link structure is correct, otherwise an error may be result.

\subsubsection{disc nodes}

Valid fields: \showfields{disc}\crlf
Id: \showid{disc}

\starttabulate[|lT|l|p|]
\NC \ssbf field \NC \bf type        \NC \bf explanation \NC \NR
\NC subtype     \NC number          \NC indicates the source of a discretionary:
                                       \type {0} = the \type {\discretionary} command,
                                       \type {1} = the \type {\-} command,
                                       \type {2} = added automatically following a \type {-},
                                       \type {3} = added by the hyphenation algorithm (simple),
                                       \type {4} = added by the hyphenation algorithm (hard, first item),
                                       \type {5} = added by the hyphenation algorithm (hard, second item) \NC \NR
\NC attr        \NC \syntax{<node>} \NC \NC \NR
\NC pre         \NC \syntax{<node>} \NC pointer to the pre|-|break text \NC \NR
\NC post        \NC \syntax{<node>} \NC pointer to the post|-|break text \NC \NR
\NC replace     \NC \syntax{<node>} \NC pointer to the no|-|break text \NC \NR
\NC penalty     \NC number          \NC the penalty associated with the break,
                                        normally \type {\hyphenpenalty} or \type
                                        {\exhyphenpenalty} \NC \NR
\stoptabulate

The subtype numbers~4 and~5 belong to the \quote {of-f-ice} explanation given
elsewhere.

Warning: never assign a node list to the \type {pre}, \type {post} or \type
{replace} field unless you are sure its internal link structure is correct,
otherwise an error may be result. This limnitation will disappear in the future,

\subsubsection{math nodes}

Valid fields: \showfields{math}\crlf
Id: \showid{math}

\starttabulate[|lT|l|p|]
\NC \ssbf field \NC \bf type        \NC \bf explanation \NC \NR
\NC subtype     \NC number          \NC \type {0} = on,
                                        \type {1} = off \NC \NR
\NC attr        \NC \syntax{<node>} \NC \NC \NR
\NC surround    \NC number          \NC  width of the \type {\mathsurround} kern \NC \NR
\stoptabulate

\subsubsection{glue nodes}

Skips are about the only type of data objects in traditional \TEX\ that are not a
simple value. The structure that represents the glue components of a skip is
called a \type {glue_spec}, and it has the following accessible fields:

\starttabulate[|lT|l|p|]
\NC \ssbf key     \NC \bf type \NC \bf explanation \NC \NR
\NC width         \NC number   \NC \NC \NR
\NC stretch       \NC number   \NC \NC \NR
\NC stretch_order \NC number   \NC \NC \NR
\NC shrink        \NC number   \NC \NC \NR
\NC shrink_order  \NC number   \NC \NC \NR
\NC writable      \NC boolean  \NC If this is true, you can't assign to this
                                   \type {glue_spec} because it is one of the
                                   preallocated special cases. \NC \NR
\stoptabulate

% These objects are reference counted, so there is actually an extra read|-|only
% field named \type {ref_count} as well. This item type will likely disappear in
% the future, and the glue fields themselves will become part of the nodes
% referencing glue items.

The effective width of some glue subtypes depends on the stretch or shrink needed
to make the encapsulating box fit its dimensions. For instance, in a paragraph
lines normally have glue representing spaces and these stretch of shrink to make
the content fit in the available space. The \type {effective_glue} function that
takes a glue node and a parent (hlist or vlist) returns the effective width of
that glue item.

A spec node is normally references to from a glue node:

Valid fields: \showfields{glue}\crlf
Id: \showid{glue}

\starttabulate[|lT|l|p|]
\NC \ssbf field \NC \bf type        \NC \bf explanation \NC \NR
\NC subtype     \NC number          \NC \type {0} = \type {\skip},
                                        \type {1-18} = internal glue parameters,
                                        \type {100-103} = \quote {leader} subtypes \NC \NR
\NC attr        \NC \syntax{<node>} \NC \NC \NR
\NC spec        \NC \syntax{<node>} \NC pointer to a glue_spec item \NC \NR
\NC leader      \NC \syntax{<node>} \NC pointer to a box or rule for leaders \NC \NR
\stoptabulate

The indirect spec approach is an optimization in the original \TEX\ code. First
of all it saves quite some memory because all these spaces that become glue now
share the same specification, and zero testing is also faster because only the
pointer has to be checked.

The exact meanings of the subtypes are as follows:

\starttabulate[|rT|l|]
\NC   1 \NC \type {\lineskip}              \NC \NR
\NC   2 \NC \type {\baselineskip}          \NC \NR
\NC   3 \NC \type {\parskip}               \NC \NR
\NC   4 \NC \type {\abovedisplayskip}      \NC \NR
\NC   5 \NC \type {\belowdisplayskip}      \NC \NR
\NC   6 \NC \type {\abovedisplayshortskip} \NC \NR
\NC   7 \NC \type {\belowdisplayshortskip} \NC \NR
\NC   8 \NC \type {\leftskip}              \NC \NR
\NC   9 \NC \type {\rightskip}             \NC \NR
\NC  10 \NC \type {\topskip}               \NC \NR
\NC  11 \NC \type {\splittopskip}          \NC \NR
\NC  12 \NC \type {\tabskip}               \NC \NR
\NC  13 \NC \type {\spaceskip}             \NC \NR
\NC  14 \NC \type {\xspaceskip}            \NC \NR
\NC  15 \NC \type {\parfillskip}           \NC \NR
\NC  16 \NC \type {\thinmuskip}            \NC \NR
\NC  17 \NC \type {\medmuskip}             \NC \NR
\NC  18 \NC \type {\thickmuskip}           \NC \NR
\NC 100 \NC \type {\leaders}               \NC \NR
\NC 101 \NC \type {\cleaders}              \NC \NR
\NC 102 \NC \type {\xleaders}              \NC \NR
\NC 103 \NC \type {\gleaders}              \NC \NR
\stoptabulate

For convenience we provide access to the spec fields directly so that you can
avoid the spec lookup. So, the following fields can also be queried or set. When
you set a field and no spec is set, a spec will automatically be created.

\starttabulate[|lT|l|p|]
\NC \ssbf key     \NC \bf type \NC \bf explanation \NC \NR
\NC width         \NC number   \NC \NC \NR
\NC stretch       \NC number   \NC \NC \NR
\NC stretch_order \NC number   \NC \NC \NR
\NC shrink        \NC number   \NC \NC \NR
\NC shrink_order  \NC number   \NC \NC \NR
\stoptabulate

When you assign the properties to a spec using the above keys the advantage is
that when needed a new spec is allocated. if you access the spec node directly
you can get an error message with respect to a non|-|writable spec node.

By using the accessors in the glue node you are more future proof as we might
decide at some point to carry all information in the glue nodes themselves. Of
course we can then also decide to make the spec field kind of virtual to keep
compatibility (for a while).

\subsubsection{kern nodes}

Valid fields: \showfields{kern}\crlf
Id: \showid{kern}

\starttabulate[|lT|l|p|]
\NC \ssbf field \NC \bf type        \NC \bf explanation \NC \NR
\NC subtype     \NC number          \NC \type {0} = from font,
                                        \type {1} = from \type {\kern},
                                        \type {2} = from \type {\accent},
                                        \type {3} = from \type {\/} \NC \NR
\NC attr        \NC \syntax{<node>} \NC \NC \NR
\NC kern        \NC number          \NC \NC \NR
\stoptabulate

\subsubsection{penalty nodes}

Valid fields: \showfields{penalty}\crlf
Id: \showid{penalty}

\starttabulate[|lT|l|p|]
\NC \ssbf field  \NC \bf type        \NC \bf explanation \NC \NR
\NC subtype      \NC number          \NC not used \NC \NR
\NC attr         \NC \syntax{<node>} \NC \NC \NR
\NC penalty      \NC number          \NC \NC \NR
\stoptabulate

\subsubsection[glyphnodes]{glyph nodes}

Valid fields: \showfields{glyph}\crlf
Id: \showid{glyph}

\starttabulate[|lT|l|p|]
\NC \ssbf field      \NC \ssbf type      \NC \ssbf explanation \NC \NR
\NC subtype          \NC number          \NC bitfield \NC \NR
\NC attr             \NC \syntax{<node>} \NC \NC \NR
\NC char             \NC number          \NC \NC \NR
\NC font             \NC number          \NC \NC \NR
\NC lang             \NC number          \NC \NC \NR
\NC left             \NC number          \NC \NC \NR
\NC right            \NC number          \NC \NC \NR
\NC uchyph           \NC boolean         \NC \NC \NR
\NC components       \NC \syntax{<node>} \NC pointer to ligature components \NC \NR
\NC xoffset          \NC number          \NC \NC \NR
\NC yoffset          \NC number          \NC \NC \NR
\NC width            \NC number          \NC \NC \NR
\NC height           \NC number          \NC \NC \NR
\NC depth            \NC number          \NC \NC \NR
\NC expansion_factor \NC number          \NC \NC \NR
\stoptabulate

A warning: never assign a node list to the components field unless you are sure
its internal link structure is correct, otherwise an error may be result. Valid
bits for the \type {subtype} field are:

\starttabulate[|c|l|]
\NC \ssbf bit \NC \bf meaning \NC \NR
\NC 0 \NC character \NC \NR
\NC 1 \NC ligature  \NC \NR
\NC 2 \NC ghost     \NC \NR
\NC 3 \NC left      \NC \NR
\NC 4 \NC right     \NC \NR
\stoptabulate

See \in {section} [charsandglyphs] for a detailed description of the \type
{subtype} field.

The \type {expansion_factor} has been introduced as part of the separation
between font- and backend. It is the result of extensive experiments with a more
efficient implementation of expansion. Early versions of \LUATEX\ already
replaced multiple instances of fonts in the backend by scaling but contrary to
\PDFTEX\ in \LUATEX\ we now also got rid of font copies in the frontend and
replaced them by expansion factors that travel with glyph nodes. Apart from a
cleaner approach this is also a step towards a better separation between front-
and backend.

The \type {is_char} function checks if a node is a glyphnode with a subtype still
less than 256. This function can be used to determine if applying font logic to a
glyph node makes sense.

\subsubsection{margin_kern nodes}

Valid fields: \showfields{margin_kern}\crlf
Id: \showid{margin_kern}

\starttabulate[|lT|l|p|]
\NC \ssbf field \NC \bf type        \NC \bf explanation \NC \NR
\NC subtype     \NC number          \NC \type {0} = left side,
                                        \type {1} = right side \NC \NR
\NC attr        \NC \syntax{<node>} \NC \NC \NR
\NC width       \NC number          \NC \NC \NR
\NC glyph       \NC \syntax{<node>} \NC \NC \NR
\stoptabulate

\subsection{Math nodes}

These are the so||called \quote {noad}s and the nodes that are specifically
associated with math processing. Most of these nodes contain subnodes so that the
list of possible fields is actually quite small. First, the subnodes:

\subsubsection{Math kernel subnodes}

Many object fields in math mode are either simple characters in a specific family
or math lists or node lists. There are four associated subnodes that represent
these cases (in the following node descriptions these are indicated by the word
\type {<kernel>}).

The \type {next} and \type {prev} fields for these subnodes are unused.

\subsubsubsection{math_char and math_text_char subnodes}

Valid fields: \showfields{math_char}\crlf
Id: \showid{math_char}

\starttabulate[|lT|l|p|]
\NC \ssbf field \NC \bf type \NC \bf explanation \NC \NR
\NC attr \NC \syntax{<node>} \NC \NC \NR
\NC char \NC number          \NC \NC \NR
\NC fam  \NC number          \NC \NC \NR
\stoptabulate

The \type {math_char} is the simplest subnode field, it contains the character
and family for a single glyph object. The \type {math_text_char} is a special
case that you will not normally encounter, it arises temporarily during math list
conversion (its sole function is to suppress a following italic correction).

\subsubsubsection{sub_box and sub_mlist subnodes}

Valid fields: \showfields{sub_box}\crlf
Id: \showid{sub_box}

\starttabulate[|lT|l|p|]
\NC \ssbf field \NC \bf type       \NC \bf explanation \NC \NR
\NC attr        \NC \syntax{<node>}\NC \NC \NR
\NC head        \NC \syntax{<node>}\NC \NC \NR
\stoptabulate

These two subnode types are used for subsidiary list items. For \type {sub_box},
the \type {head} points to a \quote {normal} vbox or hbox. For \type {sub_mlist},
the \type {head} points to a math list that is yet to be converted.

A warning: never assign a node list to the \type {head} field unless you are sure
its internal link structure is correct, otherwise an error may be result.

\subsubsection{Math delimiter subnode}

There is a fifth subnode type that is used exclusively for delimiter fields. As
before, the \type {next} and \type {prev} fields are unused.

\subsubsubsection{delim subnodes}

Valid fields: \showfields{delim}\crlf
Id: \showid{delim}

\starttabulate[|lT|l|p|]
\NC \ssbf field \NC \bf type        \NC\bf explanation \NC \NR
\NC attr        \NC \syntax{<node>} \NC \NC \NR
\NC small_char  \NC number          \NC \NC \NR
\NC small_fam   \NC number          \NC \NC \NR
\NC large_char  \NC number          \NC \NC \NR
\NC large_fam   \NC number          \NC \NC \NR
\stoptabulate

The fields \type {large_char} and \type {large_fam} can be zero, in that case the
font that is sed for the \type {small_fam} is expected to provide the large
version as an extension to the \type {small_char}.

\subsubsection{Math core nodes}

First, there are the objects (the \TEX book calls then \quote {atoms}) that are
associated with the simple math objects: Ord, Op, Bin, Rel, Open, Close, Punct,
Inner, Over, Under, Vcent. These all have the same fields, and they are combined
into a single node type with separate subtypes for differentiation.

\subsubsubsection{simple nodes}

Valid fields: \showfields{noad}\crlf
Id: \showid{noad}

\starttabulate[|lT|l|p|]
\NC \ssbf field \NC \bf type          \NC \bf explanation \NC \NR
\NC subtype     \NC number            \NC see below \NC \NR
\NC attr        \NC \syntax{<node>}   \NC \NC \NR
\NC nucleus     \NC \syntax{<kernel>} \NC \NC \NR
\NC sub         \NC \syntax{<kernel>} \NC \NC \NR
\NC sup         \NC \syntax{<kernel>} \NC \NC \NR
\stoptabulate

Operators are a bit special because they occupy three subtypes. \type {subtype}.

\starttabulate[|lT|p|]
\NC \ssbf number \NC \bf node subtype \NC \NR
\NC  0           \NC Ord \NC \NR
\NC  1           \NC Op: \type {\displaylimits} \NC \NR
\NC  2           \NC Op: \type {\limits} \NC \NR
\NC  3           \NC Op: \type {\nolimits} \NC \NR
\NC  4           \NC Bin \NC \NR
\NC  5           \NC Rel \NC \NR
\NC  6           \NC Open \NC \NR
\NC  7           \NC Close \NC \NR
\NC  8           \NC Punct \NC \NR
\NC  9           \NC Inner \NC \NR
\NC 10           \NC Under \NC \NR
\NC 11           \NC Over \NC \NR
\NC 12           \NC Vcent \NC \NR
\stoptabulate

\subsubsubsection{accent nodes}

Valid fields: \showfields{accent}\crlf
Id: \showid{accent}

\starttabulate[|lT|l|p|]
\NC \ssbf field \NC \bf type          \NC \bf explanation \NC \NR
\NC subtype     \NC number            \NC the first bit is used for a fixed top
                                          accent flag (if the \type {accent}
                                          field is present), the second bit for a
                                          fixed bottom accent flag (if the \type
                                          {bot_accent} field is present); example:
                                          the actual value \type {3} means: do
                                          not stretch either accent \NC \NR
\NC attr        \NC \syntax{<node>}   \NC \NC \NR
\NC nucleus     \NC \syntax{<kernel>} \NC \NC \NR
\NC sub         \NC \syntax{<kernel>} \NC \NC \NR
\NC sup         \NC \syntax{<kernel>} \NC \NC \NR
\NC accent      \NC \syntax{<kernel>} \NC \NC \NR
\NC bot_accent  \NC \syntax{<kernel>} \NC \NC \NR
\stoptabulate

\subsubsubsection{style nodes}

Valid fields: \showfields{style}\crlf
Id: \showid{style}

\starttabulate[|lT|l|p|]
\NC \ssbf field \NC \bf type \NC \bf explanation    \NC \NR
\NC style       \NC string   \NC contains the style \NC \NR
\stoptabulate

There are eight possibilities for the string value: one of \quote {display},
\quote {text}, \quote {script}, or \quote {scriptscript}. Each of these can have
a trailing \type {'} to signify \quote {cramped} styles.

\subsubsubsection{choice nodes}

Valid fields: \showfields{choice}\crlf
Id: \showid{choice}

\starttabulate[|lT|l|p|]
\NC \ssbf field  \NC \bf type        \NC \bf explanation \NC \NR
\NC attr         \NC \syntax{<node>} \NC \NC \NR
\NC display      \NC \syntax{<node>} \NC \NC \NR
\NC text         \NC \syntax{<node>} \NC \NC \NR
\NC script       \NC \syntax{<node>} \NC \NC \NR
\NC scriptscript \NC \syntax{<node>} \NC \NC \NR
\stoptabulate

A warning: never assign a node list to the display, text, script, or
scriptscript field unless you are sure its internal link structure is
correct, otherwise an error may be result.

\subsubsubsection{radical nodes}

Valid fields: \showfields{radical}\crlf
Id: \showid{radical}

\starttabulate[|lT|l|p|]
\NC \ssbf field \NC \bf type          \NC \bf explanation \NC \NR
\NC attr        \NC \syntax{<node>}   \NC \NC \NR
\NC nucleus     \NC \syntax{<kernel>} \NC \NC \NR
\NC sub         \NC \syntax{<kernel>} \NC \NC \NR
\NC sup         \NC \syntax{<kernel>} \NC \NC \NR
\NC left        \NC \syntax{<delim>}  \NC \NC \NR
\NC degree      \NC \syntax{<kernel>} \NC
    Only set by \type {\Uroot}
\NC \NR
\stoptabulate

A warning: never assign a node list to the nucleus, sub, sup, left, or degree
field unless you are sure its internal link structure is correct, otherwise an
error may be result.

The radical noad is also used for under- and overdelimiters, which is indicated
by the subtypes:

\starttabulate[|lT|l|]
\NC 0 \NC \type {\radical}         \NC \NR
\NC 1 \NC \type {\Uradical}        \NC \NR
\NC 2 \NC \type {\Uroot}           \NC \NR
\NC 3 \NC \type {\Uunderdelimiter} \NC \NR
\NC 4 \NC \type {\Uoverdelimiter}  \NC \NR
\NC 5 \NC \type {\Udelimiterunder} \NC \NR
\NC 6 \NC \type {\Udelimiterover}  \NC \NR
\stoptabulate

\subsubsubsection{fraction nodes}

Valid fields: \showfields{fraction}\crlf
Id: \showid{fraction}

\starttabulate[|lT|l|p|]
\NC \ssbf field \NC \bf type          \NC \bf explanation \NC \NR
\NC attr        \NC \syntax{<node>}   \NC \NC \NR
\NC width       \NC number            \NC \NC \NR
\NC num         \NC \syntax{<kernel>} \NC \NC \NR
\NC denom       \NC \syntax{<kernel>} \NC \NC \NR
\NC left        \NC \syntax{<delim>}  \NC \NC \NR
\NC right       \NC \syntax{<delim>}  \NC \NC \NR
\stoptabulate

A warning: never assign a node list to the num, or denom field unless you are
sure its internal link structure is correct, otherwise an error may be result.

\subsubsubsection{fence nodes}

Valid fields: \showfields{fence}\crlf
Id: \showid{fence}

\starttabulate[|lT|l|p|]
\NC \ssbf field \NC \bf type         \NC \bf explanation \NC \NR
\NC subtype     \NC number           \NC
    \type {1} = \type {\left},
    \type {2} = \type {\middle},
    \type {3} = \type {\right}
\NC \NR
\NC attr        \NC \syntax{<node>}  \NC \NC \NR
\NC delim       \NC \syntax{<delim>} \NC \NC \NR
\stoptabulate

\subsection{whatsit nodes}

Whatsit nodes come in many subtypes that you can ask for by running
\type {node.whatsits()}:
\startluacode
    for id, name in table.sortedpairs(node.whatsits()) do
        context.type(name)
        context(" (%s), ",id)
    end
    context.removeunwantedspaces()
    context.removepunctuation()
\stopluacode
. % period

\subsubsection{open nodes}

Valid fields: \showfields{whatsit,open}\crlf
Id: \showid{whatsit,open}

\starttabulate[|lT|l|p|]
\NC \ssbf field \NC \bf type        \NC \bf explanation \NC \NR
\NC attr        \NC \syntax{<node>} \NC \NC \NR
\NC stream      \NC number          \NC \TEX's stream id number \NC \NR
\NC name        \NC string          \NC file name \NC \NR
\NC ext         \NC string          \NC file extension \NC \NR
\NC area        \NC string          \NC file area (this may become obsolete) \NC \NR
\stoptabulate

\subsubsection{write nodes}

Valid fields: \showfields{whatsit,write}\crlf
Id: \showid{whatsit,write}

\starttabulate[|lT|l|p|]
\NC \ssbf field \NC \bf type        \NC \bf explanation \NC \NR
\NC attr        \NC \syntax{<node>} \NC \NC \NR
\NC stream      \NC number          \NC \TEX's stream id number \NC \NR
\NC data        \NC table           \NC a table representing the token list
                                        to be written \NC \NR
\stoptabulate

\subsubsection{close nodes}

Valid fields: \showfields{whatsit,close}\crlf
Id: \showid{whatsit,close}

\starttabulate[|lT|l|p|]
\NC \ssbf field \NC \bf type        \NC \bf explanation \NC \NR
\NC attr        \NC \syntax{<node>} \NC \NC \NR
\NC stream      \NC number          \NC \TEX's stream id number \NC \NR
\stoptabulate

\subsubsection{special nodes}

Valid fields: \showfields{whatsit,special}\crlf
Id: \showid{whatsit,special}

\starttabulate[|lT|l|p|]
\NC \ssbf field \NC \bf type        \NC \bf explanation \NC \NR
\NC attr        \NC \syntax{<node>} \NC \NC \NR
\NC data        \NC string          \NC the \type {\special} information \NC \NR
\stoptabulate

\subsubsection{boundary nodes}

Valid fields: \showfields{boundary}\crlf
Id: \showid{boundary}

This node relates to the \type {\noboundary} primitive but you can use it for
your own purpose too, in which case \type {\boundary} can come in handy.

\subsubsection{language nodes}

\LUATEX\ does not have language whatsits any more. All language information is
already present inside the glyph nodes themselves. This whatsit subtype will be
removed in the next release.

\subsubsection{local_par nodes}

Valid fields: \showfields{local_par}\crlf
Id: \showid{local_par}

\starttabulate[|lT|l|p|]
\NC \ssbf field     \NC \bf type        \NC \bf explanation \NC \NR
\NC attr            \NC \syntax{<node>} \NC \NC \NR
\NC pen_inter       \NC number          \NC local interline penalty (from \type
                                            {\localinterlinepenalty}) \NC \NR
\NC pen_broken      \NC number          \NC local broken penalty (from \type
                                            {\localbrokenpenalty}) \NC \NR
\NC dir             \NC string          \NC the direction of this par. see~\in
                                            [dirnodes] \NC \NR
\NC box_left        \NC \syntax{<node>} \NC the \type {\localleftbox} \NC \NR
\NC box_left_width  \NC number          \NC width of the \type {\localleftbox} \NC \NR
\NC box_right       \NC \syntax{<node>} \NC the \type {\localrightbox}
\NC \NR
\NC box_right_width \NC number          \NC width of the \type {\localrightbox} \NC \NR
\stoptabulate

A warning: never assign a node list to the \type {box_left} or \type {box_right}
field unless you are sure its internal link structure is correct, otherwise an
error may be result.

\subsubsection[dirnodes]{dir nodes}

Valid fields: \showfields{dir}\crlf
Id: \showid{dir}

\starttabulate[|lT|l|p|]
\NC \ssbf field \NC \bf type        \NC \bf explanation \NC \NR
\NC attr        \NC \syntax{<node>} \NC \NC \NR
\NC dir         \NC string          \NC the direction (but see below) \NC \NR
\NC level       \NC number          \NC nesting level of this direction whatsit \NC \NR
\NC dvi_ptr     \NC number          \NC a saved dvi buffer byte offset \NC \NR
\NC dir_h       \NC number          \NC a saved dvi position \NC \NR
\stoptabulate

A note on \type {dir} strings. Direction specifiers are three|-|letter
combinations of \type {T}, \type {B}, \type {R}, and \type {L}.

These are built up out of three separate items:

\startitemize[packed]
\startitem
    the first  is the direction of the \quote{top}   of paragraphs.
\stopitem
\startitem
    the second is the direction of the \quote{start} of lines.
\stopitem
\startitem
    the third  is the direction of the \quote{top}   of glyphs.
\stopitem
\stopitemize

However, only four combinations are accepted: \type {TLT}, \type {TRT}, \type
{RTT}, and \type {LTL}.

Inside actual \type {dir} whatsit nodes, the representation of \type {dir} is not
a three-letter but a four|-|letter combination. The first character in this case
is always either \type {+} or \type {-}, indicating whether the value is pushed
or popped from the direction stack.

\subsubsection{pdf_literal nodes}

Valid fields: \showfields{whatsit,pdf_literal}\crlf
Id: \showid{whatsit,pdf_literal}

\starttabulate[|lT|l|p|]
\NC \ssbf field \NC \bf type        \NC \bf explanation \NC \NR
\NC attr        \NC \syntax{<node>} \NC \NC \NR
\NC mode        \NC number          \NC the \quote {mode} setting of this
                                        literal \NC \NR
\NC data        \NC string          \NC the \type {\pdfliteral} information \NC \NR
\stoptabulate

Mode values:

\starttabulate[|lT|p|]
\NC \ssbf value \NC \ssbf corresponding \type {\pdftex} keyword \NC \NR
\NC 0           \NC setorigin                                \NC \NR
\NC 1           \NC page                                     \NC \NR
\NC 2           \NC direct                                   \NC \NR
\stoptabulate

\subsubsection{pdf_refobj nodes}

Valid fields: \showfields{whatsit,pdf_refobj}\crlf
Id: \showid{whatsit,pdf_refobj}

\starttabulate[|lT|l|p|]
\NC \ssbf field \NC \bf type        \NC \bf explanation \NC \NR
\NC attr        \NC \syntax{<node>} \NC \NC \NR
\NC objnum      \NC number          \NC the referenced \PDF\ object number \NC \NR
\stoptabulate

\subsubsection{pdf_annot nodes}

Valid fields: \showfields{whatsit,pdf_annot}\crlf
Id: \showid{whatsit,pdf_annot}

\starttabulate[|lT|l|p|]
\NC \ssbf field \NC \bf type       \NC \bf explanation \NC \NR
\NC attr       \NC \syntax{<node>} \NC \NC \NR
\NC width      \NC number          \NC \NC \NR
\NC height     \NC number          \NC \NC \NR
\NC depth      \NC number          \NC \NC \NR
\NC objnum     \NC number          \NC the referenced \PDF\ object number \NC \NR
\NC data       \NC string          \NC the annotation data \NC \NR
\stoptabulate

\subsubsection{pdf_start_link nodes}

Valid fields: \showfields{whatsit,pdf_start_link}\crlf
Id: \showid{whatsit,pdf_start_link}

\starttabulate[|lT|l|p|]
\NC \ssbf field \NC \bf type        \NC \bf explanation \NC \NR
\NC attr        \NC \syntax{<node>} \NC \NC \NR
\NC width       \NC number          \NC \NC \NR
\NC height      \NC number          \NC \NC \NR
\NC depth       \NC number          \NC \NC \NR
\NC objnum      \NC number          \NC the referenced \PDF\ object number \NC \NR
\NC link_attr   \NC table           \NC the link attribute token list \NC \NR
\NC action      \NC \syntax{<node>} \NC the action to perform \NC \NR
\stoptabulate

\subsubsection{pdf_end_link nodes}

Valid fields: \showfields{whatsit,pdf_end_link}\crlf
Id: \showid{whatsit,pdf_end_link}

\starttabulate[|lT|l|p|]
\NC \ssbf field \NC \bf type        \NC \bf explanation \NC \NR
\NC attr        \NC \syntax{<node>} \NC \NC \NR
\stoptabulate

\subsubsection{pdf_dest nodes}

Valid fields: \showfields{whatsit,pdf_dest}\crlf
Id: \showid{whatsit,pdf_dest}

\starttabulate[|lT|l|p|]
\NC \ssbf field \NC \bf type        \NC \bf explanation \NC \NR
\NC attr        \NC \syntax{<node>} \NC \NC \NR
\NC width       \NC number          \NC \NC \NR
\NC height      \NC number          \NC \NC \NR
\NC depth       \NC number          \NC \NC \NR
\NC named_id    \NC number          \NC is the dest_id a string value? \NC \NR
\NC dest_id     \NC number          \NC the destination id \NC \NR
\NC             \NC string          \NC the destination name \NC \NR
\NC dest_type   \NC number          \NC type of destination \NC \NR
\NC xyz_zoom    \NC number          \NC \NC \NR
\NC objnum      \NC number          \NC the \PDF\ object number \NC \NR
\stoptabulate

\subsubsection{pdf_action nodes}

Valid fields: \showfields{whatsit,pdf_action}\crlf
Id: \showid{whatsit,pdf_action}

These are a special kind of item that only appears inside \PDF\ start link
objects.

\starttabulate[|lT|l|p|]
\NC \ssbf field \NC \bf type         \NC \bf explanation \NC \NR
\NC action_type \NC number           \NC \NC \NR
\NC action_id   \NC number or string \NC \NC \NR
\NC named_id    \NC number           \NC \NC \NR
\NC file        \NC string           \NC \NC \NR
\NC new_window  \NC number           \NC \NC \NR
\NC data        \NC string           \NC \NC \NR
\NC ref_count   \NC number           \NC read-only \NC \NR
\stoptabulate

\subsubsection{pdf_thread nodes}

Valid fields: \showfields{whatsit,pdf_thread}\crlf
Id: \showid{whatsit,pdf_thread}

\starttabulate[|lT|l|p|]
\NC \ssbf field \NC \bf type        \NC \bf explanation \NC \NR
\NC attr        \NC \syntax{<node>} \NC \NC \NR
\NC width       \NC number          \NC \NC \NR
\NC height      \NC number          \NC \NC \NR
\NC depth       \NC number          \NC \NC \NR
\NC named_id    \NC number          \NC is the tread_id a string value? \NC \NR
\NC tread_id    \NC number          \NC the thread id \NC \NR
\NC             \NC string          \NC the thread name \NC \NR
\NC thread_attr \NC number          \NC extra thread information \NC \NR
\stoptabulate

\subsubsection{pdf_start_thread nodes}

Valid fields: \showfields{whatsit,pdf_start_thread}\crlf
Id: \showid{whatsit,pdf_start_thread}

\starttabulate[|lT|l|p|]
\NC \ssbf field \NC \bf type        \NC \bf explanation \NC \NR
\NC attr        \NC \syntax{<node>} \NC \NC \NR
\NC width       \NC number          \NC \NC \NR
\NC height      \NC number          \NC \NC \NR
\NC depth       \NC number          \NC \NC \NR
\NC named_id    \NC number          \NC is the tread_id a string value? \NC \NR
\NC tread_id    \NC number          \NC the thread id \NC \NR
\NC             \NC string          \NC the thread name \NC \NR
\NC thread_attr \NC number          \NC extra thread information \NC \NR
\stoptabulate

\subsubsection{pdf_end_thread nodes}

Valid fields: \showfields{whatsit,pdf_end_thread}\crlf
Id: \showid{whatsit,pdf_end_thread}

\starttabulate[|lT|l|p|]
\NC \ssbf field \NC \bf type        \NC \bf explanation \NC \NR
\NC attr        \NC \syntax{<node>} \NC \NC \NR
\stoptabulate

\subsubsection{save_pos nodes}

Valid fields: \showfields{whatsit,save_pos}\crlf
Id: \showid{whatsit,save_pos}

\starttabulate[|lT|l|p|]
\NC \ssbf field \NC \bf type        \NC \bf explanation \NC \NR
\NC attr        \NC \syntax{<node>} \NC \NC \NR
\stoptabulate

\subsubsection{late_lua nodes}

Valid fields: \showfields{whatsit,late_lua}\crlf
Id: \showid{whatsit,late_lua}

\starttabulate[|lT|l|p|]
\NC \ssbf field \NC \bf type        \NC \bf explanation \NC \NR
\NC attr        \NC \syntax{<node>} \NC \NC \NR
\NC data        \NC string          \NC data to execute \NC \NR
\NC string      \NC string          \NC data to execute \NC \NR
\NC name        \NC string          \NC the name to use for lua error reporting \NC \NR
\stoptabulate

The difference between \type {data} and \type {string} is that on assignment, the
\type {data} field is converted to a token list, cf. use as \type {\latelua}. The
\type {string} version is treated as a literal string.

\subsubsection{pdf_colorstack nodes}

Valid fields: \showfields{whatsit,pdf_colorstack}\crlf
Id: \showid{whatsit,pdf_colorstack}

\starttabulate[|lT|l|p|]
\NC \ssbf field \NC \bf type        \NC \bf explanation \NC \NR
\NC attr        \NC \syntax{<node>} \NC \NC \NR
\NC stack       \NC number          \NC colorstack id number \NC \NR
\NC command     \NC number          \NC command to execute \NC \NR
\NC data        \NC string          \NC data \NC \NR
\stoptabulate

\subsubsection{pdf_setmatrix nodes}

Valid fields: \showfields{whatsit,pdf_setmatrix}\crlf
Id: \showid{whatsit,pdf_setmatrix}

\starttabulate[|lT|l|p|]
\NC \ssbf field \NC \bf type        \NC \bf explanation \NC \NR
\NC attr        \NC \syntax{<node>} \NC \NC \NR
\NC data        \NC string          \NC data \NC \NR
\stoptabulate

\subsubsection{pdf_save nodes}

Valid fields: \showfields{whatsit,pdf_save}\crlf
Id: \showid{whatsit,pdf_save}

\starttabulate[|lT|l|p|]
\NC \ssbf field \NC \bf type        \NC \bf explanation \NC \NR
\NC attr        \NC \syntax{<node>} \NC \NC \NR
\stoptabulate

\subsubsection{pdf_restore nodes}

Valid fields: \showfields{whatsit,pdf_restore}\crlf
Id: \showid{whatsit,pdf_restore}

\starttabulate[|lT|l|p|]
\NC \ssbf field \NC \bf type        \NC \bf explanation \NC \NR
\NC attr        \NC \syntax{<node>} \NC \NC \NR
\stoptabulate

\subsubsection{user_defined nodes}

User|-|defined whatsit nodes can only be created and handled from \LUA\ code. In
effect, they are an extension to the extension mechanism. The \LUATEX\ engine
will simply step over such whatsits without ever looking at the contents.

Valid fields: \showfields{whatsit,user_defined}\crlf
Id: \showid{whatsit,user_defined}

\starttabulate[|lT|l|p|]
\NC \ssbf field \NC \bf type        \NC  \bf explanation \NC \NR
\NC attr        \NC \syntax{<node>} \NC  \NC \NR
\NC user_id     \NC number          \NC id number \NC \NR
\NC type        \NC number          \NC type of the value \NC \NR
\NC value       \NC number          \NC \NC \NR
\NC             \NC string          \NC \NC \NR
\NC             \NC \syntax{<node>} \NC \NC \NR
\NC             \NC table           \NC \NC \NR
\stoptabulate

The \type {type} can have one of five distinct values:

\starttabulate[|lT|p|]
\NC \ssbf value \NC \bf explanation \NC \NR
\NC   97        \NC the value is an attribute node list \NC \NR
\NC  100        \NC the value is a number \NC \NR
\NC  110        \NC the value is a node list \NC \NR
\NC  115        \NC the value is a string \NC \NR
\NC  116        \NC the value is a token list in \LUA\ table form \NC \NR
\stoptabulate

\section{Two access models}

Deep down in \TEX\ a node has a number which is an numeric entry in a memory
table. In fact, this model, where \TEX\ manages memory is real fast and one of
the reasons why plugging in callbacks that operate on nodes is quite fast. So, if
you use the direct model, even if you know that you deal with numbers, you should
not depend on that property but treat it an abstraction just like traditional
nodes. In fact, the fact that we use a simple basic datatype has the penalty that
less checking can be done, but less checking is also the reason why it's somewhat
faster. An important aspect is that one cannot mix both methods, but you can cast
both models.

So our advice is: use the indexed (table) approach when possible and investigate
the direct one when speed might be an issue. For that reason we also provide the
\type {get*} and \type {set*} functions in the top level node namespace. There is
a limited set of getters. When implementing this direct approach the regular
index by key variant was also optimized, so direct access only makes sense when
we're accessing nodes millions of times (which happens in some font processing
for instance).

We're talking mostly of getters because setters are less important. Documents
have not that many content related nodes and setting many thousands of properties
is hardly a burden contrary to millions of consultations.

Normally you will access nodes like this:

\starttyping
local next = current.next
if next then
    -- do something
end
\stoptyping

Here \type {next} is not a real field, but a virtual one. Accessing it results in
a metatable method being called. In practice it boils down to looking up the node
type and based on the node type checking for the field name. In a worst case you
have a node type that sits at the end of the lookup list and a field that is last
in the lookup chain. However, in successive versions of \LUATEX\ these lookups
have been optimized and the most frequently accessed nodes and fields have a
higher priority.

Because in practice the \type {next} accessor results in a function call, there
is some overhead involved. The next code does the same and performs a tiny bit
faster (but not that much because it is still a function call but one that knows
what to look up).

\starttyping
local next = node.next(current)
if next then
    -- do something
end
\stoptyping

If performance matters you can use an function instead:

\starttabulate[|T|p|]
\NC getnext    \NC parsing nodelist always involves this one \NC \NR
\NC getprev    \NC used less but is logical companion to getnext \NC \NR
\NC getboth    \NC returns the next and prev pointer of a node \NC \NR
\NC getid      \NC consulted a lot \NC \NR
\NC getsubtype \NC consulted less but also a topper \NC \NR
\NC getfont    \NC used a lot in otf handling (glyph nodes are consulted a lot) \NC \NR
\NC getchar    \NC idem and also in other places \NC \NR
\NC getdisc    \NC returns the \type {pre}, \type {post} and \type {replace} fields and
                   optionally when true is passed also the tail fields. \NC \NR
\NC getlist    \NC we often parse nested lists so this is a convenient one too
                   (only works for hlist and vlist!) \NC \NR
\NC getleader  \NC comparable to list, seldom used in \TEX\ (but needs frequent consulting
                   like lists; leaders could have been made a dedicated node type) \NC \NR
\NC getfield   \NC generic getter, sufficient for the rest (other field names are
                   often shared so a specific getter makes no sense then) \NC \NR
\stoptabulate

The direct variants also have setters, where the discretionary setter takes three
(optional) arguments plus an optional fourth indicating the subtype.

It doesn't make sense to add more. Profiling demonstrated that these fields can
get accesses way more times than other fields. Even in complex documents, many
node and fields types never get seen, or seen only a few times. Most functions in
the \type {node} namespace have a companion in \type {node.direct}, but of course
not the ones that don't deal with nodes themselves. The following table
summarized this:

\start \def\yes{$+$} \def\nop{$-$}

\starttabulate[|T|c|c|]
\HL
\NC \bf function                \NC \bf node \NC \bf direct \NC \NR
\HL
\NC \type {copy_list}            \NC \yes \NC \yes  \NC \NR
\NC \type {copy}                 \NC \yes \NC \yes  \NC \NR
\NC \type {count}                \NC \yes \NC \yes  \NC \NR
\NC \type {current_attr}         \NC \yes \NC \yes  \NC \NR
\NC \type {dimensions}           \NC \yes \NC \yes  \NC \NR
\NC \type {do_ligature_n}        \NC \yes \NC \yes  \NC \NR
\NC \type {effective_glue}       \NC \yes \NC \yes  \NC \NR
\NC \type {end_of_math}          \NC \yes \NC \yes  \NC \NR
\NC \type {family_font}          \NC \yes \NC \nop  \NC \NR
\NC \type {fields}               \NC \yes \NC \nop  \NC \NR
\NC \type {first_character}      \NC \yes \NC \nop  \NC \NR
\NC \type {first_glyph}          \NC \yes \NC \yes  \NC \NR
\NC \type {flush_list}           \NC \yes \NC \yes  \NC \NR
\NC \type {flush_node}           \NC \yes \NC \yes  \NC \NR
\NC \type {free}                 \NC \yes \NC \yes  \NC \NR
\NC \type {getboth}              \NC \yes \NC \yes  \NC \NR
\NC \type {getbox}               \NC \nop \NC \yes  \NC \NR
\NC \type {getchar}              \NC \yes \NC \yes  \NC \NR
\NC \type {getdisc}              \NC \yes \NC \yes  \NC \NR
\NC \type {getfield}             \NC \yes \NC \yes  \NC \NR
\NC \type {getfont}              \NC \yes \NC \yes  \NC \NR
\NC \type {getid}                \NC \yes \NC \yes  \NC \NR
\NC \type {getleader}            \NC \yes \NC \yes  \NC \NR
\NC \type {getlist}              \NC \yes \NC \yes  \NC \NR
\NC \type {getnext}              \NC \yes \NC \yes  \NC \NR
\NC \type {getprev}              \NC \yes \NC \yes  \NC \NR
\NC \type {getsubtype}           \NC \yes \NC \yes  \NC \NR
\NC \type {has_attribute}        \NC \yes \NC \yes  \NC \NR
\NC \type {has_field}            \NC \yes \NC \yes  \NC \NR
\NC \type {has_glyph}            \NC \yes \NC \yes  \NC \NR
\NC \type {hpack}                \NC \yes \NC \yes  \NC \NR
\NC \type {id}                   \NC \yes \NC \nop  \NC \NR
\NC \type {insert_after}         \NC \yes \NC \yes  \NC \NR
\NC \type {insert_before}        \NC \yes \NC \yes  \NC \NR
\NC \type {is_char}              \NC \yes \NC \yes  \NC \NR
\NC \type {is_direct}            \NC \nop \NC \yes  \NC \NR
\NC \type {is_node}              \NC \yes \NC \yes  \NC \NR
\NC \type {kerning}              \NC \yes \NC \yes  \NC \NR
\NC \type {last_node}            \NC \yes \NC \yes  \NC \NR
\NC \type {length}               \NC \yes \NC \yes  \NC \NR
\NC \type {ligaturing}           \NC \yes \NC \yes  \NC \NR
\NC \type {mlist_to_hlist}       \NC \yes \NC \nop  \NC \NR
\NC \type {new}                  \NC \yes \NC \yes  \NC \NR
\NC \type {next}                 \NC \yes \NC \nop  \NC \NR
\NC \type {prev}                 \NC \yes \NC \nop  \NC \NR
\NC \type {protect_glyph}        \NC \yes \NC \yes  \NC \NR
\NC \type {protect_glyphs}       \NC \yes \NC \yes  \NC \NR
\NC \type {protrusion_skippable} \NC \yes \NC \yes  \NC \NR
\NC \type {remove}               \NC \yes \NC \yes  \NC \NR
\NC \type {set_attribute}        \NC \yes \NC \yes  \NC \NR
\NC \type {setboth}              \NC \yes \NC \yes  \NC \NR
\NC \type {setbox}               \NC \yes \NC \yes  \NC \NR
\NC \type {setchar}              \NC \yes \NC \yes  \NC \NR
\NC \type {setdisc}              \NC \yes \NC \yes  \NC \NR
\NC \type {setfield}             \NC \yes \NC \yes  \NC \NR
\NC \type {setlink}              \NC \yes \NC \yes  \NC \NR
\NC \type {setnext}              \NC \yes \NC \yes  \NC \NR
\NC \type {setprev}              \NC \yes \NC \yes  \NC \NR
\NC \type {slide}                \NC \yes \NC \yes  \NC \NR
\NC \type {subtype}              \NC \yes \NC \nop  \NC \NR
\NC \type {subtypes}             \NC \yes \NC \nop  \NC \NR
\NC \type {tail}                 \NC \yes \NC \yes  \NC \NR
\NC \type {todirect}             \NC \yes \NC \yes  \NC \NR
\NC \type {tonode}               \NC \yes \NC \yes  \NC \NR
\NC \type {tostring}             \NC \yes \NC \yes  \NC \NR
\NC \type {traverse_id}          \NC \yes \NC \yes  \NC \NR
\NC \type {traverse_char}        \NC \yes \NC \yes  \NC \NR
\NC \type {traverse}             \NC \yes \NC \yes  \NC \NR
\NC \type {types}                \NC \yes \NC \nop  \NC \NR
\NC \type {type}                 \NC \yes \NC \nop  \NC \NR
\NC \type {unprotect_glyphs}     \NC \yes \NC \yes  \NC \NR
\NC \type {unset_attribute}      \NC \yes \NC \yes  \NC \NR
\NC \type {usedlist}             \NC \yes \NC \yes  \NC \NR
\NC \type {vpack}                \NC \yes \NC \yes  \NC \NR
\NC \type {whatsits}             \NC \yes \NC \nop  \NC \NR
\NC \type {whatsitsubtypes}      \NC \yes \NC \nop  \NC \NR
\NC \type {write}                \NC \yes \NC \yes  \NC \NR
\stoptabulate

\stop

The \type {node.next} and \type {node.prev} functions will stay but for
consistency there are variants called \type {getnext} and \type {getprev}. We had
to use \type {get} because \type {node.id} and \type {node.subtype} are already
taken for providing meta information about nodes. Note: The getters do only basic
checking for valid keys. You should just stick to the keys mentioned in the
sections that describe node properties.

\stopchapter

\stopcomponent
