% language=uk

\environment luatex-style
\environment luatex-logos

\startcomponent luatex-math

\startchapter[reference=math,title={Math}]

The handling of mathematics in \LUATEX\ differs quite a bit from how \TEX82 (and
therefore \PDFTEX) handles math. First, \LUATEX\ adds primitives and extends some
others so that \UNICODE\ input can be used easily. Second, all of \TEX82's
internal special values (for example for operator spacing) have been made
accessible and changeable via control sequences. Third, there are extensions that
make it easier to use \OPENTYPE\ math fonts. And finally, there are some
extensions that have been proposed or considered in the past that are now added
to the engine.

\section{The current math style}

It is possible to discover the math style that will be used for a formula in an
expandable fashion (while the math list is still being read). To make this
possible, \LUATEX\ adds the new primitive: \type {\mathstyle}. This is a \quote
{convert command} like e.g. \type {\romannumeral}: its value can only be read,
not set.

\subsection{\type {\mathstyle}}

The returned value is between 0 and 7 (in math mode), or $-1$ (all other modes).
For easy testing, the eight math style commands have been altered so that the can
be used as numeric values, so you can write code like this:

\starttyping
\ifnum\mathstyle=\textstyle
    \message{normal text style}
\else \ifnum\mathstyle=\crampedtextstyle
    \message{cramped text style}
\fi \fi
\stoptyping

\subsection{\type {\Ustack}}

There are a few math commands in \TEX\ where the style that will be used is not
known straight from the start. These commands (\type {\over}, \type {\atop},
\type {\overwithdelims}, \type {\atopwithdelims}) would therefore normally return
wrong values for \type {\mathstyle}. To fix this, \LUATEX\ introduces a special
prefix command: \type {\Ustack}:

\starttyping
$\Ustack {a \over b}$
\stoptyping

The \type {\Ustack} command will scan the next brace and start a new math group
with the correct (numerator) math style.

\section{Unicode math characters}

Character handling is now extended up to the full \UNICODE\ range (the \type {\U}
prefix), which is compatible with \XETEX.

The math primitives from \TEX\ are kept as they are, except for the ones that
convert from input to math commands: \type {mathcode}, and \type {delcode}. These
two now allow for a 21-bit character argument on the left hand side of the equals
sign.

Some of the new \LUATEX\ primitives read more than one separate value. This is
shown in the tables below by a plus sign in the second column.

The input for such primitives would look like this:

\starttyping
\def\overbrace{\Umathaccent 0 1 "23DE }
\stoptyping

The altered \TEX82 primitives are:

\starttabulate[|l|l|r|c|l|r|]
\NC \bf primitive     \NC \bf min \NC \bf max \NC \kern 2em  \NC \bf min \NC \bf max \NC \NR
\NC \type {\mathcode} \NC 0       \NC 10FFFF  \NC =          \NC 0       \NC 8000    \NC \NR
\NC \type {\delcode}  \NC 0       \NC 10FFFF  \NC =          \NC 0       \NC FFFFFF  \NC \NR
\stoptabulate

The unaltered ones are:

\starttabulate[|l|l|r|]
\NC \bf primitive        \NC \bf min \NC \bf max \NC \NR
\NC \type {\mathchardef} \NC 0       \NC    8000 \NC \NR
\NC \type {\mathchar}    \NC 0       \NC    7FFF \NC \NR
\NC \type {\mathaccent}  \NC 0       \NC    7FFF \NC \NR
\NC \type {\delimiter}   \NC 0       \NC 7FFFFFF \NC \NR
\NC \type {\radical}     \NC 0       \NC 7FFFFFF \NC \NR
\stoptabulate

For practical reasons \type {\mathchardef} will silently accept values larger
that \type {0x8000} and interpret it as \type {\Umathcharnumdef}. This is needed
to satisfy older macro packages.

The following new primitives are compatible with \XETEX:

% somewhat fuzzy:

\starttabulate[|l|l|r|c|l|r|]
\NC \bf primitive                           \NC \bf min   \NC \bf max             \NC \kern 2em  \NC \bf min   \NC \bf max                    \NC \NR
\NC \type {\Umathchardef}                   \NC 0+0+0     \NC 7+FF+10FFFF\rlap{\high{1}}   \NC   \NC           \NC                            \NC \NR
\NC \type {\Umathcharnumdef}\rlap{\high{5}} \NC -80000000 \NC    7FFFFFFF\rlap{\high{3}}   \NC   \NC           \NC                            \NC \NR
\NC \type {\Umathcode}                      \NC 0         \NC      10FFFF                  \NC = \NC 0+0+0     \NC 7+FF+10FFFF\rlap{\high{1}} \NC \NR
\NC \type {\Udelcode}                       \NC 0         \NC      10FFFF                  \NC = \NC 0+0       \NC   FF+10FFFF\rlap{\high{2}} \NC \NR
\NC \type {\Umathchar}                      \NC 0+0+0     \NC 7+FF+10FFFF                  \NC   \NC           \NC                            \NC \NR
\NC \type {\Umathaccent}                    \NC 0+0+0     \NC 7+FF+10FFFF\rlap{\high{2,4}} \NC   \NC           \NC                            \NC \NR
\NC \type {\Udelimiter}                     \NC 0+0+0     \NC 7+FF+10FFFF\rlap{\high{2}}   \NC   \NC           \NC                            \NC \NR
\NC \type {\Uradical}                       \NC 0+0       \NC   FF+10FFFF\rlap{\high{2}}   \NC   \NC           \NC                            \NC \NR
\NC \type {\Umathcharnum}                   \NC -80000000 \NC    7FFFFFFF\rlap{\high{3}}   \NC   \NC           \NC                            \NC \NR
\NC \type {\Umathcodenum}                   \NC 0         \NC      10FFFF                  \NC = \NC -80000000 \NC    7FFFFFFF\rlap{\high{3}} \NC \NR
\NC \type {\Udelcodenum}                    \NC 0         \NC      10FFFF                  \NC = \NC -80000000 \NC    7FFFFFFF\rlap{\high{3}} \NC \NR
\stoptabulate

Specifications typically look like:

\starttyping
\Umathchardef\xx="1"0"456
\Umathcode   123="1"0"789
\stoptyping

Note 1: The new primitives that deal with delimiter|-|style objects do not set up a
\quote {large family}. Selecting a suitable size for display purposes is expected
to be dealt with by the font via the \type {\Umathoperatorsize} parameter (more
information can be found in a following section).

Note 2: For these three primitives, all information is packed into a single
signed integer. For the first two (\type {\Umathcharnum} and \type
{\Umathcodenum}), the lowest 21 bits are the character code, the 3 bits above
that represent the math class, and the family data is kept in the topmost bits
(This means that the values for math families 128--255 are actually negative).
For \type {\Udelcodenum} there is no math class. The math family information is
stored in the bits directly on top of the character code. Using these three
commands is not as natural as using the two- and three|-|value commands, so
unless you know exactly what you are doing and absolutely require the speedup
resulting from the faster input scanning, it is better to use the verbose
commands instead.

Note 3: The \type {\Umathaccent} command accepts optional keywords to control
various details regarding math accents. See \in {section} [mathacc] below for
details.

New primitives that exist in \LUATEX\ only (all of these will be explained
in following sections):

\starttabulate[|l|l|l|l|]
\NC \bf primitive         \NC \bf value range (in hex) \NC \NR
\NC \type {\Uroot}           \NC 0+0--FF+10FFFF$^2$       \NC \NR
\NC \type {\Uoverdelimiter}  \NC 0+0--FF+10FFFF$^2$       \NC \NR
\NC \type {\Uunderdelimiter} \NC 0+0--FF+10FFFF$^2$       \NC \NR
\NC \type {\Udelimiterover}  \NC 0+0--FF+10FFFF$^2$       \NC \NR
\NC \type {\Udelimiterunder} \NC 0+0--FF+10FFFF$^2$       \NC \NR
\stoptabulate

\section{Cramped math styles}

\LUATEX\ has four new primitives to set the cramped math styles directly:

\starttyping
\crampeddisplaystyle
\crampedtextstyle
\crampedscriptstyle
\crampedscriptscriptstyle
\stoptyping

These additional commands are not all that valuable on their own, but they come
in handy as arguments to the math parameter settings that will be added shortly.

In Eijkhouts \quotation {\TEX\ by Topic} the rules for handling styles in scripts
are described as follows:

\startitemize
\startitem
    In any style superscripts and subscripts are taken from the next smaller style.
    Exception: in display style they are taken in script style.
\stopitem
\startitem
    Subscripts are always in the cramped variant of the style; superscripts are only
    cramped if the original style was cramped.
\stopitem
\startitem
    In an \type {..\over..} formula in any style the numerator and denominator are
    taken from the next smaller style.
\stopitem
\startitem
    The denominator is always in cramped style; the numerator is only in cramped
    style if the original style was cramped.
\stopitem
\startitem
    Formulas under a \type {\sqrt} or \type {\overline} are in cramped style.
\stopitem
\stopitemize

In \LUATEX\ one can set the styles in more detail which means that you sometimes
have to set both normal and cramped styles to get the effect you want. If we
force styles in the script using \type {\scriptstyle} and \type {\crampedscriptstyle}
we get this:

\startbuffer[demo]
\starttabulate
\NC default       \NC $b_{x=xx}^{x=xx}$ \NC \NR
\NC script        \NC $b_{\scriptstyle x=xx}^{\scriptstyle x=xx}$ \NC \NR
\NC crampedscript \NC $b_{\crampedscriptstyle x=xx}^{\crampedscriptstyle x=xx}$ \NC \NR
\stoptabulate
\stopbuffer

\getbuffer[demo]

Now we set the following parameters

\startbuffer[setup]
\Umathordrelspacing\scriptstyle=30mu
\Umathordordspacing\scriptstyle=30mu
\stopbuffer

\typebuffer[setup]

This gives:

\start\getbuffer[setup,demo]\stop

But, as this is not what is expected (visually) we should say:

\startbuffer[setup]
\Umathordrelspacing\scriptstyle=30mu
\Umathordordspacing\scriptstyle=30mu
\Umathordrelspacing\crampedscriptstyle=30mu
\Umathordordspacing\crampedscriptstyle=30mu
\stopbuffer

\typebuffer[setup]

Now we get:

\start\getbuffer[setup,demo]\stop

\section{Math parameter settings}

In \LUATEX, the font dimension parameters that \TEX\ used in math typesetting are
now accessible via primitive commands. In fact, refactoring of the math engine
has resulted in many more parameters than were accessible before.

\starttabulate
\NC \bf primitive name               \NC \bf description \NC \NR
\NC \type {\Umathquad}               \NC the width of 18 mu's \NC \NR
\NC \type {\Umathaxis}               \NC height of the vertical center axis of
                                         the math formula above the baseline \NC \NR
\NC \type {\Umathoperatorsize}       \NC minimum size of large operators in display mode \NC \NR
\NC \type {\Umathoverbarkern}        \NC vertical clearance above the rule \NC \NR
\NC \type {\Umathoverbarrule}        \NC the width of the rule \NC \NR
\NC \type {\Umathoverbarvgap}        \NC vertical clearance below the rule \NC \NR
\NC \type {\Umathunderbarkern}       \NC vertical clearance below the rule \NC \NR
\NC \type {\Umathunderbarrule}       \NC the width of the rule \NC \NR
\NC \type {\Umathunderbarvgap}       \NC vertical clearance above the rule \NC \NR
\NC \type {\Umathradicalkern}        \NC vertical clearance above the rule \NC \NR
\NC \type {\Umathradicalrule}        \NC the width of the rule \NC \NR
\NC \type {\Umathradicalvgap}        \NC vertical clearance below the rule \NC \NR
\NC \type {\Umathradicaldegreebefore}\NC the forward kern that takes place before placement of
                                         the radical degree \NC \NR
\NC \type {\Umathradicaldegreeafter} \NC the backward kern that takes place after placement of
                                         the radical degree \NC \NR
\NC \type {\Umathradicaldegreeraise} \NC this is the percentage of the total height and depth of
                                         the radical sign that the degree is raised by; it is
                                         expressed in \type {percents}, so 60\% is expressed as the
                                         integer $60$ \NC \NR
\NC \type {\Umathstackvgap}          \NC vertical clearance between the two
                                         elements in a \type {\atop} stack \NC \NR
\NC \type {\Umathstacknumup}         \NC numerator shift upward in \type {\atop} stack \NC \NR
\NC \type {\Umathstackdenomdown}     \NC denominator shift downward in \type {\atop} stack \NC \NR
\NC \type {\Umathfractionrule}       \NC the width of the rule in a \type {\over} \NC \NR
\NC \type {\Umathfractionnumvgap}    \NC vertical clearance between the numerator and the rule \NC \NR
\NC \type {\Umathfractionnumup}      \NC numerator shift upward in \type {\over} \NC \NR
\NC \type {\Umathfractiondenomvgap}  \NC vertical clearance between the denominator and the rule \NC \NR
\NC \type {\Umathfractiondenomdown}  \NC denominator shift downward in \type {\over} \NC \NR
\NC \type {\Umathfractiondelsize}    \NC minimum delimiter size for \type {\...withdelims} \NC \NR
\NC \type {\Umathlimitabovevgap}     \NC vertical clearance for limits above operators \NC \NR
\NC \type {\Umathlimitabovebgap}     \NC vertical baseline clearance for limits above operators \NC \NR
\NC \type {\Umathlimitabovekern}     \NC space reserved at the top of the limit \NC \NR
\NC \type {\Umathlimitbelowvgap}     \NC vertical clearance for limits below operators \NC \NR
\NC \type {\Umathlimitbelowbgap}     \NC vertical baseline clearance for limits below operators \NC \NR
\NC \type {\Umathlimitbelowkern}     \NC space reserved at the bottom of the limit \NC \NR
\NC \type {\Umathoverdelimitervgap}  \NC vertical clearance for limits above delimiters \NC \NR
\NC \type {\Umathoverdelimiterbgap}  \NC vertical baseline clearance for limits above delimiters \NC \NR
\NC \type {\Umathunderdelimitervgap} \NC vertical clearance for limits below delimiters \NC \NR
\NC \type {\Umathunderdelimiterbgap} \NC vertical baseline clearance for limits below delimiters \NC \NR
\NC \type {\Umathsubshiftdrop}       \NC subscript drop for boxes and subformulas \NC \NR
\NC \type {\Umathsubshiftdown}       \NC subscript drop for characters \NC \NR
\NC \type {\Umathsupshiftdrop}       \NC superscript drop (raise, actually) for boxes and subformulas \NC \NR
\NC \type {\Umathsupshiftup}         \NC superscript raise for characters \NC \NR
\NC \type {\Umathsubsupshiftdown}    \NC subscript drop in the presence of a superscript \NC \NR
\NC \type {\Umathsubtopmax}          \NC the top of standalone subscripts cannot be higher than this
                                         above the baseline \NC \NR
\NC \type {\Umathsupbottommin}       \NC the bottom of standalone superscripts cannot be less than
                                         this above the baseline \NC \NR
\NC \type {\Umathsupsubbottommax}    \NC the bottom of the superscript of a combined super- and subscript
                                         be at least as high as this above the baseline \NC \NR
\NC \type {\Umathsubsupvgap}         \NC vertical clearance between super- and subscript \NC \NR
\NC \type {\Umathspaceafterscript}   \NC additional space added after a super- or subscript \NC \NR
\NC \type {\Umathconnectoroverlapmin}\NC minimum overlap between parts in an extensible recipe \NC \NR
\stoptabulate

Each of the parameters in this section can be set by a command like this:

\starttyping
\Umathquad\displaystyle=1em
\stoptyping

they obey grouping, and you can use \type {\the\Umathquad\displaystyle} if
needed.

\section{Skips around display math}

The injection of \type {\abovedisplayskip} and \type {\belowdisplayskip} is not
symmetrical. An above one is always inserted, also when zero, but the below is
only inserted when larger than zero. Especially the later mkes it sometimes hard
to fully control spacing. Therefore \LUATEX\ comes with a new directive: \type
{\mathdisplayskipmode}. The following values apply:

\starttabulate
\NC 0 \NC normal \TEX\ behaviour: always above, only below when larger than zero \NC \NR
\NC 1 \NC always \NC \NR
\NC 2 \NC only when not zero \NC \NR
\NC 3 \NC never, not even when not zero \NC \NR
\stoptabulate

\section{Font-based Math Parameters}

While it is nice to have these math parameters available for tweaking, it would
be tedious to have to set each of them by hand. For this reason, \LUATEX\
initializes a bunch of these parameters whenever you assign a font identifier to
a math family based on either the traditional math font dimensions in the font
(for assignments to math family~2 and~3 using \TFM|-|based fonts like \type
{cmsy} and \type {cmex}), or based on the named values in a potential \type
{MathConstants} table when the font is loaded via Lua. If there is a \type
{MathConstants} table, this takes precedence over font dimensions, and in that
case no attention is paid to which family is being assigned to: the \type
{MathConstants} tables in the last assigned family sets all parameters.

In the table below, the one|-|letter style abbreviations and symbolic tfm font
dimension names match those using in the \TeX book. Assignments to \type
{\textfont} set the values for the cramped and uncramped display and text styles,
\type {\scriptfont} sets the script styles, and \type {\scriptscriptfont} sets
the scriptscript styles, so we have eight parameters for three font sizes. In the
\TFM\ case, assignments only happen in family~2 and family~3 (and of course only
for the parameters for which there are font dimensions).

Besides the parameters below, \LUATEX\ also looks at the \quote {space} font
dimension parameter. For math fonts, this should be set to zero.

\start

\switchtobodyfont[8pt]

\starttabulate[|l|l|l|p|]
\NC \bf variable                      \NC \bf style             \NC \bf default value opentype               \NC \bf default value tfm \NC \NR
\NC \type {\Umathaxis}                \NC --                    \NC AxisHeight                               \NC axis_height \NC \NR
\NC \type {\Umathoperatorsize}        \NC D, D'                 \NC DisplayOperatorMinHeight                 \NC $^6$ \NC \NR
\NC \type {\Umathfractiondelsize}     \NC D, D'                 \NC FractionDelimiterDisplayStyleSize$^9$    \NC delim1 \NC \NR
\NC                                   \NC T, T', S, S', SS, SS' \NC FractionDelimiterSize$^9$                \NC delim2 \NC \NR
\NC \type {\Umathfractiondenomdown}   \NC D, D'                 \NC FractionDenominatorDisplayStyleShiftDown \NC denom1 \NC \NR
\NC                                   \NC T, T', S, S', SS, SS' \NC FractionDenominatorShiftDown             \NC denom2 \NC \NR
\NC \type {\Umathfractiondenomvgap}   \NC D, D'                 \NC FractionDenominatorDisplayStyleGapMin    \NC 3*default_rule_thickness \NC \NR
\NC                                   \NC T, T', S, S', SS, SS' \NC FractionDenominatorGapMin                \NC default_rule_thickness \NC \NR
\NC \type {\Umathfractionnumup}       \NC D, D'                 \NC FractionNumeratorDisplayStyleShiftUp     \NC num1 \NC \NR
\NC                                   \NC T, T', S, S', SS, SS' \NC FractionNumeratorShiftUp                 \NC num2 \NC \NR
\NC \type {\Umathfractionnumvgap}     \NC D, D'                 \NC FractionNumeratorDisplayStyleGapMin      \NC 3*default_rule_thickness \NC \NR
\NC                                   \NC T, T', S, S', SS, SS' \NC FractionNumeratorGapMin                  \NC default_rule_thickness \NC \NR
\NC \type {\Umathfractionrule}        \NC --                    \NC FractionRuleThickness                    \NC default_rule_thickness \NC \NR
\NC \type {\Umathskewedfractionhgap}  \NC --                    \NC SkewedFractionHorizontalGap              \NC math_quad/2 \NC \NR
\NC \type {\Umathskewedfractionvgap}  \NC --                    \NC SkewedFractionVerticalGap                \NC math_x_height \NC \NR
\NC \type {\Umathlimitabovebgap}      \NC --                    \NC UpperLimitBaselineRiseMin                \NC big_op_spacing3 \NC \NR
\NC \type {\Umathlimitabovekern}      \NC --                    \NC 0$^1$                                    \NC big_op_spacing5 \NC \NR
\NC \type {\Umathlimitabovevgap}      \NC --                    \NC UpperLimitGapMin                         \NC big_op_spacing1 \NC \NR
\NC \type {\Umathlimitbelowbgap}      \NC --                    \NC LowerLimitBaselineDropMin                \NC big_op_spacing4 \NC \NR
\NC \type {\Umathlimitbelowkern}      \NC --                    \NC 0$^1$                                    \NC big_op_spacing5 \NC \NR
\NC \type {\Umathlimitbelowvgap}      \NC --                    \NC LowerLimitGapMin                         \NC big_op_spacing2 \NC \NR
\NC \type {\Umathoverdelimitervgap}   \NC --                    \NC StretchStackGapBelowMin                  \NC big_op_spacing1 \NC \NR
\NC \type {\Umathoverdelimiterbgap}   \NC --                    \NC StretchStackTopShiftUp                   \NC big_op_spacing3 \NC \NR
\NC \type {\Umathunderdelimitervgap}  \NC--                     \NC StretchStackGapAboveMin                  \NC big_op_spacing2 \NC \NR
\NC \type {\Umathunderdelimiterbgap}  \NC--                     \NC StretchStackBottomShiftDown              \NC big_op_spacing4 \NC \NR
\NC \type {\Umathoverbarkern}         \NC --                    \NC OverbarExtraAscender                     \NC default_rule_thickness \NC \NR
\NC \type {\Umathoverbarrule}         \NC --                    \NC OverbarRuleThickness                     \NC default_rule_thickness \NC \NR
\NC \type {\Umathoverbarvgap}         \NC --                    \NC OverbarVerticalGap                       \NC 3*default_rule_thickness \NC \NR
\NC \type {\Umathquad}                \NC --                    \NC <font_size(f)>$^1$                       \NC math_quad \NC \NR
\NC \type {\Umathradicalkern}         \NC --                    \NC RadicalExtraAscender                     \NC default_rule_thickness \NC \NR
\NC \type {\Umathradicalrule}         \NC --                    \NC RadicalRuleThickness                     \NC <not set>$^2$ \NC \NR
\NC \type {\Umathradicalvgap}         \NC D, D'                 \NC RadicalDisplayStyleVerticalGap           \NC (default_rule_thickness+\crlf
                                                                                                                 (abs(math_x_height)/4))$^3$ \NC \NR
\NC                                   \NC T, T', S, S', SS, SS' \NC RadicalVerticalGap                       \NC (default_rule_thickness+\crlf
                                                                                                                 (abs(default_rule_thickness)/4))$^3$ \NC \NR
\NC \type {\Umathradicaldegreebefore} \NC --                    \NC RadicalKernBeforeDegree                  \NC <not set>$^2$ \NC \NR
\NC \type {\Umathradicaldegreeafter}  \NC --                    \NC RadicalKernAfterDegree                   \NC <not set>$^2$ \NC \NR
\NC \type {\Umathradicaldegreeraise}  \NC --                    \NC RadicalDegreeBottomRaisePercent          \NC <not set>$^{2,7}$ \NC \NR
\NC \type {\Umathspaceafterscript}    \NC --                    \NC SpaceAfterScript                         \NC script_space$^4$ \NC \NR
\NC \type {\Umathstackdenomdown}      \NC D, D'                 \NC StackBottomDisplayStyleShiftDown         \NC denom1 \NC \NR
\NC                                   \NC T, T', S, S', SS, SS' \NC StackBottomShiftDown                     \NC denom2 \NC \NR
\NC \type {\Umathstacknumup}          \NC D, D'                 \NC StackTopDisplayStyleShiftUp              \NC num1 \NC \NR
\NC                                   \NC T, T', S, S', SS, SS' \NC StackTopShiftUp                          \NC num3 \NC \NR
\NC \type {\Umathstackvgap}           \NC D, D'                 \NC StackDisplayStyleGapMin                  \NC 7*default_rule_thickness \NC \NR
\NC                                   \NC T, T', S, S', SS, SS' \NC StackGapMin                              \NC 3*default_rule_thickness \NC \NR
\NC \type {\Umathsubshiftdown}        \NC --                    \NC SubscriptShiftDown                       \NC sub1 \NC \NR
\NC \type {\Umathsubshiftdrop}        \NC --                    \NC SubscriptBaselineDropMin                 \NC sub_drop \NC \NR
\NC \type {\Umathsubsupshiftdown}     \NC --                    \NC SubscriptShiftDownWithSuperscript$^8$    \NC \NC \NR
\NC                                   \NC                       \NC \quad\ or SubscriptShiftDown             \NC sub2 \NC \NR
\NC \type {\Umathsubtopmax}           \NC --                    \NC SubscriptTopMax                          \NC (abs(math_x_height * 4) / 5) \NC \NR
\NC \type {\Umathsubsupvgap}          \NC --                    \NC SubSuperscriptGapMin                     \NC 4*default_rule_thickness \NC \NR
\NC \type {\Umathsupbottommin}        \NC --                    \NC SuperscriptBottomMin                     \NC (abs(math_x_height) / 4) \NC \NR
\NC \type {\Umathsupshiftdrop}        \NC --                    \NC SuperscriptBaselineDropMax               \NC sup_drop \NC \NR
\NC \type {\Umathsupshiftup}          \NC D                     \NC SuperscriptShiftUp                       \NC sup1 \NC \NR
\NC                                   \NC T, S, SS,             \NC SuperscriptShiftUp                       \NC sup2 \NC \NR
\NC                                   \NC D', T', S', SS'       \NC SuperscriptShiftUpCramped                \NC sup3 \NC \NR
\NC \type {\Umathsupsubbottommax}     \NC --                    \NC SuperscriptBottomMaxWithSubscript        \NC (abs(math_x_height * 4) / 5) \NC \NR
\NC \type {\Umathunderbarkern}        \NC --                    \NC UnderbarExtraDescender                   \NC default_rule_thickness \NC \NR
\NC \type {\Umathunderbarrule}        \NC --                    \NC UnderbarRuleThickness                    \NC default_rule_thickness \NC \NR
\NC \type {\Umathunderbarvgap}        \NC --                    \NC UnderbarVerticalGap                      \NC 3*default_rule_thickness \NC \NR
\NC \type {\Umathconnectoroverlapmin} \NC --                    \NC MinConnectorOverlap                      \NC 0$^5$ \NC \NR
\stoptabulate

\stop

Note 1: \OPENTYPE\ fonts set \type {\Umathlimitabovekern} and \type
{\Umathlimitbelowkern} to zero and set \type {\Umathquad} to the font size of the
used font, because these are not supported in the \type {MATH} table,

Note 2: Traditional \TFM\ fonts do not set \type {\Umathradicalrule} because
\TEX82\ uses the height of the radical instead. When this parameter is indeed not
set when \LUATEX\ has to typeset a radical, a backward compatibility mode will
kick in that assumes that an oldstyle \TEX\ font is used. Also, they do not set
\type {\Umathradicaldegreebefore}, \type {\Umathradicaldegreeafter}, and \type
{\Umathradicaldegreeraise}. These are then automatically initialized to
$5/18$quad, $-10/18$quad, and 60.

Note 3: If \TFM\ fonts are used, then the \type {\Umathradicalvgap} is not set
until the first time \LUATEX\ has to typeset a formula because this needs
parameters from both family~2 and family~3. This provides a partial backward
compatibility with \TEX82, but that compatibility is only partial: once the \type
{\Umathradicalvgap} is set, it will not be recalculated any more.

Note 4: When \TFM\ fonts are used a similar situation arises with respect to
\type {\Umathspaceafterscript}: it is not set until the first time \LUATEX\ has
to typeset a formula. This provides some backward compatibility with \TEX82. But
once the \type {\Umathspaceafterscript} is set, \type {\scriptspace} will never
be looked at again.

Note 5: Traditional \TFM\ fonts set \type {\Umathconnectoroverlapmin} to zero
because \TEX82\ always stacks extensibles without any overlap.

Note 6: The \type {\Umathoperatorsize} is only used in \type {\displaystyle}, and
is only set in \OPENTYPE\ fonts. In \TFM\ font mode, it is artificially set to
one scaled point more than the initial attempt's size, so that always the \quote
{first next} will be tried, just like in \TEX82.

Note 7: The \type {\Umathradicaldegreeraise} is a special case because it is the
only parameter that is expressed in a percentage instead of as a number of scaled
points.

Note 8: \type {SubscriptShiftDownWithSuperscript} does not actually exist in the
\quote {standard} \OPENTYPE\ math font Cambria, but it is useful enough to be
added.

Note 9: \type {FractionDelimiterDisplayStyleSize} and \type
{FractionDelimiterSize} do not actually exist in the \quote {standard} \OPENTYPE\
math font Cambria, but were useful enough to be added.

\section{Math spacing setting}

Besides the parameters mentioned in the previous sections, there are also 64 new
primitives to control the math spacing table (as explained in Chapter~18 of the
\TEX book). The primitive names are a simple matter of combining two math atom
types, but for completeness' sake, here is the whole list:

\starttwocolumns
\starttyping
\Umathordordspacing
\Umathordopspacing
\Umathordbinspacing
\Umathordrelspacing
\Umathordopenspacing
\Umathordclosespacing
\Umathordpunctspacing
\Umathordinnerspacing
\Umathopordspacing
\Umathopopspacing
\Umathopbinspacing
\Umathoprelspacing
\Umathopopenspacing
\Umathopclosespacing
\Umathoppunctspacing
\Umathopinnerspacing
\Umathbinordspacing
\Umathbinopspacing
\Umathbinbinspacing
\Umathbinrelspacing
\Umathbinopenspacing
\Umathbinclosespacing
\Umathbinpunctspacing
\Umathbininnerspacing
\Umathrelordspacing
\Umathrelopspacing
\Umathrelbinspacing
\Umathrelrelspacing
\Umathrelopenspacing
\Umathrelclosespacing
\Umathrelpunctspacing
\Umathrelinnerspacing
\Umathopenordspacing
\Umathopenopspacing
\Umathopenbinspacing
\Umathopenrelspacing
\Umathopenopenspacing
\Umathopenclosespacing
\Umathopenpunctspacing
\Umathopeninnerspacing
\Umathcloseordspacing
\Umathcloseopspacing
\Umathclosebinspacing
\Umathcloserelspacing
\Umathcloseopenspacing
\Umathcloseclosespacing
\Umathclosepunctspacing
\Umathcloseinnerspacing
\Umathpunctordspacing
\Umathpunctopspacing
\Umathpunctbinspacing
\Umathpunctrelspacing
\Umathpunctopenspacing
\Umathpunctclosespacing
\Umathpunctpunctspacing
\Umathpunctinnerspacing
\Umathinnerordspacing
\Umathinneropspacing
\Umathinnerbinspacing
\Umathinnerrelspacing
\Umathinneropenspacing
\Umathinnerclosespacing
\Umathinnerpunctspacing
\Umathinnerinnerspacing
\stoptyping
\stoptwocolumns

These parameters are of type \type {\muskip}, so setting a parameter can be done
like this:

\starttyping
\Umathopordspacing\displaystyle=4mu plus 2mu
\stoptyping

They are all initialized by \type {initex} to the values mentioned in the table
in Chapter~18 of the \TEX book.

Note 1: for ease of use as well as for backward compatibility, \type
{\thinmuskip}, \type {\medmuskip} and \type {\thickmuskip} are treated
especially. In their case a pointer to the corresponding internal parameter is
saved, not the actual \type {\muskip} value. This means that any later changes to
one of these three parameters will be taken into account.

Note 2: Careful readers will realise that there are also primitives for the items
marked \type {*} in the \TEX book. These will not actually be used as those
combinations of atoms cannot actually happen, but it seemed better not to break
orthogonality. They are initialized to zero.

\section[mathacc]{Math accent handling}

\LUATEX\ supports both top accents and bottom accents in math mode, and math
accents stretch automatically (if this is supported by the font the accent comes
from, of course). Bottom and combined accents as well as fixed-width math accents
are controlled by optional keywords following \type {\Umathaccent}.

The keyword \type {bottom} after \type {\Umathaccent} signals that a bottom accent
is needed, and the keyword \type {both} signals that both a top and a bottom
accent are needed (in this case two accents need to be specified, of course).

Then the set of three integers defining the accent is read. This set of integers
can be prefixed by the \type {fixed} keyword to indicate that a non-stretching
variant is requested (in case of both accents, this step is repeated).

A simple example:

\starttyping
\Umathaccent both fixed 0 0 "20D7 fixed 0 0 "20D7 {example}
\stoptyping

If a math top accent has to be placed and the accentee is a character and has a
non-zero \type {top_accent} value, then this value will be used to place the
accent instead of the \type {\skewchar} kern used by \TEX82.

The \type {top_accent} value represents a vertical line somewhere in the
accentee. The accent will be shifted horizontally such that its own \type
{top_accent} line coincides with the one from the accentee. If the \type
{top_accent} value of the accent is zero, then half the width of the accent
followed by its italic correction is used instead.

The vertical placement of a top accent depends on the \type {x_height} of the
font of the accentee (as explained in the \TEX book), but if value that turns out
to be zero and the font had a \type {MathConstants} table, then \type
{AccentBaseHeight} is used instead.

The vertical placement of a bottom accent is straight below the accentee, no
correction takes place.

Possible locations are \type {top}, \type {bottom}, \type {both} and \type
{center}. When no location is given \type {top} is assumed. An additional
parameter \type {fraction} can be specified followed by a number; a value of for
instance 1200 means that the criterium is 1.2 times the width of the nuclues. The
fraction only applies to the stepwise selected shapes and is mostly meant for the
\type {overlay} location. It also works for the other locations but then it
concerns the width.

\section{Math root extension}

The new primitive \type {\Uroot} allows the construction of a radical noad
including a degree field. Its syntax is an extension of \type {\Uradical}:

\starttyping
\Uradical <fam integer> <char integer> <radicand>
\Uroot    <fam integer> <char integer> <degree> <radicand>
\stoptyping

The placement of the degree is controlled by the math parameters \type
{\Umathradicaldegreebefore}, \type {\Umathradicaldegreeafter}, and \type
{\Umathradicaldegreeraise}. The degree will be typeset in \type
{\scriptscriptstyle}.

\section{Math kerning in super- and subscripts}

The character fields in a \LUA|-|loaded \OPENTYPE\ math font can have a \quote
{mathkern} table. The format of this table is the same as the \quote {mathkern}
table that is returned by the \type {fontloader} library, except that all height
and kern values have to be specified in actual scaled points.

When a super- or subscript has to be placed next to a math item, \LUATEX\ checks
whether the super- or subscript and the nucleus are both simple character items.
If they are, and if the fonts of both character items are \OPENTYPE\ fonts (as
opposed to legacy \TEX\ fonts), then \LUATEX\ will use the \OPENTYPE\ math
algorithm for deciding on the horizontal placement of the super- or subscript.

This works as follows:

\startitemize
    \startitem
        The vertical position of the script is calculated.
    \stopitem
    \startitem
        The default horizontal position is flat next to the base character.
    \stopitem
    \startitem
        For superscripts, the italic correction of the base character is added.
    \stopitem
    \startitem
        For a superscript, two vertical values are calculated: the bottom of the
        script (after shifting up), and the top of the base. For a subscript, the two
        values are the top of the (shifted down) script, and the bottom of the base.
    \stopitem
    \startitem
        For each of these two locations:
        \startitemize
            \startitem
                find the math kern value at this height for the base (for a subscript
                placement, this is the bottom_right corner, for a superscript
                placement the top_right corner)
            \stopitem
            \startitem
                find the math kern value at this height for the script (for a
                subscript placement, this is the top_left corner, for a superscript
                placement the bottom_left corner)
            \stopitem
            \startitem
                add the found values together to get a preliminary result.
            \stopitem
        \stopitemize
    \stopitem
    \startitem
        The horizontal kern to be applied is the smallest of the two results from
        previous step.
    \stopitem
\stopitemize

The math kern value at a specific height is the kern value that is specified by the
next higher height and kern pair, or the highest one in the character (if there is no
value high enough in the character), or simply zero (if the character has no math kern
pairs at all).

\section{Scripts on horizontally extensible items like arrows}

The primitives \type {\Uunderdelimiter} and \type {\Uoverdelimiter} allow the
placement of a subscript or superscript on an automatically extensible item and
\type {\Udelimiterunder} and \type {\Udelimiterover} allow the placement of an
automatically extensible item as a subscript or superscript on a nucleus. The
input:

% these produce radical noads .. in fact the code base has the numbers wrong for
% quite a while, so no one seems to use this

\startbuffer
$\Uoverdelimiter  0 "2194 {\hbox{\strut  overdelimiter}}$
$\Uunderdelimiter 0 "2194 {\hbox{\strut underdelimiter}}$
$\Udelimiterover  0 "2194 {\hbox{\strut  delimiterover}}$
$\Udelimiterunder 0 "2194 {\hbox{\strut delimiterunder}}$
\stopbuffer

\typebuffer will render this:

\blank \startnarrower \getbuffer \stopnarrower \blank

The vertical placements are controlled by \type {\Umathunderdelimiterbgap}, \type
{\Umathunderdelimitervgap}, \type {\Umathoverdelimiterbgap}, and \type
{\Umathoverdelimitervgap} in a similar way as limit placements on large operators.
The superscript in \type {\Uoverdelimiter} is typeset in a suitable scripted style,
the subscript in \type {\Uunderdelimiter} is cramped as well.

These primitives accepts an option \type {width} specification. When used the
also optional keywords \type {left}, \type {middle} and \type {right} will
determine what happens when a requested size can't be met (which can happen when
we step to successive larger variants).

An extra primitive \type {\Uhextensible} is available that can be used like this:

\startbuffer
$\Uhextensible width 10cm 0 "2194$
\stopbuffer

\typebuffer This will render this:

\blank \startnarrower \getbuffer \stopnarrower \blank

Here you can also pass options, like:

\startbuffer
$\Uhextensible width 1pt middle 0 "2194$
\stopbuffer

\typebuffer This gives:

\blank \startnarrower \getbuffer \stopnarrower \blank

\LUATEX\ internally uses a structure that supports \OPENTYPE\ \quote
{MathVariants} as well as \TFM\ \quote {extensible recipes}. In most cases where
font metrics are involved we have a different code path for traditional fonts end
\OPENTYPE\ fonts.

\section {Extracting values}

You can extract the components of a math character. Say that we have defined:

\starttyping
\Umathcode 1 2 3 4
\stoptyping

then

\starttyping
[\Umathcharclass1] [\Umathcharfam1] [\Umathcharslot1]
\stoptyping

will return:

\starttyping
[2] [3] [4]
\stoptyping

These commands are provides as convenience. Before they came available you could
do the following:

\starttyping
\def\Umathcharclass{\directlua{tex.print(tex.getmathcode(token.scan_int())[1])}}
\def\Umathcharfam  {\directlua{tex.print(tex.getmathcode(token.scan_int())[2])}}
\def\Umathcharslot {\directlua{tex.print(tex.getmathcode(token.scan_int())[3])}}
\stoptyping

\section{fractions}

The \type {\abovewithdelims} command accepts a keyword \type {exact}. When issued
the extra space relative to the rule thickness is not added. One can of course
use the \type {\Umathfraction..gap} commands to influence the spacing. Also the
rule is still positioned around the math axis.

\starttyping
$$ { {a} \abovewithdelims() exact 4pt {b} }$$
\stoptyping

The math parameter table contains some parameters that specify a horizontal and
vertical gap for skewed fractions. Of course some guessing is needed in order to
implement something that uses them. And so we now provide a primitive similar to the
other fraction related ones but with a few options so that one can influence the
rendering. Of course a user can also mess around a bit with the parameters
\type {\Umathskewedfractionhgap} and \type {\Umathskewedfractionvgap}.

The syntax used here is:

\starttyping
{ {1} \Uskewed / <options> {2} }
{ {1} \Uskewedwithdelims / () <options> {2} }
\stoptyping

where the options can be \type {noaxis} and \type {exact}. By default we add half
the axis to the shifts and by default we zero the width of the middle character.
For Latin Modern The result looks as follows:

\def\ShowA#1#2#3{$x + { {#1} \Uskewed           /    #3 {#2} } + x$}
\def\ShowB#1#2#3{$x + { {#1} \Uskewedwithdelims / () #3 {#2} } + x$}

\start
    \switchtobodyfont[modern]
    \starttabulate[||||||]
        \NC \NC
            \ShowA{a}{b}{} \NC
            \ShowA{1}{2}{} \NC
            \ShowB{a}{b}{} \NC
            \ShowB{1}{2}{} \NC
        \NR
        \NC \type{exact} \NC
            \ShowA{a}{b}{exact} \NC
            \ShowA{1}{2}{exact} \NC
            \ShowB{a}{b}{exact} \NC
            \ShowB{1}{2}{exact} \NC
        \NR
        \NC \type{noaxis} \NC
            \ShowA{a}{b}{noaxis} \NC
            \ShowA{1}{2}{noaxis} \NC
            \ShowB{a}{b}{noaxis} \NC
            \ShowB{1}{2}{noaxis} \NC
        \NR
        \NC \type{exact noaxis} \NC
            \ShowA{a}{b}{exact noaxis} \NC
            \ShowA{1}{2}{exact noaxis} \NC
            \ShowB{a}{b}{exact noaxis} \NC
            \ShowB{1}{2}{exact noaxis} \NC
        \NR
    \stoptabulate
\stop

\section {Other Math changes}

\subsection {Verbose versions of single-character math commands}

\LUATEX\ defines six new primitives that have the same function as
\type {^}, \type {_}, \type {$}, and \type {$$}: %$

\starttabulate[|l|l|l|l|]
\NC \bf primitive           \NC \bf explanation \NC \NR
\NC \type {\Usuperscript}      \NC Duplicates the functionality of \type {^} \NC \NR
\NC \type {\Usubscript}        \NC Duplicates the functionality of \type {_} \NC \NR
\NC \type {\Ustartmath}        \NC Duplicates the functionality of \type {$}, % $
                                   when used in non-math mode. \NC \NR
\NC \type {\Ustopmath}         \NC Duplicates the functionality of \type {$}, % $
                                   when used in inline math mode. \NC \NR
\NC \type {\Ustartdisplaymath} \NC Duplicates the functionality of \type {$$}, % $$
                                   when used in non-math mode. \NC \NR
\NC \type {\Ustopdisplaymath}  \NC Duplicates the functionality of \type {$$}, % $$
                                   when used in display math mode. \NC \NR
\stoptabulate

The \type {\Ustopmath} and \type {\Ustopdisplaymath} primitives check if the current
math mode is the correct one (inline vs.\ displayed), but you can freely intermix
the four mathon|/|mathoff commands with explicit dollar sign(s).

\subsection{Allowed math commands in non-math modes}

The commands \type {\mathchar}, and \type {\Umathchar} and control sequences that
are the result of \type {\mathchardef} or \type {\Umathchardef} are also
acceptable in the horizontal and vertical modes. In those cases, the \type
{\textfont} from the requested math family is used.

\section{Math surrounding skips}

Inline math is surrounded by (optional) \type {\mathsurround} spacing but that is fixed
dimension. There is now an additional parameter \type {\mathsurroundskip}. When set to a
non|-|zero value (or zero with some stretch or shrink) this parameter will replace
\type {\mathsurround}. By using an additional parameter instead of changing the nature
of \type {\mathsurround}, we can remain compatible.

% \section{Math todo}
%
% The following items are still todo.
%
% \startitemize
% \startitem
%     Pre-scripts.
% \stopitem
% \startitem
%     Multi-story stacks.
% \stopitem
% \startitem
%     Flattened accents for high characters (maybe).
% \stopitem
% \startitem
%     Better control over the spacing around displays and handling of equation numbers.
% \stopitem
% \startitem
%     Support for multi|-|line displays using \MATHML\ style alignment points.
% \stopitem
% \stopitemize

\subsection {Delimiters: \type{\Uleft}, \type {\Umiddle} and \type {\Uright}}

Normally you will force delimiters to certain sizes by putting an empty box or
rule next to it. The resulting delimiter will either be a character from the
stepwise size range or an extensible. The latter can be quite differently
positioned that the characters as it depends on the fit as well as the fact if
the used characters in the font have depth or height. Commands like (plain \TEX
s) \type {\big} need use this feature. In \LUATEX\ we provide a bit more control
by three variants that supporting optional parameters \type {height}, \type
{depth} and \type {axis}. The following example uses this:

\startbuffer
\Uleft   height 30pt depth 10pt      \Udelimiter "0 "0 "000028
\quad x\quad
\Umiddle height 40pt depth 15pt      \Udelimiter "0 "0 "002016
\quad x\quad
\Uright  height 30pt depth 10pt      \Udelimiter "0 "0 "000029
\quad \quad \quad
\Uleft   height 30pt depth 10pt axis \Udelimiter "0 "0 "000028
\quad x\quad
\Umiddle height 40pt depth 15pt axis \Udelimiter "0 "0 "002016
\quad x\quad
\Uright  height 30pt depth 10pt axis \Udelimiter "0 "0 "000029
\stopbuffer

\typebuffer

\startlinecorrection
\ruledhbox{\mathematics{\getbuffer}}
\stoplinecorrection

The keyword \type {exact} can be used as directive that the real dimensions
should be applied when the criteria can't be met which can happen when we're
still stepping through the successively larger variants. When no dimensions are
given the \type {noaxis} command can be used to prevent shifting over the axis.

You can influence the final class with the keyword \type {class} which will
influence the spacing.

\subsection{Fixed scripts}

We have three parameters that are used for this fixed anchoring:

\starttabulate[|l|l|]
\NC $d$ \NC \type {\Umathsubshiftdown}    \NC \NR
\NC $u$ \NC \type {\Umathsupshiftup}      \NC \NR
\NC $s$ \NC \type {\Umathsubsupshiftdown} \NC \NR
\stoptabulate

When we set \type {\mathscriptsmode} to a value other than zero these are used
for calculating fixed positions. This is something that is needed for instance
for chemistry. You can manipulate the mentioned variables to achive different
effects.

\def\SampleMath#1%
  {$\mathscriptsmode#1\mathupright CH_2 + CH^+_2 + CH^2_2$}

\starttabulate[|c|c|c|l|]
\NC \bf mode \NC \bf down      \NC \bf up        \NC                \NC \NR
\NC 0        \NC dynamic       \NC dynamic       \NC \SampleMath{0} \NC \NR
\NC 1        \NC $d$           \NC $u$           \NC \SampleMath{1} \NC \NR
\NC 2        \NC $s$           \NC $u$           \NC \SampleMath{2} \NC \NR
\NC 3        \NC $s$           \NC $u + s - d$   \NC \SampleMath{3} \NC \NR
\NC 4        \NC $d + (s-d)/2$ \NC $u + (s-d)/2$ \NC \SampleMath{4} \NC \NR
\NC 5        \NC $d$           \NC $u + s - d$   \NC \SampleMath{5} \NC \NR
\stoptabulate

The value of this parameter obeys grouping but applies to the whole current
formula.

% if needed we can put the value in stylenodes but maybe more should go there

\subsection {Tracing}

Because there are quite some math related parameters and values, it is possible
to limit tracing. Only when \type {tracingassigns} and|/|or \type
{tracingrestores} are set to~2 or more they will be traced.

\subsection {Math options}

The logic in the math engine is rather complex and there are often no universal
solutions (read: what works out well for one font, fails for another). Therefore
some variations in the implementation will be driven by options for which a new
primitive \type {\mathoption} has been introduced (so that we don't end up with
many new commands). The approach of options also permits us to see what effect a
specific solution has.

\subsubsection {\type {\mathoption noitaliccompensation}}

This option compensates placement for characters with a built|-|in italic
correction.

\startbuffer
{\showboxes\int}\quad
{\showboxes\int_{|}^{|}}\quad
{\showboxes\int\limits_{|}^{|}}
\stopbuffer

\typebuffer

Gives (with computer modern that has such italics):

\startlinecorrection[blank]
    \switchtobodyfont[modern]
    \startcombination[nx=2,ny=2,distance=5em]
        {\mathoption noitaliccompensation 0\relax \mathematics{\getbuffer}}
            {\nohyphens\type{0:inline}}
        {\mathoption noitaliccompensation 0\relax \mathematics{\displaymath\getbuffer}}
            {\nohyphens\type{0:display}}
        {\mathoption noitaliccompensation 1\relax \mathematics{\getbuffer}}
            {\nohyphens\type{1:inline}}
        {\mathoption noitaliccompensation 1\relax \mathematics{\displaymath\getbuffer}}
            {\nohyphens\type{1:display}}
    \stopcombination
\stoplinecorrection

\subsubsection {\type {\mathoption nocharitalic}}

When two characters follow each other italic correction can interfere. The
following example shows what this option does:

\startbuffer
\catcode"1D443=11
\catcode"1D444=11
\catcode"1D445=11
P( PP PQR
\stopbuffer

\typebuffer

Gives (with computer modern that has such italics):

\startlinecorrection[blank]
    \switchtobodyfont[modern]
    \startcombination[nx=2,ny=2,distance=5em]
        {\mathoption nocharitalic 0\relax \mathematics{\getbuffer}}
            {\nohyphens\type{0:inline}}
        {\mathoption nocharitalic 0\relax \mathematics{\displaymath\getbuffer}}
            {\nohyphens\type{0:display}}
        {\mathoption nocharitalic 1\relax \mathematics{\getbuffer}}
            {\nohyphens\type{1:inline}}
        {\mathoption nocharitalic 1\relax \mathematics{\displaymath\getbuffer}}
            {\nohyphens\type{1:display}}
    \stopcombination
\stoplinecorrection

\subsubsection {\type {\mathoption useoldfractionscaling}}

This option has been introduced as solution for tracker item 604 for fuzzy cases
around either or not present fraction related settings for new fonts.

\stopchapter

\stopcomponent
