% language=uk

\environment luatex-style

\startcomponent luatex-backend

\startchapter[reference=backend,title={The backend libraries}]

\section{The \type {pdf} library}

\topicindex{backend}
\topicindex{\PDF}

This library contains variables and functions that are related to the \PDF\
backend. You can find more details about the expected values to setters in \in
{section} [backendprimitives].

\subsection{\type {mapfile}, \type {mapline}}

\topicindex{map files}
\topicindex{files+map}

\startfunctioncall
pdf.mapfile(<string> map file)
pdf.mapline(<string> map line)
\stopfunctioncall

These two functions can be used to replace primitives \orm {pdfmapfile} and
\orm {pdfmapline} inherited from \PDFTEX. They expect a string as only
parameter and have no return value. The first character in a map line can be
\type {-}, \type {+} or \type {=} which means as much as remove, add or replace
this line.

\subsection {\type {[set|get][catalog|info|names|trailer]}}

\topicindex{\PDF+trailer}
\topicindex{\PDF+catalog}
\topicindex{\PDF+info}

These functions complement the corresponding \PDF\ backend token lists dealing
with metadata. The value types are strings and they are written to the \PDF\
file directly after the token registers set at the \TEX\ end are written.

\subsection {\type {[set|get][pageattributes|pageresources|pagesattributes]}}

\topicindex{\PDF+page attributes}
\topicindex{\PDF+page resources}

These functions complement the corresponding \PDF\ backend token lists dealing
with page resources. The variables have no interaction with the corresponding \PDF\
backend token register. They are written to the \PDF\ file directly after the
token registers set at the \TEX\ end are written.

\subsection{\type {[set|get][xformattributes|xformresources]}}

\topicindex{\PDF+xform attributes}
\topicindex{\PDF+xform resources}

These functions complement the corresponding \PDF\ backend token lists dealing
with reuseable boxes and images. The variables have no interaction with the
corresponding \PDF\ backend token register. They are written to the \PDF\
file directly after the token registers set at the \TEX\ end are written.

\subsection{\type {getversion} and \type {[set|get]minorversion}}

\topicindex{\PDF+version}

The version is frozen in the binary but you can set the minor version. What minor
version you set depends on what \PDF\ features you use. This is out of control
of \LUATEX.

\subsection{\type {getcreationdate}}

\topicindex{\PDF+date}

This function returns a string with the date in the format that ends up in the
\PDF\ file, in this case it's: {\tttf \cldcontext{pdf.getcreationdate()}}.

\subsection{\type {[set|get]inclusionerrorlevel}, \type {[set|get]ignoreunknownimages}}

\topicindex{\PDF+options}

These variable control how error in included image are treated. They are modeled
after the \PDFTEX\ equivalents.

\subsection{\type {[set|get]suppressoptionalinfo}}

\topicindex{\PDF+options}

This bitset determines what kind of info gets flushed. By default we flush all.

\subsection{\type {[set|get]trailerid}}

\topicindex{\PDF+trailer}

You can set your own trailer id. This has to be a valid array string
with checksums.

\subsection{\type {[set|get]compresslevel}}

\topicindex{\PDF+compression}

These two functions set the level of compression. The minimum value is~0,
the maximum is~9.

\subsection{\type {[set|get]objcompresslevel}}

\topicindex{\PDF+compression}

These two functions set the level of compression. The minimum value is~0,
the maximum is~9.

\subsection{\type {[set|get]recompress}}

\topicindex{\PDF+compression}

When set to~1 compressed objects will be decompressed and when compresslevel is
larger than zero be recompressed. This is mostly a debugging feature and should
not be relied upon.

\subsection{\type {[set|get]gentounicode}}

\topicindex{\PDF+unicode}

This flag enables tounicode generation (like in \PDFTEX).

\subsection{\type {[set|get]omitcidset}}

\topicindex{\PDF+options}

This flag disables inclusion of a so called CIDSet which can be handy when aiming
at some of the many \PDF\ substandards.

\subsection{\type {[set|get]decimaldigits}}

\topicindex{\PDF+precision}

These two functions set the accuracy of floats written to the \PDF file. You can
set any value but the backend will not go below~3 and above~6.

\subsection{\type {[set|get]pkresolution}}

\topicindex{\PDF+resolution}

These setter takes two arguments: the resolution and an optional zero or one that
indicates if this is a fixed one. The getter returns these two values.

\subsection{\type {getlast[obj|link|annot]} and \type {getretval}}

\topicindex{\PDF+objects}
\topicindex{\PDF+annotations}

These status variables are similar to the ones traditionally used in the backend
interface at the \TEX\ end.

\subsection{\type {maxobjnum} and \type {objtype}, \type {fontname}, \type {fontobjnum},
\type {fontsize}, \type {xformname}}.

These (and some other) introspective helpers were moved from the \type {tex}
namespace to the \type {pdf} namespace but kept their original names. They are
mostly used when you construct \PDF\ objects yourself and need for instance
information about a (to be) embedded font.

\subsection{\type {[set|get]origin}}

\topicindex{\PDF+positioning}

This one is used to set the horizonal and|/|or vertical offset, a traditional
backend property.

\starttyping
pdf.setorigin() -- sets both to 0pt
pdf.setorigin(tex.sp("1in")) -- sets both to 1in
pdf.setorigin(tex.sp("1in"),tex.sp("1in"))
\stoptyping

The counterpart of this function returns two values.

\subsection{\type {[set|get]imageresolution}}

\topicindex{\PDF+resolution}

These two functions relate to the imageresolution that is used when the image
itself doesn't provide a non|-|zero x or y resolution.

\subsection {\type {[set|get][link|dest|thread|xform]margin}}

\topicindex{\PDF+margins}

These functions can be used to set and retrieve the margins that are added to the
natural bounding boxes of the respective objects.

\subsection {\type {get[pos|hpos|vpos]}}

\topicindex{\PDF+positions}

These functions get current location on the output page, measured from its lower
left corner. The values return scaled points as units.

\starttyping
local h, v = pdf.getpos()
\stoptyping

\subsection {\type {[has|get]matrix}}

\topicindex{\PDF+matrix}

The current matrix transformation is available via the \type {getmatrix} command,
which returns 6 values: \type {sx}, \type {rx}, \type {ry}, \type {sy}, \type
{tx}, and \type {ty}. The \type {hasmatrix} function returns \type {true} when a
matrix is applied.

\starttyping
if pdf.hasmatrix() then
    local sx, rx, ry, sy, tx, ty = pdf.getmatrix()
    -- do something useful or not
end
\stoptyping

\subsection {\type {print}}

\topicindex{\PDF+print to}

You can print a string to the \PDF\ document from within a \lpr {latelua} call.
This function is not to be used inside \prm {directlua} unless you know {\it
exactly} what you are doing.

\startfunctioncall
pdf.print(<string> s)
pdf.print(<string> type, <string> s)
\stopfunctioncall

The optional parameter can be used to mimic the behavior of \PDF\ literals: the
\type {type} is \type {direct} or \type {page}.

\subsection {\type {immediateobj}}

\topicindex{\PDF+objects}

This function creates a \PDF\ object and immediately writes it to the \PDF\ file.
It is modelled after \PDFTEX's \prm {immediate} \orm {pdfobj} primitives. All
function variants return the object number of the newly generated object.

\startfunctioncall
<number> n =
    pdf.immediateobj(<string> objtext)
<number> n =
    pdf.immediateobj("file", <string> filename)
<number> n =
    pdf.immediateobj("stream", <string> streamtext, <string> attrtext)
<number> n =
    pdf.immediateobj("streamfile", <string> filename, <string> attrtext)
\stopfunctioncall

The first version puts the \type {objtext} raw into an object. Only the object
wrapper is automatically generated, but any internal structure (like \type {<<
>>} dictionary markers) needs to be provided by the user. The second version with
keyword \type {file} as first argument puts the contents of the file with name
\type {filename} raw into the object. The third version with keyword \type
{stream} creates a stream object and puts the \type {streamtext} raw into the
stream. The stream length is automatically calculated. The optional \type
{attrtext} goes into the dictionary of that object. The fourth version with
keyword \type {streamfile} does the same as the third one, it just reads the
stream data raw from a file.

An optional first argument can be given to make the function use a previously
reserved \PDF\ object.

\startfunctioncall
<number> n =
    pdf.immediateobj(<integer> n, <string> objtext)
<number> n =
    pdf.immediateobj(<integer> n, "file", <string> filename)
<number> n =
    pdf.immediateobj(<integer> n, "stream", <string> streamtext, <string> attrtext)
<number> n =
    pdf.immediateobj(<integer> n, "streamfile", <string> filename, <string> attrtext)
\stopfunctioncall

\subsection{\type{obj}}

\topicindex{\PDF+objects}

This function creates a \PDF\ object, which is written to the \PDF\ file only
when referenced, e.g., by \type {refobj()}.

All function variants return the object number of the newly generated object, and
there are two separate calling modes. The first mode is modelled after \PDFTEX's
\orm {pdfobj} primitive.

\startfunctioncall
<number> n =
    pdf.obj(<string> objtext)
<number> n =
    pdf.obj("file", <string> filename)
<number> n =
    pdf.obj("stream", <string> streamtext, <string> attrtext)
<number> n =
    pdf.obj("streamfile", <string> filename, <string> attrtext)
\stopfunctioncall

An optional first argument can be given to make the function use a previously
reserved \PDF\ object.

\startfunctioncall
<number> n =
    pdf.obj(<integer> n, <string> objtext)
<number> n =
    pdf.obj(<integer> n, "file", <string> filename)
<number> n =
    pdf.obj(<integer> n, "stream", <string> streamtext, <string> attrtext)
<number> n =
    pdf.obj(<integer> n, "streamfile", <string> filename, <string> attrtext)
\stopfunctioncall

The second mode accepts a single argument table with key--value pairs.

\startfunctioncall
<number> n = pdf.obj {
    type           = <string>,
    immmediate     = <boolean>,
    objnum         = <number>,
    attr           = <string>,
    compresslevel  = <number>,
    objcompression = <boolean>,
    file           = <string>,
    string         = <string>,
    nolength       = <boolean>,
}
\stopfunctioncall

The \type {type} field can have the values \type {raw} and \type {stream}, this
field is required, the others are optional (within constraints). When \type
{nolength} is set, there will be no \type {/Length} entry added to the
dictionary.

Note: this mode makes \type{obj} look more flexible than it actually is: the
constraints from the separate parameter version still apply, so for example you
can't have both \type {string} and \type {file} at the same time.

\subsection {\type {refobj}}

\topicindex{\PDF+objects}

This function, the \LUA\ version of the \orm {pdfrefobj} primitive, references an
object by its object number, so that the object will be written to the \PDF\ file.

\startfunctioncall
pdf.refobj(<integer> n)
\stopfunctioncall

This function works in both the \prm {directlua} and \lpr {latelua} environment.
Inside \prm {directlua} a new whatsit node \quote {pdf_refobj} is created, which
will be marked for flushing during page output and the object is then written
directly after the page, when also the resources objects are written to the \PDF\
file. Inside \lpr {latelua} the object will be marked for flushing.

This function has no return values.

\subsection {\type {reserveobj}}

\topicindex{\PDF+objects}

This function creates an empty \PDF\ object and returns its number.

\startfunctioncall
<number> n = pdf.reserveobj()
<number> n = pdf.reserveobj("annot")
\stopfunctioncall

\subsection {\type {registerannot}}

This function adds an object number to the \type {/Annots} array for the current
page without doing anything else. This function can only be used from within
\lpr {latelua}.

\startfunctioncall
pdf.registerannot (<number> objnum)
\stopfunctioncall

\subsection {\type {newcolorstack}}

\topicindex{\PDF+color stack}

This function allocates a new color stack and returns it's id. The arguments
are the same as for the similar backend extension primitive.

\startfunctioncall
pdf.newcolorstack("0 g","page",true) -- page|direct|origin
\stopfunctioncall

\subsection {\type {setfontattributes}}

\topicindex{\PDF+fonts}

This function will force some additional code into the font resource. It can for
instance be used to add a custom \type {ToUnicode} vector to a bitmap file.

\startfunctioncall
pdf.setfontattributes(<number> font id, <string> pdf code)
\stopfunctioncall

\section{The \type {pdfe} library}

\subsection{Introduction}

\topicindex{\PDF+objects}

\topicindex{\PDF+analyze}
\topicindex{\PDF+\type{pdfe}}

The \type {pdfe} library replaces the \type {epdf} library and provides an
interface to \PDF\ files. It uses the same code as is used for \PDF\ image
inclusion. The \type {pplib} library by Paweł Jackowski replaces the \type
{poppler} (derived from \type {xpdf}) library.

A \PDF\ file is basically a tree of objects and one descends into the tree via
dictionaries (key/value) and arrays (index/value). There are a few topmost
dictionaries that start at root that are accessed more directly.

Although everything in \PDF\ is basically an object we only wrap a few in so
called userdata \LUA\ objects.

\starttabulate
\BC \PDF          \BC \LUA \NC \NR
\NC null          \NC nil \NC \NR
\NC boolean       \NC boolean \NC \NR
\NC integer       \NC integer \NC \NR
\NC float         \NC number \NC \NR
\NC name          \NC string \NC \NR
\NC string        \NC string \NC \NR
\NC array         \NC array userdatum \NC \NR
\NC dictionary    \NC dictionary userdatum \NC \NR
\NC stream        \NC stream userdatum (with related dictionary) \NC \NR
\NC reference     \NC reference userdatum \NC \NR
\stoptabulate

The regular getters return these \LUA\ data types but one can also get more
detailed information.

\subsection {\type {open}, \type {new}, \type {status}, \type {close}, \type {unencrypt}}

A document is loaded from a file or string

\starttyping
<pdfe document> = pdfe.open(filename)
<pdfe document> = pdfe.new(somestring,somelength)
\stoptyping

Such a document is closed with:

\starttyping
pdfe.close(<pdfe document>)
\stoptyping

You can check if a document opened well by:

\starttyping
pdfe.status(<pdfe document>)
\stoptyping

The returned codes are:

\starttabulate[|c|l|]
\DB value       \BC explanation \NC \NR
\TB
\NC \type {-2}  \NC the document failed to open \NC \NR
\NC \type {-1}  \NC the document is (still) protected \NC \NR
\NC \type {0}   \NC the document is not encrypted \NC \NR
\NC \type {2}   \NC the document has been unencrypted \NC \NR
\LL
\stoptabulate

An encrypted document can be unencrypted by the next command where instead of
either password you can give \type {nil}:

\starttyping
pdfe.unencrypt(<pdfe document>,userpassword,ownerpassword)
\stoptyping

\subsection {\type {size}, \type {version}, \type {nofobjects}, \type {nofpages}}

A successfully opened document can provide some information:

\starttyping
bytes = size(<pdfe document>)
major, minor = version(<pdfe document>)
n = nofobjects(<pdfe document>)
n = nofpages(<pdfe document>)
bytes, waste = nofpages(<pdfe document>)
\stoptyping

\subsection {\type {get[catalog|trailer|info}}

For accessing the document structure you start with the so called catalog, a
dictionary:

\starttyping
<pdfe dictionary> = pdfe.getcatalog(<pdfe document>)
\stoptyping

The other two root dictionaries are accessed with:

\starttyping
<pdfe dictionary> = pdfe.gettrailer(<pdfe document>)
<pdfe dictionary> = pdfe.getinfo(<pdfe document>)
\stoptyping

\subsection {\type {getpage}, \type {getbox}}

A specific page can conveniently be reached with the next command, which
returns a dictionary.

\starttyping
<pdfe dictionary> = pdfe.getpage(<pdfe document>,pagenumber)
\stoptyping

Another convenience command gives you the (bounding) box of a (normally page)
which can be inheritted from the document itself. An example of a valid box name
is \type {MediaBox}.

\starttyping
pages = pdfe.getbox(<pdfe document>,boxname)
\stoptyping

\subsection {\type {get[string|integer|number|boolean|string]}}

Common values in dictionaries and arrays are strings, integers, floats, booleans
and names (which are also strings) and these are also normal \LUA\ objects:

\starttyping
s = getstring (<pdfe array|dictionary>,index|key)
i = getinteger(<pdfe array|dictionary>,index|key)
n = getnumber (<pdfe array|dictionary>,index|key)
b = getboolean(<pdfe array|dictionary>,index|key)
n = getname   (<pdfe array|dictionary>,index|key)
\stoptyping

\subsection {\type {get[from][dictionary|array|stream]}}

Normally you will use an index in an array and key in a dictionary but dictionaries
also accept an index. The size of an array or dictionary is available with the
usual \type {#} operator.

\starttyping
<pdfe dictionary>   = getdictionary(<pdfe array|dictionary>,index|key)
<pdfe array>        = getarray     (<pdfe array|dictionary>,index|key)
<pdfe stream>,
<pdfe dictionary>   = getstream    (<pdfe array|dictionary>,index|key)
\stoptyping

These commands return dictionaries, arrays and streams, which are dictionaries
with a blob of data attached.

Before we come to an alternative access mode, we mention that the objects provide
access in a different way too, for instance this is valid:

\starttyping
print(pdfe.open("foo.pdf").Catalog.Type)
\stoptyping

At the topmost level there are \type {Catalog}, \type {Info}, \type {Trailer}
and \type {Pages}, so this is also okay:

\starttyping
print(pdfe.open("foo.pdf").Pages[1])
\stoptyping

Streams are sort of special. When your index or key hits a stream you get back a
stream object and dictionary object. The dictionary you can access in the usual
way and for the stream there are the following methods:

\starttyping
okay   = openstream(<pdfe stream>,[decode])
         closestream(<pdfe stream>)
str, n = readfromstream(<pdfe stream>)
str, n = readwholestream(<pdfe stream>,[decode])
\stoptyping

You either read in chunks, or you ask for the whole. When reading in chunks, you
need to open and close the stream yourself. The \type {n} value indicates the
length read. The \type {decode} parameter controls if the stream data gets
uncompressed.

As with dictionaries, you can access fields in a stream dictionary in the usual
\LUA\ way too. You get the content when you \quote {call} the stream. You can
pass a boolean that indicates if the stream has to be decompressed.

% pdfe.objectcodes      = objectcodes
% pdfe.stringcodes      = stringcodes
% pdfe.encryptioncodes  = encryptioncodes

In addition to the interface described before, there is also a bit lower level
interface available.

\starttyping
key, type, value, detail = getfromdictionary(<pdfe dictionary>,index)
type, value, detail = getfromarray(<pdfe array>,index)
\stoptyping

\starttabulate[|c|l|l|l|]
\DB type       \BC meaning    \BC value            \BC detail \NC \NR
\NC \type {0}  \NC none       \NC nil              \NC \NC \NR
\NC \type {1}  \NC null       \NC nil              \NC \NC \NR
\NC \type {2}  \NC boolean    \NC boolean          \NC \NC \NR
\NC \type {3}  \NC boolean    \NC integer          \NC \NC \NR
\NC \type {4}  \NC number     \NC float            \NC \NC \NR
\NC \type {5}  \NC name       \NC string           \NC \NC \NR
\NC \type {6}  \NC string     \NC string           \NC hex \NC \NR
\NC \type {7}  \NC array      \NC arrayobject      \NC size \NC \NR
\NC \type {8}  \NC dictionary \NC dictionaryobject \NC size \NC \NR
\NC \type {9}  \NC stream     \NC streamobject     \NC dictionary size \NC \NR
\NC \type {10} \NC reference  \NC integer          \NC \NC \NR
\LL
\stoptabulate

A \type {hex} string is (in the \PDF\ file) surrounded by \type {<>} while plain
strings are bounded by \type {<>}.

\subsection {\type {[dictionary|array]totable}}

All entries in a dictionary or table can be fetched with the following commands
where the return values are a hashed or indexed table.

\starttyping
hash = dictionarytotable(<pdfe dictionary>)
list = arraytotable(<pdfe array>)
\stoptyping

You can get a list of pages with:

\starttyping
{ { <pdfe dictionary>, size, objnum }, ... } = pagestotable(<pdfe document>)
\stoptyping

\subsection {\type {getfromreference}}

Because you can have unresolved references, a reference object can be resolved
with:

\starttyping
<pdfe dictionary|array|stream> = getfromreference(<pdfe reference>)
\stoptyping

\section {Memory streams}

\topicindex{\PDF+memory streams}

The \type {pdfe.new} that takes three arguments:

\starttabulate
\DB value           \BC explanation      \NC \NR
\TB
\NC \type {stream}  \NC this is a (in low level \LUA\ speak) light userdata
                        object, i.e.\ a pointer to a sequence of bytes \NC \NR
\NC \type {length}  \NC this is the length of the stream in bytes (the stream can
                        have embedded zeros) \NC \NR
\NC \type {name}    \NC optional, this is a unique identifier that is used for
                        hashing the stream, so that multiple doesn't use more
                        memory \NC \NR
\LL
\stoptabulate

The third argument is optional. When it is not given the function will return an
\type {pdfe} document object as with a regular file, otherwise it will return a
filename that can be used elsewhere (e.g.\ in the image library) to reference the
stream as pseudo file.

Instead of a light userdata stream (which is actually fragile but handy when you
come from a library) you can also pass a \LUA\ string, in which case the given
length is (at most) the string length.

The function returns an \type {pdfe} object and a string. The string can be used in
the \type {img} library instead of a filename. You need to prevent garbage
collection of the object when you use it as image (for instance by storing it
somewhere).

Both the memory stream and it's use in the image library is experimental and can
change. In case you wonder where this can be used: when you use the swiglib
library for \type {graphicmagick}, it can return such a userdata object. This
permits conversion in memory and passing the result directly to the backend. This
might save some runtime in one|-|pass workflows. This feature is currently not
meant for production and we might come up with a better implementation.

\section {The \type {pdfscanner} library}

\topicindex{\PDF+scanner}

The \type {pdfscanner} library allows interpretation of \PDF\ content streams and
\type {/ToUnicode} (cmap) streams. You can get those streams from the \type
{pdfe} library, as explained in an earlier section. There is only a single
top|-|level function in this library:

\startfunctioncall
pdfscanner.scan (<pdfe stream>, <table> operatortable, <table> info)
pdfscanner.scan (<pdfe array>, <table> operatortable, <table> info)
pdfscanner.scan (<string>, <table> operatortable, <table> info)
\stopfunctioncall

The first argument should be a \LUA\ string or a stream or array onject coming
from the \type {pdfe} library. The second argument, \type {operatortable}, should
be a \LUA\ table where the keys are \PDF\ operator name strings and the values
are \LUA\ functions (defined by you) that are used to process those operators.
The functions are called whenever the scanner finds one of these \PDF\ operators
in the content stream(s). The functions are called with two arguments: the \type
{scanner} object itself, and the \type {info} table that was passed are the third
argument to \type {pdfscanner.scan}.

Internally, \type {pdfscanner.scan} loops over the \PDF\ operators in the
stream(s), collecting operands on an internal stack until it finds a \PDF\
operator. If that \PDF\ operator's name exists in \type {operatortable}, then the
associated function is executed. After the function has run (or when there is no
function to execute) the internal operand stack is cleared in preparation for the
next operator, and processing continues.

The \type {scanner} argument to the processing functions is needed because it
offers various methods to get the actual operands from the internal operand
stack.

A simple example of processing a \PDF's document stream could look like this:

\starttyping
local operatortable = { }

operatortable.Do = function(scanner,info)
    local resources = info.resources
    if resources then
        local val     = scanner:pop()
        local name    = val[2]
        local xobject = resources.XObject
        print(info.space .. "Uses XObject " .. name)
        local resources = xobject.Resources
        if resources then
            local newinfo =  {
                space     = info.space .. "  ",
                resources = resources,
            }
            pdfscanner.scan(entry, operatortable, newinfo)
        end
    end
end

local function Analyze(filename)
    local doc = pdfe.open(filename)
    if doc then
        local pages = doc.Pages
        for i=1,#pages do
            local page = pages[i]
            local info = {
              space     = "  " ,
              resources = page.Resources,
            }
            print("Page " .. i)
         -- pdfscanner.scan(page.Contents,operatortable,info)
            pdfscanner.scan(page.Contents(),operatortable,info)
        end
    end
end

Analyze("foo.pdf")
\stoptyping

This example iterates over all the actual content in the \PDF, and prints out the
found \type {XObject} names. While the code demonstrates quite some of the \type
{pdfe} functions, let's focus on the type \type {pdfscanner} specific code
instead.

From the bottom up, the following line runs the scanner with the \PDF\ page's
top|-|level content given in the first argument.

The third argument, \type {info}, contains two entries: \type {space} is used to
indent the printed output, and \type {resources} is needed so that embedded \type
{XForms} can find their own content.

The second argument, \type {operatortable} defines a processing function for a
single \PDF\ operator, \type {Do}.

The function \type {Do} prints the name of the current \type {XObject}, and then
starts a new scanner for that object's content stream, under the condition that
the \type {XObject} is in fact a \type {/Form}. That nested scanner is called
with new \type {info} argument with an updated \type {space} value so that the
indentation of the output nicely nests, and with a new \type {resources} field
to help the next iteration down to properly process any other, embedded \type
{XObject}s.

Of course, this is not a very useful example in practice, but for the purpose of
demonstrating \type {pdfscanner}, it is just long enough. It makes use of only
one \type {scanner} method: \type {scanner:pop()}. That function pops the top
operand of the internal stack, and returns a \LUA\ table where the object at index
one is a string representing the type of the operand, and object two is its
value.

The list of possible operand types and associated \LUA\ value types is:

\starttabulate[|l|l|]
\DB types           \BC type      \NC \NR
\TB
\NC \type{integer}  \NC <number>  \NC \NR
\NC \type{real}     \NC <number>  \NC \NR
\NC \type{boolean}  \NC <boolean> \NC \NR
\NC \type{name}     \NC <string>  \NC \NR
\NC \type{operator} \NC <string>  \NC \NR
\NC \type{string}   \NC <string>  \NC \NR
\NC \type{array}    \NC <table>   \NC \NR
\NC \type{dict}     \NC <table>   \NC \NR
\LL
\stoptabulate

In case of \type {integer} or \type {real}, the value is always a \LUA\ (floating
point) number. In case of \type {name}, the leading slash is always stripped.

In case of \type {string}, please bear in mind that \PDF\ actually supports
different types of strings (with different encodings) in different parts of the
\PDF\ document, so you may need to reencode some of the results; \type {pdfscanner}
always outputs the byte stream without reencoding anything. \type {pdfscanner}
does not differentiate between literal strings and hexadecimal strings (the
hexadecimal values are decoded), and it treats the stream data for inline images
as a string that is the single operand for \type {EI}.

In case of \type {array}, the table content is a list of \type {pop} return
values and in case of \type {dict}, the table keys are \PDF\ name strings and the
values are \type {pop} return values.

There are a few more methods defined that you can ask \type {scanner}:

\starttabulate[|l|p|]
\DB method           \BC explanation \NC \NR
\TB
\NC \type{pop}       \NC see above \NC \NR
\NC \type{popNumber} \NC return only the value of a \type {real} or \type {integer} \NC \NR
\NC \type{popName}   \NC return only the value of a \type {name} \NC \NR
\NC \type{popString} \NC return only the value of a \type {string} \NC \NR
\NC \type{popArray}  \NC return only the value of a \type {array} \NC \NR
\NC \type{popDict}   \NC return only the value of a \type {dict} \NC \NR
\NC \type{popBool}   \NC return only the value of a \type {boolean} \NC \NR
\NC \type{done}      \NC abort further processing of this \type {scan()} call \NC \NR
\LL
\stoptabulate

The \type {popXXX} are convenience functions, and come in handy when you know the
type of the operands beforehand (which you usually do, in \PDF). For example, the
\type {Do} function could have used \type {local name = scanner:popName()}
instead, because the single operand to the \type {Do} operator is always a \PDF\
name object.

The \type {done} function allows you to abort processing of a stream once you
have learned everything you want to learn. This comes in handy while parsing
\type {/ToUnicode}, because there usually is trailing garbage that you are not
interested in. Without \type {done}, processing only ends at the end of the
stream, possibly wasting \CPU\ cycles.

\stopchapter

\stopcomponent
