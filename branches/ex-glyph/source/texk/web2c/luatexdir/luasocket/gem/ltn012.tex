\documentclass[10pt]{article}
\usepackage{fancyvrb}
\usepackage{url}
\DefineVerbatimEnvironment{lua}{Verbatim}{fontsize=\small,commandchars=\@\#\%}
\DefineVerbatimEnvironment{C}{Verbatim}{fontsize=\small,commandchars=\@\#\%}
\DefineVerbatimEnvironment{mime}{Verbatim}{fontsize=\small,commandchars=\$\#\%}
\newcommand{\stick}[1]{\vbox{\setlength{\parskip}{0pt}#1}}
\newcommand{\bl}{\ensuremath{\mathtt{\backslash}}}
\newcommand{\CR}{\texttt{CR}}
\newcommand{\LF}{\texttt{LF}}
\newcommand{\CRLF}{\texttt{CR~LF}}
\newcommand{\nil}{\texttt{nil}}

\title{Filters, sources, sinks, and pumps\\
      {\large or Functional programming for the rest of us}}
\author{Diego Nehab}

\begin{document}

\maketitle

\begin{abstract}
Certain data processing operations can be implemented in the
form of filters. A filter is a function that can process
data received in consecutive invocations, returning partial
results each time it is called.  Examples of operations that
can be implemented as filters include the end-of-line
normalization for text, Base64 and Quoted-Printable transfer
content encodings, the breaking of text into lines, SMTP
dot-stuffing, and there are many others.  Filters become
even more powerful when we allow them to be chained together
to create composite filters. In this context, filters can be
seen as the internal links in a chain of data transformations.
Sources and sinks are the corresponding end points in these
chains. A source is a function that produces data, chunk by
chunk, and a sink is a function that takes data, chunk by
chunk. Finally, pumps are procedures that actively drive
data from a source to a sink, and indirectly through all 
intervening filters.  In this article, we describe the design of an
elegant interface for filters, sources, sinks, chains, and
pumps, and we illustrate each step with concrete examples. 
\end{abstract}

\section{Introduction}

Within the realm of networking applications, we are often
required to apply transformations to streams of data. Examples
include the end-of-line normalization for text, Base64 and
Quoted-Printable transfer content encodings, breaking text
into lines with a maximum number of columns, SMTP
dot-stuffing, \texttt{gzip} compression, HTTP chunked
transfer coding, and the list goes on.

Many complex tasks require a combination of two or more such
transformations, and therefore a general mechanism for
promoting reuse is desirable. In the process of designing
\texttt{LuaSocket~2.0}, we repeatedly faced this problem.
The solution we reached proved to be very general and
convenient. It is based on the concepts of filters, sources,
sinks, and pumps, which we introduce below. 

\emph{Filters} are functions that can be repeatedly invoked
with chunks of input, successively returning processed
chunks of output. Naturally, the result of
concatenating all the output chunks must be the same as the
result of applying the filter to the concatenation of all
input chunks. In fancier language, filters \emph{commute}
with the concatenation operator. More importantly, filters
must handle input data correctly no matter how the stream
has been split into chunks. 

A \emph{chain} is a function that transparently combines the
effect of one or more filters. The interface of a chain is
indistinguishable from the interface of its component
filters.  This  allows a chained filter to be used wherever
an atomic filter is accepted. In particular, chains can be
themselves chained to create arbitrarily complex operations.

Filters can be seen as internal nodes in a network through
which data will flow, potentially being transformed many
times along the way.  Chains connect these nodes together.
The initial and final nodes of the network are
\emph{sources} and \emph{sinks}, respectively.  Less
abstractly, a source is a function that produces new chunks
of data every time it is invoked.  Conversely, sinks are
functions that give a final destination to the chunks of
data they receive in sucessive calls.  Naturally, sources
and sinks can also be chained with filters to produce
filtered sources and sinks.

Finally, filters, chains, sources, and sinks are all passive
entities: they must be repeatedly invoked in order for
anything to happen.  \emph{Pumps} provide the driving force
that pushes data through the network, from a source to a
sink, and indirectly through all intervening filters.

In the following sections, we start with a simplified
interface, which we later refine. The evolution we present
is not contrived: it recreates the steps we ourselves
followed as we consolidated our understanding of these
concepts within our application domain. 

\subsection{A simple example}

The end-of-line normalization of text is a good
example to motivate our initial filter interface. 
Assume we are given text in an unknown end-of-line
convention (including possibly mixed conventions) out of the
commonly found Unix (\LF), Mac OS (\CR), and
DOS (\CRLF) conventions. We would like to be able to 
use the folowing code to normalize the end-of-line markers: 
\begin{quote}
\begin{lua}
@stick#
local CRLF = "\013\010"
local input = source.chain(source.file(io.stdin), normalize(CRLF))
local output = sink.file(io.stdout)
pump.all(input, output)
%
\end{lua}
\end{quote}

This program should read data from the standard input stream
and normalize the end-of-line markers to the canonic
\CRLF\ marker, as defined by the MIME standard. 
Finally, the normalized text should be sent to the standard output
stream.  We use a \emph{file source} that produces data from
standard input, and chain it with a filter that normalizes
the data. The pump then repeatedly obtains data from the
source, and passes it to the \emph{file sink}, which sends
it to the standard output.

In the code above, the \texttt{normalize} \emph{factory} is a
function that creates our normalization filter, which 
replaces any end-of-line marker with the canonic marker. 
The initial filter interface is
trivial: a filter function receives a chunk of input data,
and returns a chunk of processed data.  When there are no
more input data left, the caller notifies the filter by invoking
it with a \nil\ chunk. The filter responds by returning
the final chunk of processed data (which could of course be
the empty string).

Although the interface is extremely simple, the
implementation is not so obvious. A normalization filter
respecting this interface needs to keep some kind of context
between calls. This is because a chunk boundary may lie between 
the \CR\ and \LF\ characters marking the end of a single line. This
need for contextual storage motivates the use of
factories: each time the factory is invoked, it returns a
filter with its own context so that we can have several
independent filters being used at the same time.  For
efficiency reasons, we must avoid the obvious solution of 
concatenating all the input into the context before
producing any output chunks. 

To that end, we break the implementation into two parts:
a low-level filter, and a factory of high-level filters. The
low-level filter is implemented in C and does not maintain
any context between function calls. The high-level filter
factory, implemented in Lua, creates and returns a
high-level filter that maintains whatever context the low-level
filter needs, but isolates the user from its internal
details. That way, we take advantage of C's efficiency to
perform the hard work, and take advantage of Lua's
simplicity for the bookkeeping.

\subsection{The Lua part of the filter}

Below is the complete implementation of the factory of high-level
end-of-line normalization filters:
\begin{quote}
\begin{lua}
@stick#
function filter.cycle(lowlevel, context, extra)
  return function(chunk)
    local ret
    ret, context = lowlevel(context, chunk, extra)
    return ret
  end
end
%

@stick#
function normalize(marker)
  return filter.cycle(eol, 0, marker)
end
%
\end{lua}
\end{quote}

The \texttt{normalize} factory simply calls a more generic
factory, the \texttt{cycle}~factory, passing the low-level
filter~\texttt{eol}. The \texttt{cycle}~factory receives a
low-level filter, an initial context, and an extra
parameter, and returns a new high-level filter.  Each time
the high-level filer is passed a new chunk, it invokes the
low-level filter with the previous context, the new chunk,
and the extra argument.  It is the low-level filter that
does all the work, producing the chunk of processed data and
a new context. The high-level filter then replaces its
internal context, and returns the processed chunk of data to
the user.  Notice that we take advantage of Lua's lexical
scoping to store the context in a closure between function
calls.  

\subsection{The C part of the filter}

As for the low-level filter, we must first accept
that there is no perfect solution to the end-of-line marker
normalization problem. The difficulty comes from an
inherent ambiguity in the definition of empty lines within
mixed input. However, the following solution works well for
any consistent input, as well as for non-empty lines in
mixed input. It also does a reasonable job with empty lines
and serves as a good example of how to implement a low-level
filter.

The idea is to consider both \CR\ and~\LF\ as end-of-line
\emph{candidates}.  We issue a single break if any candidate
is seen alone, or if it is followed by a different
candidate.  In other words, \CR~\CR~and \LF~\LF\ each issue
two end-of-line markers, whereas \CR~\LF~and \LF~\CR\ issue
only one marker each.  It is easy to see that this method
correctly handles the most common end-of-line conventions.

With this in mind, we divide the low-level filter into two
simple functions.  The inner function~\texttt{pushchar} performs the
normalization itself. It takes each input character in turn,
deciding what to output and how to modify the context. The
context tells if the last processed character was an
end-of-line candidate, and if so, which candidate it was.
For efficiency, we use Lua's auxiliary library's buffer
interface: 
\begin{quote}
\begin{C}
@stick#
@#define candidate(c) (c == CR || c == LF)
static int pushchar(int c, int last, const char *marker, 
    luaL_Buffer *buffer) {
  if (candidate(c)) {
    if (candidate(last)) {
      if (c == last) 
        luaL_addstring(buffer, marker);
      return 0;
    } else {
      luaL_addstring(buffer, marker);
      return c;
    }
  } else {
    luaL_pushchar(buffer, c);
    return 0;
  }
}
%
\end{C}
\end{quote}

The outer function~\texttt{eol} simply interfaces with Lua.
It receives the context and input chunk (as well as an
optional custom end-of-line marker), and returns the
transformed output chunk and the new context.
Notice that if the input chunk is \nil, the operation
is considered to be finished. In that case, the loop will
not execute a single time and the context is reset to the
initial state.  This allows the filter to be reused many
times: 
\begin{quote}
\begin{C}
@stick#
static int eol(lua_State *L) {
  int context = luaL_checkint(L, 1);
  size_t isize = 0;
  const char *input = luaL_optlstring(L, 2, NULL, &isize);
  const char *last = input + isize;
  const char *marker = luaL_optstring(L, 3, CRLF);
  luaL_Buffer buffer;
  luaL_buffinit(L, &buffer);
  if (!input) {
    lua_pushnil(L);
    lua_pushnumber(L, 0);
    return 2;
  }
  while (input < last)
    context = pushchar(*input++, context, marker, &buffer);
  luaL_pushresult(&buffer);
  lua_pushnumber(L, context);
  return 2;
}
%
\end{C}
\end{quote}

When designing filters, the challenging part is usually 
deciding what to store in the context. For line breaking, for
instance, it could be the number of bytes that still fit in the
current line.  For Base64 encoding, it could be a string
with the bytes that remain after the division of the input
into 3-byte atoms. The MIME module in the \texttt{LuaSocket}
distribution has many other examples. 

\section{Filter chains}

Chains greatly increase the power of filters.  For example,
according to the standard for Quoted-Printable encoding,
text should be normalized to a canonic end-of-line marker
prior to encoding.  After encoding, the resulting text must
be broken into lines of no more than 76 characters, with the
use of soft line breaks (a line terminated by the \texttt{=}
sign).  To help specifying complex transformations like
this, we define a chain factory that creates a composite
filter from one or more filters.  A chained filter passes
data through all its components, and can be used wherever a
primitive filter is accepted.

The chaining factory is very simple. The auxiliary
function~\texttt{chainpair} chains two filters together,
taking special care if the chunk is the last.  This is
because the final \nil\ chunk notification has to be
pushed through both filters in turn:  
\begin{quote}
\begin{lua}
@stick#
local function chainpair(f1, f2)
  return function(chunk)
    local ret = f2(f1(chunk))
    if chunk then return ret
    else return ret .. f2() end
  end
end
%

@stick#
function filter.chain(...)
  local f = select(1, ...) 
  for i = 2, select('@#', ...) do
    f = chainpair(f, select(i, ...))
  end
  return f
end
%
\end{lua}
\end{quote}

Thanks to the chain factory, we can
define the Quoted-Printable conversion as such:
\begin{quote}
\begin{lua}
@stick#
local qp = filter.chain(normalize(CRLF), encode("quoted-printable"), 
  wrap("quoted-printable"))
local input = source.chain(source.file(io.stdin), qp)
local output = sink.file(io.stdout)
pump.all(input, output)
%
\end{lua}
\end{quote}

\section{Sources, sinks, and pumps}

The filters we introduced so far act as the internal nodes
in a network of transformations. Information flows from node
to node (or rather from one filter to the next) and is
transformed along the way. Chaining filters together is our
way to connect nodes in this network. As the starting point
for the network, we need a source node that produces the
data. In the end of the network, we need a sink node that
gives a final destination to the data.

\subsection{Sources}

A source returns the next chunk of data each time it is
invoked. When there is no more data, it simply returns~\nil.  
In the event of an error, the source can inform the
caller by returning \nil\ followed by the error message.

Below are two simple source factories. The \texttt{empty} source
returns no data, possibly returning an associated error
message. The \texttt{file} source yields the contents of a file 
in a chunk by chunk fashion:
\begin{quote}
\begin{lua}
@stick#
function source.empty(err)
  return function()
    return nil, err
  end
end
%

@stick#
function source.file(handle, io_err)
  if handle then 
    return function()
      local chunk = handle:read(2048)
      if not chunk then handle:close() end
      return chunk
    end
  else return source.empty(io_err or "unable to open file") end
end
%
\end{lua}
\end{quote}

\subsection{Filtered sources}

A filtered source passes its data through the
associated filter before returning it to the caller. 
Filtered sources are useful when working with
functions that get their input data from a source (such as
the pumps in our examples). By chaining a source with one or
more filters, such functions can be transparently provided
with filtered data, with no need to change their interfaces. 
Here is a factory that does the job:
\begin{quote}
\begin{lua}
@stick#
function source.chain(src, f)
  return function()
    if not src then 
      return nil 
    end
    local chunk, err = src()
    if not chunk then 
      src = nil
      return f(nil)
    else 
      return f(chunk) 
    end
  end
end
%
\end{lua}
\end{quote}

\subsection{Sinks}

Just as we defined an interface for a source of data, we can
also define an interface for a data destination.  We call
any function respecting this interface a sink. In our first
example, we used a file sink connected to the standard
output. 

Sinks receive consecutive chunks of data, until the end of
data is signaled by a \nil\ input chunk. A sink can be
notified of an error with an optional extra argument that
contains the error message, following a \nil\ chunk.  
If a sink detects an error itself, and
wishes not to be called again, it can return \nil,
followed by an error message. A return value that
is not \nil\ means the sink will accept more data.

Below are two useful sink factories. 
The table factory creates a sink that stores
individual chunks into an array. The data can later be
efficiently concatenated into a single string with Lua's
\texttt{table.concat} library function. The \texttt{null} sink 
simply discards the chunks it receives:
\begin{quote}
\begin{lua}
@stick#
function sink.table(t)
  t = t or {}
  local f = function(chunk, err)
    if chunk then table.insert(t, chunk) end
    return 1
  end
  return f, t
end
%

@stick#
local function null()
  return 1
end

function sink.null()
  return null
end
%
\end{lua}
\end{quote}

Naturally, filtered sinks are just as useful as filtered
sources. A filtered sink passes each chunk it receives
through the associated filter before handing it down to the
original sink.  In the following example, we use a source
that reads from the standard input.  The input chunks are
sent to a table sink, which has been coupled with a
normalization filter.  The filtered chunks are then
concatenated from the output array, and finally sent to
standard out:
\begin{quote}
\begin{lua}
@stick#
local input = source.file(io.stdin)
local output, t = sink.table()
output = sink.chain(normalize(CRLF), output)
pump.all(input, output)
io.write(table.concat(t))
%
\end{lua}
\end{quote}

\subsection{Pumps}

Although not on purpose, our interface for sources is
compatible with Lua iterators. That is, a source can be
neatly used in conjunction with \texttt{for} loops.  Using
our file source as an iterator, we can write the following
code:
\begin{quote}
\begin{lua}
@stick#
for chunk in source.file(io.stdin) do
  io.write(chunk)
end
%
\end{lua}
\end{quote}

Loops like this will always be present because everything 
we designed so far is passive. Sources, sinks, filters: none
of them can do anything on their own. The operation of
pumping all data a source can provide into a sink is so
common that it deserves its own function:
\begin{quote}
\begin{lua}
@stick#
function pump.step(src, snk)
  local chunk, src_err = src()
  local ret, snk_err = snk(chunk, src_err)
  if chunk and ret then return 1
  else return nil, src_err or snk_err end
end
%

@stick#
function pump.all(src, snk, step)
    step = step or pump.step
    while true do
        local ret, err = step(src, snk)
        if not ret then
            if err then return nil, err
            else return 1 end
        end 
    end
end
%
\end{lua}
\end{quote}

The \texttt{pump.step} function moves one chunk of data from
the source to the sink. The \texttt{pump.all} function takes
an optional \texttt{step} function and uses it to pump all the
data from the source to the sink. 
Here is an example that uses the Base64 and the
line wrapping filters from the \texttt{LuaSocket}
distribution.  The program reads a binary file from
disk and stores it in another file, after encoding it to the
Base64 transfer content encoding:
\begin{quote}
\begin{lua}
@stick#
local input = source.chain(
  source.file(io.open("input.bin", "rb")), 
  encode("base64"))
local output = sink.chain(
  wrap(76),
  sink.file(io.open("output.b64", "w")))
pump.all(input, output)
%
\end{lua}
\end{quote}

The way we split the filters here is not intuitive, on
purpose.  Alternatively, we could have chained the Base64
encode filter and the line-wrap filter together, and then
chain the resulting filter with either the file source or
the file sink. It doesn't really matter. 

\section{Exploding filters}

Our current filter interface has one serious shortcoming.
Consider for example a \texttt{gzip} decompression filter.
During decompression, a small input chunk can be exploded
into a huge amount of data. To address this problem, we
decided to change the filter interface and allow exploding
filters to return large quantities of output data in a chunk
by chunk manner. 

More specifically, after passing each chunk of input to
a filter, and collecting the first chunk of output, the
user must now loop to receive other chunks from the filter until no
filtered data is left. Within these secondary calls, the
caller passes an empty string to the filter. The filter
responds with an empty string when it is ready for the next
input chunk. In the end, after the user passes a
\nil\ chunk notifying the filter that there is no
more input data, the filter might still have to produce too
much output data to return in a single chunk. The user has
to loop again, now passing \nil\ to the filter each time,
until the filter itself returns \nil\ to notify the
user it is finally done.

Fortunately, it is very easy to modify a filter to respect
the new interface. In fact, the end-of-line translation
filter we presented earlier already conforms to it.  The
complexity is encapsulated within the chaining functions,
which must now include a loop. Since these functions only
have to be written once, the user is rarely affected.
Interestingly, the modifications do not have a measurable
negative impact in the performance of filters that do
not need the added flexibility. On the other hand, for a
small price in complexity, the changes make exploding
filters practical.

\section{A complex example}

The LTN12 module in the \texttt{LuaSocket} distribution
implements all the ideas we have described. The MIME
and SMTP modules are tightly integrated with LTN12, 
and can be used to showcase the expressive power of filters,
sources, sinks, and pumps. Below is an example 
of how a user would proceed to define and send a
multipart message, with attachments, using \texttt{LuaSocket}:
\begin{quote}
\begin{mime}
local smtp = require"socket.smtp"
local mime = require"mime"
local ltn12 = require"ltn12"

local message = smtp.message{
  headers = {
    from = "Sicrano <sicrano@example.com>",
    to = "Fulano <fulano@example.com>",
    subject = "A message with an attachment"},
  body = {
    preamble = "Hope you can see the attachment" .. CRLF,
    [1] = {
      body = "Here is our logo" .. CRLF},
    [2] = {
      headers = {
        ["content-type"] = 'image/png; name="luasocket.png"',
        ["content-disposition"] = 
          'attachment; filename="luasocket.png"',
        ["content-description"] = 'LuaSocket logo',
        ["content-transfer-encoding"] = "BASE64"},
      body = ltn12.source.chain(
        ltn12.source.file(io.open("luasocket.png", "rb")),
        ltn12.filter.chain(
          mime.encode("base64"),
          mime.wrap()))}}}

assert(smtp.send{
  rcpt = "<fulano@example.com>",
  from = "<sicrano@example.com>",
  source = message})
\end{mime}
\end{quote}

The \texttt{smtp.message} function receives a table
describing the message, and returns a source. The
\texttt{smtp.send} function takes this source, chains it with the
SMTP dot-stuffing filter, connects a socket sink
with the server, and simply pumps the data. The message is never 
assembled in memory.  Everything is produced on demand, 
transformed in small pieces, and sent to the server in chunks, 
including the file attachment which is loaded from disk and 
encoded on the fly. It just works.

\section{Conclusions}

In this article, we introduced the concepts of filters,
sources, sinks, and pumps to the Lua language. These are
useful tools for stream processing in general. Sources provide
a simple abstraction for data acquisition. Sinks provide an
abstraction for final data destinations. Filters define an
interface for data transformations.  The chaining of
filters, sources and sinks provides an elegant way to create
arbitrarily complex data transformations from simpler
components. Pumps simply push the data through.  

\section{Acknowledgements}

The concepts described in this text are the result of  long
discussions with David Burgess. A version of this text has
been released on-line as the Lua Technical Note 012, hence
the name of the corresponding LuaSocket module, LTN12.  Wim
Couwenberg contributed to the implementation of the module,
and Adrian Sietsma was the first to notice the
correspondence between sources and Lua iterators. 


\end{document}
