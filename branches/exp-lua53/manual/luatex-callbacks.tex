% language=uk

\environment luatex-style
\environment luatex-logos

\startcomponent luatex-callbacks

\startchapter[reference=callbacks,title={\LUATEX\ \LUA\ callbacks}]

\section {Registering callbacks}

This library has functions that register, find and list callbacks. Callbacks are
\LUA\ functions that are called in well defined places. There are two kind of
callbacks: those that mix with existing functionality, and those that (when
enabled) replace functionality. In mosty cases the second category is expected to
behave similar to the built in functionality because in a next step specific
data is expected. For instance, you can replace the hyphenation routine. The
function gets a list that can be hyphenated (or not). The final list should be
valid and is (normally) used for constructing a paragraph. Another function can
replace the ligature builder and|/|or kerner. Doing something else is possible
but in the end might not give the user the expected outcome.

The first thing you need to do is registering a callback:

\startfunctioncall
id, error =
    callback.register (<string> callback_name, <function> func)
id, error =
    callback.register (<string> callback_name, nil)
id, error =
    callback.register (<string> callback_name, false)
\stopfunctioncall

Here the \syntax {callback_name} is a predefined callback name, see below. The
function returns the internal \type {id} of the callback or \type {nil}, if the
callback could not be registered. In the latter case, \type {error} contains an
error message, otherwise it is \type {nil}.

\LUATEX\ internalizes the callback function in such a way that it does not matter
if you redefine a function accidentally.

Callback assignments are always global. You can use the special value \type {nil}
instead of a function for clearing the callback.

For some minor speed gain, you can assign the boolean \type {false} to the
non|-|file related callbacks, doing so will prevent \LUATEX\ from executing
whatever it would execute by default (when no callback function is registered at
all). Be warned: this may cause all sorts of grief unless you know \notabene
{exactly} what you are doing!

\startfunctioncall
<table> info =
    callback.list()
\stopfunctioncall

The keys in the table are the known callback names, the value is a boolean where
\type {true} means that the callback is currently set (active).

\startfunctioncall
<function> f = callback.find (callback_name)
\stopfunctioncall

If the callback is not set, \type {callback.find} returns \type {nil}.

\section{File discovery callbacks}

The behaviour documented in this subsection is considered stable in the sense that
there will not be backward|-|incompatible changes any more.

\subsection{\type {find_read_file} and \type {find_write_file}}

Your callback function should have the following conventions:

\startfunctioncall
<string> actual_name =
    function (<number> id_number, <string> asked_name)
\stopfunctioncall

Arguments:

\startitemize

\sym{id_number}

This number is zero for the log or \type {\input} files. For \TEX's \type {\read}
or \type {\write} the number is incremented by one, so \type {\read0} becomes~1.

\sym{asked_name}

This is the user|-|supplied filename, as found by \type {\input}, \type {\openin}
or \type {\openout}.

\stopitemize

Return value:

\startitemize

\sym{actual_name}

This is the filename used. For the very first file that is read in by \TEX, you
have to make sure you return an \type {actual_name} that has an extension and
that is suitable for use as \type {jobname}. If you don't, you will have to
manually fix the name of the log file and output file after \LUATEX\ is finished,
and an eventual format filename will become mangled. That is because these file
names depend on the jobname.

You have to return \type {nil} if the file cannot be found.

\stopitemize

\subsection{\type {find_font_file}}

Your callback function should have the following conventions:

\startfunctioncall
<string> actual_name =
    function (<string> asked_name)
\stopfunctioncall

The \type {asked_name} is an \OTF\ or \TFM\ font metrics file.

Return \type {nil} if the file cannot be found.

\subsection{\type {find_output_file}}

Your callback function should have the following conventions:

\startfunctioncall
<string> actual_name =
    function (<string> asked_name)
\stopfunctioncall

The \type {asked_name} is the \PDF\ or \DVI\ file for writing.

\subsection{\type {find_format_file}}

Your callback function should have the following conventions:

\startfunctioncall
<string> actual_name =
    function (<string> asked_name)
\stopfunctioncall

The \type {asked_name} is a format file for reading (the format file for writing
is always opened in the current directory).

\subsection{\type {find_vf_file}}

Like \type {find_font_file}, but for virtual fonts. This applies to both \ALEPH's
\OVF\ files and traditional Knuthian \VF\ files.

\subsection{\type {find_map_file}}

Like \type {find_font_file}, but for map files.

\subsection{\type {find_enc_file}}

Like \type {find_font_file}, but for enc files.

\subsection{\type {find_pk_file}}

Like \type {find_font_file}, but for pk bitmap files. This callback takes two
arguments: \type {name} and \type {dpi}. In your callback you can decide to
look for:

\starttyping
<base res>dpi/<fontname>.<actual res>pk
\stoptyping

but other strategies are possible. It is up to you to find a \quote {reasonable}
bitmap file to go with that specification.

\subsection{\type {find_data_file}}

Like \type {find_font_file}, but for embedded files (\type {\pdfobj file '...'}).

\subsection{\type {find_opentype_file}}

Like \type {find_font_file}, but for \OPENTYPE\ font files.

\subsection{\type {find_truetype_file} and \type {find_type1_file}}

Your callback function should have the following conventions:

\startfunctioncall
<string> actual_name =
    function (<string> asked_name)
\stopfunctioncall

The \type {asked_name} is a font file. This callback is called while \LUATEX\ is
building its internal list of needed font files, so the actual timing may
surprise you. Your return value is later fed back into the matching \type
{read_file} callback.

Strangely enough, \type {find_type1_file} is also used for \OPENTYPE\ (\OTF)
fonts.

\subsection{\type {find_image_file}}

Your callback function should have the following conventions:

\startfunctioncall
<string> actual_name =
    function (<string> asked_name)
\stopfunctioncall

The \type {asked_name} is an image file. Your return value is used to open a file
from the hard disk, so make sure you return something that is considered the name
of a valid file by your operating system.

\subsection[iocallback]{File reading callbacks}

The behavior documented in this subsection is considered stable in the sense that
there will not be backward-incompatible changes any more.

\subsection{\type {open_read_file}}

Your callback function should have the following conventions:

\startfunctioncall
<table> env =
    function (<string> file_name)
\stopfunctioncall

Argument:

\startitemize

\sym{file_name}

The filename returned by a previous \type {find_read_file} or the return value of
\type {kpse.find_file()} if there was no such callback defined.

\stopitemize

Return value:

\startitemize

\sym{env}

This is a table containing at least one required and one optional callback
function for this file. The required field is \type {reader} and the associated
function will be called once for each new line to be read, the optional one is
\type {close} that will be called once when \LUATEX\ is done with the file.

\LUATEX\ never looks at the rest of the table, so you can use it to store your
private per|-|file data. Both the callback functions will receive the table as
their only argument.

\stopitemize

\subsubsection{\type {reader}}

\LUATEX\ will run this function whenever it needs a new input line from the file.

\startfunctioncall
function(<table> env)
    return <string> line
end
\stopfunctioncall

Your function should return either a string or \type {nil}. The value \type {nil}
signals that the end of file has occurred, and will make \TEX\ call the optional
\type {close} function next.

\subsubsection{\type {close}}

\LUATEX\ will run this optional function when it decides to close the file.

\startfunctioncall
function(<table> env)
end
\stopfunctioncall

Your function should not return any value.

\subsection{General file readers}

There is a set of callbacks for the loading of binary data files. These all use
the same interface:

\startfunctioncall
function(<string> name)
    return <boolean> success, <string> data, <number> data_size
end
\stopfunctioncall

The \type {name} will normally be a full path name as it is returned by either
one of the file discovery callbacks or the internal version of \type
{kpse.find_file()}.

\startitemize

\sym{success}

Return \type {false} when a fatal error occurred (e.g.\ when the file cannot be
found, after all).

\sym{data}

The bytes comprising the file.

\sym{data_size}

The length of the \type {data}, in bytes.

\stopitemize

Return an empty string and zero if the file was found but there was a
reading problem.

The list of functions is as follows:

\starttabulate[|l|p|]
\NC \type{read_font_file}     \NC ofm or tfm files \NC \NR
\NC \type{read_vf_file}       \NC virtual fonts \NC \NR
\NC \type{read_map_file}      \NC map files \NC \NR
\NC \type{read_enc_file}      \NC encoding files \NC \NR
\NC \type{read_pk_file}       \NC pk bitmap files \NC \NR
\NC \type{read_data_file}     \NC embedded files (as is possible with \PDF\ objects) \NC \NR
\NC \type{read_truetype_file} \NC \TRUETYPE\ font files \NC \NR
\NC \type{read_type1_file}    \NC \TYPEONE\ font files \NC \NR
\NC \type{read_opentype_file} \NC \OPENTYPE\ font files \NC \NR
\stoptabulate

\section{Data processing callbacks}

\subsection{\type {process_input_buffer}}

This callback allows you to change the contents of the line input buffer just
before \LUATEX\ actually starts looking at it.

\startfunctioncall
function(<string> buffer)
    return <string> adjusted_buffer
end
\stopfunctioncall

If you return \type {nil}, \LUATEX\ will pretend like your callback never
happened. You can gain a small amount of processing time from that. This callback
does not replace any internal code.

\subsection{\type {process_output_buffer}}

This callback allows you to change the contents of the line output buffer just
before \LUATEX\ actually starts writing it to a file as the result of a \type
{\write} command. It is only called for output to an actual file (that is,
excluding the log, the terminal, and \type {\write18} calls).

\startfunctioncall
function(<string> buffer)
    return <string> adjusted_buffer
end
\stopfunctioncall

If you return \type {nil}, \LUATEX\ will pretend like your callback never
happened. You can gain a small amount of processing time from that. This callback
does not replace any internal code.

\subsection{\type {process_jobname}}

This callback allows you to change the jobname given by \type {\jobname} in \TEX\
and \type {tex.jobname} in Lua. It does not affect the internal job name or the
name of the output or log files.

\startfunctioncall
function(<string> jobname)
    return <string> adjusted_jobname
end
\stopfunctioncall

The only argument is the actual job name; you should not use \type {tex.jobname}
inside this function or infinite recursion may occur. If you return \type {nil},
\LUATEX\ will pretend your callback never happened. This callback does not
replace any internal code.

\section{Node list processing callbacks}

The description of nodes and node lists is in~\in{chapter}[nodes].

\subsection{\type {contribute_filter}}

This callback is called when \LUATEX\ adds contents to list:

\startfunctioncall
function(<string> extrainfo)
end
\stopfunctioncall

The string reports the group code. From this you can deduce from
what list you can give a treat.

\starttabulate[|l|p|]
\BC value             \BC explanation                                  \NC \NR
\NC \type{pre_box}    \NC interline material is being added            \NC \NR
\NC \type{pre_adjust} \NC \type {\vadjust} material is being added     \NC \NR
\NC \type{box}        \NC a typeset box is being added (always called) \NC \NR
\NC \type{adjust}     \NC \type {\vadjust} material is being added     \NC \NR
\stoptabulate

\subsection{\type {buildpage_filter}}

This callback is called whenever \LUATEX\ is ready to move stuff to the main
vertical list. You can use this callback to do specialized manipulation of the
page building stage like imposition or column balancing.

\startfunctioncall
function(<string> extrainfo)
end
\stopfunctioncall

The string \type {extrainfo} gives some additional information about what \TEX's
state is with respect to the \quote {current page}. The possible values for the
\type {buildpage_filter} callback are:

\starttabulate[|l|p|]
\BC value                  \BC explanation                               \NC \NR
\NC \type{alignment}       \NC a (partial) alignment is being added      \NC \NR
\NC \type{after_output}    \NC an output routine has just finished       \NC \NR
\NC \type{new_graf}        \NC the beginning of a new paragraph          \NC \NR
\NC \type{vmode_par}       \NC \type {\par} was found in vertical mode   \NC \NR
\NC \type{hmode_par}       \NC \type {\par} was found in horizontal mode \NC \NR
\NC \type{insert}          \NC an insert is added                        \NC \NR
\NC \type{penalty}         \NC a penalty (in vertical mode)              \NC \NR
\NC \type{before_display}  \NC immediately before a display starts       \NC \NR
\NC \type{after_display}   \NC a display is finished                     \NC \NR
\NC \type{end}             \NC \LUATEX\ is terminating (it's all over)   \NC \NR
\stoptabulate

\subsection{\type {build_page_insert}}

This callback is called when the pagebuilder adds an insert. There is not much
control over this mechanism but this callback permits some last minute
manipulations of the spacing before an insert, something that might be handy when
for instance multiple inserts (types) are appended in a row.

\startfunctioncall
function(<number> n, <number> i)
    return <number> register
end
\stopfunctioncall

with

\starttabulate[|l|p|]
\BC value    \BC explanation             \NC \NR
\NC \type{n} \NC the insert class        \NC \NR
\NC \type{i} \NC the order of the insert \NC \NR
\stoptabulate

The return value is a number indicating the skip register to use for the
prepended spacing. This permits for instance a different top space (when \type
{i} equals one) and intermediate space (when \type {i} is larger than one). Of course
you can mess with the insert box but you need to make sure that \LUATEX\ is happy
afterwards.

\subsection{\type {pre_linebreak_filter}}

This callback is called just before \LUATEX\ starts converting a list of nodes
into a stack of \type {\hbox}es, after the addition of \type {\parfillskip}.

\startfunctioncall
function(<node> head, <string> groupcode)
    return true | false | <node> newhead
end
\stopfunctioncall

The string called \type {groupcode} identifies the nodelist's context within
\TEX's processing. The range of possibilities is given in the table below, but
not all of those can actually appear in \type {pre_linebreak_filter}, some are
for the \type {hpack_filter} and \type {vpack_filter} callbacks that will be
explained in the next two paragraphs.

\starttabulate[|l|p|]
\BC value                \BC explanation                                     \NC \NR
\NC \type{<empty>}       \NC main vertical list                              \NC \NR
\NC \type{hbox}          \NC \type {\hbox} in horizontal mode                \NC \NR
\NC \type{adjusted_hbox} \NC \type {\hbox} in vertical mode                  \NC \NR
\NC \type{vbox}          \NC \type {\vbox}                                   \NC \NR
\NC \type{vtop}          \NC \type {\vtop}                                   \NC \NR
\NC \type{align}         \NC \type {\halign} or \type {\valign}              \NC \NR
\NC \type{disc}          \NC discretionaries                                 \NC \NR
\NC \type{insert}        \NC packaging an insert                             \NC \NR
\NC \type{vcenter}       \NC \type {\vcenter}                                \NC \NR
\NC \type{local_box}     \NC \type {\localleftbox} or \type {\localrightbox} \NC \NR
\NC \type{split_off}     \NC top of a \type {\vsplit}                        \NC \NR
\NC \type{split_keep}    \NC remainder of a \type {\vsplit}                  \NC \NR
\NC \type{align_set}     \NC alignment cell                                  \NC \NR
\NC \type{fin_row}       \NC alignment row                                   \NC \NR
\stoptabulate

As for all the callbacks that deal with nodes, the return value can be one of
three things:

\startitemize
\startitem
    boolean \type {true} signals successful processing
\stopitem
\startitem
    \type {<node>} signals that the \quote {head} node should be replaced by the
    returned node
\stopitem
\startitem
    boolean \type {false} signals that the \quote {head} node list should be
    ignored and flushed from memory
\stopitem
\stopitemize

This callback does not replace any internal code.

\subsection{\type {linebreak_filter}}

This callback replaces \LUATEX's line breaking algorithm.

\startfunctioncall
function(<node> head, <boolean> is_display)
    return <node> newhead
end
\stopfunctioncall

The returned node is the head of the list that will be added to the main vertical
list, the boolean argument is true if this paragraph is interrupted by a
following math display.

If you return something that is not a \type {<node>}, \LUATEX\ will apply the
internal linebreak algorithm on the list that starts at \type {<head>}.
Otherwise, the \type {<node>} you return is supposed to be the head of a list of
nodes that are all allowed in vertical mode, and at least one of those has to
represent a hbox. Failure to do so will result in a fatal error.

Setting this callback to \type {false} is possible, but dangerous, because it is
possible you will end up in an unfixable \quote {deadcycles loop}.

\subsection{\type {append_to_vlist_filter}}

This callback is called whenever \LUATEX\ adds a box to a vertical list:

\startfunctioncall
function(<node> box, <string> locationcode, <number prevdepth>,
    <boolean> mirrored)
    return list, prevdepth
end
\stopfunctioncall

It is ok to return nothing in which case you also need to flush the box or deal
with it yourself. The prevdepth is also optional. Locations are \type {box},
\type {alignment}, \type {equation}, \type {equation_number} and \type
{post_linebreak}.

\subsection{\type {post_linebreak_filter}}

This callback is called just after \LUATEX\ has converted a list of nodes into a
stack of \type {\hbox}es.

\startfunctioncall
function(<node> head, <string> groupcode)
    return true | false | <node> newhead
end
\stopfunctioncall

This callback does not replace any internal code.

\subsection{\type {hpack_filter}}

This callback is called when \TEX\ is ready to start boxing some horizontal mode
material. Math items and line boxes are ignored at the moment.

\startfunctioncall
function(<node> head, <string> groupcode, <number> size,
         <string> packtype [, <string> direction] [, <node> attributelist])
    return true | false | <node> newhead
end
\stopfunctioncall

The \type {packtype} is either \type {additional} or \type {exactly}. If \type
{additional}, then the \type {size} is a \type {\hbox spread ...} argument. If
\type {exactly}, then the \type {size} is a \type {\hbox to ...}. In both cases,
the number is in scaled points.

The \type {direction} is either one of the three-letter direction specifier
strings, or \type {nil}.

This callback does not replace any internal code.

\subsection{\type {vpack_filter}}

This callback is called when \TEX\ is ready to start boxing some vertical mode
material. Math displays are ignored at the moment.

This function is very similar to the \type {hpack_filter}. Besides the fact
that it is called at different moments, there is an extra variable that matches
\TEX's \type {\maxdepth} setting.

\startfunctioncall
function(<node> head, <string> groupcode, <number> size, <string> packtype,
        <number> maxdepth [, <string> direction] [, <node> attributelist]))
    return true | false | <node> newhead
end
\stopfunctioncall

This callback does not replace any internal code.

\subsection{\type {hpack_quality}}

This callback can be used to intercept the overfull messages that can result from
packing a horizontal list (as happens in the par builder). The function takes a
few arguments:

\startfunctioncall
function(<string> incident, <number> detail, <node> head, <number> first,
         <number> last)
    return <node> whatever
end
\stopfunctioncall

The incident is one of \type {overfull}, \type {underfull}, \type {loose} or
\type {tight}. The detail is either the amount of overflow in case of \type
{overfull}, or the badness otherwise. The head is the list that is constructed
(when protrusion or expansion is enabled, this is an intermediate list).
Optionally you can return a node, for instance an overfull rule indicator. That
node will be appended to the list (just like \TEX's own rule would).

\subsection{\type {vpack_quality}}

This callback can be used to intercept the overfull messages that can result from
packing a vertical list (as happens in the page builder). The function takes a
few arguments:

\startfunctioncall
function(<string> incident, <number> detail, <node> head, <number> first,
         <number> last)
end
\stopfunctioncall

The incident is one of \type {overfull}, \type {underfull}, \type {loose} or
\type {tight}. The detail is either the amount of overflow in case of \type
{overfull}, or the badness otherwise. The head is the list that is constructed.

\subsection{\type {process_rule}}

This is an experimental callback. It can be used with rules of subtype~4
(user). The callback gets three arguments: the node, the width and the
height. The callback can use \type {pdf.print} to write code to the \PDF\
file but beware of not messing up the final result. No checking is done.

\subsection{\type {pre_output_filter}}

This callback is called when \TEX\ is ready to start boxing the box 255 for \type
{\output}.

\startfunctioncall
function(<node> head, <string> groupcode, <number> size, <string> packtype,
        <number> maxdepth [, <string> direction])
    return true | false | <node> newhead
end
\stopfunctioncall

This callback does not replace any internal code.

\subsection{\type {hyphenate}}

\startfunctioncall
function(<node> head, <node> tail)
end
\stopfunctioncall

No return values. This callback has to insert discretionary nodes in the node
list it receives.

Setting this callback to \type {false} will prevent the internal discretionary
insertion pass.

\subsection{\type {ligaturing}}

\startfunctioncall
function(<node> head, <node> tail)
end
\stopfunctioncall

No return values. This callback has to apply ligaturing to the node list it
receives.

You don't have to worry about return values because the \type {head} node that is
passed on to the callback is guaranteed not to be a glyph_node (if need be, a
temporary node will be prepended), and therefore it cannot be affected by the
mutations that take place. After the callback, the internal value of the \quote
{tail of the list} will be recalculated.

The \type {next} of \type {head} is guaranteed to be non-nil.

The \type {next} of \type {tail} is guaranteed to be nil, and therefore the
second callback argument can often be ignored. It is provided for orthogonality,
and because it can sometimes be handy when special processing has to take place.

Setting this callback to \type {false} will prevent the internal ligature
creation pass.

You must not ruin the node list. For instance, the head normally is a local par node,
and the tail a glue. Messing too much can push \LUATEX\ into panic mode.

\subsection{\type {kerning}}

\startfunctioncall
function(<node> head, <node> tail)
end
\stopfunctioncall

No return values. This callback has to apply kerning between the nodes in the
node list it receives. See \type {ligaturing} for calling conventions.

Setting this callback to \type {false} will prevent the internal kern insertion
pass.

You must not ruin the node list. For instance, the head normally is a local par node,
and the tail a glue. Messing too much can push \LUATEX\ into panic mode.

\subsection{\type {insert_local_par}}

Each paragraph starts with a local par node that keeps track of for instance
the direction. You can hook a callback into the creator:

\startfunctioncall
function(<node> local_par, <string> location)
end
\stopfunctioncall

There is no return value and you should make sure that the node stays valid
as otherwise \TEX\ can get confused.

\subsection{\type {mlist_to_hlist}}

This callback replaces \LUATEX's math list to node list conversion algorithm.

\startfunctioncall
function(<node> head, <string> display_type, <boolean> need_penalties)
    return <node> newhead
end
\stopfunctioncall

The returned node is the head of the list that will be added to the vertical or
horizontal list, the string argument is either \quote {text} or \quote {display}
depending on the current math mode, the boolean argument is \type {true} if
penalties have to be inserted in this list, \type {false} otherwise.

Setting this callback to \type {false} is bad, it will almost certainly result in
an endless loop.

\section{Information reporting callbacks}

\subsection{\type {pre_dump}}

\startfunctioncall
function()
end
\stopfunctioncall

This function is called just before dumping to a format file starts. It does not
replace any code and there are neither arguments nor return values.

\subsection{\type {start_run}}

\startfunctioncall
function()
end
\stopfunctioncall

This callback replaces the code that prints \LUATEX's banner. Note that for
successful use, this callback has to be set in the \LUA\ initialization script,
otherwise it will be seen only after the run has already started.

\subsection{\type {stop_run}}

\startfunctioncall
function()
end
\stopfunctioncall

This callback replaces the code that prints \LUATEX's statistics and \quote
{output written to} messages. The engine can still do housekeeping and therefore
you should not rely on this hook for postprocessing the \PDF\ or log file.

\subsection{\type {start_page_number}}

\startfunctioncall
function()
end
\stopfunctioncall

Replaces the code that prints the \type {[} and the page number at the begin of
\type {\shipout}. This callback will also override the printing of box information
that normally takes place when \type {\tracingoutput} is positive.

\subsection{\type {stop_page_number}}

\startfunctioncall
function()
end
\stopfunctioncall

Replaces the code that prints the \type {]} at the end of \type {\shipout}.

\subsection{\type {show_error_hook}}

\startfunctioncall
function()
end
\stopfunctioncall

This callback is run from inside the \TEX\ error function, and the idea is to
allow you to do some extra reporting on top of what \TEX\ already does (none of
the normal actions are removed). You may find some of the values in the \type
{status} table useful. This callback does not replace any internal code.

\subsection{\type {show_error_message}}

\startfunctioncall
function()
end
\stopfunctioncall

This callback replaces the code that prints the error message. The usual
interaction after the message is not affected.

\subsection{\type {show_lua_error_hook}}

\startfunctioncall
function()
end
\stopfunctioncall

This callback replaces the code that prints the extra \LUA\ error message.

\subsection{\type {start_file}}

\startfunctioncall
function(category,filename)
end
\stopfunctioncall

This callback replaces the code that prints \LUATEX's when a file is opened like
\type {(filename} for regular files. The category is a number:

\starttabulate[|||]
\NC 1 \NC a normal data file, like a \TEX\ source \NC \NR
\NC 2 \NC a font map coupling font names to resources \NC \NR
\NC 3 \NC an image file (\type {png}, \type {pdf}, etc) \NC \NR
\NC 4 \NC an embedded font subset \NC \NR
\NC 5 \NC a fully embedded font \NC \NR
\stoptabulate

\subsection{\type {stop_file}}

\startfunctioncall
function(category)
end
\stopfunctioncall

This callback replaces the code that prints \LUATEX's when a file is closed like
the \type {)} for regular files.

\subsection{\type {call_edit}}

\startfunctioncall
function(filename,linenumber)
end
\stopfunctioncall

This callback replaces the call to an external editor when \quote{E} is pressed
in reply to an error message. Processing will end immediately after the callback
returns control to the main program.

\subsection{\type {finish_synctex_callback}}

This callback can be used to wrap up alternative synctex methods. It kicks in
after the normal synctex finalizer (that happens to remove the synctex files
after a run when native synctex is not enabled).

\section{PDF-related callbacks}

\subsection{\type {finish_pdffile}}

\startfunctioncall
function()
end
\stopfunctioncall

This callback is called when all document pages are already written to the \PDF\
file and \LUATEX\ is about to finalize the output document structure. Its
intended use is final update of \PDF\ dictionaries such as \type {/Catalog} or
\type {/Info}. The callback does not replace any code. There are neither
arguments nor return values.

\subsection{\type {finish_pdfpage}}

\startfunctioncall
function(shippingout)
end
\stopfunctioncall

This callback is called after the \PDF\ page stream has been assembled and before
the page object gets finalized.

\section{Font-related callbacks}

\subsection{\type {define_font}}

\startfunctioncall
function(<string> name, <number> size, <number> id)
    return <table> font | <number> id
end
\stopfunctioncall

The string \type {name} is the filename part of the font specification, as given
by the user.

The number \type {size} is a bit special:

\startitemize[packed]
\startitem
    If it is positive, it specifies an \quote{at size} in scaled points.
\stopitem
\startitem
    If it is negative, its absolute value represents a \quote {scaled} setting
    relative to the designsize of the font.
\stopitem
\stopitemize

The \type {id} is the internal number assigned to the font.

The internal structure of the \type {font} table that is to be returned is
explained in \in {chapter} [fonts]. That table is saved internally, so you can
put extra fields in the table for your later \LUA\ code to use. In alternative,
\type {retval} can be a previously defined fontid. This is useful if a previous
definition can be reused instead of creating a whole new font structure.

Setting this callback to \type {false} is pointless as it will prevent font
loading completely but will nevertheless generate errors.

\subsection{\type {glyph_not_found}}

This callback kicks in when the backend cannot insert a glyph. When no callback
is defined a message is written to the log.

\startfunctioncall
function(<number> id, <number> char)
    -- do something with font id and char code
end
\stopfunctioncall

\stopchapter

\stopcomponent
