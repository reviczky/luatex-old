\startenvironment luatex-style

% todo: use \useMPlibrary[lua]

\enabletrackers[fonts.usage]

\usemodule[fonts-statistics]

\setuplayout
  [height=middle,
   width=middle,
   backspace=2cm,
   topspace=10mm,
   bottomspace=10mm,
   header=10mm,
   footer=10mm,
   footerdistance=10mm,
   headerdistance=10mm]

\setuppagenumbering
  [alternative=doublesided]

\setuptolerance
  [stretch,tolerant]

\setuptype
  [lines=hyphenated]

\setuptyping
  [lines=hyphenated]

\setupitemize
  [each]
  [packed]

\definesymbol[1][\Uchar"2023]
\definesymbol[2][\endash]
\definesymbol[3][\wait]      % we want to catch it

\setupitemize
  [each]
  [headcolor=maincolor,
   symbolcolor=maincolor,
   color=maincolor]

\setupwhitespace
  [medium]

\setuptabulate
  [blank={small,samepage},
   headstyle=bold,
   rulecolor=maincolor,
   rulethickness=1pt,
   foregroundcolor=white,
   foregroundstyle=\ss\bfx\WORD,
   backgroundcolor=maincolor]

\setupcaptions
  [headcolor=darkblue]

\startluacode
    local skipped = table.tohash { 'id', 'subtype', 'next', 'prev' }

    function document.functions.showfields(s)
        local t = string.split(s,',')
        local f = node.fields(t[1],t[2])
        if f then
            local d = false
            for i=1,#f do
                local fi = f[i]
                if skipped[fi] then
                    -- okay
                elseif d then
                    context(', {\tttf %s}', fi)
                else
                    context('{\tttf %s}', fi)
                    d = true
                end
            end
        end
    end

    function document.functions.showid(s)
        local t = string.split(s,',')
        context('{tttf %s}',node.id(t[1]))
        if t[2] then
            context(', {tttf %s}',node.subtype(t[2]))
        end
    end

    function document.functions.showsubtypes(s)
        local s = node.subtypes(s)
        local d = false
        for k, v in table.sortedhash(s) do
            if d then
                context(', %s = {\\tttf %s}',k,v)
            else
                context('%s = {\\tttf %s}',k,v)
                d = true
            end
        end
    end
\stopluacode

\unexpanded\def\showfields  #1{\ctxlua{document.functions.showfields("#1")}}
\unexpanded\def\showid      #1{\ctxlua{document.functions.showid("#1")}}
\unexpanded\def\showsubtypes#1{\ctxlua{document.functions.showsubtypes("#1")}}

\definecolor[blue]      [b=.5]
\definecolor[red]       [r=.5]
\definecolor[green]     [g=.5]
\definecolor[maincolor] [b=.5]
\definecolor[keptcolor] [b=.5]
\definecolor[othercolor][r=.5,g=.5]

\writestatus{luatex manual}{}
\writestatus{luatex manual}{defining lucodaot} \usebodyfont  [lucidaot]
\writestatus{luatex manual}{defining pagella}  \usebodyfont  [pagella]
\writestatus{luatex manual}{defining cambria}  \usebodyfont  [cambria]
\writestatus{luatex manual}{defining modern}   \usebodyfont  [modern]
\writestatus{luatex manual}{defining dejavu}   \setupbodyfont[dejavu,10pt]
\writestatus{luatex manual}{}

\setuphead [chapter]      [align={flushleft,broad},style=\bfd]
\setuphead [section]      [align={flushleft,broad},style=\bfb]
\setuphead [subsection]   [align={flushleft,broad},style=\bfa]
\setuphead [subsubsection][align={flushleft,broad},style=\bf]

\setuphead [chapter]      [color=maincolor]
\setuphead [section]      [color=maincolor]
\setuphead [subsection]   [color=maincolor]
\setuphead [subsubsection][color=maincolor]

\definehead
  [remark]
  [subsubsubject]

\setupheadertexts
  []

% \setuplayout
%   [style=bold,
%    color=maincolor]

\definemixedcolumns
  [twocolumns]
  [n=2,
   balance=yes,
   before=\blank,
   after=\blank]

\definemixedcolumns
  [threecolumns]
  [twocolumns]
  [n=3]

\definemixedcolumns
  [fourcolumns]
  [threecolumns]
  [n=4]

% if we do this we also need to do it in table cells
%
% \setuptyping
%   [color=maincolor]
%
% \setuptype
%   [color=maincolor]

\definetyping
  [functioncall]

\startMPdefinitions

    color   luaplanetcolor ; luaplanetcolor := \MPcolor{maincolor} ;
    color   luaholecolor   ; luaholecolor   := white ;
    numeric luaextraangle  ; luaextraangle  := 0 ;
    numeric luaorbitfactor ; luaorbitfactor := .25 ;

    vardef lualogo = image (

        % Graphic design by A. Nakonechnyj. Copyright (c) 1998, All rights reserved.

        save d, r, p ; numeric d, r, p ;

        d := sqrt(2)/4 ; r := 1/4 ; p := r/8 ;

        fill fullcircle scaled 1
            withcolor luaplanetcolor ;
        draw fullcircle rotated 40.5 scaled (1+r)
            dashed evenly scaled p
            withpen pencircle scaled (p/2)
            withcolor (luaorbitfactor * luaholecolor) ;
        fill fullcircle scaled r shifted (d+1/8,d+1/8)
            rotated - luaextraangle
            withcolor luaplanetcolor ;
        fill fullcircle scaled r shifted (d-1/8,d-1/8)
            withcolor luaholecolor   ;
        luaorbitfactor := .25 ;
    ) enddef ;

\stopMPdefinitions

\startuseMPgraphic{luapage}
    StartPage ;

        fill Page withcolor \MPcolor{othercolor} ;

        luaorbitfactor := 1 ;

        picture p ; p := lualogo ysized (5*\measure{paperheight}/10) ;
        draw p
            shifted - center p
            shifted (
                \measure{spreadwidth} - .5*\measure{paperwidth} + \measure{spinewidth},
                .375*\measure{paperheight}
            )
        ;

    StopPage ;
\stopuseMPgraphic

% \starttexdefinition luaextraangle
%     % we can also just access the last page and so in mp directly
%     \ctxlua {
%         context(\lastpage == 0 and 0 or \realfolio*360/\lastpage)
%     }
% \stoptexdefinition

\startuseMPgraphic{luanumber}
  % luaextraangle  := \luaextraangle;
    luaextraangle  := if (LastPageNumber == 0) : 0 else : (RealPageNumber / LastPageNumber) * 360  fi;
    luaorbitfactor := 0.25 ;
    picture p ; p := lualogo ;
    setbounds p to boundingbox fullcircle ;
    draw p ysized 1cm ;
\stopuseMPgraphic

\definelayer
  [page]
  [width=\paperwidth,
   height=\paperheight]

\setupbackgrounds
  [leftpage]
  [background=page]

\setupbackgrounds
  [rightpage]
  [background=page]

\definemeasure[banneroffset][\bottomspace-\footerheight-\footerdistance+2cm]

\startsetups pagenumber:right
  \setlayerframed
    [page]
    [preset=rightbottom,x=1.0cm,y=\measure{banneroffset}]
    [frame=off,height=1cm,offset=overlay]
    {\strut\useMPgraphic{luanumber}}
  \setlayerframed
    [page]
    [preset=rightbottom,x=2.5cm,y=\measure{banneroffset}]
    [frame=off,height=1cm,width=1cm,offset=overlay,
     foregroundstyle=bold,foregroundcolor=maincolor]
    {\strut\pagenumber}
  \setlayerframed
    [page]
    [preset=rightbottom,x=3.5cm,y=\measure{banneroffset}]
    [frame=off,height=1cm,offset=overlay,
     foregroundstyle=bold,foregroundcolor=maincolor]
    {\strut\getmarking[chapter]}
\stopsetups

\startsetups pagenumber:left
  \setlayerframed
    [page]
    [preset=leftbottom,x=3.5cm,y=\measure{banneroffset}]
    [frame=off,height=1cm,offset=overlay,
     foregroundstyle=bold,foregroundcolor=maincolor]
    {\strut\getmarking[chapter]}
  \setlayerframed
    [page]
    [preset=leftbottom,x=2.5cm,y=\measure{banneroffset}]
    [frame=off,height=1cm,width=1cm,offset=overlay,
     foregroundstyle=bold,foregroundcolor=maincolor]
    {\strut\pagenumber}
  \setlayerframed
    [page]
    [preset=leftbottom,x=1.0cm,y=\measure{banneroffset}]
    [frame=off,height=1cm,offset=overlay]
    {\strut\useMPgraphic{luanumber}}
\stopsetups

\unexpanded\def\nonterminal#1>{\mathematics{\langle\hbox{\rm #1}\rangle}}

% taco's brainwave -) .. todo: create a typing variant so that we can avoid the !crlf

\newcatcodetable\syntaxcodetable

\unexpanded\def\makesyntaxcodetable
  {\begingroup
   \catcode`\<=13 \catcode`\|=12
   \catcode`\!= 0 \catcode`\\=12
   \savecatcodetable\syntaxcodetable
   \endgroup}

\makesyntaxcodetable

\unexpanded\def\startsyntax {\begingroup\catcodetable\syntaxcodetable  \dostartsyntax}
\unexpanded\def\syntax      {\begingroup\catcodetable\syntaxcodetable  \dosyntax}
           \let\stopsyntax   \relax

\unexpanded\def\syntaxenvbody#1%
  {\par
   \tt
   \startnarrower
 % \maincolor
   #1
   \stopnarrower
   \par}

\unexpanded\def\syntaxbody#1%
  {\begingroup
 % \maincolor
   \tt #1%
   \endgroup}

\bgroup \catcodetable\syntaxcodetable

!gdef!dostartsyntax#1\stopsyntax{!let<!nonterminal!syntaxenvbody{#1}!endgroup}
!gdef!dosyntax     #1{!let<!nonterminal!syntaxbody{#1}!endgroup}

!egroup

\definetyping
  [texsyntax]
% [color=maincolor]

% end of wave

\setupinteraction
  [state=start,
   focus=standard,
   style=,
   color=,
   contrastcolor=]

\placebookmarks
  [chapter,section,subsection]

\setuplist
  [chapter,section,subsection,subsubsection]
  [interaction=all,
   width=3em]

\setuplist
  [chapter]
  [style=bold,
   before={\testpage[4]\blank},
   color=keptcolor]

\setuplist
  [section]
  [before={\testpage[3]}]

\setuplist
  [subsection,subsubsection]
  [margin=3em,
   width=5em]

\definestartstop
  [notabene]
  [style=slanted]

\definestartstop
  [preamble]
  [style=normal,
   before=\blank,
   after=\blank]

% Hans doesn't like the bookmarks opening by default so we comment this:
%
% \setupinteractionscreen
%   [option=bookmark]

\startbuffer[stylecalculations]

    \normalexpanded{\definemeasure[spinewidth] [0pt]}
    \normalexpanded{\definemeasure[paperwidth] [\the\paperwidth ]}
    \normalexpanded{\definemeasure[paperheight][\the\paperheight]}
    \normalexpanded{\definemeasure[spreadwidth][\measure{paperwidth}]}

\stopbuffer

\getbuffer[stylecalculations]

\dontcomplain

\environment luatex-logos

\defineregister[topicindex]
\defineregister[primitiveindex]
\defineregister[callbackindex]
\defineregister[nodeindex]

\unexpanded\def\lpr#1{\doifmode{*bodypart}{\primitiveindex[#1]{\bf\tex {#1}}}\tex {#1}}
\unexpanded\def\prm#1{\doifmode{*bodypart}{\primitiveindex[#1]{\tex    {#1}}}\tex {#1}}
\unexpanded\def\orm#1{\doifmode{*bodypart}{\primitiveindex[#1]{\tex    {#1}}}\tex {#1}}
\unexpanded\def\cbk#1{\doifmode{*bodypart}{\callbackindex [#1]{\type   {#1}}}\type{#1}}
\unexpanded\def\nod#1{\doifmode{*bodypart}{\nodeindex     [#1]{\bf\type{#1}}}\type{#1}}
\unexpanded\def\whs#1{\doifmode{*bodypart}{\nodeindex     [#1]{\type   {#1}}}\type{#1}}
\unexpanded\def\noa#1{\doifmode{*bodypart}{\nodeindex     [#1]{\type   {#1}}}\type{#1}}

\stopenvironment
