
\nonknuthmode
\catcode`\|=12

\setuplayout[backspace=1cm,edge=0pt,topspace=1cm,
  margin=0pt,location=middle,height=fit,footer=0pt,width=middle,
  header=24pt]

\setuptabulate[before=]
\setuppagenumbering[location=,alternative=doublesided]
\setupheadertexts[Luatex 0.64 short reference][pagenumber]
\setupheader[style=\ssbf]

\setupalign[flushleft]

\setuphead[section][number=no,style=\ssbfb,after=,before=\blank]
\setuphead[subsection][number=no,after=,style=\ssbf]

\switchtobodyfont[times,9pt]
\usemodule[abr-01]
\def\DPI{DPI}

\setuptype[style=\ss]
\setuptyping[style=\ss]

\def\afunction#1{{\ss\bf #1}}
\def\aliteral#1{{\ss "#1"}}

\def\showitemnoret#1#2#3#4#5#6%
  {\hangafter=1\hangindent=12pt
   \noindent {\ss \afunction{#1#6#2}#3}%
   \hskip 12pt plus .5\hsize\penalty0 \hskip 0pt plus -.5\hsize 
   #5\par}

\def\showitemret#1#2#3#4#5#6%
  {\hangafter=1\hangindent=12pt
   \noindent {\ss #4 = \afunction{#1#6#2}#3}%
   \hskip 12pt plus .5\hsize\penalty0\hskip 0pt plus -.5\hsize 
   #5\par}

\def\luafunction#1#2#3#4#5%
  {\doifemptyelse{#4}
    {\showitemnoret{#1}{#2}{#3}{#4}{#5}{.}}
    {\showitemret{#1}{#2}{#3}{#4}{#5}{.}}\par}

\def\luamethod#1#2#3#4#5%
  {\doifemptyelse{#4}
    {\showitemnoret{#1}{#2}{#3}{#4}{#5}{:}}
    {\showitemret{#1}{#2}{#3}{#4}{#5}{:}}\par}

\def\luacallback#1#2#3#4#5%
  {\doifemptyelse{#4}
    {\showitemnoret{}{#2}{#3}{#4}{#5}{}}
    {\showitemret{}{#2}{#3}{#4}{#5}{}}\par}

\startluacode 
local fdata = dofile('fdata.lua')
function showfunction (t,f)
         local val = fdata[t][f]
         local r = {}
         local a = {}
         if #val.returnvalues > 0 then
            for i,v in ipairs (val.returnvalues) do
               if v.optional then
		  r[#r+1] = '[' .. v.type .. ']'
               else
                 r[#r+1] = v.type
               end
            end
         end
         if #val.arguments > 0 then
            for i,v in ipairs (val.arguments) do
               if v.optional then
		  a[#a+1] = '[' ..  v.type .. ']'
               else
                 a[#a+1] = v.type
               end
            end
         end
         tex.sprint(string.format('\\lua%s{%s}{%s}{(%s)}{%s}{%s}', val.type, t, f, 
			       table.concat(a,', '),table.concat(r,', '), val.shortdesc))
end

\stopluacode

\def\presentfunction#1#2{\directlua{showfunction("#1","#2")}}

\starttext


\startcolumns

\section{Getting started}

\type{luatex} is a typesetter; \type{texlua} and \type{luatex --luaonly} are lua interpreters.
In lua interpreter mode, the lua tables \type{tex}, \type{token}, \type{node}, and \type{pdf}
are unavailable.

\blank

\tex{directlua}\type{} expands immediately, \tex{latelua}\type{}
is processed during \tex{output}.

\blank

\starttabulate[|l|p|]
\NC \tex{luatexversion}:   \NC {\bf\the\luatexversion} \NC\NR
\NC \tex{luatexrevision}: \NC {\bf\luatexrevision}    \NC\NR
\NC \tex{luatexdatestamp}: \NC {\bf\luatexdatestamp}  \NC\NR
\stoptabulate

\subsection{Attributes}

There are 65536 attribute registers, each of which obeys grouping
and can have any integer value other than the most negative number
(which indicates the value is unset).

\subsection{Catcode tables}

There are 65536 category code tables, each of which can hold the full range
of Unicode. Category table manipulation:

\starttabulate[|l|p|]
\NC \tex{initcatcodetable} $N$ \NC initialize an \quote{initex} table in $N$ \NC\NR
\NC \tex{savecatcodetable} $N$\NC save the current category codes to table $N$ globally\NC\NR
\NC \tex{catcodetable} $N$ \NC  switch to predefined table $N$\NC\NR
\stoptabulate

\subsection{Filenames}

\tex{input}, \tex{openin}, and \tex{font} accept braced file names to allow embedded spaces.

\subsection{Images and forms}

\type{\pdfrefximage} and \type{\pdfrefxform} accept optional dimension parameters 
in the same format as \type{\pdfximage}.

\section{Preloaded lua modules}

\starttabulate[|l|l|]
\NC\type{slnunicode}\NC\hyphenatedurl{http://luaforge.net/projects/sln}\NC\NR
\NC\type{luazip}    \NC\hyphenatedurl{http://www.keplerproject.org/luazip/}\NC\NR
\NC\type{luafilesystem} \NC\hyphenatedurl{http://www.keplerproject.org/luafilesystem/}\NC\NR
\NC\type{lpeg}      \NC\hyphenatedurl{http://www.inf.puc-rio.br/~roberto/lpeg.html}\NC\NR
\NC\type{lzlib}     \NC\hyphenatedurl{http://luaforge.net/projects/lzlib/}\NC\NR
\NC\type{md5}       \NC\hyphenatedurl{http://www.inf.puc-rio.br/~roberto/md5/md5-5/md5.html}\NC\NR
\NC\type{luasocket} \NC \kern -12pt\hyphenatedurl{http://www.tecgraf.puc-rio.br/~diego/professional/luasocket/}\NC\NR
\stoptabulate

\section{String extensions}

\presentfunction{string}{explode}
\presentfunction{string}{utfvalues}
\presentfunction{string}{utfcharacters}
\presentfunction{string}{characters}
\presentfunction{string}{characterpairs}
\presentfunction{string}{bytes}
\presentfunction{string}{bytepairs}

\section{Operating system extensions}

\presentfunction{os}{exec}
\presentfunction{os}{spawn}
\presentfunction{os}{setenv}
\presentfunction{os}{gettimeofday}
\presentfunction{os}{times}
\presentfunction{os}{tmpdir}
\presentfunction{os}{uname}
\presentfunction{os}{selfdir}

\section{Lua file system extensions}

\presentfunction{lfs}{isdir}
\presentfunction{lfs}{isfile}
\presentfunction{lfs}{shortname}
\presentfunction{lfs}{readlink}

\section{Callback table}
\presentfunction{callback}{register}
\presentfunction{callback}{list}
\presentfunction{callback}{find}

\subsection{File discovery callbacks}

\presentfunction{callback}{find_read_file}
\presentfunction{callback}{find_write_file}
\presentfunction{callback}{find_font_file}
\presentfunction{callback}{find_output_file}
\presentfunction{callback}{find_format_file}
\presentfunction{callback}{find_vf_file}
\presentfunction{callback}{find_map_file}
\presentfunction{callback}{find_enc_file}
\presentfunction{callback}{find_subfont_file}
\presentfunction{callback}{find_pk_file}
\presentfunction{callback}{find_data_file}
\presentfunction{callback}{find_opentype_file}
\presentfunction{callback}{find_truetype_file}
\presentfunction{callback}{find_type1_file}
\presentfunction{callback}{find_image_file}

\subsection{File reading  callbacks}

\presentfunction{callback}{open_read_file}
\presentfunction{callback}{reader}
\presentfunction{callback}{close}

\blank

\presentfunction{callback}{read_font_file}
\presentfunction{callback}{read_vf_file}
\presentfunction{callback}{read_map_file}
\presentfunction{callback}{read_enc_file}
\presentfunction{callback}{read_sfd_file}
\presentfunction{callback}{read_pk_file}
\presentfunction{callback}{read_data_file}
\presentfunction{callback}{read_truetype_file}
\presentfunction{callback}{read_type1_file}
\presentfunction{callback}{read_opentype_file}

\subsection{Tokenisation changes  callbacks}

\presentfunction{callback}{process_input_buffer}
\presentfunction{callback}{process_output_buffer}
\presentfunction{callback}{token_filter}

\subsection{Node list  callbacks}

\presentfunction{callback}{buildpage_filter}

\afunction{buildpage_filter} context information:
\starttabulate[|lf\ss|p|]
\NC \ssbf value     \NC \bf explanation                        \NC\NR
\NC alignment       \NC a (partial) alignment is being added   \NC\NR
\NC after_output    \NC an output routine has just finished    \NC\NR
\NC box             \NC a typeset box is being added           \NC\NR
\NC new_graf        \NC the beginning of a new paragraph       \NC\NR
\NC vmode_par       \NC \tex{par} was found in vertical mode   \NC\NR
\NC hmode_par       \NC \tex{par} was found in horizontal mode \NC\NR
\NC insert          \NC an insert is added                     \NC\NR
\NC penalty         \NC a penalty (in vertical mode)           \NC\NR
\NC before_display  \NC immediately before a display starts    \NC\NR
\NC after_display   \NC a display is finished                  \NC\NR
\NC end             \NC \LUATEX\ is terminating (it's all over)\NC\NR
\stoptabulate

\presentfunction{callback}{pre_linebreak_filter}

\afunction{pre_linebreak_filter} context information:
\starttabulate[|lf\ss|p|]
\NC \ssbf value        \NC \bf explanation                     \NC\NR
\NC <empty>      \NC main vertical list                        \NC\NR
\NC hbox         \NC \tex{hbox} in horizontal mode             \NC\NR
\NC adjusted_hbox\NC \tex{hbox} in vertical mode               \NC\NR
\NC vbox         \NC \tex{vbox}                                \NC\NR
\NC vtop         \NC \tex{vtop}                                \NC\NR
\NC align        \NC \tex{halign} or \tex{valign}              \NC\NR
\NC disc         \NC discretionaries                           \NC\NR
\NC insert       \NC packaging an insert                       \NC\NR
\NC vcenter      \NC \tex{vcenter}                             \NC\NR
\NC local_box    \NC \tex{localleftbox} or \tex{localrightbox} \NC\NR
\NC split_off    \NC top of a \tex{vsplit}                     \NC\NR
\NC split_keep   \NC remainder of a \tex{vsplit}               \NC\NR
\NC align_set    \NC alignment cell                            \NC\NR
\NC fin_row      \NC alignment row                             \NC\NR
\stoptabulate

\presentfunction{callback}{linebreak_filter}
\presentfunction{callback}{post_linebreak_filter}
\presentfunction{callback}{hpack_filter}
\presentfunction{callback}{vpack_filter}
\presentfunction{callback}{pre_output_filter}
\presentfunction{callback}{hyphenate}
\presentfunction{callback}{ligaturing}
\presentfunction{callback}{kerning}
\presentfunction{callback}{mlist_to_hlist}

\subsection{Font definition  callback}
\presentfunction{callback}{define_font}

\subsection{Event  callbacks}

\presentfunction{callback}{pre_dump}
\presentfunction{callback}{stop_run}
\presentfunction{callback}{start_run}
\presentfunction{callback}{start_page_number}
\presentfunction{callback}{stop_page_number}
\presentfunction{callback}{show_error_hook}
\presentfunction{callback}{finish_pdffile}

\section{Font table}

\presentfunction{font}{read_tfm}
\presentfunction{font}{read_vf}
\presentfunction{font}{getfont}
\presentfunction{font}{setfont}
\presentfunction{font}{frozen}
\presentfunction{font}{define}
\presentfunction{font}{nextid}
\presentfunction{font}{id}
\presentfunction{font}{current}
\presentfunction{font}{max}
\presentfunction{font}{each}

\section{Font loader table}

\presentfunction{fontloader}{info}

\afunction{fontloader.info} returned information:
\starttabulate[|lf\ss|lf\ss|p|]
\NC \ssbf key                      \NC \bf type \NC \bf explanation \NC\NR
\NC fontname                     \NC string   \NC the \POSTSCRIPT\ name of the font\NC\NR
\NC fullname                     \NC string   \NC the formal name of the font\NC\NR
\NC familyname                   \NC string   \NC the family name this font belongs to\NC\NR
\NC weight                       \NC string   \NC a string indicating the color value of the font\NC\NR
\NC version                      \NC string   \NC the internal font version\NC\NR
\NC italicangle                  \NC float    \NC the slant angle\NC\NR
\stoptabulate

\presentfunction{fontloader}{open}

Listing all of the substructure returned from
\afunction{fontloader.open} would take too much room, see the big
reference manual.

\presentfunction{fontloader}{apply_featurefile}
\presentfunction{fontloader}{apply_afmfile}


\section{Image table}

Full list of \type{<image>} object fields:
\starttabulate[|lf\ss|lf\ss|p|]
\NC \bf field name\NC \bf type \NC description \NC \NR
\NC depth         \NC number   \NC the image depth for \LUATEX\ (in scaled points)\NC \NR
\NC height        \NC number   \NC the image height for \LUATEX\ (in scaled points)\NC \NR
\NC width         \NC number   \NC the image width for \LUATEX\  (in scaled points)\NC \NR
\NC transform     \NC number   \NC the image transform, integer number 0..7\NC \NR
\NC attr          \NC string   \NC the image attributes for \LUATEX \NC \NR
\NC filename      \NC string   \NC the image file name \NC \NR
\NC stream        \NC string   \NC the raw stream data for an \type{/Xobject} \type{/Form} object\NC \NR
\NC page          \NC ??       \NC the identifier for the requested image
                                   page (type is number or string,
                                   default is the number 1)\NC \NR
\NC pagebox       \NC string   \NC the requested bounding box,
                                   one of \type {none}, \type {media}, \type {crop}, \type {bleed}, \type {trim}, \type {art} \NC \NR
\NC bbox          \NC table    \NC table with 4 boundingbox dimensions \type{llx}, \type{lly},
                                   \type{urx}, and \type{ury} overruling the \type{pagebox} entry\NC \NR
\NC filepath      \NC string   \NC the full (expanded) file name of the image \NC \NR
\NC colordepth    \NC number   \NC the number of bits used by the color space \NC \NR
\NC colorspace    \NC number   \NC the color space object number \NC \NR
\NC imagetype     \NC string   \NC one of \type {pdf}, \type {png}, \type {jpg}, \type {jbig2}, or \type{nil} \NC \NR
\NC objnum        \NC number   \NC the \PDF\ image object number \NC \NR
\NC index         \NC number   \NC the \PDF\ image name suffix \NC \NR
\NC pages         \NC number   \NC the total number of available pages \NC \NR
\NC xsize         \NC number   \NC the natural image width \NC \NR
\NC ysize         \NC number   \NC the natural image height \NC \NR
\NC xres          \NC number   \NC the horizontal natural image resolution (in \DPI) \NC \NR
\NC yres          \NC number   \NC the vertical natural image resolution (in \DPI) \NC \NR
\stoptabulate

\presentfunction{img}{new}
\presentfunction{img}{keys}
\presentfunction{img}{scan}
\presentfunction{img}{copy}
\presentfunction{img}{write}
\presentfunction{img}{immediatewrite}
\presentfunction{img}{node}
\presentfunction{img}{types}
\presentfunction{img}{boxes}

\section{Kpathsea table}

\presentfunction{kpse}{set_program_name}
\presentfunction{kpse}{new}
\presentfunction{kpse}{find_file}
\presentfunction{kpse}{lookup}

The \afunction{kpse.lookup} options match commandline arguments from
\type{kpsewhich}:

\starttabulate[|lf\ss|lf\ss|p|]
\NC \ssbf key \NC \ssbf type \NC \ssbf description \NC \NR
\NC debug     \NC number     \NC set debugging flags for this lookup\NC     \NR
\NC format    \NC string     \NC use specific file type (see list above)\NC \NR
\NC dpi       \NC number     \NC use this resolution for this lookup; default 600\NC \NR
\NC path      \NC string     \NC search in the given path\NC \NR
\NC all       \NC boolean    \NC output all matches, not just the first\NC \NR
\NC must-exist\NC boolean    \NC search the disk as well as ls-R if necessary\NC \NR
\NC mktexpk   \NC boolean    \NC disable/enable mktexpk generation for this lookup\NC \NR
\NC mktextex  \NC boolean    \NC disable/enable mktextex generation for this lookup\NC \NR
\NC mktexmf   \NC boolean    \NC disable/enable mktexmf generation for this lookup\NC \NR
\NC mktextfm  \NC boolean    \NC disable/enable mktextfm generation for this lookup\NC \NR
\NC subdir    \NC string 
                  or table   \NC only output matches whose directory part
                                 ends with the given string(s) \NC \NR
\stoptabulate

\presentfunction{kpse}{init_prog}
\presentfunction{kpse}{readable_file}
\presentfunction{kpse}{expand_path}
\presentfunction{kpse}{expand_var}
\presentfunction{kpse}{expand_braces}
\presentfunction{kpse}{show_path}
\presentfunction{kpse}{var_value}
\presentfunction{kpse}{version}

\section{Language table}
\presentfunction{lang}{new}
\presentfunction{lang}{id}
\presentfunction{lang}{hyphenation}
\presentfunction{lang}{clear_hyphenation}
\presentfunction{lang}{clean}
\presentfunction{lang}{patterns}
\presentfunction{lang}{clear_patterns}
\presentfunction{lang}{prehyphenchar}
\presentfunction{lang}{posthyphenchar}
\presentfunction{lang}{preexhyphenchar}
\presentfunction{lang}{postexhyphenchar}
\presentfunction{lang}{hyphenate}

\section{Lua table}

There are 65536 bytecode registers, that are saved in the format file.
Assignments are always global.

\presentfunction{lua}{getbytecode}
\presentfunction{lua}{setbytecode}

They also be accessed via the virtual array \type{lua.bytecode[]}.

The virtual array \type{lua.name[]} can be used to give names to lua
chunks.  To use \type{lua.name[1]}, set \type{lua.name[1] = 'testname'} and \type{\directlua1{rubbish}}.

\section{Metapost table}

\presentfunction{mplib}{version}

\presentfunction{mplib}{new}
\presentfunction{mp}{execute}
\presentfunction{mp}{finish}

The return value of \type{mp:execute} and \type{mp:finish} is a table
with a few possible keys (only \type {status} is always guaranteed to be present).

\starttabulate[|lf\ss|lf\ss|p|]
\NC log    \NC string \NC output to the \quote {log} stream \NC \NR
\NC term   \NC string \NC output to the \quote {term} stream \NC \NR
\NC error  \NC string \NC output to the \quote {error} stream (only used for \quote {out of memory})\NC \NR
\NC status \NC number \NC the return value: 0=good, 1=warning, 2=errors, 3=fatal error \NC \NR
\NC fig    \NC table \NC an array of generated figures (if any)\NC \NR
\stoptabulate

Handling of \type{fig} objects would take too much room here, please
see the big reference manual.

\presentfunction{mp}{statistics}
\presentfunction{mp}{char_width}
\presentfunction{mp}{char_height}
\presentfunction{mp}{char_depth}

\section{Node table}

\presentfunction{node}{types}
\presentfunction{node}{whatsits}
\presentfunction{node}{is_node}
\presentfunction{node}{id}
\presentfunction{node}{subtype}
\presentfunction{node}{type}
\presentfunction{node}{fields}
\presentfunction{node}{has_field}
\presentfunction{node}{new}
\presentfunction{node}{free}
\presentfunction{node}{flush_list}
\presentfunction{node}{copy}
\presentfunction{node}{copy_list}
\presentfunction{node}{hpack}
\presentfunction{node}{vpack}
\presentfunction{node}{dimensions}
\presentfunction{node}{mlist_to_hlist}
\presentfunction{node}{slide}
\presentfunction{node}{tail}
\presentfunction{node}{length}
\presentfunction{node}{count}
\presentfunction{node}{traverse}
\presentfunction{node}{traverse_id}
\presentfunction{node}{remove}
\presentfunction{node}{insert_before}
\presentfunction{node}{insert_after}
\presentfunction{node}{first_character}
\presentfunction{node}{ligaturing}
\presentfunction{node}{kerning}
\presentfunction{node}{unprotect_glyphs}
\presentfunction{node}{protect_glyphs}
\presentfunction{node}{last_node}
\presentfunction{node}{write}
\presentfunction{node}{protrusion_skippable}

\presentfunction{node}{has_attribute}
\presentfunction{node}{set_attribute}
\presentfunction{node}{unset_attribute}

\section{Pdf table}
\presentfunction{pdf}{immediateobj}
\presentfunction{pdf}{mapfile}
\presentfunction{pdf}{mapline}
\presentfunction{pdf}{obj}
\presentfunction{pdf}{refobj}
\presentfunction{pdf}{pageref}
\presentfunction{pdf}{print}
\presentfunction{pdf}{reserveobj}
\presentfunction{pdf}{registerannot}

\section{Status table}
\presentfunction{status}{list}

The current list is:
\starttabulate[|lf\ss|p|]
\NC \ssbf key    \NC  \bf explanation \NC\NR
\NC pdf_gone\NC                  written \PDF\ bytes      \NC \NR
\NC pdf_ptr\NC                   not yet written \PDF\ bytes      \NC \NR
\NC dvi_gone\NC                  written \DVI\ bytes      \NC \NR
\NC dvi_ptr\NC                   not yet written \DVI\ bytes      \NC \NR
\NC total_pages\NC               number of written pages      \NC \NR
\NC output_file_name\NC          name of the \PDF\ or \DVI\ file      \NC \NR
\NC log_name\NC                  name of the log file      \NC \NR
\NC banner\NC                    terminal display banner      \NC \NR
\NC var_used\NC                  variable (one|-|word) memory in use \NC \NR
\NC dyn_used\NC                  token (multi|-|word) memory in use  \NC \NR
\NC str_ptr\NC                   number of strings      \NC \NR
\NC init_str_ptr\NC              number of \INITEX\ strings      \NC \NR
\NC max_strings\NC               maximum allowed strings      \NC \NR
\NC pool_ptr\NC                  string pool index      \NC \NR
\NC init_pool_ptr\NC             \INITEX\ string pool index      \NC \NR
\NC pool_size\NC                 current size allocated for string characters \NC \NR
\NC node_mem_usage\NC            a string giving insight into currently used nodes\NC\NR
\NC var_mem_max\NC               number of allocated words for nodes\NC \NR
\NC fix_mem_max\NC               number of allocated words for tokens\NC \NR
\NC fix_mem_end\NC               maximum number of used tokens\NC \NR
\NC cs_count\NC                  number of control sequences      \NC \NR
\NC hash_size\NC                 size of hash       \NC \NR
\NC hash_extra\NC                extra allowed hash  \NC \NR
\NC font_ptr\NC                  number of active fonts      \NC \NR
\NC max_in_stack\NC              max used input stack entries      \NC \NR
\NC max_nest_stack\NC            max used nesting stack entries     \NC \NR
\NC max_param_stack\NC           max used parameter stack entries     \NC \NR
\NC max_buf_stack\NC             max used buffer position      \NC \NR
\NC max_save_stack\NC            max used save stack entries      \NC \NR
\NC stack_size\NC                input stack size      \NC \NR
\NC nest_size\NC                 nesting stack size      \NC \NR
\NC param_size\NC                parameter stack size      \NC \NR
\NC buf_size\NC                  current allocated size of the line buffer \NC \NR
\NC save_size\NC                 save stack size      \NC \NR
\NC obj_ptr\NC                   max \PDF\ object pointer      \NC \NR
\NC obj_tab_size\NC              \PDF\ object table size      \NC \NR
\NC pdf_os_cntr\NC               max \PDF\ object stream pointer      \NC \NR
\NC pdf_os_objidx\NC             \PDF\ object stream index \NC \NR
\NC pdf_dest_names_ptr\NC        max \PDF\ destination pointer       \NC \NR
\NC dest_names_size\NC           \PDF\ destination table size      \NC \NR
\NC pdf_mem_ptr\NC               max \PDF\ memory used      \NC \NR
\NC pdf_mem_size\NC              \PDF\ memory size      \NC \NR
\NC largest_used_mark\NC         max referenced marks class        \NC \NR
\NC filename\NC                  name of the current input file    \NC \NR
\NC inputid\NC                   numeric id of the current input    \NC \NR
\NC linenumber\NC                location in the current input file\NC \NR
\NC lasterrorstring\NC           last error string\NC \NR
\NC luabytecodes\NC              number of active \LUA\ bytecode registers\NC \NR
\NC luabytecode_bytes\NC         number of bytes in \LUA\ bytecode registers\NC \NR
\NC luastate_bytes\NC            number of bytes in use by \LUA\ interpreters\NC \NR
\NC output_active\NC             \type{true} if the \tex{output} routine is active\NC \NR
\NC callbacks\NC                 total number of executed callbacks so far\NC \NR
\NC indirect_callbacks\NC        number of those that were themselves
                                 a result of other callbacks (e.g. file readers)\NC \NR
\NC luatex_svn\NC                the luatex repository id  (added in 0.51)\NC\NR
\NC luatex_version\NC            the luatex version number (added in 0.38)\NC\NR
\NC luatex_revision\NC           the luatex revision string (added in 0.38)\NC\NR
\NC ini_version\NC               \type{true} if this is an \INITEX\ run (added in 0.38)\NC\NR
\stoptabulate

\section{Typesetting table}

\presentfunction{tex}{set}
\presentfunction{tex}{get}

Many of \LUATEX's internal parameters can be queried and set this way,
but not nearly all.  The big reference manual has an extensive list.

\blank
\presentfunction{tex}{setattribute}
\presentfunction{tex}{getattribute}
\presentfunction{tex}{setbox}
\presentfunction{tex}{getbox}
\presentfunction{tex}{setcount}
\presentfunction{tex}{getcount}
\presentfunction{tex}{setdimen}
\presentfunction{tex}{getdimen}
\presentfunction{tex}{setskip}
\presentfunction{tex}{getskip}
\presentfunction{tex}{settoks}
\presentfunction{tex}{gettoks}
\presentfunction{tex}{setcatcode}
\presentfunction{tex}{getcatcode}
\presentfunction{tex}{setlccode}
\presentfunction{tex}{getlccode}
\presentfunction{tex}{setsfcode}
\presentfunction{tex}{getsfcode}
\presentfunction{tex}{setuccode}
\presentfunction{tex}{getuccode}
\presentfunction{tex}{setmathcode}
\presentfunction{tex}{getmathcode}
\presentfunction{tex}{setdelcode}
\presentfunction{tex}{getdelcode}

In all the \afunction{tex.set...} functions above, the optional string is
the literal \aliteral{global}. The items can also be accessed directly via virtual arrays:
\type{tex.attributes[]}, \type{tex.box[]}, \type{tex.count[]}, \type{tex.dimen[]},
\type{tex.skip[]}, \type{tex.toks[]}; \type{tex.catcode[]}, \type{tex.lccode[]}, 
\type{tex.sfcode[]}, \type{tex.uccode[]}, \type{tex.mathcode[]}, \type{tex.delcode[]}.

\blank

\presentfunction{tex}{setmath}
\presentfunction{tex}{getmath}

\presentfunction{tex}{print}
\presentfunction{tex}{sprint}
\presentfunction{tex}{tprint}
\presentfunction{tex}{write}
\presentfunction{tex}{round}
\presentfunction{tex}{scale}
\presentfunction{tex}{sp}
\presentfunction{tex}{definefont}
\presentfunction{tex}{error}
\presentfunction{tex}{enableprimitives}
\presentfunction{tex}{extraprimitives}
\presentfunction{tex}{primitives}
\presentfunction{tex}{badness}
\presentfunction{tex}{linebreak}

The \afunction{tex.linebreak} parameters:

\starttabulate[|lf\ss|lf\ss|p|]
\NC \bf name                 \NC \bf type \NC \bf description \NC \NR
\NC pardir                   \NC string   \NC \NC \NR
\NC pretolerance             \NC number   \NC \NC \NR
\NC tracingparagraphs        \NC number   \NC \NC \NR
\NC tolerance                \NC number   \NC \NC \NR
\NC looseness                \NC number   \NC \NC \NR
\NC hyphenpenalty            \NC number   \NC \NC \NR
\NC exhyphenpenalty          \NC number   \NC \NC \NR
\NC pdfadjustspacing         \NC number   \NC \NC \NR
\NC adjdemerits              \NC number   \NC \NC \NR
\NC pdfprotrudechars         \NC number   \NC \NC \NR
\NC linepenalty              \NC number   \NC \NC \NR
\NC lastlinefit              \NC number   \NC \NC \NR
\NC doublehyphendemerits     \NC number   \NC \NC \NR
\NC finalhyphendemerits      \NC number   \NC \NC \NR
\NC hangafter                \NC number   \NC \NC \NR
\NC interlinepenalty         \NC number or table \NC if a table, then it is an array like \type{\interlinepenalties}\NC \NR
\NC clubpenalty              \NC number or table \NC if a table, then it is an array like \type{\clubpenalties}\NC \NR
\NC widowpenalty             \NC number or table \NC if a table, then it is an array like \type{\widowpenalties}\NC \NR
\NC brokenpenalty            \NC number   \NC \NC \NR
\NC emergencystretch         \NC number   \NC in scaled points \NC \NR
\NC hangindent               \NC number   \NC in scaled points \NC \NR
\NC hsize                    \NC number   \NC in scaled points \NC \NR
\NC leftskip                 \NC glue_spec node   \NC \NC \NR
\NC rightskip                \NC glue_spec node   \NC \NC \NR
\NC pdfeachlineheight        \NC number   \NC in scaled points \NC \NR
\NC pdfeachlinedepth         \NC number   \NC in scaled points \NC \NR
\NC pdffirstlineheight       \NC number   \NC in scaled points \NC \NR
\NC pdflastlinedepth         \NC number   \NC in scaled points \NC \NR
\NC pdfignoreddimen          \NC number   \NC in scaled points \NC \NR
\NC parshape                 \NC table   \NC \NC \NR
\stoptabulate


The \afunction{tex.linebreak} returned table data:

\starttabulate[|lf\ss|p|]
\NC prevdepth   \NC depth of the last line in the broken paragraph \NC \NR
\NC prevgraf    \NC number of lines in the broken paragraph \NC \NR
\NC looseness   \NC the actual looseness value in the broken paragraph \NC \NR
\NC demerits    \NC the total demerits of the chosen solution  \NC \NR
\stoptabulate

\presentfunction{tex}{shipout}

\blank
The virtual table  \type{tex.lists} contains the set of internal
registers that keep track of building page lists.

\starttabulate[|lf\ss|p|]
\NC \bf field          \NC \bf description \NC \NR
\NC page_ins_head      \NC  circular list of pending insertions \NC \NR
\NC contrib_head       \NC  the recent contributions \NC \NR
\NC page_head          \NC  the page-so-far \NC \NR
\NC hold_head          \NC used for held-over items for next page\NC \NR
\NC adjust_head        \NC head of the current \tex{adjust} list \NC \NR
\NC pre_adjust_head    \NC head of the current \tex{adjust pre} list\NC \NR
\stoptabulate

\blank
The virtual table \type{tex.nest} contains the currently active
semantic nesting state. It has two main parts: an zero-based array of
userdata for the semantic nest itself, and the numerical value
\type{tex.nest.ptr}. Known fields:

\starttabulate[|lf\ss|lf\ss|l|p|]
\NC \ssbf key   \NC \bf type  \NC \bf modes  \NC \bf explanation \NC\NR
\NC mode        \NC number    \NC all        \NC The current mode. 
                                                 0 = no mode, 
                                                 1 = vertical,
                                                 127 = horizontal,
                                                 253 = display math.
                                                 $-1$ = internal vertical,
                                                 $-127$ = restricted horizontal,
                                                 $-253$ = inline math.\NC\NR
\NC modeline    \NC number    \NC all        \NC source input line where this mode was entered in,
                                                 negative inside the output routine.\NC\NR
\NC head        \NC node      \NC all        \NC the head of the current list\NC\NR
\NC tail        \NC node      \NC all        \NC the tail of the current list\NC\NR
\NC prevgraf    \NC number    \NC vmode      \NC number of lines in the previous paragraph\NC\NR
\NC prevdepth   \NC number    \NC vmode      \NC depth of the previous paragraph\NC\NR
\NC spacefactor \NC number    \NC hmode      \NC the current space factor\NC\NR
\NC dirs        \NC node      \NC hmode      \NC internal use only\NC\NR
\NC noad        \NC node      \NC mmode      \NC internal use only\NC\NR
\NC delimptr    \NC node      \NC mmode      \NC internal use only\NC\NR
\NC mathdir     \NC boolean   \NC mmode      \NC true when during math processing the \type{\mathdir} is not 
                                                 the same as the surrounding \type{\textdir}\NC\NR
\NC mathstyle   \NC number    \NC mmode      \NC the current \type{\mathstyle} \NC\NR
\stoptabulate


\section{Texconfig table}

This is a table that is created empty. A startup \LUA\ script could
fill this table with a number of settings that are read out by
the executable after loading and executing the startup file.

\starttabulate[|lf\ss|lf\ss|l|p|]
\NC \ssbf key      \NC \bf type     \NC \bf default \NC \bf explanation \NC\NR
\NC kpse_init \NC boolean \NC true    \NC \type{false} totally disables \KPATHSEA\ initialisation\NC \NR
\NC shell_escape    \NC    string\NC \NC cf.\ web2c docs \NC\NR
\NC shell_escape_commands \NC string\NC \NC cf.\ web2c docs \NC\NR
\NC string_vacancies \NC   number\NC  75000\NC cf.\ web2c docs \NC \NR
\NC pool_free \NC               number\NC  5000\NC cf.\ web2c docs \NC \NR
\NC max_strings \NC        number\NC  15000\NC cf.\ web2c docs \NC \NR
\NC strings_free \NC       number\NC  100\NC cf.\ web2c docs \NC \NR
\NC nest_size \NC               number\NC  50\NC cf.\ web2c docs \NC \NR
\NC max_in_open \NC        number\NC  15\NC cf.\ web2c docs \NC \NR
\NC param_size \NC         number\NC  60\NC cf.\ web2c docs \NC \NR
\NC save_size \NC               number\NC  4000\NC cf.\ web2c docs \NC \NR
\NC stack_size \NC         number\NC  300\NC cf.\ web2c docs \NC \NR
\NC dvi_buf_size \NC       number\NC  16384\NC cf.\ web2c docs \NC \NR
\NC error_line \NC         number\NC  79\NC cf.\ web2c docs \NC \NR
\NC half_error_line \NC    number\NC  50\NC cf.\ web2c docs \NC \NR
\NC max_print_line \NC     number\NC  79\NC cf.\ web2c docs \NC \NR
\NC hash_extra \NC         number\NC  0\NC cf.\ web2c docs \NC \NR
\NC pk_dpi \NC             number\NC  72\NC cf.\ web2c docs \NC \NR
\NC trace_file_names \NC boolean \NC true \NC \type{false} disables \TEX's normal file feedback\NC \NR
\NC file_line_error  \NC boolean \NC false \NC \type{file:line} style error messages\NC \NR
\NC halt_on_error    \NC boolean \NC false \NC abort run on the first encountered error\NC \NR
\NC formatname       \NC string \NC \NC if no format name was given
                                             on the commandline, this will be used\NC \NR
\NC jobname          \NC string \NC \NC as \type{formatname}.\NC \NR
\stoptabulate


\section{IO table}
\presentfunction{texio}{write}
\presentfunction{texio}{write_nl}

\section{Token table}

A token is represented in \LUA\ as a small table. For the moment, this
table consists of three numeric entries:

\starttabulate[|l|l|p|]
\NC \bf index\NC \bf meaning         \NC \bf description \NC \NR
\NC 1      \NC command code        \NC this is a value between~$0$ and~$130$\NC \NR
\NC 2      \NC command modifier    \NC this is a value between~$0$ and~$2^{21}$ \NC \NR
\NC 3      \NC control sequence id \NC for commands that are not the result of control
                                       sequences, like letters and characters, it is zero,
                                       otherwise, it is a number pointing into the \quote
                                       {equivalence table} \NC \NR
\stoptabulate

\presentfunction{token}{get_next}
\presentfunction{token}{is_expandable}
\presentfunction{token}{expand}
\presentfunction{token}{is_activechar}
\presentfunction{token}{create}
\presentfunction{token}{command_name}
\presentfunction{token}{command_id}
\presentfunction{token}{csname_name}
\presentfunction{token}{csname_id}


\stopcolumns
\stoptext
