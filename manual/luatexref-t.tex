% engine=luatex language=uk

% FOR TACO:

% example of nodes

% function upcase(n) -- slow variant, ascii only
%     while n do
%         if n.id == node.id('glyph') then
%             n.char = string.byte(string.downcase(string.char(n.char)))
%         end
%         n = n.next
%     end
% end

% example of nodes with attributes

% function upcase(n) -- slow variant, ascii only
%     while n do
%         if n.id == node.id('glyph') and node.has_attribute(n,10,1) then
%             n.char = string.byte(string.downcase(string.char(n.char)))
%         end
%         n = n.next
%     end
% end

 % TODO: fix layout of function legend descriptions

\environment luatexref-env

\catcode`\_=12
 
\setvariables
  [document]
  [snapshot=2007-07-06]

\starttext

\setups[titlepage]

\title{Contents}

\placecontent[criterium=text,level=subsection]

\chapter{Introduction}

\startframedtext[framecolor=red,foregroundcolor=red,width=\hsize]

This book will eventually become the reference manual of \LUATEX. At
the moment, it simply reports the behaviour of the executable
matching the snapshot date in the title page.

\blank

Features may come and go. The current version of \LUATEX\ is not meant
for production and users cannot depend on functionality staying the
same.

\blank

Nothing in the API is considered stable just yet. This manual
therefore simply reflects the current state of the executable. {\bs
Absolutely nothing\/} on the following pages is set in stone. When the
need arises, anything can (and will) be changed without prior notice.

\blank

\bf If you are unhappy with this situation, wait for the public betas.

\stopframedtext

\blank[2*line]

\LUATEX\ consists of a number of interrelated but (still) distinguishable
 parts:

\startitemize[packed]
\item \PDFTEX\ version 1.40.3
\item \ALEPH\ RC4 (from the \TEXLIVE\ repository)
\item Functionality of \ETEX\ 2.2
\item \LUA\ 5.1.2
\item Dedicated \LUA\ libraries
\item Various \TEX\ extensions
\item The (\OPENTYPE) Font Parser from \FONTFORGE\ 2007.06.07
\item Compiled source code to glue it all together
\stopitemize

Neither I/O translation processes, nor tcx files, nor
\ENCTEX\ can be used. All these encoding|-|related functions are
superseded by a \LUA|-|based solution (reader callbacks).

\chapter{Basic \TEX\ enhancements}

\section{\UNICODE\ support}

Text input and output is now considered to be \UNICODE\ text, so
characters can use the full range of \UNICODE\ ($2^{20}+2^{16} =
\hbox{10FFFF} = 1114111$).

For now, it only makes sense to use values above the base plane (FFFF) for \tex{mathcode} and \tex{catcode} assignments, since the
hyphenation patterns are still limited to at the most 16-bit values,
so the other commands will not know what to do with those high values.

Many primitives are affected by this. For instance,
\tex{char} now accepts values between~0 and 1114111. This should not
be a problem for well|-|behaved input files, but it could create
incompatibilities for input that would have generated an error when
processed by older \TEX|-|based engines. The maximum number of
allocations is \type {"10FFFF} or $2^{20}+2^{16}$ (21 bits). The maximum
value that can be assigned are:

\starttabulate[|l|l|l|l|]
\NC \bf primitive     \NC \bf bits \NC \bf hex     \NC \bf numeric                     \NC\NR
\NC \tex{char}        \NC 21       \NC \tt ~10FFFF \NC $2^{20}+2^{16}$                 \NC\NR
\NC \tex{chardef}     \NC 21       \NC \tt ~10FFFF \NC $2^{20}+2^{16}$                 \NC\NR
\NC \tex{lccode}      \NC 21       \NC \tt ~10FFFF \NC $2^{20}+2^{16}$                 \NC\NR
\NC \tex{uccode}      \NC 21       \NC \tt ~10FFFF \NC $2^{20}+2^{16}$                 \NC\NR
\NC \tex{sfcode}      \NC 15       \NC \tt ~~~7FFF \NC $2^{15}$                        \NC\NR
\NC \tex{catcode}     \NC ~4       \NC \tt ~~~~~~F \NC $2^{4}$                         \NC\NR
\NC \tex{mathchardef} \NC 15       \NC \tt ~~~8000 \NC $2^{3}\times2^{8}\times2^{4}$   \NC\NR
\NC \tex{mathcode}    \NC 15       \NC \tt ~~~8000 \NC $2^{3}\times2^{8}\times2^{4}$   \NC\NR
\NC \tex{delcode}     \NC 27       \NC \tt 7FFFFFF \NC $2^{3}\times2^{4}\times2^{8}\times2^{4}\times2^{8}$ \NC\NR
\stoptabulate

As far as the core engine is concerned, all input and output to text
files is \UTF-8 encoded. Input files can be pre|-|processed using the
\luatex{reader} callback. This will be explained in a later chapter.

Output in byte|-|sized chunks can be achieved by using characters just
outside of the valid unicode range, starting at the value 1.114.112 
(0x110000). When the times comes to print a character $c>=1.114.112$, \LUATEX\ will
actually print the single byte corresponding to $c-1.114.112$.

Output to the terminal uses \type{^^} notation for the lower control
range ($c<32$), with the exception of \type{^^I}, \type{^^J} and
\type{^^M}.  These are considered \quote{safe} and therefore printed as-is.

Normalization of the \UNICODE\ input can be handled by a macro package
during callback processing (will be explained below).

\section{Wide math characters}

Text is now extended up to the full \UNICODE\ range, but math mode deals
mostly with glyphs in fonts directly, and fonts tend to be 16-bit at
maximum.

Therefore, the math primitives from \ALEPH\ are kept mostly as-is,
except for the ones that convert from input to math commands. The
extended commands (with the \type{o} prefix) accept 16-bit glyph
indices in one of 256 possible families. The traditional \TEX\
primitives are unchanged, their arguments are upscaled internally.

\starttabulate[|l|l|l|l|]
\NC \bf primitive     \NC max index/bits \NC \bf hex              \NC \bf numeric                          \NC\NR
\NC \tex{mathchar}    \NC 15             \NC \tt~~~7FFF           \NC $2^{3}*2^{8}*2^{4}$                  \NC\NR
\NC \tex{delimiter}   \NC 27             \NC \tt7FFFFFF           \NC $2^{3}*2^{4}*2^{8}*2^{4}*2^{8}$      \NC\NR
\NC \tex{omathchar}   \NC 27             \NC \tt7FFFFFF           \NC $2^{3}*2^{16}*2^{8}$                 \NC\NR
\NC \tex{odelimiter}  \NC 27+24          \NC \tt7FFFFFF + ~FFFFFF \NC $2^{3}*2^{8}*2^{16}+2^{8}*2^{16}$    \NC\NR
\NC \tex{omathchardef}\NC 21=27          \NC \tt~10FFFF = 8000000 \NC $2^{20}+2^{16} = 2^{3}*2^{16}*2^{8}$ \NC\NR
\NC \tex{omathcode}   \NC 21=27          \NC \tt~10FFFF = 8000000 \NC $2^{20}+2^{16} = 2^{3}*2^{16}*2^{8}$ \NC\NR
\NC \tex{odelcode}    \NC 21=27+24       \NC \tt~10FFFF = 7FFFFFF \NC $2^{20}+2^{16} = 2^{3}*2^{8}*2^{16}$ \NC\NR
\NC                            \NC                \NC \tt~~~~~~~ + ~FFFFFF \NC \quad $+ 2^{8}*2^{16}$               \NC\NR
\stoptabulate

\section{Extended register tables}

All registers can be 16 bit numbers as in \ALEPH. The affected commands
are:

\startcolumns[n=4]
\starttyping
\count
\dimen
\skip
\muskip
\marks
\toks
\countdef
\dimendef
\skipdef
\muskipdef
\toksdef
\box
\unhbox
\unvbox
\copy
\unhcopy
\unvcopy
\wd
\ht
\dp
\setbox
\vsplit
\stoptyping
\stopcolumns

\section{Attribute registers}

Attributes are a completely new concept to \LUATEX. Syntactically,
they behave a lot like counters: attributes obey \TEX's nesting stack
and can be used after \tex{the} etc. just like the normal
\tex{count} registers.

\startsyntax
\attribute <16-bit number> <optional equals> <31-bit number>\crlf
\attributedef <csname> <optional equals> <16-bit number>
\stopsyntax

However, there are some differences as well. Conceptually, an
attribute is either \quote{set} or \quote{unset}. Set attributes can only have
values of~0 or more, otherwise they are considered unset and
automatically remapped to $-1$. All attributes start out in the
\quote{unset} state (in \INITEX).

Attributes can be used as extra counter values, but their usefulness
comes from the fact that the numbers and values of all \quote{set}
attributes are attached to all nodes created in their scope. These can
then be queried from any \LUA\ code that deals with node
processing. Further information is given in the chapter~\in{on}[nodes].

\section{\LUA\ related primitives}

In order to merge \LUA\ code with \TEX\ input, a few new primitives are
needed. \LUATEX\ has support for 65536 separate \LUA\ interpreter
states. States are automatically created based on the integer argument
to the primitives \tex{directlua} and \tex{latelua}.

\subsection{\tex{directlua}}

The primitive \tex{directlua} is used to execute \LUA\ code immediately.
The syntax is

\startsyntax
\directlua <16-bit number> <general text>
\stopsyntax

The \syntax{<general text>} is fed into the \LUA\ interpreter state
indicated by the \syntax{<16-bit number>}. If the state does not exist
yet, then it will be initialized automatically.

This command is expandable.

\subsection{\tex{latelua}}

\tex{latelua} stores \LUA\ code in a whatsit that will be processed
inside the output routine. It's intended use is very similar to
\tex{pdfliteral}.

Within the \LUA\ code, you should use \luatex{pdf.print} to print stuff
directly to the pdffile.

\startsyntax
\latelua <16-bit number> <general text>
\stopsyntax

\subsection{\tex{luaescapestring}}

This primitive converts a \TEX\ token string so that it can be safely
used as the contents of a \LUA\ string: embedded backslashes, double
quotes and single quotes are escaped by prepending an extra token
consisting of a backslash with catcode~12.

\startsyntax
\luaescapestring <general text>
\stopsyntax


\subsection{\tex{closelua}}

This primitive allows you to close a \LUA\ state, freeing all of its
used memory.

\startsyntax
\closelua <16-bit number>
\stopsyntax

You cannot close \LUA\ state zero (0), any attempt to do so will be
silently ignored.

States are only closed automatically when a fatal (out of memory)
error occurs, but at that point \LUATEX\ will exit anyway.

States are not closed immediately, but only when the output routine
comes into play next (because there may be pending \tex{latelua}
calls)

\section{New \ETEX\ primitives}

\subsection{\tex{clearmarks}}

This primitive clears a marks class completely, resetting all three
connected mark texts to empty.

\startsyntax
\clearmarks <16-bit number>
\stopsyntax

\subsection{\tex{noligs} and \tex{nokerns}}

These primitives prohibit ligature and kerning insertion at the time
when the initial node list is built by \LUATEX's main control loop
(It is a temporary trick that will be removed soon).

\startsyntax
\noligs <integer>
\nokerns <integer>
\stopsyntax


\subsection{\tex{formatname}}

\tex{formatname}'s syntax is identical to \tex{jobname}.

In \INITEX, the expansion is empty. Otherwise, the expansion is the
value that \tex{jobname} had during the \INITEX\ run that dumped the
currently loaded format.

\subsection{\tex{scantextokens}}

The syntax of \tex{scantextokens} is identical to \tex{scantokens}.

This is a slightly adapted version of \ETEX's \tex{scantokens}. The
differences are:

\startitemize
\item The last (and usually only) line does not have a
      \tex{endlinechar} appended
\item \tex{scantextokens} never raises an EOF error,
      and it does not execute \tex{everyeof} tokens.
\item The \quote{while end of file} tests are not executed, allowing
      the expansion to end on a different grouping level or while a
      conditional is still incomplete
\stopitemize


\subsection{Catcode tables}

Catcode tables are a new feature that allows you to switch to a
predefined catcode regime in a single statement. You can have a
practically unlimited number of different tables (at this moment up to
$268.435.456$. The limit depends on an array allocation).

The subsystem is backward compatible: if you never use the following
commands, your document will not notice any difference in behavior
compared to traditional \TEX.

The contents of each catcode table is independent of any other
catcode tables, and their contents is stored and retrieved from the
format file.

\subsubsection{\tex{catcodetable}}

\startsyntax
\catcodetable <28-bit number>
\stopsyntax

The \tex{catcodetable} switches to a different catcode table.
Such a table has to be previously created using one of the two
primitives below, or it has to be zero (table zero is initialized by
\INITEX).

\subsubsection{\tex{initcatcodetable}}

\startsyntax
\initcatcodetable <28-bit number>
\stopsyntax

The \tex{initcatcodetable} creates a new table with catcodes
identical to those defined by \INITEX:

\starttabulate[|l|l|l|l|]
\NC~0\NC \tt\textbackslash         \NC         \NC \tt escape        \NC\NR
\NC~5\NC \tt\letterhat\letterhat M \NC return  \NC \tt car\_ret      \NC\NR
\NC~9\NC \tt\letterhat\letterhat @ \NC null    \NC \tt ignore        \NC\NR
\NC10\NC \tt <space>               \NC space   \NC \tt spacer        \NC\NR
\NC11\NC {\tt a} -- {\tt z}        \NC         \NC \tt letter        \NC\NR
\NC11\NC {\tt A} -- {\tt Z}        \NC         \NC \tt letter        \NC\NR
\NC12\NC everything else           \NC         \NC \tt other         \NC\NR
\NC14\NC \tt\letterpercent         \NC         \NC \tt comment       \NC\NR
\NC15\NC \tt\letterhat\letterhat ? \NC delete  \NC \tt invalid\_char \NC\NR
\stoptabulate

The new catcode table is allocated globally: it will not go away after
the current group has ended. If the supplied number is the currently
active table, an error is raised.

\subsubsection{\tex{savecatcodetable}}

\startsyntax
\savecatcodetable <28-bit number>
\stopsyntax

\tex{savecatcodetable} copies the current set of catcodes to a
new table with the requested number. The definitions in this new table
are all treated as if they were made in the outermost level.

The new table is allocated globally: it will not go away after the
current group has ended. If the supplied number is the currently
active table, an error is raised.

\subsection{Font syntax}

\LUATEX\ will accept a braced argument as a font name:

\starttyping
\font\myfont = {cmr10}
\stoptyping

This allows for embedded spaces, without the need for double quotes.
Macro expansion takes place in the argument.

\chapter {\LUA\ general}

\section{Initialization}

\subsection{\LUATEX\ as a \LUA\ interpreter}

There are two situations that make \LUATEX\ behaves like it is a \LUA\
interpreter only:
\startitemize[packed]
\item If a \type{--luaonly} option is given
\item If the executable is named \type{texlua} (or \type{luatexlua})
%\item if the non|-|option (file) on the commandline has the extension
%	\type{lua} or \type{luc}, as a fallback.
\stopitemize

In this mode, it will set \LUA's \type{arg[0]} to the found script
name, pushing preceding options in negative values and the rest of the
commandline in the positive values, just like the \LUA\
interpreter.

\LUATEX\ will exit immediately after executing the specified \LUA\
script and is, in effect, a somehwat bulky standalone \LUA\ interpreter.

\subsection{Other commandline processing}

Whenever the \LUATEX\ executable starts, it looks for a \type{--lua}
commandline option. If such an option is present, it will enter an
alternative mode of commandline parsing.

In this mode, it will only interpret a very small subset of the
commandline directly:

\starttabulate[|l|l|]
\NC --luaonly    \NC execute a \LUA\ script, then exit         \NC\NR
\NC --lua=s      \NC load and execute a \LUA\ init script      \NC\NR
\NC --safer      \NC disable easily exploitable \LUA\ commands \NC\NR
\NC --help       \NC display help and exit                     \NC\NR
\NC --version    \NC display version and exit                  \NC\NR
\stoptabulate

If a requested \LUA\ script can not be found using the actual name given
on the commandline, a second attempt is made by prepending the value
of the environment variable \type{LUATEXDIR}, if that variable is
defined.

Then the script is loaded and executed. It will find the entire
commandline in the table \lua{arg}, beginning with \lua
{arg[0]}, that is the name of the executable.

\LUATEX\ will fetch some of the other commandline options from
the \luatex{texconfig} table at the end of script execution (see the
description of the \luatex{texconfig} table later on in this document).

Commandline processing happens very early on. So early, in fact, that
none of \TEX's initializations have taken place yet. For that reason,
the \luatex{tex}, \luatex{token}, \luatex{node} and
\luatex{pdf} tables are off|-|limits during the execution of the
startup file (they are nilled). Special
care is taken that \luatex{texio.write} and \luatex{texio.write_nl}
function properly, so that you can at least report your actions to the
log file when (and if) it eventually becomes opened (note that \TEX\
does not even know it's \tex{jobname} yet at this point).

The file is loaded into \LUA\ state~0, and everything you do will remain
visible during the rest of the run, with the exception of the
\luatex{tex}, \luatex{token}, \luatex{node} and \luatex{pdf} tables: those
will be restored to their normal meaning right after the execution
of the script.

We recommend you use the startup file only for your own
\TEX|-|independant initializations (if you need any), to parse the
commandline, set values in the \luatex{texconfig} table, and
register the callbacks you need.


You can use the \type{--safer} switch to disable some commands that
can easily be abused by a malicious document. At the moment, this
switch \type{nil}s the following functions:

\starttabulate[|l|l|]
\NC \bf library \NC \bf functions                         \NC \NR
\NC \tt os      \NC \tt execute exec setenv rename remove \NC \NR
\NC \tt io      \NC \tt popen output tmpfile              \NC \NR
\NC \tt lfs     \NC \tt rmdir mkdir chdir lock touch      \NC \NR
\stoptabulate

And it makes \lua{io.open()} fail on files that are opened for
anything besides reading.

Unless the \luatex{texconfig} table tells it not to start \KPATHSEA\ at
all (set \luatex{texconfig.kpse_init} to \type{false} for that), it also
acts on three other commandline options:

\starttabulate[|l|l|]
\NC \bf flag         \NC \bf meaning                           \NC \NR
\NC \tt --fmt=s      \NC set the format name                   \NC \NR
\NC \tt --progname=s \NC set the progname (only for \KPATHSEA) \NC \NR
\NC \tt --ini        \NC enable \INITEX\ mode                  \NC \NR
\stoptabulate

In order to initialize the built|-|in \KPATHSEA\ library properly,
\LUATEX\ needs to know the correct \quote{progname} to use, and for that it
needs to check \type{-progname} (and \type{-ini} and \type{-fmt}, if
\type{-progname} is missing).

If there is no \type{--lua} option, the commandline is interpreted
in a similar fashion as in traditional \PDFTEX\ and \ALEPH.

\section{\LUA\ changes}

The \lua{read("*line")} function from the io library has been
adjusted so that it is line|-|ending neutral: any of \type{LF}, \type
{CR} or \type{CR+LF} are accepted.

The \lua{tostring()} printer for numbers has been changed so that it
return~\type{0} instead of something like~\hbox{\type{2e-5}} (which confused \TEX\
enormously) when the value is so small that \TEX\ cannot distinguish
it from zero.

Dynamic loading of \type{.so} and \type{.dll} files is disabled on
all platforms.

\lua{luafilesystem} has been extended with two extra boolean functions
(\luatex{isdir(filename)} and \luatex{isfile(filename)}) and one extra
string field in the attributes table (\type{permissions}).

The \lua{string} library has an extra function:
\luatex{string.explode(s[,m])}. This function returns an array containing
the string argument \type{s} split into substrings based on the value
of the string argument \type{m}. The second argument is a string that
is either empty (this splits the string into characters), a single
character (this splits on each occurrence of that character, possibly
introducing empty strings), or a single character followed by the plus
sign \type{+} (this special version does not create empty
substrings). The default value for \type{m} is `\type{ +}' (multiple
spaces).


The \lua{string} library also has six extra iterators that return strings
piecemeal:

\startitemize
\item \luatex{string.utfvalues(s)} (returns an integer value in the
\UNICODE\ range)
\item \luatex{string.utfcharacters(s)} (returns a string with a single
\UTF-8 token in it)
\item \luatex{string.characters(s)} (a string of length one)
\item \luatex{string.characterpairs(s)} (two strings of length one) will
produce an empty second string in the string length was odd.
\item \luatex{string.bytes(s)} (a single byte value)
\item \luatex{string.bytepairs(s)} (two byte values) Will produce nil instead of a
number as its second return value if the string length was odd.
\stopitemize

The \lua{os} library has a few extra functions and variables:

\startitemize
\item \luatex{os.exec('command')} is a non|-|returning version of \lua{os.execute}.
  The advantage of this command is that it cleans out the current
  process before starting the new one, making it especially useful for
  use in \TEXLUA.

\item \luatex{os.setenv('key','value')}
  This sets a variable in the environment. Passing \lua{nil} instead of a
  value string will remove the variable.

\item \luatex{os.environ}
  This is a read|-|only hash table containing all of the variables and
  values in the process environment.

\stopitemize

\section {\LUA\ Modules}

Some modules that are normally external to Lua are statically linked
in with \LUATEX, because they offer useful functionality: 

\startitemize
\item \lua{slnunicode}, from the Selene libraries, \hyphenatedurl{http://luaforge.net/projects/sln}. (version 1.1)
\item \lua{luazip}, from the kepler project, \hyphenatedurl{http://www.keplerproject.org/luazip/}.  
  (version 1.2.1, but patched for compilation with lua 5.1)
\item \lua{luafilesystem}, also from the kepler project, \hyphenatedurl{http://www.keplerproject.org/luafilesystem/}.  
  (version 1.2, but patched for compilation with lua 5.1)
\item \lua{lpeg}, by Roberto Ierusalimschy, \hyphenatedurl{http://www.inf.puc-rio.br/~roberto/lpeg.html}. (version 0.6)
\item \lua{lzlib}, by Tiago Dionizio, \hyphenatedurl{http://mega.ist.utl.pt/~tngd/lua/}. (version 0.2)
\item \lua{md5}, by Roberto Ierusalimschy \hyphenatedurl{http://www.inf.puc-rio.br/~roberto/md5/md5-5/md5.html}.
\stopitemize


\chapter{\LUATEX\ \LUA\ Libraries}

The interfacing between \TEX\ and \LUA\ is facilitated by a set of
\LUA\ modules.

\section{The \luatex{tex} library}

The \luatex{tex} table contains a large list of virtual internal \TEX\
parameters that are partially writable.

The designation \quote{virtual} means that these items are not properly
defined in \LUA, but are only front\-ends that are handled by a metatable
that operates on the actual \TEX\ values. As a result, most of the \LUA\
table operators (like \type{pairs} and \type{#}) do not work on such
items.

At the moment, it is possible to access almost every parameter
that has these characteristics:

\startitemize[packed]
\item You can use it after \tex{the}
\item It is a single token.
\stopitemize

This excludes parameters that need extra arguments, like
\tex{the}\tex{scriptfont}.

The subset comprising simple integer and dimension registers are
writable as well as readable (stuff like \tex{tracingcommands} and
\tex{parindent}).

\subsection{Integer parameters}

The integer parameters accept and return \LUA\ numbers.

Read-write:

\startcolumns[n=2]
\starttyping
tex.adjdemerits
tex.binoppenalty
tex.brokenpenalty
tex.catcodetable
tex.clubpenalty
tex.day
tex.defaulthyphenchar
tex.defaultskewchar
tex.delimiterfactor
tex.displaywidowpenalty
tex.doublehyphendemerits
tex.endlinechar
tex.errorcontextlines
tex.escapechar
tex.exhyphenpenalty
tex.fam
tex.finalhyphendemerits
tex.floatingpenalty
tex.globaldefs
tex.hangafter
tex.hbadness
tex.holdinginserts
tex.hyphenpenalty
tex.interlinepenalty
tex.language
tex.lastlinefit
tex.lefthyphenmin
tex.linepenalty
tex.localbrokenpenalty
tex.localinterlinepenalty
tex.looseness
tex.mag
tex.maxdeadcycles
tex.month
tex.newlinechar
tex.outputpenalty
tex.pausing
tex.pdfadjustinterwordglue
tex.pdfadjustspacing
tex.pdfappendkern
tex.pdfcompresslevel
tex.pdfdecimaldigits
tex.pdfforcepagebox
tex.pdfgamma
tex.pdfgentounicode
tex.pdfimageapplygamma
tex.pdfimagegamma
tex.pdfimagehicolor
tex.pdfimageresolution
tex.pdfinclusionerrorlevel
tex.pdfminorversion
tex.pdfmovechars
tex.pdfobjcompresslevel
tex.pdfoptionalwaysusepdfpagebox
tex.pdfoptionpdfinclusionerrorlevel
tex.pdfoptionpdfminorversion
tex.pdfoutput
tex.pdfpagebox
tex.pdfpkresolution
tex.pdfprependkern
tex.pdfprotrudechars
tex.pdftracingfonts
tex.pdfuniqueresname
tex.postdisplaypenalty
tex.predisplaydirection
tex.predisplaypenalty
tex.pretolerance
tex.relpenalty
tex.righthyphenmin
tex.savinghyphcodes
tex.savingvdiscards
tex.showboxbreadth
tex.showboxdepth
tex.time
tex.tolerance
tex.tracingassigns
tex.tracingcommands
tex.tracinggroups
tex.tracingifs
tex.tracinglostchars
tex.tracingmacros
tex.tracingnesting
tex.tracingonline
tex.tracingoutput
tex.tracingpages
tex.tracingparagraphs
tex.tracingrestores
tex.tracingscantokens
tex.tracingstats
tex.uchyph
tex.vbadness
tex.widowpenalty
tex.year
\stoptyping
\stopcolumns

Read|-|only:

\startcolumns[n=3]
\starttyping
tex.deadcycles
tex.insertpenalties
tex.parshape
tex.prevgraf
tex.spacefactor
\stoptyping
\stopcolumns

\subsection{Dimension parameters}

The dimension parameters accept \LUA\ numbers (signifying scaled points)
or strings (with included dimension). The result is always a string.

Read-write:

\startcolumns[n=3]
\starttyping
tex.boxmaxdepth
tex.delimitershortfall
tex.displayindent
tex.displaywidth
tex.emergencystretch
tex.hangindent
tex.hfuzz
tex.hoffset
tex.hsize
tex.lineskiplimit
tex.mathsurround
tex.maxdepth
tex.nulldelimiterspace
tex.overfullrule
tex.pagebottomoffset
tex.pageheight
tex.pagerightoffset
tex.pagewidth
tex.parindent
tex.pdfdestmargin
tex.pdfeachlinedepth
tex.pdfeachlineheight
tex.pdffirstlineheight
tex.pdfhorigin
tex.pdflastlinedepth
tex.pdflinkmargin
tex.pdfpageheight
tex.pdfpagewidth
tex.pdfpxdimen
tex.pdfthreadmargin
tex.pdfvorigin
tex.predisplaysize
tex.scriptspace
tex.splitmaxdepth
tex.vfuzz
tex.voffset
tex.vsize
\stoptyping
\stopcolumns

Read|-|only:

\startcolumns[n=3]
\starttyping
tex.pagedepth
tex.pagefilllstretch
tex.pagefillstretch
tex.pagefilstretch
tex.pagegoal
tex.pageshrink
tex.pagestretch
tex.pagetotal
tex.prevdepth
\stoptyping
\stopcolumns

\subsection{Direction parameters}

All direction parameters are read|-|only and return a \LUA\ string

\startcolumns[n=3]
\starttyping
tex.bodydir
tex.mathdir
tex.pagedir
tex.pardir
tex.textdir
\stoptyping
\stopcolumns

\subsection{Glue parameters}

All glue parameters are read|-|only and return a \LUA\ string

\startcolumns[n=3]
\starttyping
tex.abovedisplayshortskip
tex.abovedisplayskip
tex.baselineskip
tex.belowdisplayshortskip
tex.belowdisplayskip
tex.leftskip
tex.lineskip
tex.parfillskip
tex.parskip
tex.rightskip
tex.spaceskip
tex.splittopskip
tex.tabskip
tex.topskip
tex.xspaceskip
\stoptyping
\stopcolumns

\subsection{Muglue parameters}

All muglue parameters are read|-|only and return a \LUA\ string

\startcolumns[n=3]
\starttyping
tex.medmuskip
tex.thickmuskip
tex.thinmuskip
\stoptyping
\stopcolumns

\subsection{Tokenlist parameters}

All tokenlist parameters are read|-|only and return a \LUA\ string

\startcolumns[n=3]
\starttyping
tex.errhelp
tex.everycr
tex.everydisplay
tex.everyeof
tex.everyhbox
tex.everyjob
tex.everymath
tex.everypar
tex.everyvbox
tex.output
tex.pdfpageattr
tex.pdfpageresources
tex.pdfpagesattr
tex.pdfpkmode
\stoptyping
\stopcolumns

\subsection{Convert commands}

The supported commands at this moment are:

\startcolumns[n=3]
\starttyping
tex.AlephVersion
tex.Alephrevision
tex.OmegaVersion
tex.Omegarevision
tex.eTeXVersion
tex.eTeXrevision
tex.formatname
tex.jobname
tex.pdfnormaldeviate
tex.pdftexbanner
tex.pdftexrevision
\stoptyping
\stopcolumns

All \quote{convert} commands are read|-|only and return a \LUA\ string

This list looks haphazard, but it really is not. These are all the
cases of the \quote{convert} internal command that do not require an
argument.

\subsection{attribute, count, dimension and token registers}

\TEX's attributes (\tex{attribute}), counters (\tex{count}),
dimensions (\tex{dimen}) and token (\tex{toks}) registers can be
accessed and written to using four virtual sub|-|tables of the
\luatex{tex} table:

\startcolumns[n=3]
\starttyping
tex.attribute
tex.count
tex.dimen
tex.toks
\stoptyping
\stopcolumns

It is possible to use the names of relevant \tex{attributedef}, \tex{countdef},
\tex{dimendef}, or \tex{toksdef} control sequences as indices
to these tables:

\starttyping
tex.count.scratchcounter = 0
enormous = tex.dimen["maxdimen"]
\stoptyping

In this case, \LUATEX\ looks up the value for you on the fly. You have
to use a valid \tex{countdef} (or \tex{attributedef}, or
\tex{dimendef}, or \tex{toksdef}), anything else will generate an error
(the intent is to eventually also allow \type{<chardef tokens>} and even
macros that expand into a number)

The attribute and count registers accept and return \LUA\ numbers.

The dimension registers accept \LUA\ numbers (in scaled points) or
strings (with an included absolute dimension; \type {em} and \type {ex} and \type {px}
are forbidden). The result is always a number in scaled points.

The token registers accept and return \LUA\ strings. \LUA\ strings are
converted to token lists using \tex{the}\tex{toks} style expansion.

As an alternative to array addressing, there are also accessor
functions defined:

\startfunctioncall
tex.setdimen(number n, string s)
tex.setdimen(string s, string s)
tex.setdimen(number n, number n)
tex.setdimen(string s, number n)
number n = tex.getdimen(number n)
number n = tex.getdimen(string s)

tex.setcount(number n, number n)
tex.setcount(string s, number n)
number n = tex.getcount(number n)
number n = tex.getcount(string s)

tex.settoks (number n, string s)
tex.settoks (string s, string s)
string s = tex.gettoks (number n)
string s = tex.gettoks (string s)
\stopfunctioncall

\subsection{Box registers}

The current dimensions of \tex{box} registers can be read and
altered using three other virtual sub|-|tables :

\starttyping
tex.wd
tex.ht
tex.dp
\stoptyping

These are indexed strictly by number.

The box size registers accept \LUA\ numbers (in scaled points)
or strings (with included dimension). The result is always a number
in scaled points.

As an alternative to array addressing, there are also accessor
functions defined:

\startfunctioncall
tex.setboxwd(number n, number n)
number n = tex.getboxwd(number n)

tex.setboxht(number n, number n)
number n = tex.getboxht(number n)

tex.setboxdp(number n, number n)
number n = tex.getboxdp(number n)
\stopfunctioncall

It is also possible to set and query actual boxes, using the node
interface as defined in the \luatex{node} library:

\starttyping
tex.box
\stoptyping

for array access, or

\starttyping
tex.setbox(number n, node s)
node n = tex.getbox(number n)
\stoptyping

for function|-|based access

Be warned that an assignment like

\starttyping
tex.box[0] = tex.box[2]
\stoptyping

does not copy the node list, it just duplicates a node pointer.  If
\tex{box2} will be cleared by \TEX\ commands later on, the contents
of \tex{box0} becomes invalid as well. To prevent this from
happening, always use \luatex{node.copy_list()} unless you are
assigning to a temporary variable:

\starttyping
tex.box[0] = node.copy_list(tex.box[2])
\stoptyping

\subsection{Print functions}

The \luatex{tex} table also contains the three print functions that
are the major interface from \LUA\ scripting to \TEX.

The arguments to these three functions are all stored in an in|-|memory
virtual file that is fed to the \TEX\ scanner as the result of the
expansion of \tex{directlua}.

The total amount of returnable text from a \tex{directlua} command
is only limited by available system \RAM. However, each separate
printed string has to fit completely in \TEX's input buffer.

\subsubsection{\luatex{tex.print}}

\startfunctioncall
tex.print(<string s>, ...)
tex.print(<number n>, <string s>, ...)
\stopfunctioncall

Each string argument is treated by \TEX\ as a separate input line.

The optional parameter can be used to print the strings using the
catcode regime defined by \tex{catcodetable}~\type{n}. If \type{n} is not
a valid catcode table, then it is ignored, and the currently
active catcode regime is used instead.

The very last string of the very last \luatex{tex.print()} command in a
\tex{directlua} will not have the \tex{endlinechar} appended, all
others do.

\subsubsection{\luatex{tex.sprint}}

\startfunctioncall
tex.sprint(<string s>, ...)
tex.sprint(<number n>, <string s>, ...)
\stopfunctioncall

Each string argument is treated by \TEX\ as a special kind of input line
that makes it suitable for use as a partial line input mechanism:

\startitemize[packed]
\item \TEX\ does not switch to the \quote{new line} state, so
   that leading spaces are not ignored.
\item No \tex{endlinechar} is inserted.
\item Trailing spaces are not removed.
\stopitemize

\subsubsection{\luatex{tex.write}}

\startfunctioncall
tex.write(<string s>, ...)
\stopfunctioncall

Each string argument is treated by \TEX\ as a special kind of input
line that makes is suitable for use as a quick way to dump
information:

\startitemize
\item All catcodes on that line are either \quote{space} (for "~") or
     \quote{character} (for all others).
\item There is no \tex{endlinechar} appended.
\stopitemize

\subsection{Helper functions}

\subsubsection{\luatex{tex.round}}

\startfunctioncall
number n =  tex.round(number o)
\stopfunctioncall

Rounds lua number \type{o}, and returns a number that is in the range
of a valid \TEX\ register value. If the number starts out of range, it
generates a `Number to big' error as well.

\subsubsection{\luatex{tex.scale}}

\startfunctioncall
number n =  tex.scale(number o, number delta)
table n  =  tex.scale(table o, number delta)
\stopfunctioncall

Multiplies the lua numbers \type{o} and \type{delta}, and returns a
rounded number that is in the range of a valid \TEX\ register value.
In the table version, it creates a copy of the table with all numeric
top||level values scaled in that manner. If the mutiplied number(s) are
of range, it generates `Number to big' error(s) as well.

\section{The \luatex{token} library}

The \luatex{token} table contains interface functions to \TEX's
handling of tokens. These functions are most useful when combined with
the \luatex{token_filter} callback, but they could be used standalone as
well.

A token is represented in \LUA\ as a small table. For the moment, this
table consists of three numeric entries:

\starttabulate[|l|l|p|]
\NC \bf nr \NC \bf meaning         \NC \bf description \NC \NR
\NC 1      \NC command code        \NC this is a value between~$0$ and~$130$ (approximately)\NC \NR
\NC 2      \NC command modifier    \NC this is a value between~$0$ and~$2^{21}$ \NC \NR
\NC 3      \NC control sequence id \NC for commands that are not te result of control
                                       sequences, like letters and characters, it is zero,
                                       otherwise, it is  number pointing into the \quote
                                       {equivalence table} \NC \NR
\stoptabulate

\subsection{\luatex{token.get_next}}

\startfunctioncall
token t = token.get_next()
\stopfunctioncall

This fetches the next input token from the current input source,
without expansion.

\subsection{\luatex{token.is_expandable}}

\startfunctioncall
boolean b = token.is_expandable(token t)
\stopfunctioncall

This tests if the token \type{t} could be expanded.

\subsection{\luatex{token.expand}}

\startfunctioncall
token.expand()
\stopfunctioncall

If a token is expandable, this will expand one level of it, so that
the first token of the expansion will now be the next token to be read
by \luatex{tex.get_next()}.

\subsection{\luatex{token.is_activechar}}

\startfunctioncall
boolean b = token.is_activechar(token t)
\stopfunctioncall

This is a special test that is sometimes handy. Discovering whether
some token is the result of an active character turned out to be very
hard otherwise.

\subsection{\luatex{token.create}}

\startfunctioncall
token t = token.create(string csname)
token t = token.create(number charcode)
token t = token.create(number charcode, number catcode)
\stopfunctioncall

This is the token factory. If you feed it a string, then it is a
control sequence csname, and it will be looked up in the equivalence
table.

If you feed it number, then this is assumed to be an input character,
and an optional second number gives its category code.  This means it
is possible to overrule a character's category code, with a few
exceptions: the category codes~0 (escape), 9~(ignored), 13~(active),
14~(comment), and 15 (invalid) cannot occur inside a token. The values~0, 9, 14
and~15 are therefore illegal as input to \luatex{token.create()}, and
active characters will be resolved immediately.

Note: unknown string sequences and never defined active characters
will result in a token representing an \quote{undefined control sequence}
with a near|-|random name. It is {\it not\/} possible to define brand
new control sequences using \luatex{token.create}!

\subsection{\luatex{token.command_name}}

\startfunctioncall
string commandname = token.command_name(token t)
\stopfunctioncall

This returns the name associated with the \quote{command} value of the token
in \LUATEX. There is no direct connection between these names and
primitives. For instance, all \tex{ifxxx} tests are grouped under
\type {if_fest}, and the \quote{command modifier} defines which test is to be run.

\subsection{\luatex{token.command_id}}

\startfunctioncall
number i = token.command_name(string commandname)
\stopfunctioncall

This returns a number that is the inverse operation of the previous
command, to be used as the first item in a token table.

\subsection{\luatex{token.csname_name}}

\startfunctioncall
string csname = token.csname_name(token t)
\stopfunctioncall

This returns the name associated with the \quote{equivalence table} value of
the token in \LUATEX. It returns the string value of the command used
to create the current token, or an empty string if there is no
associated control sequence.

\subsection{\luatex{token.csname_id}}

\startfunctioncall
number i = token.csname_id(string csname)
\stopfunctioncall

This returns a number that is the inverse operation of the previous
command, to be used as the third item in a token table.

\section{The \luatex{node} library}

The \luatex{node} library contains functions that facilitate dealing
with (lists of) nodes and their values.

\LUATEX\ nodes are represented in \LUA\ as userdata with
the metadata type \luatex{luatex.node}. The various fields within
a node can be accessed using fields that are indexed by strings.

Each node has at least the three fields \type{next}, \type{id}, and
\type{subtype}:

\startitemize[intro]

\item The \type{next} field returns the userdata
object for the next node in a linked list of nodes, or
nil, if there is no next node.

\item The \type{id} indicates \TEX's \quote{node type}. The field \type{id}
has a numeric value for efficiency reasons, but some of the library
functions also accept a string value instead of \type{id}.

\item The \type{subtype} is another number. It often gives further information
about a node of a particular \type{id}, but it is most important when dealing
with \quote{whatsits}, because they are differentiated solely based on their
\type{subtype}.
\stopitemize

The other available fields depend on the \type{id} (and for \quote{whatsits}, the
\type{subtype}) of the node. Further details on the various fields and their
meanings are given in the \in{chapter}[nodes].

\TEX's math nodes are not yet supported: there is not yet an interface
to the internals of the math list and it is not possible to create
them from \LUA. Support for \type{unset} (alignment) nodes is partial:
they can be queried and modified from \LUA\ code, but not created.

Nodes can be compared to each other, but: you are actually comparing
indices into the node memory. This means that equality tests can only
be trusted under very limited conditions. It will not work correctly
in any situation where one of the two nodes has been freed and|/|or
reallocated: in that case, there will be false positives.

\subsection{Node handling functions}

\subsubsection{\luatex{node.types}}

\startfunctioncall
table t = node.types()
\stopfunctioncall

This function returns an array that maps node id numbers to node type
strings, providing an overview of the possible top|-|level \type{id}
types.

\subsubsection{\luatex{node.whatsits}}

\startfunctioncall
table t = node.whatsits()
\stopfunctioncall

\TEX's \quote{whatsits} all have the same \type{id}. The various subtypes
are defined by their \type{subtype}. The function is much like
\luatex{node.id}, except that it provides an array of \type{subtype}
mappings.

\subsubsection{\luatex{node.id}}

\startfunctioncall
number id = node.id(string type)
\stopfunctioncall

This converts a single type name to it's internal numeric
representation.

\subsubsection{\luatex{node.subtype}}

\startfunctioncall
number subtype = node.subtype(string type)
\stopfunctioncall

This converts a single whatsit name to it's internal numeric
representation (\type{subtype}).

\subsubsection{\luatex{node.type}}

\startfunctioncall
string type = node.type(number id)
\stopfunctioncall

This converts a internal numeric representation to an external string
representation.

\subsubsection{\luatex{node.fields}}

\startfunctioncall
table t = node.fields(string type)
table t = node.fields(string type, string subtype)
\stopfunctioncall

This function returns an array of valid field names for a particular
type of node. If you want to get the valid fields for a \quote{whatsit},
then you have to supply the second argument also. Otherwise, the
second argument will be silently ignored.

This function accepts numeric \type{id} and \type{subtype} values as
well.

\subsubsection{\luatex{node.has_field}}

\startfunctioncall
boolean t = node.has_field(node n, string field)
\stopfunctioncall

This function returns a boolean that is only true if \type{n} is
actually a node, and it has the field. 


\subsubsection{\luatex{node.new}}

\startfunctioncall
node n = node.new(string type)
node n = node.new(string type, string subtype)
\stopfunctioncall

Creates a new node. All fields are initialized to either zero or null
(depending on wether the field is a number or a pointer), except for
\type{id} and \type{subtype} (if supplied). If you need a new whatsit,
then the second argument is required.

This function accepts numeric \type{id} and \type{subtype} values as
well.

\subsubsection{\luatex{node.free}}

\startfunctioncall
node.free(node n)
\stopfunctioncall

Removes the node \type{m} from \TEX's memory. Be careful: no checks
are done on whether this node is still pointed to from a register or some
\type{next} field: it is up to you to make sure that the internal data
structures remain correct.

\subsubsection{\luatex{node.flush_list}}

\startfunctioncall
node.flush_list(node n)
\stopfunctioncall

Removes the node list \type{m} and the complete node list following
\type{m} from \TEX's memory. Be careful: no checks are done on whether
any of these nodes is still pointed to from a register or some
\type{next} field: it is up to you to make sure that the internal data
structures remain correct.

\subsubsection{\luatex{node.copy}}

\startfunctioncall
node m = node.copy(node n)
\stopfunctioncall

Creates a deep copy of node \type{n}. Only the \type{next} field is
not copied.

\subsubsection{\luatex{node.copy_list}}

\startfunctioncall
node m = node.copy_list(node n)
\stopfunctioncall

Creates a deep copy of the node list that starts at \type{n}.

\subsubsection{\luatex{node.hpack}}

\startfunctioncall
node h = node.hpack(node n)
node h = node.hpack(node n, number w, string info)
\stopfunctioncall

This function creates a new hlist by packagin the list that with node
\type{n} into a horizontal box. With only a single argument, this box 
is created using the natural width of it's components. In the three
argument form, \type{info} must be either \type{additional} or
\type{exactly}, and \type{w} is the additional (\type{\hbox spread})
or exact (\type{\hbox to}) width to be used.

Caveat: at this moment, there can be unintended side|-|effects to this
function, like updating some of the \type{\marks} and \type{\inserts}.

\subsubsection{\luatex{node.slide}}

\startfunctioncall
node m = node.slide(node n)
\stopfunctioncall

Returns the last node of the node list that starts at \type{n}.

\subsubsection{\luatex{node.length}}

\startfunctioncall
number i = node.length(node n)
number i = node.length(node n, node m)
\stopfunctioncall

Returns the number of nodes contained in the node list that starts at
\type{n}. If \type{m} is also supplied it stops at \type{m} instead of
at the end of the list. The node \type{m} is not counted.

\subsubsection{\luatex{node.count}}

\startfunctioncall
number i = node.count(string type, node n)
number i = node.count(string type, node n, node m)
\stopfunctioncall

Returns the number of nodes contained in the node list that starts at
\type{n} that have an \type{id} matching \type{type}.
If \type{m} is also supplied, counting stops at \type{m} instead of
at the end of the list. The node \type{m} is not counted.

\subsubsection{\luatex{node.traverse}}

\startfunctioncall
node t = node.traverse(node n)
node t = node.traverse(node n, node m)
\stopfunctioncall

This is an iterator that loops over the node list that starts at \type{n}.
If \type{m} is also supplied, the iterator stops at \type{m} instead
of at the end of the list. The node \type{m} is not processed.

\subsubsection{\luatex{node.traverse_id}}

\startfunctioncall
node t = node.traverse_id(string type, node n, node m)
node t = node.traverse_id(string type, node n)
\stopfunctioncall

This is an iterator that loops over all the nodes in the list that
starts at \type{n} that have an \type{id} matching \type{type}.  If
\type{m} is also supplied, the iterator stops at \type{m} instead of
at the end of the list. The node \type{m} is not processed.


\subsubsection{\luatex{node.remove}}

\startfunctioncall
node head, current = node.remove(node head, node current)
\stopfunctioncall

This function removes the node \type{current} from the list following
\type{head}. It is your responsibility to make sure it is really part
of that list. The return values are the new \type{head} and
\type{current} nodes. The returned \type{current} is the 
node in the calling argument, and is only passed back as a convenience
(it's \type{next} field will be cleared). The returned \type{head} is
more important, because: if the function is called with \type{current}
equal to \type{head}, it will be changed.


\subsubsection{\luatex{node.insert_before}}

\startfunctioncall
node head, new = node.insert_before(node head, node current, node new)
\stopfunctioncall

This function inserts the node \type{new} before \type{current} into
the list following \type{head}. It is your responsibility to make sure
that \type{current} is really part of that list. The return values are
the (potentially mutated) \type{head} and the \type{new}, set up to
be part of the list (with correct \type{next} field). If \type{head}
is initially \type{nil}, it will become \type{new}.


\subsubsection{\luatex{node.insert_after}}

\startfunctioncall
node head, new = node.insert_after(node head, node current, node new)
\stopfunctioncall

This function inserts the node \type{new} after \type{current} into
the list following \type{head}. It is your responsibility to make sure
that \type{current} is really part of that list. The return values are
the \type{head} and the \type{new}, set up to be part of the list
(with correct \type{next} field). If \type{head} is initially
\type{nil}, it will become \type{new}.


\subsection{Attribute handling}

Attributes appear as linked list of userdata objects in the
\type{attr} field of individual nodes. They can be handled
individually, but it much safer and more efficient to use the
dedicated functions associated with them.

\subsubsection{\luatex{node.has_attribute}}

\startfunctioncall
number v = node.has_attribute(node n, number id)
number v = node.has_attribute(node n, number id, number val)
\stopfunctioncall

Tests if a node has the attribute with number \type{id} set. If
\type{val} is also supplied, also tests if the value matches \type{val}.
It returns the value, or, if no match is found, nil.

\subsubsection{\luatex{node.set_attribute}}

\startfunctioncall
node.set_attribute(node n, number id, number val)
\stopfunctioncall

Sets the attribute with number \type{id} to the value
\type{val}. Duplicate assignments are ignored.

\subsubsection{\luatex{node.unset_attribute}}

\startfunctioncall
number v = node.unset_attribute(node n, number id, number val)
number v = node.unset_attribute(node n, number id)
\stopfunctioncall

Unsets the attribute with number \type{id}. If \type{val} is also supplied,
it will only perform this operation if the value matches \type{val}.
Missing attributes or attribute|-|value pairs are ignored.

If the attribute was actually deleted, returns its old
value. Otherwise, returns nil.

\section{The \luatex{texio} library}

This library takes care of the low|-|level I/O interface.

\subsection{Printing functions}

\subsubsection{\luatex{texio.write}}

\startfunctioncall
texio.write(string target, string s)
texio.write(string s)
\stopfunctioncall

Without the \type{target} argument, Writes the string to the same
location(s) \TEX\ writes messages to at this moment. If
\tex{batchmode} is in effect, it writes only to the log,
otherwise  it writes to the log and the terminal.

The optional \type{target} can be one of three possibilities:
\type{term}, \type{log} or \type {term and log}.

\subsubsection{\luatex{tex.write_nl}}

\startfunctioncall
texio.write_nl(string target, string s)
texio.write_nl(string s)
\stopfunctioncall

Like \luatex{texio.write}, but make sure that the string~\type {s} will
appear at the beginning of a line. You can use an empty string if you
only want to move to the next line.

\section{The \luatex{pdf} library}

This table contains the current \type{h} en \type{v} values that
define the location on the output page. The values can be queried
and set using scaled points as units.

\starttyping
pdf.v
pdf.h
\stoptyping

The associated function calls are

\startfunctioncall
pdf.setv(number n)
number n = pdf.getv()
pdf.seth(number n)
number n = pdf.geth()
\stopfunctioncall


It also holds a print function to write stuff to the \PDF\ document,
that can be used from within a \tex{latelua} argument. This function
is not to be used inside \tex{directlua} unless you know {\it
exactly} what you are doing.

\subsubsubject{\luatex{pdf.print}}

\startfunctioncall
pdf.print(<string s>)
pdf.print(<string type>, <string s>)
\stopfunctioncall

The optional parameter can be used to mimic the behaviour of
\tex{pdfliteral}: the \type{type} is \type{direct} or \type{page}.

\section{The \luatex{callback} library}

This library has functions that register, find and list callbacks.

The \luatex{callback} library is only available in \LUA\ state zero (0).

\startfunctioncall
callback.register(string <callback name>,function <callback_func>)
callback.register(string <callback name>,nil)
\stopfunctioncall

where the \syntax{<callback name>} is a predefined callback name, see
below.

\LUATEX\ internalizes the callback function in such a way that
it does not matter if you redefine a function accidentally.

Callback assignments are always global. You can use the special value
\type {nil} instead of a function for clearing the callback.

\startfunctioncall
table <info> = callback.list()
\stopfunctioncall

The keys in the table are the known callback names, the value is a
boolean where \type{true} means that the callback is currently set
(active).

\startfunctioncall
function <f> = callback.find(<callback name>)
\stopfunctioncall

If the callback is not set, \luatex{callback.find} returns \type{nil}.

\subsection{File discovery callbacks}

\subsubsection{\luatex{find_read_file} and \luatex{find_write_file}}

You callback function should have the following conventions:

\startfunctioncall
string <actual_name> = function (number <id_number>, string <asked_name>)
\stopfunctioncall

Arguments:

\startitemize

\sym{id_number} 

This number is zero for the log or \tex {input} files. For \TEX's \tex{read} or
\tex{write} the number is incremented by one, so \tex{read0} becomes~1.

\sym{asked_name} 

This is the user|-|supplied filename, as found by \tex{input}, \tex{openin}
or \tex{openout}.

\stopitemize

Return value:

\startitemize

\sym{actual_name} 

This is the filename used. For the very first file that is read in by \TEX,
you have to make sure you return an \type{actual_name} that has an
extension and that is suitable for use as \type{jobname}. If you
don't, you will have to manually fix the name for the log file and
output file, and an eventual format filename will become mangled,
since these depend on the jobname.

You have to return \type{nil} if the file cannot be found.

\stopitemize

\subsubsection{\luatex{find_font_file}}

Your callback function should have the following conventions:

\startfunctioncall
string <actual_name> = function (string <asked_name>)
\stopfunctioncall

The \type{asked_name} is an \OTF\ or \TFM\ font metrics file.

Return \type{nil} if the file cannot be found.

\subsubsection{\luatex{find_output_file}}

You callback function should have the following conventions:

\startfunctioncall
string <actual_name> = function (string <asked_name>)
\stopfunctioncall

The \type{asked_name} is the \PDF\ or \DVI\ file for writing.

\subsubsection{\luatex{find_format_file}}

You callback function should have the following conventions:

\startfunctioncall
string <actual_name> = function (string <asked_name>)
\stopfunctioncall

The \type{asked_name} is a format file for reading (the format file
for writing is always opened in the current directory).

\subsubsection{\luatex{find_vf_file}}

Like \luatex{find_font_file}, but for virtual fonts. This applies to
both \ALEPH's \OVF\ files and traditional Knuthian \VF\ files.

\subsubsection{\luatex{find_ocp_file}}

Like \luatex{find_font_file}, but for ocp files.

\subsubsection{\luatex{find_map_file}}

Like \luatex{find_font_file}, but for map files.

\subsubsection{\luatex{find_enc_file}}

Like \luatex{find_font_file}, but for enc files.

\subsubsection{\luatex{find_sfd_file}}

Like \luatex{find_font_file}, but for subfont definition files.

\subsubsection{\luatex{find_pk_file}}

Like \luatex{find_font_file}, but for pk bitmap files. The argument
\type{<name>} is a bit special in this case. It's form is

\starttyping
<base res>dpi/<fontname>.<actual res>pk
\stoptyping

So you may be asked for \type{600dpi/manfnt.720pk}.  It is up to you
to find a \quote{reasonable} bitmap file to go with that specification.

\subsubsection{\luatex{find_data_file}}

Like \luatex{find_font_file}, but for embedded files (\tex{pdfobj file "..."}).

\subsubsection{\luatex{find_opentype_file}}

Like \luatex{find_font_file}, but for \OPENTYPE\ font files.

\subsubsection{\luatex{find_truetype_file} and \luatex{find_type1_file}}

You callback function should have the following conventions:

\startfunctioncall
string <actual_name> = function (string <asked_name>)
\stopfunctioncall

The \type{asked_name} is a font file. This callback is called while
\LUATEX\ is building its internal list of needed font files, so the
actual timing may surprise you. Your return value is later fed back
into the matching \luatex{read_file} callback.

Strangely enough, \luatex{find_type1_file} is also used for \OPENTYPE\
(\OTF) fonts.

\subsubsection{\luatex{find_image_file}}

You callback function should have the following conventions:

\startfunctioncall
string <actual_name> = function (string <asked_name>)
\stopfunctioncall

The \type{asked_name} is an image file. Your return value is used to
open a file from the harddisk, so make sure you return something that
is considered the name of a valid file by your operating system.

\subsection{File reading callbacks}

\subsubsection{\luatex{open_read_file}}

You callback function should have the following conventions:

\startfunctioncall
table <env> = function (string <file_name>)
\stopfunctioncall

Argument:

\startitemize

\sym{file_name} 

the filename returned by a previous \luatex{find_read_file} or the return
value of \luatex{kpse_find_file()} if there was no such callback defined.

\stopitemize

Return value:

\startitemize

\sym{env} 

this is a table containing at least one required and one optional
callback functions for this file. The required field is
\luatex{reader} and the associated function will be called once
for each new line to be read, the optional one is \luatex{close}
that will be called once when \LUATEX\ is done with the file.

\LUATEX\ never looks at the rest of the table, so you can use it to
store your private per|-|file data. Both the callback functions will
receive the table as their only argument.

\stopitemize

\subsubsubsection{\luatex{reader}}

\LUATEX\ will run this function whenever it needs a new input line
from the file.

\startfunctioncall
function(table <env>)
    return string <line>
end
\stopfunctioncall

Your function should return either a string or \type{nil}. The value \type{nil}
signals that the end of file has occurred, and will make \TEX\ call
the optional \luatex{close} function next.

\subsubsubsection{\luatex{close}}

\LUATEX\ will optionally run this function when it needs to close the file.

\startfunctioncall
function(table <env>)
    return
end
\stopfunctioncall

Your function should not return any value.

\subsubsection{\luatex{read_font_file}}

This function is called when \TEX\ needs to read a \type{ofm} or
\type{tfm} file.

\startfunctioncall
function(string <name>)
    return boolean <success>, string <data>, number <data_size>
end
\stopfunctioncall

\startitemize

\sym{success} 

return false when a fatal error occured (e.g. when the file cannot be
found, after all).

\sym{data} 

the bytes comprising the file.

\sym{data_size} 

the length of the \type{data}, in bytes.

\stopitemize

return an empty string and zero if the file was found but there was a
reading problem.

\subsubsection{\luatex{read_vf_file}}

Like \luatex{read_font_file}, but for virtual fonts.

\subsubsection{\luatex{read_ocp_file}}

Like \luatex{read_font_file}, but for ocp files.

\subsubsection{\luatex{read_map_file}}

Like \luatex{read_font_file}, but for map files.

\subsubsection{\luatex{read_enc_file}}

Like \luatex{read_font_file}, but for enc files.

\subsubsection{\luatex{read_sfd_file}}

Like \luatex{read_font_file}, but for subfont definition files.

\subsubsection{\luatex{read_pk_file}}

Like \luatex{read_font_file}, but for pk bitmap files.

\subsubsection{\luatex{read_data_file}}

Like \luatex{read_font_file}, but for embedded files (\tex{pdfobj file "..."}).

\subsubsection{\luatex{read_truetype_file}}

Like \luatex{read_font_file}, but for \TRUETYPE\ font files. The
\type{name} is a path name as returned by \luatex{find_truetype_file}
or \luatex{kpse_find_file}.

\subsubsection{\luatex{read_type1_file}}

Like \luatex{read_font_file}, but for type1 font files. The
\type{name} is a path name as returned by \luatex{find_type1_file}
or \luatex{kpse_find_file}.

\subsubsection{\luatex{read_opentype_file}}

Like \luatex{read_font_file}, but for \OPENTYPE\ font files. The
\type{name} is a path name as returned by \luatex{find_type1_file}
or \luatex{kpse_find_file}.


\subsection{Data processing callbacks}

\subsubsection{\luatex{process_input_buffer}}


This callback allows you to change the contents of the line input
buffer just before \LUATEX\ actually starts looking at it.

\startfunctioncall
function(string <buffer>)
    return string <adjusted_buffer>
end
\stopfunctioncall

If you return \type{nil}, \LUATEX\ will pretend like your callback
never happened. You can gain a small amount of processing time from
that.


\subsubsection{\luatex{token_filter}}

This callback allows you to change the fetch and preprocess any
lexical token that enters \LUATEX, before \LUATEX\ executes or expands
the associated command.

\startfunctioncall
function()
    return table <token>
end
\stopfunctioncall

The calling convention for this callback is bit more complicated then
for most other callbacks.  The function should either return a \LUA\
table representing a valid to|-|be|-|processed token or tokenlist, or
something else like nil or an empty table.

If your \LUA\ function does not return a table representing a valid
token, it will be immediately called again, until it eventually does
return a useful token or tokenlist (or until you reset the callback
value to nil). See the description of \luatex{token} for some
handy functions to be used in conjunction with this callback.

If your function returns a single usable token, then that token will
be processed by \LUATEX\ immediately. If the function returns a token
list (a table consisting of a list of consecutive token tables), then
that list will be pushed to the input stack as completely new token
list level, with it's token type set to \quote{inserted}. In either case,
the returned token(s) will not be fed back into the callback function.

\subsection{Node list processing callbacks}

The description of nodes and node lists is in the~\in{chapter}[nodes].

\subsubsection{\luatex{buildpage_filter}}

This callback is called whenever \LUATEX\ is ready to move stuff to
the main vertical list. You can use this callback to do specialized
manipulation of the page building stage like imposition or column
balancing.

\startfunctioncall
function(node <head>, string <extrainfo>)
    return true | false | node <newhead>
end
\stopfunctioncall

As for all the callbacks that deal with nodes, the return value can be
one of three things:

\startitemize[packed]
\item \type{boolean true} signals succesful processing
\item \type{node} signals that the \quote{head} node should be
replaced by this node
\item \type{boolean false} signals that the \quote{head} node list should be
ignored and flushed from memory
\stopitemize

The string \type{extrainfo} gives some additional information about
what \TEX's state is with respect to the `current page'. The possible
values are:

\starttabulate[|lT|p|]
\NC \bf value           \NC \bf explanation                        \NC\NR
\NC \tt alignment       \NC a (partial) alignment is being added   \NC\NR
\NC \tt box             \NC a typeset box is being added           \NC\NR
\NC \tt begin_of_par    \NC the beginning of a new paragraph       \NC\NR
\NC \tt vmode_par       \NC \tex{par} was found in vertical mode   \NC\NR
\NC \tt hmode_par       \NC \tex{par} was found in horizontal mode \NC\NR
\NC \tt insert          \NC an insert is added                     \NC\NR
\NC \tt penalty         \NC a penalty (in vertical mode)           \NC\NR
\NC \tt before_display  \NC immediately before a display starts    \NC\NR
\NC \tt after_display   \NC a display is finished                  \NC\NR
\stoptabulate

\subsubsection{\luatex{pre_linebreak_filter}}

This callback is called just before \LUATEX\ starts converting a list
of nodes into a stack of \tex{hbox}es. The removal of any final skip
and the subsequent insertion of \tex{parfillskip} has not happened
yet at that moment.

\startfunctioncall
function(node <head>, string <groupcode>, int <glyph_count>)
    return true | false | node <newhead>
end
\stopfunctioncall

The string called \type {groupcode} identifies the nodelist's context
within \TEX's processing. The range of possibilities is given in the
table below, but not all of those can actually appear in
\luatex {pre_linebreak_filter}, some are for the
\luatex {Xpack_filter} callbacks that will be explained in the next
two paragraphs.

\starttabulate[|lT|p|]
\NC \bf value        \NC \bf explanation                           \NC\NR
\NC \tt hbox         \NC \tex{hbox} in horizontal mode             \NC\NR
\NC \tt adjusted_hbox\NC \tex{hbox} in vertical mode               \NC\NR
\NC \tt vbox         \NC \tex{vbox}                                \NC\NR
\NC \tt vtop         \NC \tex{vtop}                                \NC\NR
\NC \tt align        \NC \tex{halign} or \tex{valign}              \NC\NR
\NC \tt disc         \NC discretionaries                           \NC\NR
\NC \tt insert       \NC packaging an insert                       \NC\NR
\NC \tt vcenter      \NC \tex{vcenter}                             \NC\NR
\NC \tt local_box    \NC \tex{localleftbox} or \tex{localrightbox} \NC\NR
\NC \tt split_off    \NC top of a \tex{vsplit}                     \NC\NR
\NC \tt split_keep   \NC remainder of a \tex{vsplit}               \NC\NR
\NC \tt preamble     \NC alignment preamble                        \NC\NR
\NC \tt align_set    \NC alignment cell                            \NC\NR
\NC \tt fin_row      \NC alignment row                             \NC\NR
\stoptabulate

\subsubsection{\luatex{hpack_filter}}

This callback is called when \TEX\ is ready to start boxing some
horizontal mode material. Math items are ignored at the moment.

\startfunctioncall
function(node <head>, number <size>, string <packtype>,
        string <groupcode>, number <glyphcount>)
    return true | false | node <newhead>
end
\stopfunctioncall

The \type{packtype} is either \type{additional} or \type{exactly}. If
\type{additional}, then the \type{size} is a \tex{hbox spread ...}
argument. If \type{exactly}, then the \type{size} is a \tex{hbox to ...}.
In both cases, the number is in scaled points.

\subsubsection{\luatex{vpack_filter}}

This callback is called when \TEX\ is ready to start boxing some
vertical mode material. Math displays are ignored at the moment.

This function is very similar to the \luatex{hpack_filter}. Besides
the fact that it is called at different moments, there is an extra
variable that matches \TEX's \tex{maxdepth} setting, but there is
no glyph count.

\startfunctioncall
function(node <head>, number <size>, string <packtype>,
        number <maxdepth>, string <groupcode>)
    return true | false | node <newhead>
end
\stopfunctioncall

\subsubsection{\luatex{pre_output_filter}}

This callback is called when \TEX\ is ready to start boxing the
box 255 for \tex{output}.

\startfunctioncall
function(node <head>, number <size>, string <packtype>,
        number <maxdepth>, string <groupcode>)
    return true | false | node <newhead>
end
\stopfunctioncall

\subsection{Information reporting callbacks}

\subsubsection{\luatex{start_run}}

\startfunctioncall
function()
\stopfunctioncall

Replaces the code that prints \LUATEX's banner

\subsubsection{\luatex{stop_run}}

\startfunctioncall
function()
\stopfunctioncall

Replaces the code that prints \LUATEX's statistics and \quote{output written
to} messages.

\subsubsection{\luatex{start_page_number}}

\startfunctioncall
function()
\stopfunctioncall

Replaces the code that prints the \type{[} and the page number at the
begin of \tex{shipout}. This callback will also override the
printing of box information that normally takes place when
\tex{tracingoutput} is positive.

\subsubsection{\luatex{stop_page_number}}

\startfunctioncall
function()
\stopfunctioncall

Replaces the code that prints the \type{]} at the end of \tex{shipout}

\subsubsection{\luatex{show_error_hook}}

\startfunctioncall
function()
    return
end
\stopfunctioncall

This callback is run from inside the \TEX\ error function, and the idea
is to allow you to do some extra reporting on top of what \TEX\ already
does (none of the normal actions are removed). You may find some of
the values in the \luatex{status} table useful.

\startitemize

\sym{message} 

is the formal error message \TEX\ has given to the user
(the line after the "!")

\sym{indicator} 

is either a filename (when it is a string) or a location indicator (a
number) that can means lots of different things like a token list id
or a \tex{read} number.

\sym{lineno} 

is the current line number
\stopitemize

This is an investigative item only, only for 'testing the water'.

The final goal is the total replacement of \TEX's error handling
routines, but that needs lots of adjustments in the web source because
\TEX\ deals with errors in a somewhat haphazard fashion.

\subsection{Font-related callbacks}

\subsubsection{\luatex{define_font}}

\startfunctioncall
function(string <name>, number <size>, number <id>)
    return table <font>
end
\stopfunctioncall

The string \type{<name>} is the filename part of the font
specification, as given by the user.

The number \type{<size>} is a bit special:

\startitemize[packed]
\item if it is positive, it specifies an \quote{at size} in scaled points.
\item if it is negative, its absolute value represents a \quote{scaled}
setting relative to the designsize of the font.
\stopitemize

The internal structure of the \type{<font>} table that is to be
returned is explained in \in{chapter}[fonts]. That table is saved
internally, so you can put extra fields in the table for your
later \LUA\ code to use.

\section{The \luatex{lua} library}

This library contains two read|-|only  items:

\subsection{Variables}

\starttyping
number n = lua.id
\stoptyping

This sets the id number of the instance.

\starttyping
string s = lua.version
\stoptyping

This sets a \LUATEX\ version identifier string (currently \type{"0.1"}).

\subsection{\LUA\ bytecode registers}

\LUA\ registers can be used to communicate \LUA\ functions across \LUA\
states. The accepted values for assignments are functions and
nil. Likewise, the retrieved value is either a function or nil.

\starttyping
lua.bytecode[n] = function () .. end
lua.bytecode[n]()
\stoptyping

The contents of the \luatex{lua.bytecode} array is stored inside the format
file as actual \LUA\ bytecode, so it can also be used to preload lua code.

The associated function calls are

\startfunctioncall
function f = lua.getbytecode(number n)
lua.setbytecode(number n, function f)
\stopfunctioncall

\section{The \luatex{kpse} library}

\subsection{\luatex{kpse.find_file}}

The most important function in the library is find_file:

\startfunctioncall
string f = kpse.find_file(string filename)
string f = kpse.find_file(string filename, string ftype)
string f = kpse.find_file(string filename, boolean mustexist)
string f = kpse.find_file(string filename, string ftype, boolean mustexist)
string f = kpse.find_file(string filename, string ftype, number dpi)
\stopfunctioncall

Arguments:

\startitemize

\sym{filename} 

the name of the file you want to find, with or without extension.

\sym{type}

maps to the \type {-format} argument of \KPSEWHICH.  The supported
values are:

\startcolumns[n=2]
\starttyping
gf
pk
bitmap font
tfm
afm
base
bib
bst
cnf
ls-R
fmt
map
mem
mf
mfpool
mft
mp
mppool
MetaPost support
ocp
ofm
opl
otp
ovf
ovp
graphic/figure
tex
TeX system documentation
texpool
TeX system sources
PostScript header
Troff fonts
type1 fonts
vf
dvips config
ist
truetype fonts
type42 fonts
web2c files
other text files
other binary files
misc fonts
web
cweb
enc files
cmap files
subfont definition files
opentype fonts
pdftex config
lig files
texmfscripts
\stoptyping
\stopcolumns

The default type is \type{tex}.

\sym{mustexist}

is similar to \KPSEWHICH's \type{-must-exist}, and the default is \type{false}.
If you specify \type{true} (or a non|-|zero integer), then the \KPSE\ library
will search the disk as well as the \type {ls-R} databases.

\sym{dpi}

This is used for the size argument of the formats \type{pk}, \type{gf}, and \type{bitmap font}.
\stopitemize

\subsection{\luatex{kpse.set_program_name}}

Sets the \KPATHSEA\ executable (and optionally program) name

\startfunctioncall
kpse.set_program_name(string name)
kpse.set_program_name(string name, string progname)
\stopfunctioncall

The second argument controls the use of the \quote{dotted} values in the
\type{texmf.cnf} configuration file, and defaults to the first argument.

\subsection{\luatex{kpse.init_prog}}

Extra initialization for programs that need to generate bitmap fonts.

\startfunctioncall
kpse.init_prog(string prefix, number base_dpi, string mfmode)
kpse.init_prog(string prefix, number base_dpi, string mfmode, string fallback)
\stopfunctioncall


\subsection{\luatex{kpse.readable_file}}

Test is a (absolute) file name is a readable file

\startfunctioncall
string f = kpse.readable_file(string name)
\stopfunctioncall

The return value is the actual absolute filename you should use,
because the disk name is not always the same as the requested name,
due to aliases and system|-|specific handling under e.g.\ \MSDOS.

Returns \lua {nil} if the file does not exist or is not readable.

\subsection{\luatex{kpse.expand_path}}

Like kpsewhich's \type {-expand-path}:

\startfunctioncall
string r = kpse.expand_path(string s)
\stopfunctioncall

\subsection{\luatex{kpse.expand_var}}

Like kpsewhich's  \type{-expand-var}:

\startfunctioncall
string r = kpse.expand_var(string s)
\stopfunctioncall

\subsection{\luatex{kpse.expand_braces}}

Like kpsewhich's \type{-expand-braces}:

\startfunctioncall
string r = kpse.expand_braces(string s)
\stopfunctioncall

\subsection{\luatex{kpse.var_value}}

Like kpsewhich's \type{-var-value}:

\startfunctioncall
string r = kpse.var_value(string s)
\stopfunctioncall

\section{The \luatex{status} library}

This contains a number of run|-|time configuration items that
you may find useful in message reporting, as well as an iterator
function that gets all of the names and values as a table.

\startfunctioncall
table <info> = status.list()
\stopfunctioncall

The keys in the table are the known items, the value is the current
value.

Almost all of the values in \type{status} are fetched through a
metatable at run|-|time whenever they are accessed, so you cannot use
\type{pairs} on \type{status}, but you {\it can\/} use \type{pairs}
on \type{<info>}, of course.

If you do not need the full list, you can also ask for a single item
by using it's name as an index into \type{status}.

The current list is:

\starttabulate[|lT|p|]
\NC \bf key    \NC  \bf explanation \NC\NR
\NC pdf_gone\NC                  written \PDF\ bytes      \NC \NR
\NC pdf_ptr\NC                   not yet written pdf bytes      \NC \NR
\NC dvi_gone\NC                  written dvi bytes      \NC \NR
\NC dvi_ptr\NC                   not yet written dvi bytes      \NC \NR
\NC total_pages\NC               number of written pages      \NC \NR
\NC output_file_name\NC          name of the \PDF\ or \DVI\ file      \NC \NR
\NC log_name\NC                  name of the log file      \NC \NR
\NC banner\NC                    terminal display banner      \NC \NR
\NC pdftex_banner\NC                  \NC \NR
\NC var_used\NC                  variable (one|-|word) memory in use \NC \NR
\NC dyn_used\NC                  token (multi|-|word) memory in use  \NC \NR
\NC str_ptr\NC                   number of strings      \NC \NR
\NC init_str_ptr\NC              number of \INITEX\ strings      \NC \NR
\NC max_strings\NC               maximum allowed strings      \NC \NR
\NC pool_ptr\NC                  string pool index      \NC \NR
\NC init_pool_ptr\NC             \INITEX\ string pool index      \NC \NR
\NC pool_size\NC                 maximum allowed string characters \NC \NR
\NC var_mem_max\NC               number of allocated words for nodes\NC \NR
\NC fix_mem_max\NC               number of allocated words for tokens\NC \NR
\NC fix_mem_end\NC               maximum number of used tokens\NC \NR
\NC cs_count\NC                  number of control sequences      \NC \NR
\NC hash_size\NC                 size of hash       \NC \NR
\NC hash_extra\NC                extra allowed hash  \NC \NR
\NC font_ptr\NC                  number of active fonts      \NC \NR
\NC hyph_count\NC                hyphenation exceptions      \NC \NR
\NC hyph_size\NC                 max used hyphenation exceptions  \NC \NR
\NC max_in_stack\NC              max used input stack entries      \NC \NR
\NC max_nest_stack\NC            max used nesting stack entries     \NC \NR
\NC max_param_stack\NC           max used parameter stack entries     \NC \NR
\NC max_buf_stack\NC             max used buffer position      \NC \NR
\NC max_save_stack\NC            max used save stack entries      \NC \NR
\NC stack_size\NC                input stack size      \NC \NR
\NC nest_size\NC                 nesting stack size      \NC \NR
\NC param_size\NC                parameter stack size      \NC \NR
\NC buf_size\NC                  line buffer size      \NC \NR
\NC save_size\NC                 save stack size      \NC \NR
\NC obj_ptr\NC                   max \PDF\ object pointer      \NC \NR
\NC obj_tab_size\NC              \PDF\ object table size      \NC \NR
\NC pdf_os_cntr\NC               max \PDF\ object stream pointer      \NC \NR
\NC pdf_os_objidx\NC             \PDF\ object stream index \NC \NR
\NC pdf_dest_names_ptr\NC        max \PDF\ destination pointer       \NC \NR
\NC dest_names_size\NC           \PDF\ destination table size      \NC \NR
\NC pdf_mem_ptr\NC               max \PDF\ memory used      \NC \NR
\NC pdf_mem_size\NC              \PDF\ memory size      \NC \NR
\NC largest_used_mark\NC         max referenced marks class        \NC \NR
\NC filename\NC                  name of the current input file    \NC \NR
\NC inputid\NC                   numeric id of the current input    \NC \NR
\NC linenumber\NC                location in the current input file\NC \NR
\NC lasterrorstring\NC           last error string\NC \NR
\NC luabytecodes\NC              number of active \LUA\ bytecode registers\NC \NR
\NC luabytecode_bytes\NC         number of bytes in \LUA\ bytecode registers\NC \NR
\NC luastates\NC                 number of active \LUA\ interpreters\NC \NR
\NC luastate_bytes\NC            number of bytes in use by \LUA\ interpreters\NC \NR
\NC output_active\NC             \type{true} if the \tex{output} routine is active\NC \NR
\stoptabulate

\section{The \luatex{texconfig} table}

This is a table that is created empty. A startup \LUA\ script could
fill this table with a number of settings that are read out by
the executable after loading and executing the startup file.

\starttabulate[|lT|l|l|p|]
\NC \bf key      \NC \bf type     \NC \bf default \NC \bf explanation \NC\NR
\NC string_vacancies \NC   number\NC  75000\NC cf.\ web2c docs \NC \NR
\NC pool_free \NC 	        number\NC  5000\NC cf.\ web2c docs \NC \NR
\NC max_strings \NC        number\NC  15000\NC cf.\ web2c docs \NC \NR
\NC strings_free \NC       number\NC  100\NC cf.\ web2c docs \NC \NR
\NC trie_size \NC 	        number\NC  20000\NC cf.\ web2c docs \NC \NR
\NC hyph_size \NC 	        number\NC  659\NC cf.\ web2c docs \NC \NR
\NC nest_size \NC 	        number\NC  50\NC cf.\ web2c docs \NC \NR
\NC max_in_open \NC        number\NC  15\NC cf.\ web2c docs \NC \NR
\NC param_size \NC         number\NC  60\NC cf.\ web2c docs \NC \NR
\NC save_size \NC 	        number\NC  4000\NC cf.\ web2c docs \NC \NR
\NC stack_size \NC         number\NC  300\NC cf.\ web2c docs \NC \NR
\NC dvi_buf_size \NC       number\NC  16384\NC cf.\ web2c docs \NC \NR
\NC error_line \NC         number\NC  79\NC cf.\ web2c docs \NC \NR
\NC half_error_line \NC    number\NC  50\NC cf.\ web2c docs \NC \NR
\NC max_print_line \NC     number\NC  79\NC cf.\ web2c docs \NC \NR
\NC ocp_list_size \NC      number\NC 1000\NC cf.\ web2c docs \NC \NR
\NC ocp_buf_size \NC       number\NC 1000\NC cf.\ web2c docs \NC \NR
\NC ocp_stack_size \NC     number\NC 1000\NC cf.\ web2c docs \NC \NR
\NC hash_extra \NC         number\NC  0\NC cf.\ web2c docs \NC \NR
\NC pk_dpi \NC             number\NC  72\NC cf.\ web2c docs \NC \NR
\NC kpse_init \NC boolean \NC true    \NC \type{false} totally disables \KPATHSEA\ initialisation
                                          (only ever unset this if you implement {\it all\/} file
                                          find callbacks!)\NC \NR
\NC trace_file_names \NC boolean \NC true \NC \type{false} disables \TEX's normal file open|-|close
                                              feedback (the assumption is that callbacks will take care of
	                                      that) \NC \NR
\NC src_special_auto  \NC boolean \NC false \NC source specials sub|-|item \NC \NR
\NC src_special_everypar  \NC boolean \NC false \NC source specials sub|-|item \NC \NR
\NC src_special_everyparend  \NC boolean \NC false \NC source specials sub|-|item \NC \NR
\NC src_special_everycr  \NC boolean \NC false \NC source specials sub|-|item \NC \NR
\NC src_special_everymath  \NC boolean \NC false \NC source specials sub|-|item \NC \NR
\NC src_special_everyhbox  \NC boolean \NC false \NC source specials sub|-|item \NC \NR
\NC src_special_everyvbox  \NC boolean \NC false \NC source specials sub|-|item \NC \NR
\NC src_special_everydisplay  \NC boolean \NC false \NC source specials sub|-|item \NC \NR
\NC file_line_error  \NC boolean \NC false \NC Do \type{file:line} style error messages\NC \NR
\NC halt_on_error    \NC boolean \NC false \NC Abort run on the first encountered error\NC \NR
\NC formatname       \NC string \NC \NC If no format name was given
                                             on the commandline, this key will be tested first
                                             instead of simply quitting\NC \NR
\NC jobname          \NC string \NC \NC If no input file name was given
                                           on the commandline, this key will be tested first
                                           instead of simply giving up\NC \NR
\stoptabulate

\section{The \luatex{font} library}

The font library will provide the interface into the internals of the
font system, as well as contain some binary font loaders.

\subsection{Loading a \TFM\ file}

\startfunctioncall
table fnt = font.read_tfm(string name, number s)
\stopfunctioncall

The number is a bit special:

\startitemize
\item if it is positive, it specifies an \quote{at size} in scaled points.
\item if it is negative, its absolute value represents a \quote{scaled}
setting relative to the designsize of the font.
\stopitemize

The internal structure of the virtual font table that is returned is
explained in \in{chapter}[fonts].

\subsection{Loading a \VF\ file}

\startfunctioncall
table vf_fnt = font.read_vf(string name, number s)
\stopfunctioncall

The number is a bit special:

\startitemize
\item if it is positive, it specifies an \quote{at size} in scaled points.
\item if it is negative, its absolute value represents a \quote{scaled}
setting relative to the designsize of the font.
\stopitemize

\subsection{The fonts array}

The whole table of \TEX\ fonts is accessible from lua using a virtual array.

\starttyping
font.fonts[n] = { ... }
table f = font.fonts[n]
\stoptyping

See \in{chapter}[fonts] for the structure of the tables. Because this
is a virtual array, you cannot call \type{pairs} on it, but see below
for the \type{font.each} iterator.

The two metatable functions implementing the virtual array are:

\startfunctioncall
table f = font.getfont(number n)
font.setfont(number n, table f)
\stopfunctioncall

Also note the following: Assignments can only be made to fonts that
have already been defined in \TEX, but have not been accessed {\it at
all\/} since that definition. This limits the usability of the write
access to \type{font.fonts} quite a lot, a less stringent ruleset will
likely be implemented later.

\subsection{Checking a font's status}

You can test for the status of a font by calling this function:

\startfunctioncall
boolean f = font.frozen(number n)
\stopfunctioncall

The return value is one of true (unassignable), false (can be changed)
or nil (not a valid font at all).

\subsection{Defining a font directly}

You can define your own font into \luatex{font.fonts}

\startfunctioncall
number i = font.define(table f)
\stopfunctioncall

The return value is the internal id number of the defined font (the
index into \luatex{font.fonts}). If the font creation fails, an error is
raised. The table is a font structure, as explained in
\in{chapter}[fonts].

\subsection{Currently active font}

\startfunctioncall
number i = font.current();
\stopfunctioncall

This is the currently used font number.

\subsection{Maximum font id}

\startfunctioncall
number i = font.max();
\stopfunctioncall

This is the largest used index in \type{font.fonts}.

\subsection{Iterating over all fonts}

\startfunctioncall
for i,v in font.each() do
  ...
end
\stopfunctioncall

This is an iterator over each of the defined \TEX\ fonts. The first
returned value is the index in \type{font.fonts}, the second the font
itself, as a lua table. The indices are listed incrementally, but they
do not always form an array of consecutive numbers: in some cases
there can be holes in the sequence.

\section{The \luatex{fontforge} library}

\subsection{Getting quick information on a font}

\startfunctioncall
local info = fontforge.info("filename")
\stopfunctioncall

This function returns either \type{nil}, or a \type{table} containing six
fairly interesting information items from the font file:

\starttabulate[|lT|l|p|]
\NC \bf key                      \NC \bf type \NC \bf explanation \NC\NR
\NC fontname                     \NC string   \NC \NC\NR
\NC fullname                     \NC string   \NC \NC\NR
\NC familyname                   \NC string   \NC \NC\NR
\NC weight                       \NC string   \NC \NC\NR
\NC version                      \NC string   \NC \NC\NR
\NC italicangle                  \NC float    \NC \NC\NR
\stoptabulate

Getting information through this function is (sometimes much) more
efficient than loading the font properly, and is therefore handy when
you want to create a dictionary of available fonts based on a
directory contents.

\subsection{Loading an \OPENTYPE\ or \TRUETYPE\ file}

If you want to use an \OPENTYPE\ font, you have to get the metric
information from somewhere. Using the \type{fontforge} library, the
basic way to get that information is thus:


\starttyping
local font = fontforge.open("filename")
local metrics = fontforge.to_table(font)
fontforge.close(font)
\stoptyping

the \type{filename} font file is parsed and even partially interpreted
by the \OPENTYPE|/|\TRUETYPE\ loading routines from \FONTFORGE. There
are a few advantages to this approach compared to reading the actual
font file ourselves:

\startitemize

\item The font is automatically re|-|encoded, so that the \type{ttf_metrics}
   table is using \UNICODE\ for the character indices.

\item Many features are pre|-|processed into a format that is easier to handle
   than just the bare tables would be.

\item \POSTSCRIPT|-|based \OPENTYPE\ fonts do not store the character height and
  depth in the font file, so the character boundingbox has to be
  calculated in some way.

\item In the future, it may be interesting to allow \LUA\ scripts access to
  the font program itself, perhaps even creating or changing the font.

\stopitemize

The top|-|level keys in the returned table are (this documentation is
not yet finished):

\starttabulate[|lT|l|p|]
\NC \bf key                      \NC \bf type \NC \bf explanation \NC\NR
\NC table_version                \NC number   \NC indicates the metrics version\NC\NR
\NC fontname                     \NC string   \NC \NC\NR
\NC fullname                     \NC string   \NC \NC\NR
\NC familyname                   \NC string   \NC \NC\NR
\NC weight                       \NC string   \NC \NC\NR
\NC copyright                    \NC string   \NC \NC\NR
\NC filename                     \NC string   \NC \NC\NR
\NC defbasefilename              \NC string   \NC \NC\NR
\NC version                      \NC string   \NC \NC\NR
\NC italicangle                  \NC float    \NC \NC\NR
\NC upos                         \NC float    \NC \NC\NR
\NC uwidth                       \NC float    \NC \NC\NR
\NC units_per_em                 \NC number   \NC \NC\NR
\NC ascent                       \NC number   \NC \NC\NR
\NC descent                      \NC number   \NC \NC\NR
\NC vertical_origin              \NC number   \NC \NC\NR
\NC uniqueid                     \NC number   \NC \NC\NR
\NC glyphcnt                     \NC number   \NC \NC\NR
\NC glyphmax                     \NC number   \NC \NC\NR
\NC glyphs                       \NC array    \NC \NC\NR
\NC changed                      \NC number   \NC \NC\NR
\NC hasvmetrics                  \NC number   \NC \NC\NR
\NC order2                       \NC number   \NC \NC\NR
\NC strokedfont                  \NC number   \NC \NC\NR
\NC weight_width_slope_only      \NC number   \NC \NC\NR
\NC head_optimized_for_cleartype \NC number   \NC \NC\NR
\NC uni_interp                   \NC enum     \NC \type {unset}, \type {none}, \type {adobe},
                                                    \type {greek}, \type {japanese}, \type {trad_chinese},
                                                    \type {simp_chinese}, \type {korean}, \type {ams}\NC\NR
\NC map                          \NC table    \NC \NC\NR
\NC origname                     \NC string   \NC \NC\NR
\NC private                      \NC table    \NC \NC\NR
\NC xuid                         \NC string   \NC \NC\NR
\NC pfminfo                      \NC table    \NC \NC\NR
\NC names                        \NC table    \NC \NC\NR
\NC cidinfo                      \NC table    \NC \NC\NR
\NC subfonts                     \NC array    \NC \NC\NR
\NC cidmaster                    \NC array    \NC \NC\NR
\NC commments                    \NC string   \NC \NC\NR
\NC anchor_classes               \NC table    \NC \NC\NR
\NC ttf_tables                   \NC table    \NC \NC\NR
\NC kerns                        \NC table    \NC \NC\NR
\NC vkerns                       \NC table    \NC \NC\NR
\NC texdata                      \NC table    \NC \NC\NR
\NC lookups                       \NC table    \NC \NC\NR
\NC gpos                          \NC table    \NC \NC\NR
\NC gsub                          \NC table    \NC \NC\NR
\NC chosenname                   \NC string   \NC \NC\NR
\NC macstyle                     \NC number   \NC \NC\NR
\NC fondname                     \NC string   \NC \NC\NR
\NC design_size                  \NC number   \NC \NC\NR
\NC fontstyle_id                 \NC number   \NC \NC\NR
\NC fontstyle_name               \NC table    \NC \NC\NR
\NC design_range_bottom          \NC number   \NC \NC\NR
\NC design_range_top             \NC number   \NC \NC\NR
\NC strokewidth                  \NC float    \NC \NC\NR
\NC mark_classes                 \NC array    \NC \NC\NR
\NC mark_class_names             \NC array    \NC \NC\NR
\NC creationtime                 \NC number   \NC \NC\NR
\NC modificationtime             \NC number   \NC \NC\NR
\NC os2_version                  \NC number   \NC \NC\NR
\stoptabulate

\subsubsection{Glyph items}

The \type{glyphs} is an array containing the per|-|character
information (quite a few of these are only present if nonzero).

\starttabulate[|lT|l|p|]
\NC \bf key                      \NC \bf type \NC \bf explanation \NC\NR
\NC name                         \NC string   \NC \NC\NR
\NC unicodeenc                   \NC number   \NC \NC\NR
\NC boundingbox                  \NC array    \NC array of four numbers\NC\NR
\NC width                        \NC number   \NC (only for horizontal fonts)\NC\NR
\NC vwidth                       \NC number   \NC (only for vertical fonts)\NC\NR
\NC lsidebearing                 \NC number   \NC (only if nonzero)\NC\NR
\NC glyph_class                  \NC number   \NC (only if nonzero)\NC\NR
\NC kerns                        \NC array    \NC (only for horizontal fonts, if set)\NC\NR
\NC vkerns                       \NC array    \NC (only for vertical fonts, if set)\NC\NR
\NC dependents                   \NC array    \NC linear array of glyph name strings (only if nonempty)\NC\NR
\NC lookups                       \NC table    \NC (only if nonempty)\NC\NR
\NC ligofme                      \NC table    \NC (only if nonempty)\NC\NR
\NC comment                      \NC string   \NC (only if set)\NC\NR
\NC anchor                       \NC table   \NC (only if set)\NC\NR
%\NC color                        \NC number   \NC (only if set)\NC\NR
\NC tex_height                   \NC number   \NC (only if set)\NC\NR
\NC tex_depth                    \NC number   \NC (only if set)\NC\NR
\NC tex_sub_pos                  \NC number   \NC (only if set)\NC\NR
\NC tex_super_pos                \NC number   \NC (only if set)\NC\NR
\stoptabulate

The \type{kerns} and \type{vkerns} are linear arrays of small hashes:

\starttabulate[|lT|l|p|]
\NC \bf key                      \NC \bf type \NC \bf explanation \NC\NR
\NC char                         \NC string   \NC \NC\NR
\NC off                          \NC number   \NC \NC\NR
\NC lookup                       \NC string   \NC \NC\NR
\stoptabulate

The \type{lookups} is a hash based on lookup subtable names, with 
the value of each key inside that a linear array of small hashes:

% TODO: fix this description
\starttabulate[|lT|l|p|]
\NC \bf key                      \NC \bf type \NC \bf explanation \NC\NR
\NC type                         \NC enum     \NC \type {position}, \type {pair},  \type {substitution}, \type {alternate},
                                                  \type {multiple}, \type {ligature}, \type {lcaret},  \type {kerning}, \type {vkerning}, \type {anchors},
                                                  \type {contextpos}, \type {contextsub}, \type {chainpos}, \type {chainsub},
                                                  \type {reversesub}, \type {max}, \type {kernback}, \type {vkernback} \NC\NR
\stoptabulate

For the first seven values of \type{type}, there can be additional sub|-|information:

\starttabulate[|lT|l|l|p|]
\NC \bf value    \NC \bf key            \NC \bf type \NC \bf explanation \NC\NR
\NC position     \NC pos                \NC table    \NC \type {vr} table\NC\NR
\NC pair         \NC pair               \NC table    \NC one string: \type {paired}, and a \type {vr} (sub)table\NC\NR
\NC substitution \NC subs               \NC table    \NC one string: \type {variant}\NC\NR
\NC alternate    \NC alt                \NC table    \NC one string: \type {components}\NC\NR
\NC multiple     \NC mult               \NC table    \NC one string: \type {components}\NC\NR
\NC ligature     \NC lig                \NC table    \NC two strings: \type {components}, \type {char}\NC\NR
\NC lcaret       \NC lcaret             \NC array    \NC linear array of numbers\NC\NR
\stoptabulate

The \type{vr} table contains for number|-|valued fields: \type{xoff},
\type{yoff}, \type{h_adv_off} and \type{v_adv_off}.

The other values of \type{type} could probably use some extra
information as well, but I do not know which case of the union is
supposed to be selected.

The \type{ligofme} is a linear array of small hashes:

\starttabulate[|lT|l|p|]
\NC \bf key            \NC \bf type \NC \bf explanation \NC\NR
\NC lig                \NC table    \NC uses the same substructure as a single \type{possub} item\NC\NR
\NC char               \NC string   \NC \NC\NR
\NC components         \NC array    \NC linear array of named components\NC\NR
\NC ccnt               \NC number   \NC \NC\NR
\stoptabulate

The \type{anchor} table is indexed by a string signifying the 
anchor type, which is one of 

\starttabulate[|lT|l|p|]
\NC \bf key            \NC \bf type \NC \bf explanation \NC\NR
\NC mark              \NC string   \NC placement mark\NR
\NC basechar          \NC string   \NC mark for attaching combining items to a base char\NR
\NC baselig           \NC string   \NC mark for attaching combining items to a ligature\NR
\NC basemark          \NC string   \NC generic mark for attaching combining items to connect to\NR
\NC centry            \NC string   \NC cursive entry point\NR
\NC cexit             \NC string   \NC cursive exit point\NR
\stoptabulate

Each of those, if present is another hash, with the entries
anchors. The anchor name (class) is the key, and the values
are the actual data:

\starttabulate[|lT|l|p|]
\NC \bf key            \NC \bf type \NC \bf explanation \NC\NR
\NC x                  \NC number   \NC x location\NC\NR
\NC y                  \NC number   \NC y location\NC\NR
\NC ttf_pt_index       \NC number   \NC truetype point index, only if given\NC\NR
\NC lig_index          \NC number   \NC only if non-zero\NC\NR
\stoptabulate


For clarification, an anchor table could for example look like this :
\starttyping
  ["anchor"]={
    ["basemark"]={
     ["Anchor-7"]={ ["x"]=170, ["y"]=1080 }
    },
    ["mark"]={
     ["Anchor-1"] ={ ["x"]=160, ["y"]=810 },
     ["Anchor-4"]= { ["x"]=160, ["y"]=800 }
    }
   },
\stoptyping



\subsubsection{map table}

The top|-|level map is a list of encoding mappings. Each of those is a table itself.

\starttabulate[|lT|l|p|]
\NC \bf key            \NC \bf type \NC \bf explanation \NC\NR
\NC enccount           \NC number   \NC \NC\NR
\NC encmax             \NC number   \NC \NC\NR
\NC backmax            \NC number   \NC \NC\NR
\NC remap              \NC table    \NC \NC\NR
\NC map                \NC array    \NC non|-|linear array of mappings\NC\NR
\NC backmap            \NC array    \NC non|-|linear array of backward mappings\NC\NR
\NC enc                \NC table    \NC \NC\NR
\stoptabulate

The \type{remap} table is very small:

\starttabulate[|lT|l|p|]
\NC \bf key            \NC \bf type \NC \bf explanation \NC\NR
\NC firstenc           \NC number   \NC \NC\NR
\NC lastenc            \NC number   \NC \NC\NR
\NC infont             \NC number   \NC \NC\NR
\stoptabulate

The \type{enc} table is a bit more verbose:

\starttabulate[|lT|l|p|]
\NC \bf key            \NC \bf type \NC \bf explanation \NC\NR
\NC enc_name           \NC string   \NC \NC\NR
\NC char_cnt           \NC number   \NC \NC\NR
\NC char_max           \NC number   \NC \NC\NR
\NC unicode            \NC array    \NC of \UNICODE\ position numbers\NC\NR
\NC psnames            \NC array    \NC of \POSTSCRIPT\ glyph names\NC\NR
\NC builtin            \NC number   \NC \NC\NR
\NC hidden             \NC number   \NC \NC\NR
\NC only_1byte         \NC number   \NC \NC\NR
\NC has_1byte          \NC number   \NC \NC\NR
\NC has_2byte          \NC number   \NC \NC\NR
\NC is_unicodebmp      \NC number   \NC (only if nonzero)\NC\NR
\NC is_unicodefull     \NC number   \NC (only if nonzero)\NC\NR
\NC is_custom          \NC number   \NC (only if nonzero)\NC\NR
\NC is_original        \NC number   \NC (only if nonzero)\NC\NR
\NC is_compact         \NC number   \NC (only if nonzero)\NC\NR
\NC is_japanese        \NC number   \NC (only if nonzero)\NC\NR
\NC is_korean          \NC number   \NC (only if nonzero)\NC\NR
\NC is_tradchinese     \NC number   \NC (only if nonzero)\NC\NR
\NC is_simplechinese   \NC number   \NC \NC\NR
\NC low_page           \NC number   \NC \NC\NR
\NC high_page          \NC number   \NC \NC\NR
\NC iconv_name         \NC string   \NC \NC\NR
\NC iso_2022_escape    \NC string   \NC \NC\NR
\stoptabulate

\subsubsection{private table}

This is the font's private \POSTSCRIPT\ dictionary, if any. Keys and
values are both strings.

\subsubsection{cidinfo table}

\starttabulate[|l|l|p|]
\NC registry                  \NC string   \NC \NC\NR
\NC ordering                  \NC string   \NC \NC\NR
\NC supplement                \NC number   \NC \NC\NR
\NC version                   \NC number   \NC \NC\NR
\stoptabulate

\subsubsection{pfminfo table}

The \type{pfminfo} table contains most of the OS/2 information:

\starttabulate[|lT|l|p|]
\NC \bf key            \NC \bf type \NC \bf explanation \NC\NR
\NC pfmset             \NC number  \NC \NC\NR
\NC winascent_add      \NC number  \NC \NC\NR
\NC windescent_add     \NC number  \NC \NC\NR
\NC hheadascent_add    \NC number  \NC \NC\NR
\NC hheaddescent_add   \NC number  \NC \NC\NR
\NC typoascent_add     \NC number  \NC \NC\NR
\NC typodescent_add    \NC number  \NC \NC\NR
\NC subsuper_set       \NC number  \NC \NC\NR
\NC panose_set         \NC number  \NC \NC\NR
\NC hheadset           \NC number  \NC \NC\NR
\NC vheadset           \NC number  \NC \NC\NR
\NC pfmfamily          \NC number  \NC \NC\NR
\NC weight             \NC number  \NC \NC\NR
\NC width              \NC number  \NC \NC\NR
\NC avgwidth           \NC number  \NC \NC\NR
\NC firstchar          \NC number  \NC \NC\NR
\NC lastchar           \NC number  \NC \NC\NR
\NC fstype             \NC number  \NC \NC\NR
\NC linegap            \NC number  \NC \NC\NR
\NC vlinegap           \NC number  \NC \NC\NR
\NC hhead_ascent       \NC number  \NC \NC\NR
\NC hhead_descent      \NC number  \NC \NC\NR
\NC hhead_descent      \NC number  \NC \NC\NR
\NC os2_typoascent     \NC number  \NC \NC\NR
\NC os2_typodescent    \NC number  \NC \NC\NR
\NC os2_typolinegap    \NC number  \NC \NC\NR
\NC os2_winascent      \NC number  \NC \NC\NR
\NC os2_windescent     \NC number  \NC \NC\NR
\NC os2_subxsize       \NC number  \NC \NC\NR
\NC os2_subysize       \NC number  \NC \NC\NR
\NC os2_subxoff        \NC number  \NC \NC\NR
\NC os2_subyoff        \NC number  \NC \NC\NR
\NC os2_supxsize       \NC number  \NC \NC\NR
\NC os2_supysize       \NC number  \NC \NC\NR
\NC os2_supxoff        \NC number  \NC \NC\NR
\NC os2_supyoff        \NC number  \NC \NC\NR
\NC os2_strikeysize    \NC number  \NC \NC\NR
\NC os2_strikeypos     \NC number  \NC \NC\NR
\NC os2_family_class   \NC number  \NC \NC\NR
\NC os2_xheight        \NC number  \NC \NC\NR
\NC os2_capheight      \NC number  \NC \NC\NR
\NC os2_defaultchar    \NC number  \NC \NC\NR
\NC os2_breakchar      \NC number  \NC \NC\NR
\NC os2_vendor         \NC string  \NC \NC\NR
\NC panose             \NC table  \NC \NC\NR
\stoptabulate

The \type{panose} subtable has exactly 10 string keys:

\starttabulate[|lT|l|pl|]
\NC \bf key            \NC \bf type \NC \bf explanation \NC\NR
familytype             \NC string   \NC Values as in the \OPENTYPE\ font specification:
                                        \type {Any}, \type {No Fit}, \type {Text and Display}, \type {Script},
                                        \type {Decorative}, \type {Pictorial} \NC\NR
serifstyle             \NC string   \NC See the \OPENTYPE\ font specification for values\NC\NR
weight                 \NC string   \NC id. \NC\NR
proportion             \NC string   \NC id. \NC\NR
contrast               \NC string   \NC id. \NC\NR
strokevariation        \NC string   \NC id. \NC\NR
armstyle               \NC string   \NC id. \NC\NR
letterform             \NC string   \NC id. \NC\NR
midline                \NC string   \NC id. \NC\NR
xheight                \NC string   \NC id. \NC\NR
\stoptabulate

\subsubsection{names table}

Each item has two top|-|level keys:

\starttabulate[|lT|l|p|]
\NC \bf key            \NC \bf type \NC \bf explanation \NC\NR
lang                   \NC string   \NC language for this entry \NC\NR
names                  \NC table    \NC \NC\NR
\stoptabulate

The \type{names} keys are the actual \TRUETYPE\ name strings. The
possible keys are:

\starttabulate[|lT|p|]
\NC \bf key            \NC \bf type \NC \bf explanation \NC\NR
\NC copyright   \NC \NC\NR
\NC family   \NC \NC\NR
\NC subfamily   \NC \NC\NR
\NC uniqueid   \NC \NC\NR
\NC fullname   \NC \NC\NR
\NC version   \NC \NC\NR
\NC postscriptname   \NC \NC\NR
\NC trademark   \NC \NC\NR
\NC manufacturer   \NC \NC\NR
\NC designer   \NC \NC\NR
\NC descriptor   \NC \NC\NR
\NC venderurl   \NC \NC\NR
\NC designerurl   \NC \NC\NR
\NC license   \NC \NC\NR
\NC licenseurl   \NC \NC\NR
\NC idontknow   \NC \NC\NR
\NC preffamilyname   \NC \NC\NR
\NC prefmodifiers   \NC \NC\NR
\NC compatfull   \NC \NC\NR
\NC sampletext   \NC \NC\NR
\NC cidfindfontname   \NC \NC\NR
\stoptabulate

\subsubsection{anchor_classes table}

The anchor_classes classes:

\starttabulate[|lT|l|p|]
\NC \bf key            \NC \bf type \NC \bf explanation \NC\NR
\NC name                   \NC string   \NC A descriptive id of this anchor class\NC\NR
\NC lookup                 \NC string   \NC \NC\NR
\NC type                   \NC string   \NC One of "mark", "mkmk", "curs", "mklg" \NC\NR
\stoptabulate

\subsubsection{gpos table}

Th gpos table has one array entry for each lookup.

\starttabulate[|lT|l|p|]
\NC \bf key            \NC \bf type \NC \bf explanation \NC\NR
\NC type                  \NC string   \NC One of  
  %"gsub_start", "gsub_single", "gsub_multiple", "gsub_alternate", 
  %"gsub_ligature", "gsub_context",  "gsub_context_chain",
  %"gsub_reversecchain", % "morx_indic", "morx_context", "morx_insert",
  "gpos_start", "gpos_single", "gpos_pair", "gpos_cursive", 
  "gpos_mark2base", "gpos_mark2ligature", "gpos_mark2mark",  "gpos_context", 
  "gpos_contextchain" % ,  "kern_statemachine" 
\NC\NR
\NC flags                 \NC table  \NC \NC\NR
\NC name                  \NC string   \NC \NC\NR
\NC features              \NC array   \NC \NC\NR
\NC subtables             \NC array   \NC \NC\NR
\stoptabulate

The flags table has a true value for each of the lookup flags that is
actually set:

\starttabulate[|lT|l|p|]
\NC \bf key            \NC \bf type \NC \bf explanation \NC\NR
\NC r2l                    \NC boolean   \NC \NC\NR
\NC ignorebaseglyphs       \NC boolean    \NC \NC\NR
\NC ignoreligatures        \NC boolean    \NC \NC\NR
\NC ignorecombiningmarks   \NC boolean    \NC \NC\NR
\stoptabulate



The features table has:

\starttabulate[|lT|l|p|]
\NC \bf key            \NC \bf type \NC \bf explanation \NC\NR
\NC tag                    \NC string   \NC \NC\NR
\NC scripts                \NC table    \NC \NC\NR
\NC ismax                  \NC number   \NC (only if true)\NC\NR
\stoptabulate

The scripts table within features has:

\starttabulate[|lT|l|p|]
\NC \bf key            \NC \bf type \NC \bf explanation \NC\NR
\NC script	               \NC string          \NC \NC\NR
\NC langs                  \NC array of strings \NC \NC\NR
\stoptabulate



The subtables table has:

\starttabulate[|lT|l|p|]
\NC \bf key            \NC \bf type \NC \bf explanation \NC\NR
\NC name                    \NC string   \NC \NC\NR
\NC suffix                  \NC string   \NC (only if used)\NC\NR
\NC anchor_classes          \NC number   \NC (only if used)\NC\NR
\NC vertical_kerning        \NC number   \NC (only if used)\NC\NR
\NC kernclass               \NC table    \NC (only if used)\NC\NR
\stoptabulate


The kernclass with subtables table has:

\starttabulate[|lT|l|p|]
\NC \bf key            \NC \bf type \NC \bf explanation \NC\NR
\NC first_cnt             \NC number   \NC \NC\NR
\NC second_cnt            \NC number   \NC \NC\NR
\NC firsts                \NC array of strings  \NC \NC\NR
\NC seconds               \NC array of strings   \NC \NC\NR
\NC lookup                \NC string \NC associated lookup \NC \NR
\NC offsets               \NC array of numbers  \NC \NC\NR
\NC kcid                 \NC number   \NC ? \NC\NR
\stoptabulate

\subsubsection{gsub table}

This has identical layout to the \type{gpos} table, except for the
type:

\starttabulate[|lT|l|p|]
\NC \bf key            \NC \bf type \NC \bf explanation \NC\NR
\NC type                  \NC string   \NC One of  "gsub_start", "gsub_single", "gsub_multiple", "gsub_alternate", 
  "gsub_ligature", "gsub_context",  "gsub_context_chain",
  "gsub_reversecchain" %,  % "morx_indic", "morx_context", "morx_insert",
%  "gpos_start", "gpos_single", "gpos_pair", "gpos_cursive", 
%  "gpos_mark2base", "gpos_mark2ligature", "gpos_mark2mark",  "gpos_context", 
%  "gpos_contextchain" % ,  "kern_statemachine" 
\NC\NR
\stoptabulate



\subsubsection{ttf_tables table}

\starttabulate[|lT|l|p|]
\NC \bf key            \NC \bf type \NC \bf explanation \NC\NR
\NC tag                    \NC string   \NC \NC\NR
\NC len                    \NC number   \NC \NC\NR
\NC maxlen                 \NC number   \NC \NC\NR
\NC data                   \NC number   \NC \NC\NR
\stoptabulate

\subsubsection{kerns table}

Substructure is identical to the per|-|glyph subtable.

\subsubsection{vkerns table}

Substructure is identical to the per|-|glyph subtable.

\subsubsection{texdata table}


\starttabulate[|lT|l|p|]
\NC \bf key            \NC \bf type \NC \bf explanation \NC\NR
\NC type                   \NC string   \NC \type {unset}, \type {text}, \type {math}, \type {mathext}\NC\NR
\NC params                 \NC array    \NC 22 font numeric parameters\NC\NR
\stoptabulate

\subsubsection{lookups table}

Top|-|level \type{lookups} is quite different from the ones at
character level. The keys in this hash are strings, the values the
actual lookups, represented as dictionary tables.

\starttabulate[|lT|l|p|]
\NC \bf key            \NC \bf type \NC \bf explanation \NC\NR
\NC type                   \NC number   \NC \NC\NR
\NC format                 \NC enum     \NC \type {glyphs}, \type {class}, \type {coverage}, \type {reversecoverage} \NC\NR
\NC tag                    \NC string   \NC \NC\NR
\NC current_class          \NC array   \NC \NC\NR
\NC before_class           \NC array   \NC \NC\NR
\NC after_class            \NC array   \NC \NC\NR
\NC rules                  \NC array   \NC an array of rule items\NC\NR
\stoptabulate

Rule items have one common item and one specialized item:

\starttabulate[|lT|l|p|]
\NC \bf key            \NC \bf type \NC \bf explanation \NC\NR
\NC lookups                \NC array    \NC A list of \quote{lookup items}\NC\NR
\NC glyph                  \NC array     \NC Only if the parent's format is \quote{glyph}\NC\NR
\NC class                  \NC array     \NC Only if the parent's format is \quote{glyph}\NC\NR
\NC coverage               \NC array     \NC Only if the parent's format is \quote{glyph}\NC\NR
\NC reversecoverage        \NC array     \NC Only if the parent's format is \quote{glyph}\NC\NR
\stoptabulate

Each of the lookup item is:

\starttabulate[|lT|l|p|]
\NC \bf key            \NC \bf type \NC \bf explanation \NC\NR
\NC seq                    \NC number   \NC \NC\NR
\NC lookup                 \NC string   \NC \NC\NR
\stoptabulate

glyph:

\starttabulate[|lT|l|p|]
\NC \bf key            \NC \bf type \NC \bf explanation \NC\NR
\NC names                  \NC string   \NC \NC\NR
\NC back                   \NC string   \NC \NC\NR
\NC fore                   \NC string   \NC \NC\NR
\stoptabulate


class:

\starttabulate[|lT|l|p|]
\NC \bf key            \NC \bf type \NC \bf explanation \NC\NR
\NC current              \NC array    \NC of numbers \NC\NR
\NC before               \NC array    \NC of numbers  \NC\NR
\NC after                \NC array    \NC of numbers  \NC\NR
\stoptabulate

coverage:

\starttabulate[|lT|l|p|]
\NC \bf key            \NC \bf type \NC \bf explanation \NC\NR
\NC current                \NC array    \NC of strings \NC\NR
\NC before                 \NC array    \NC of strings\NC\NR
\NC after                  \NC array    \NC of strings \NC\NR
\stoptabulate

reversecoverage:

\starttabulate[|lT|l|p|]
\NC \bf key            \NC \bf type \NC \bf explanation \NC\NR
\NC current                \NC array    \NC of strings \NC\NR
\NC before                 \NC array    \NC of strings\NC\NR
\NC after                  \NC array    \NC of strings \NC\NR
\NC replacements           \NC string   \NC \NC\NR
\stoptabulate

\iffalse
\subsection{Loading \OPENTYPE\ or \TRUETYPE\ name information}

\startfunctioncall
table ttf_info = font.read_otf_info(string name)
table ttf_info = font.read_ttf_info(string name)
\stopfunctioncall

These two functions are very similar to the two commands from previous
section, but they only return a small subset of the information. The
returned table only has five keys: \type{fontname}, \type{fullname},
\type{familyname}, \type{weight} and \type{table_version}.

\fi

\chapter[fonts]{Font structure}

All \TEX\ fonts are represented to \LUA\ code as tables, an internally
as C~structures. All keys in the table below are saved in the internal
font structure if they are present in the table returned by the
\luatex{define_font} callback, or if they result from the normal \TFM|/|\VF\
reading routines if there is no \luatex{define_font} callback defined.

The column \quote{from \VF} means that this key will be created by the
\luatex{font.read_vf()} routine, \quote{from \TFM} means that the key will be created
by the \luatex{font.read_tfm()} routine, and \quote{used} means whether or not the
\LUATEX\ engine itself will do something with the key.

The top|-|level keys in the table are as follows:

\starttabulate[|l|l|l|l|l|p|]
\NC \bf key            \NC \bf from vf \NC \bf from tfm \NC \bf used\NC \bf value type \NC \bf description \NC\NR
\NC name               \NC yes      \NC yes      \NC yes \NC string \NC metric (file) name\NC\NR
\NC area               \NC no       \NC yes      \NC yes \NC string \NC (directory)location, typically empty\NC\NR
\NC used               \NC no       \NC yes      \NC yes \NC boolean\NC used already? (initial: false)\NC \NR
\NC characters         \NC yes      \NC yes      \NC yes \NC table  \NC the defined glyphs of this font \NC \NR
\NC checksum           \NC yes      \NC yes      \NC no  \NC number \NC default: 0 \NC \NR
\NC designsize         \NC no       \NC yes      \NC yes \NC number \NC expected size (default: 655360 == 10pt) \NC \NR
\NC direction          \NC no       \NC yes      \NC yes \NC number \NC default: 0 (LTR) \NR
\NC encodingbytes      \NC no       \NC no       \NC yes \NC number \NC default: depends on \type {format}\NC\NR
\NC encodingname       \NC no       \NC no       \NC yes \NC string \NC encoding name\NC\NR
\NC fonts              \NC yes      \NC no       \NC yes \NC table  \NC locally used fonts\NC \NR
\NC fullname           \NC no       \NC no       \NC yes \NC string \NC actual (\POSTSCRIPT) name\NC\NR
\NC header             \NC yes      \NC no       \NC no  \NC string \NC header comments, if any\NC \NR
\NC hyphenchar         \NC no       \NC no       \NC yes \NC number \NC default: TeX's \tex{hyphenchar} \NC \NR
\NC parameters         \NC no       \NC yes      \NC yes \NC hash   \NC default: 7 parameters, all zero \NC \NR
\NC size               \NC no       \NC yes      \NC yes \NC number \NC loaded (at) size. (default: same as designsize) \NC \NR
\NC skewchar           \NC no       \NC no       \NC yes \NC number \NC default: TeX's \tex{skewchar}  \NC \NR
\NC type               \NC yes      \NC no       \NC yes \NC string \NC basic type of this font\NC \NR
\NC format             \NC no       \NC no       \NC yes \NC string \NC disk format type\NC \NR
\NC embedding          \NC no       \NC no       \NC yes \NC string \NC \PDF\ inclusion\NC \NR
\NC filename           \NC no       \NC no       \NC yes \NC string \NC disk file name\NC\NR
\stoptabulate

The key \type{name} is always required.

The key \type{used} is set by the engine when a font is actively in
use, this makes sure that the font's definition is written to the
output file (\DVI\ or \PDF). The \TFM\ reader sets it to false.

The \type{direction} is a number signalling the \quote{normal} direction for
this font. There are sixteen possibilities:

\starttabulate[|c|c|c|c|]
\NC \bf number \NC \bf meaning \NC \bf number \NC \bf meaning \NC\NR
\NC 0          \NC LT          \NC 8          \NC TT          \NC\NR
\NC 1          \NC LL          \NC 9          \NC TL          \NC\NR
\NC 2          \NC LB          \NC 10         \NC TB          \NC\NR
\NC 3          \NC LR          \NC 11         \NC TR          \NC\NR
\NC 4          \NC RT          \NC 12         \NC BT          \NC\NR
\NC 5          \NC RL          \NC 13         \NC BL          \NC\NR
\NC 6          \NC RB          \NC 14         \NC BB          \NC\NR
\NC 7          \NC RR          \NC 15         \NC BR          \NC\NR
\stoptabulate

These are \OMEGA|-|style direction abbreviations: the first character
indicates the \quote{first} edge of the character glyphs (the edge that is
seen first in the writing direction), the second the \quote{top} side.

The \type{parameters} is a hash with mixed key types. There are seven
possible string keys, as well as a number of integer indices (these
start from 8 up). The seven strings are actually used instead of the
bottom seven indices, because that gives a nicer user interface.

The names and their internal remapping:

\starttabulate[|l|c|]
\NC \bf name      \NC \bf internal remapped number \NC\NR
\NC slant         \NC 1  \NC\NR
\NC space         \NC 2  \NC\NR
\NC space_stretch \NC 3  \NC\NR
\NC space_shrink  \NC 4  \NC\NR
\NC x_height      \NC 5  \NC\NR
\NC quad          \NC 6  \NC\NR
\NC extra_space   \NC 7  \NC\LR
\stoptabulate

The keys \type{type}, \type{format}, \type{embedding}, \type{fullname} and
\type{filename} are used to embed \OPENTYPE\ fonts in the result \PDF.

The \type{characters} table is a list of character hashes indexed by
integer number. The number is the \quote{internal code}. \TEX\ knows this
character by.

Two very special string indexes can be used also: \type{left_boundary} is a
virtual character whose ligatures and kerns are used to handle word
boundary processing. \type{right_boundary} is similar but not actually
used for anything (yet!).

Other index keys are ignored.

Each character hash itself is a hash. For example, here is the
character \quote{f} (decimal 102) in the font cmr10 at 10 points:

\starttyping
[102] = {
  ["kerns"] = {
     [63] = 50973,
     [93] = 50973,
     [39] = 50973,
     [33] = 50973,
     [41] = 50973
  },
  ["italic"] = 50973,
  ["height"] = 455111,
  ["depth"] = 0,
  ["ligatures"] = {
    [102] = {
       ["char"] = 11,
       ["type"] = 0
    },
    [108] = {
       ["char"] = 13,
       ["type"] = 0
    },
    [105] = {
       ["char"] = 12,
       ["type"] = 0
    }
  },
  ["width"] = 200250
}
\stoptyping

The following top|-|level keys can be present inside a character hash:

\starttabulate[|l|c|c|c|l|p|]
\NC \bf key    \NC \bf from vf \NC \bf from tfm \NC \bf used \NC \bf value type \NC \bf description \NC\NR
\NC width      \NC yes         \NC yes          \NC yes      \NC number         \NC character's width, in sp (default 0) \NC\NR
\NC height     \NC no          \NC yes          \NC yes      \NC number         \NC character's height, in sp (default 0) \NC\NR
\NC depth      \NC no          \NC yes          \NC yes      \NC number         \NC character's depth, in sp (default 0) \NC\NR
\NC italic     \NC no          \NC yes          \NC yes      \NC number         \NC character's italic correction, in sp (default zero) \NC\NR
\NC next       \NC no          \NC yes          \NC yes      \NC number         \NC the \quote{next larger} character index \NC\NR
\NC extensible \NC no          \NC yes          \NC yes      \NC table          \NC the constituent bits of an extensible recipe \NC\NR
\NC kerns      \NC no          \NC yes          \NC yes      \NC table          \NC kerning information \NC\NR
\NC ligatures  \NC no          \NC yes          \NC yes      \NC table          \NC ligaturing information \NC\NR
\NC commands   \NC yes         \NC no           \NC yes      \NC array          \NC virtual font commands \NC\NR
\NC name       \NC no          \NC no           \NC no       \NC string         \NC the character (\POSTSCRIPT) name \NC\NR
\NC index      \NC no          \NC no           \NC yes      \NC number         \NC the (\OPENTYPE\ or \TRUETYPE) font glyph index \NC\NR
\NC used       \NC no          \NC yes          \NC yes      \NC boolean        \NC typeset already (default: false)? \NC\NR
\stoptabulate

The presence of \type{extensible} will overrule \type{next}, if that is also present.

The \type{extensible} table is very simple:

\starttabulate[|l|l|p|]
\NC \bf key \NC \bf type \NC \bf description                    \NC\NR
\NC top     \NC number   \NC \quote{top} character index        \NC\NR
\NC mid     \NC number   \NC \quote{middle} character index     \NC\NR
\NC bot     \NC number   \NC \quote{bottom} character index     \NC\NR
\NC rep     \NC number   \NC \quote{repeatable} character index \NC\NR
\stoptabulate

The \type{kerns} table is a hash indexed by character index (and
\quote{character index} is defined as either a non|-|negative integer or the
string value \type {right_boundary}), with the values the kerning to be
applied, in scaled points.

The \type{ligatures} table is a hash indexed by character index (and
\quote{character index} is defined as either a non|-|negative integer or the
string value \type {right_boundary}), with the values being yet another small
hash, with two fields:

\starttabulate[|l|l|p|]
\NC \bf key \NC \bf type \NC \bf description \NC\NR
\NC type    \NC number   \NC the type of this ligature command, default 0 \NC\NR
\NC char    \NC number   \NC the character index of the resultant ligature \NC\NR
\stoptabulate

The \type{char} field in a ligature is required.

The \type{type} field inside a ligature is the numerical or string value of one of the eight
possible ligature types supported by \TEX.  When \TEX\ inserts a new ligature, it puts the new
glyph in the middle of the left and right glyphs. The original left and right glyphs can
optionally be retained, and when at least one of them is kept, it is also possible to move the
new \quote{insertion point} forward one or two places. The glyph that ends up to the right of the
insertion point will become the next \quote{left}.

\starttabulate[|l|c|l|l|]
\NC \bf textual (Knuth) \NC \bf number \NC \bf string    \NC result \NC\NR
\NC l + r =: n          \NC 0          \NC \type{=:}     \NC \|n    \NC\NR
\NC l + r =:\| n        \NC 1          \NC \type{=:|}    \NC \|nr   \NC\NR
\NC l + r \|=: n        \NC 2          \NC \type{|=:}    \NC \|ln   \NC\NR
\NC l + r \|=:\| n      \NC 3          \NC \type{|=:|}   \NC \|lnr  \NC\NR
\NC l + r  =:\|\> n     \NC 5          \NC \type{=:|>}   \NC n\|r   \NC\NR
\NC l + r \|=:\> n      \NC 6          \NC \type{|=:>}   \NC l\|n   \NC\NR
\NC l + r \|=:\|\> n    \NC 7          \NC \type{|=:|>}  \NC l\|nr  \NC\NR
\NC l + r \|=:\|\>\> n  \NC 11         \NC \type{|=:|>>} \NC ln\|r  \NC\NR
\stoptabulate

The default value is~0, and can be left out. That signifies a \quote{normal}
ligature where the ligature replaces both original glyphs. In this table
the~\| indicates the final insertion point.

The \type{commands} array is explained below.

\section {Real fonts}

Whether or not a \TEX\ font is a \quote{real} font that should be written to
the \PDF\ document is decided by the \type{type} value in the top|-|level
font structure. If the value is \type{real}, then this is a proper
font, and the inclusion mechanism will attempt to add the needed
font object definitions to the \PDF.

Values for \type{type}:

\starttabulate[|l|p|]
\NC \bf value     \NC \bf description        \NC\NR
\NC real          \NC this is a base font    \NC\NR
\NC virtual       \NC this is a virtual font \NC\NR
\stoptabulate

The actions to be taken depend on a number of different variables:

\startitemize[packed]
\item Whether the used font fits in an 8-bit encoding scheme or not
\item The type of the disk font file
\item The level of embedding requested
\stopitemize

A font that uses anything other than an 8-bit encoding vector has to
be written to the \PDF\ in a different way.

The rule is: if the font table has \type {encodingbytes} set to~2, then this
is a wide font, in all other cases it isn't. The value~2 is the
default for \OPENTYPE\ and \TRUETYPE\ fonts loaded via \LUA.

If no special care is needed, \LUATEX\ currently falls back to the
mapfile|-|based solution used by \PDFTEX\ and \DVIPS. This behaviour
will be removed in the future, when the existing code becomes
integrated in the new subsystem.

But if this is a \quote{wide} font, then the new subsystem kicks in, and
some extra fields have to be present in the font structure. In this
case, \LUATEX\ does not use a map file at all.

The extra fields are: \type{format}, \type{embedding}, \type{fullname},
\type{cidinfo} (as explained above), \type{filename}, and the
\type{index} key in the separate characters.

Values for \type{format}:

\starttabulate[|l|p|]
\NC \bf value \NC \bf description                                           \NC\NR
\NC type1     \NC this is a \POSTSCRIPT\ \TYPEONE\ font                     \NC\NR
\NC type3     \NC this is a bitmapped (\PK) font                            \NC\NR
\NC truetype  \NC this is a \TRUETYPE\ or \TRUETYPE|-|based \OPENTYPE\ font \NC\NR
\NC opentype  \NC this is a \POSTSCRIPT|-|based \OPENTYPE\ font             \NC\NR
\stoptabulate

Curerntly, only \TRUETYPE\ and \OPENTYPE\ fonts can be \quote{wide}
fonts (\TYPEONE\ \POSTSCRIPT\ fonts are not supported).

Values for \type{embedding}:

\starttabulate[|l|p|]
\NC \bf value \NC \bf description \NC\NR
\NC no        \NC don't embed the font at all \NC\NR
\NC subset    \NC include and atttempt to subset the font \NC\NR
\NC full      \NC include this font in it's entirety \NC\NR
\stoptabulate

At the moment, \type{subset} only works for \POSTSCRIPT|-|based non|-|\CID\
\OPENTYPE\ fonts, every other font format essentially is treated as
\type{full}.

It is not possible to artificially modify the transformation matrix
for the font at the moment.

The other fields are used as follows: The \type{fullname} will be the
\POSTSCRIPT|/|\PDF\ font name. The \type{cidinfo} will be used as the
character set (the CID \type{/Ordering} and \type{/Registry} keys). The
\type{filename} points to the actual font file. If you include the
full path in the \type{filename} or if the file is in the local
directory, \LUATEX\ will run a little bit more efficient because it
will not have to re|-|run the \type{find_xxx_file} callback in that
case.

Be careful: when mixing old and new fonts in one document, it is possible to
create name \POSTSCRIPT\ name clashes that can result in printing
errors. When this happens, you have to change the \type{fullname}
of the font.

Typeset strings are written out in a wide format using 2~bytes per
glyph, using the \type{index} key in the character information as
value. The overall effect is like having an encoding based on numbers
instead of traditional (\POSTSCRIPT) name|-|based reencoding.

This type of reencoding means that there is no longer a clear
connection between the text in your input file and the strings in the
output \PDF\ file. Dealing with this is on the agenda.

\section {Virtual fonts}

You have to take the following steps if you want \LUATEX\ to treat the
returned table from \luatex{define_font} as a virtual font:

\startitemize[packed]
\item Set the top|-|level key \type {type} to \type {virtual}.
\item Make sure there is at least one valid entry in \luatex{fonts} (see below)
\item Give a \type {commands} array to every character (see below)
\stopitemize

The presence of the toplevel \type {type} key with the specific value
\type {virtual} will trigger handling of the rest of the special virtual
font fields in the table, but the mere existance of 'type' is enough
to prevent \LUATEX\ from looking for a virtual font on its own.

Therefore, this also works \quote{in reverse}: if you are absolutely certain
that a font is not a virtual font, assigning the value \type{base} or
\type{real} to \type{type} will inhibit \LUATEX\ from looking for a virtual font
file, thereby saving you a disk search.

The \luatex{fonts} is another \LUA\ array. The values are one- or two|-|key
hashes themselves, each entry indicating one of the base fonts in a
virtual font. An example makes this easy to understand

\starttyping
fonts = { { name = "ptmr8a", size = 655360},
          { name = "psyr", size = 600000},
          { id = 38 } }
\stoptyping

says that the first referenced font (index 1) in this virtual font is
\type{ptrmr8a} loaded at 10pt, and the second is \type{psyr}  loaded
at a little over 9pt. The third one is previously defined font that
is known to \LUATEX\ as fontid \quote{38}.

The array index numbers are used by the character command definitions
that are part of each character.

The \luatex{commands} array is a hash here each item is another small array, with first
entry representing a command and the extra items the parameters to that command. The
allowed commands and their arguments are:

\starttabulate[|l|l|l|l|p|]
\NC \bf command name  \NC \bf arguments \NC \bf arg type \NC \bf description \NC\NR
\NC font              \NC 1         \NC number    \NC select a new font from the local \luatex{fonts} table\NC\NR
\NC char              \NC 1         \NC number    \NC typeset this character number from the current \NC\NR
\NC node              \NC 1         \NC node      \NC output this node (list), and move right\NC\NR
\NC slot              \NC 2         \NC number    \NC a shortcut for a font, char set\NC\NR
\NC push              \NC 0         \NC           \NC save current position\NC\NR
\NC nop               \NC 0         \NC           \NC do nothing \NC\NR
\NC pop               \NC 0         \NC           \NC pop position \NC\NR
\NC rule              \NC 2         \NC 2 numbers \NC output a rule $w*h$, and move right\NC\NR
\NC down              \NC 1         \NC number    \NC move down on the page\NC\NR
\NC right             \NC 1         \NC number    \NC move right on the page\NC\NR
\NC special           \NC 1         \NC string    \NC output a \tex{special} command\NC\NR
\NC comment           \NC any       \NC any       \NC the rest of the command is ignored\NC\NR
\stoptabulate

Here is a rather elaborate example:
\starttyping
...
commands = {
  {"push"},                     -- remember where we are
  {"right", 5000},              -- move right about 0.08pt
  {"font", 1},                  -- select the fonts[1] entry
  {"char", 97},                 -- place character 97 (a)
  {"pop"},                      -- go all the way back
  {"down", -200000},            -- move *up* about 3pt
  {"special", "pdf: 1 0 0 rg"}  -- switch to red color
  {"rule", 500000, 20000}       -- draw a bar
  {'special',"pdf: 0 g"}        -- back to black
}
...
\stoptyping

The default value for \type {font} is always~1, for each character anew. If
the virtual font is essentially only a re|-|encoding, then you do usually
do not have create an explicit \quote{font} entry.

Regardless of the amount of movement you create within the \type {commands},
the output pointer will always move by exactly the width as given in
the \type {width} key of the character hash, after running the \type {commands}.

Even in a \quote{real} font, there can be virtual characters. When \LUATEX\ encounters a \type {commands}
field inside a character when it becomes time to typeset the character, it will interpret the
commands, just like for a true virtual character. In this case, if you have created no \quote{fonts}
array, then the default and only \quote{base} font is taken to be the current font itself. In
practise, this means that you can create virtual duplicates of existing characters.

Note: this feature does {\it not\/} work the other way around. There can not be \quote{real}
characters in a virtual font!

Finally, here is a plain \TEX\ input file with a demonstration:

\starttyping
% start of virtual-demo.tex

\pdfoutput=1
\directlua0 {
  callback.register("define_font",
    function (name,area,size)
      if name == 'cmr10-red' then
        f = font.read_tfm('cmr10',size)
        f.name = 'cmr10-red'
        f.type = 'virtual'
        f.fonts = {{'cmr10', size}}
        for i,v in pairs(f.characters) do
          if (string.char(i)):find("[tacohanshartmut]") then
            v.commands = {
              {'special','pdf: 1 0 0 rg'},
              {'char',i},
              {'special','pdf: 0 g'},
            }
          else
            v.commands = {{'char',i}}
          end
        end
      else
        f = font.read_tfm(name,size)
      end
      return f
    end
  )
}

\font\myfont = cmr10-red \myfont  This is a line of text \par
\font\myfontx= cmr10 \myfontx Here is another line of text \par

\bye
% end of virtual-demo.tex
\stoptyping

\chapter[nodes]{Nodes}

\section{\LUA\ node representation}

\TEX's nodes are represented in \LUA\ as userdata object with a variable
set of fields.

The current return value of \luatex{node.types()} is: \ctxlua {for _,v
in pairs(node.types()) do tex.print('\\type{' .. v .. '} (' .. _
.. '), ') end } but as already mentioned, the math and alignment nodes
in this list are not supported at the moment. The useful list
is described in the next sections.

\subsection{Auxiliary items}

A few node|-|typed userdata objects do not occur in the \quote{normal} list of
nodes, but can be pointed to from within that list. They are not quite
really nodes, but it is easier for the library routines to treat them
as if they were.

\subsubsection{glue_spec items}

Skips are about the only type of data objects in traditional \TEX\
that are not a simple value. The structure that represents the glue
components of a skip is called a \type {glue_spec}, and it has the following
accessible fields:

\starttabulate[|lT|l|p|]
\NC \bf key            \NC \bf type \NC \bf explanation \NC\NR
\NC width          \NC number  \NC \NC\NR
\NC stretch        \NC number  \NC \NC\NR
\NC stretch_order  \NC number  \NC \NC\NR
\NC shrink         \NC number  \NC \NC\NR
\NC shrink_order   \NC number  \NC \NC\NR
\stoptabulate

These objects are reference counted, so there is actually an extra
field named \type {ref_count} as well, but that will become completely
hidden in the near future.

\subsubsection{attribute\_list items}

The newly introduced attribute registers are also non|-|trivial, because
the value that is attached to a node is essentially a sparse array of
key|-|value pairs.

It is generally easiest to deal with attributes by using the dedicated
functions in the \luatex{node} library, but for completeness, here is
the low|-|level interface:

\starttabulate[|lT|l|p|]
\NC \bf field     \NC \bf type \NC  \bf explanation \NC\NR
\NC next          \NC node  \NC pointer to the first attribute\NC\NR
\stoptabulate

There are no extra fields, this kind of item is only used as a head
pointer for attribute items, making them easier to handle.

A normal node's \type{attr}ibute field will point to a field of this
type, and the \type{next} field in that field will then point to the
first defined \quote{attribute} item, whose \type {next} will point to the second
\quote{attribute} item, etc.

\subsubsection{attribute item}

Valid fields:

\starttabulate[|lT|l|p|]
\NC \bf field     \NC \bf type \NC  \bf explanation \NC\NR
\NC next           \NC node    \NC pointer to the next attribute\NC\NR
\NC number         \NC number  \NC the LHS\NC\NR
\NC value          \NC number  \NC the RHS\NC\NR
\stoptabulate

\subsection{Main text nodes}

These are the nodes that comprise actual typesetting commands.

A few fields are present in all nodes regardless of their type, these are:

\starttabulate[|lT|l|p|]
\NC \bf field     \NC \bf type \NC  \bf explanation \NC\NR
\NC next    \NC node    \NC  The next node in a list, or nil\NC\NR
\NC id      \NC number  \NC  The node's type (\type{id}) number \NC\NR
\NC subtype \NC number  \NC  The node \type{subtype} identifier\NC\NR
\stoptabulate

The subtype is sometimes just a stub entry. Almost all nodes also have
an \type {attr} field. In the following tables \type{next} and \type{id} are
not explicitly mentioned.

\subsubsection{hlist nodes}

Valid fields: \showfields{hlist}

\starttabulate[|lT|l|p|]
\NC \bf field     \NC \bf type \NC  \bf explanation \NC\NR
\NC subtype    \NC number  \NC  unused\NC\NR
\NC attr       \NC node    \NC  The head of the associated attribute list \NC\NR
\NC width      \NC number  \NC  \NC\NR
\NC height     \NC number  \NC  \NC\NR
\NC depth      \NC number  \NC  \NC\NR
\NC shift      \NC number  \NC  a displacement perpendicular to the
                                character progression direction \NC\NR
\NC glue_order \NC number  \NC  a number in the range 0--4, indicating
                                the glue order\NC\NR
\NC glue_set   \NC number  \NC  the calculated glue ratio\NC\NR
\NC glue_sign  \NC number  \NC  \NC\NR
\NC list       \NC node    \NC  the body of this list\NC\NR
\NC dir        \NC number  \NC  the direction of this box\NC\NR
\stoptabulate

\subsubsection{vlist nodes}

Valid fields: As for hlist, except that \quote{shift} is a displacement
perpendicular to the line progression direction.

\subsubsection{rule nodes}

Valid fields: \showfields{rule}

\starttabulate[|lT|l|p|]
\NC \bf field     \NC \bf type \NC  \bf explanation \NC\NR
\NC subtype    \NC number  \NC  unused\NC\NR
\NC attr       \NC node    \NC  \NC\NR
\NC width      \NC number  \NC  Rule size. The special value $-1073741824$
                                is used for \quote{running} glue dimensions\NC\NR
\NC height     \NC number  \NC  ' '\NC\NR
\NC depth      \NC number  \NC  ' '\NC\NR
\NC dir        \NC number  \NC  the direction of this rule\NC\NR
\stoptabulate

\subsubsection{ins nodes}

Valid fields: \showfields{ins}

\starttabulate[|lT|l|p|]
\NC \bf field     \NC \bf type \NC  \bf explanation \NC\NR
\NC subtype    \NC number  \NC  the insertion class\NC\NR
\NC attr       \NC node    \NC  \NC\NR
\NC cost       \NC number  \NC  The penalty associated with this insert\NC\NR
\NC height     \NC number  \NC  \NC\NR
\NC depth      \NC number  \NC  \NC\NR
\NC list       \NC node    \NC  the body of this insert\NC\NR
\NC top_skip   \NC node    \NC a pointer to the \tex{splittopskip} glue spec\NC\NR
\stoptabulate

\subsubsection{mark nodes}

Valid fields: \showfields{mark}

\starttabulate[|lT|l|p|]
\NC \bf field     \NC \bf type \NC  \bf explanation \NC\NR
\NC subtype    \NC number  \NC  unused\NC\NR
\NC attr       \NC node    \NC  \NC\NR
\NC class      \NC number  \NC  the mark class\NC\NR
\NC mark       \NC table   \NC  a table representing a token list\NC\NR
\stoptabulate

\subsubsection{adjust nodes}

Valid fields: \showfields{adjust}

\starttabulate[|lT|l|p|]
\NC \bf field     \NC \bf type \NC  \bf explanation \NC\NR
\NC subtype    \NC number  \NC  1 = \quote{pre}\NC\NR
\NC attr       \NC node    \NC  \NC\NR
\NC list       \NC node    \NC  adjusted material\NC\NR
\stoptabulate

\subsubsection{disc nodes}

Valid fields: \showfields{disc}

\starttabulate[|lT|l|p|]
\NC \bf field     \NC \bf type \NC  \bf explanation \NC\NR
\NC subtype    \NC number  \NC  unused\NC\NR
\NC attr       \NC node    \NC  \NC\NR
\NC pre        \NC node    \NC  pointer to the pre|-|break text\NC\NR
\NC post       \NC node    \NC  pointer to the post|-|break text\NC\NR
\NC replace    \NC number  \NC  the number of nodes to skip if this
                                discretionary is chosen as a breakpoint\NC\NR
\stoptabulate

\subsubsection{math nodes}

Valid fields: \showfields{math}

\starttabulate[|lT|l|p|]
\NC \bf field     \NC \bf type \NC  \bf explanation \NC\NR
\NC subtype    \NC number  \NC  0 = \quote{on}, 1 = \quote{off}\NC\NR
\NC attr       \NC node    \NC  \NC\NR
\NC surround   \NC number  \NC  width of the \tex{mathsurround} kern\NC\NR
\stoptabulate

\subsubsection{glue nodes}

Valid fields: \showfields{glue}

\starttabulate[|lT|l|p|]
\NC \bf field     \NC \bf type \NC  \bf explanation \NC\NR
\NC subtype    \NC number  \NC  0 = \tex{skip},
                                1--18 = internal glue parameters,
                                100 = \tex{leaders},
                                101 = \tex{cleaders},
                                102 = \tex{xleaders}  \NC\NR
\NC attr       \NC node    \NC  \NC\NR
\NC spec       \NC node    \NC  pointer to a glue\_spec item \NC\NR
\NC leader     \NC node    \NC  pointer to a box or rule for leaders\NC\NR
\stoptabulate

\subsubsection{kern nodes}

Valid fields: \showfields{kern}

\starttabulate[|lT|l|p|]
\NC \bf field     \NC \bf type \NC  \bf explanation \NC\NR
\NC subtype    \NC number  \NC  0 = from font,
                                1 = from \tex{kern} or \tex{/},
                                2 = from \tex{accent}\NC\NR
\NC attr       \NC node    \NC  \NC\NR
\NC width      \NC number  \NC  \NC\NR
\stoptabulate


\subsubsection{penalty nodes}

Valid fields: \showfields{penalty}

\starttabulate[|lT|l|p|]
\NC \bf field     \NC \bf type \NC  \bf explanation \NC\NR
\NC subtype    \NC number  \NC  not used\NC\NR
\NC attr       \NC node    \NC  \NC\NR
\NC penalty    \NC number  \NC  \NC\NR
\stoptabulate

\subsubsection{glyph nodes}

Valid fields: \showfields{glyph}

\starttabulate[|lT|l|p|]
\NC \bf field     \NC \bf type \NC  \bf explanation \NC\NR
\NC subtype    \NC number  \NC  0 = normal,
	                            1 = ligature,
                                2 = leftboundary ligature,
                                3 = rightboundary ligature\NC\NR
\NC attr       \NC node    \NC  \NC\NR
\NC char       \NC number  \NC \NC\NR
\NC font       \NC number  \NC \NC\NR
\NC components \NC node    \NC pointer to ligature components\NC\NR
\NC xoffset    \NC number  \NC \NC \NR
\NC yoffset    \NC number  \NC \NC \NR
\stoptabulate

\subsubsection{margin\_kern nodes}

Valid fields: \showfields{margin_kern}

\starttabulate[|lT|l|p|]
\NC \bf field     \NC \bf type \NC  \bf explanation \NC\NR
\NC subtype    \NC number  \NC  0 = left side,
                                1 = right side\NC\NR
\NC attr       \NC node    \NC  \NC\NR
\NC width      \NC number  \NC  \NC\NR
\NC glyph      \NC node    \NC  \NC\NR
\stoptabulate

\subsection{whatsit nodes}

Whatsit nodes come in many subtypes, that you can ask for my running
\luatex{node.whatsits()}: \ctxlua {for _,v in pairs(node.whatsits()) do
tex.print('\\type{' .. v .. '} (' .. _ .. '), ') end }

\subsubsection{open nodes}

Valid fields: \showfields{whatsit,open}

\starttabulate[|lT|l|p|]
\NC \bf field     \NC \bf type \NC  \bf explanation \NC\NR
\NC attr      \NC node    \NC  \NC\NR
\NC stream    \NC number  \NC \TEX's stream id number\NC\NR
\NC name      \NC string  \NC file name \NC\NR
\NC ext       \NC string  \NC file extension \NC\NR
\NC area      \NC string  \NC file area \NC\NR
\stoptabulate

\subsubsection{write nodes}

Valid fields: \showfields{whatsit,write}

\starttabulate[|lT|l|p|]
\NC \bf field     \NC \bf type \NC  \bf explanation \NC\NR
\NC attr      \NC node    \NC  \NC\NR
\NC stream    \NC number  \NC \TEX's stream id number\NC\NR
\NC data      \NC table   \NC a table representing the token list to be written\NC\NR
\stoptabulate

\subsubsection{close nodes}

Valid fields: \showfields{whatsit,close}

\starttabulate[|lT|l|p|]
\NC \bf field     \NC \bf type \NC  \bf explanation \NC\NR
\NC attr      \NC node    \NC  \NC\NR
\NC stream    \NC number  \NC \TEX's stream id number\NC\NR
\stoptabulate

\subsubsection{special nodes}

Valid fields: \showfields{whatsit,special}

\starttabulate[|lT|l|p|]
\NC \bf field     \NC \bf type \NC  \bf explanation \NC\NR
\NC attr      \NC node    \NC  \NC\NR
\NC data      \NC string  \NC The \tex{special} information\NC\NR
\stoptabulate

\subsubsection{language nodes}

Valid fields: \showfields{whatsit,language}

\starttabulate[|lT|l|p|]
\NC \bf field     \NC \bf type \NC  \bf explanation \NC\NR
\NC attr      \NC node    \NC  \NC\NR
\NC lang      \NC number  \NC language id number\NC\NR
\NC left      \NC number  \NC value of \tex{lefthyphenmin}\NC\NR
\NC right     \NC number  \NC value of \tex{righthyphenmin}\NC\NR
\stoptabulate


\subsubsection{local_par nodes}

Valid fields: \showfields{whatsit,local_par}

\starttabulate[|lT|l|p|]
\NC \bf field     \NC \bf type \NC  \bf explanation \NC\NR
\NC attr      \NC node    \NC  \NC\NR
\NC pen_inter \NC number  \NC interline penalty\NC\NR
\NC pen_broken\NC number  \NC broken penalty\NC\NR
\NC dir       \NC number  \NC the direction of this par\NC\NR
\NC box_left  \NC node      \NC the \tex{localleftbox}\NC\NR
\NC box_left_width\NC number\NC width of the \tex{localleftbox}\NC\NR
\NC box_right  \NC node      \NC the \tex{localrightbox}\NC\NR
\NC box_right_width\NC number\NC width of the \tex{localrightbox}\NC\NR
\stoptabulate

\subsubsection{dir nodes}

Valid fields: \showfields{whatsit,dir}

\starttabulate[|lT|l|p|]
\NC \bf field     \NC \bf type \NC  \bf explanation \NC\NR
\NC attr      \NC node    \NC  \NC\NR
\NC dir       \NC number  \NC the direction\NC\NR
\NC level     \NC number  \NC nesting level of this direction whatsit\NC\NR
\NC dvi_ptr   \NC number  \NC a saved dvi buffer byte offset\NC\NR
\NC dir_h     \NC number  \NC a saved dvi position\NC\NR
\stoptabulate

\subsubsection{pdf_literal nodes}

Valid fields: \showfields{whatsit,pdf_literal}

\starttabulate[|lT|l|p|]
\NC \bf field     \NC \bf type \NC  \bf explanation \NC\NR
\NC attr      \NC node    \NC  \NC\NR
\NC mode      \NC number  \NC  The \quote{mode} setting of this literal\NC\NR
\NC data      \NC string  \NC The \tex{pdfliteral} information\NC\NR
\stoptabulate

\subsubsection{pdf_refobj nodes}

Valid fields: \showfields{whatsit,pdf_refobj}

\starttabulate[|lT|l|p|]
\NC \bf field     \NC \bf type \NC  \bf explanation \NC\NR
\NC attr      \NC node    \NC  \NC\NR
\NC objnum    \NC number  \NC the referenced \PDF\ object number\NC\NR
\stoptabulate

\subsubsection{pdf_refxform nodes}

Valid fields: \showfields{whatsit,pdf_refxform}

\starttabulate[|lT|l|p|]
\NC \bf field     \NC \bf type \NC  \bf explanation \NC\NR
\NC attr      \NC node    \NC  \NC\NR
\NC width     \NC number  \NC \NC \NR
\NC height    \NC number  \NC \NC \NR
\NC depth     \NC number  \NC \NC \NR
\NC objnum    \NC number  \NC the referenced \PDF\ object number\NC\NR
\stoptabulate

\subsubsection{pdf_refximage nodes}

Valid fields: \showfields{whatsit,pdf_refximage}

\starttabulate[|lT|l|p|]
\NC \bf field     \NC \bf type \NC  \bf explanation \NC\NR
\NC attr      \NC node    \NC  \NC\NR
\NC width     \NC number  \NC \NC \NR
\NC height    \NC number  \NC \NC \NR
\NC depth     \NC number  \NC \NC \NR
\NC objnum    \NC number  \NC the referenced \PDF\ object number\NC\NR
\stoptabulate

\subsubsection{pdf_annot nodes}

Valid fields: \showfields{whatsit,pdf_annot}

\starttabulate[|lT|l|p|]
\NC \bf field     \NC \bf type \NC  \bf explanation \NC\NR
\NC attr      \NC node    \NC  \NC\NR
\NC width     \NC number  \NC \NC \NR
\NC height    \NC number  \NC \NC \NR
\NC depth     \NC number  \NC \NC \NR
\NC objnum    \NC number  \NC the referenced \PDF\ object number\NC\NR
\NC data      \NC string  \NC the annotation data\NC\NR
\stoptabulate


\subsubsection{pdf_start_link nodes}

Valid fields: \showfields{whatsit,pdf_start_link}

\starttabulate[|lT|l|p|]
\NC \bf field     \NC \bf type \NC  \bf explanation \NC\NR
\NC attr      \NC node    \NC  \NC\NR
\NC width     \NC number  \NC \NC \NR
\NC height    \NC number  \NC \NC \NR
\NC depth     \NC number  \NC \NC \NR
\NC objnum    \NC number  \NC the referenced \PDF\ object number\NC\NR
\NC link_attr \NC table   \NC the link attribute token list\NC\NR
\NC action    \NC node    \NC the action to perform\NC\NR
\stoptabulate

\subsubsection{action items}

Valid fields: \showfields{action}

These are a special kind of item that only appears inside
pdf start link objects.

\subsubsection{pdf_end_link nodes}

Valid fields: \showfields{whatsit,pdf_end_link}

\starttabulate[|lT|l|p|]
\NC \bf field     \NC \bf type \NC  \bf explanation \NC\NR
\NC attr      \NC node    \NC  \NC\NR
\stoptabulate

\subsubsection{pdf_dest nodes}

Valid fields: \showfields{whatsit,pdf_dest}

\starttabulate[|lT|l|p|]
\NC \bf field     \NC \bf type \NC  \bf explanation \NC\NR
\NC attr      \NC node    \NC  \NC\NR
\NC width     \NC number  \NC \NC \NR
\NC height    \NC number  \NC \NC \NR
\NC depth     \NC number  \NC \NC \NR
\NC named_id  \NC number  \NC is the dest_id a string value?\NC\NR
\NC dest_id   \NC number or string \NC the destination id\NC\NR
\NC dest_type \NC number\NC type of destination\NC\NR
\NC xyz_zoom  \NC number\NC \NC\NR
\NC objnum    \NC number  \NC the \PDF\ object number\NC\NR
\stoptabulate

\subsubsection{pdf_thread nodes}

Valid fields: \showfields{whatsit,pdf_thread}

\starttabulate[|lT|l|p|]
\NC \bf field     \NC \bf type \NC  \bf explanation \NC\NR
\NC attr       \NC node    \NC  \NC\NR
\NC width      \NC number  \NC \NC \NR
\NC height     \NC number  \NC \NC \NR
\NC depth      \NC number  \NC \NC \NR
\NC named_id   \NC number  \NC is the tread_id a string value?\NC\NR
\NC tread_id   \NC number or string \NC the thread id\NC\NR
\NC thread_attr\NC number           \NC extra thread information\NC\NR
\stoptabulate

\subsubsection{pdf_start_thread nodes}

Valid fields: \showfields{whatsit,pdf_start_thread}

\starttabulate[|lT|l|p|]
\NC \bf field     \NC \bf type \NC  \bf explanation \NC\NR
\NC attr       \NC node    \NC  \NC\NR
\NC width      \NC number  \NC \NC \NR
\NC height     \NC number  \NC \NC \NR
\NC depth      \NC number  \NC \NC \NR
\NC named_id   \NC number  \NC is the tread_id a string value?\NC\NR
\NC tread_id   \NC number or string \NC the thread id\NC\NR
\NC thread_attr\NC number           \NC extra thread information\NC\NR
\stoptabulate

\subsubsection{pdf_end_thread nodes}

Valid fields: \showfields{whatsit,pdf_end_thread}

\starttabulate[|lT|l|p|]
\NC \bf field     \NC \bf type \NC  \bf explanation \NC\NR
\NC attr      \NC node    \NC  \NC\NR
\stoptabulate

\subsubsection{pdf_save_pos nodes}

Valid fields: \showfields{whatsit,pdf_save_pos}

\starttabulate[|lT|l|p|]
\NC \bf field     \NC \bf type \NC  \bf explanation \NC\NR
\NC attr      \NC node    \NC  \NC\NR
\stoptabulate

\subsubsection{late_lua nodes}

Valid fields: \showfields{whatsit,late_lua}

\starttabulate[|lT|l|p|]
\NC \bf field     \NC \bf type \NC  \bf explanation \NC\NR
\NC attr      \NC node    \NC  \NC\NR
\NC reg       \NC number  \NC \LUA\ state id number \NC\NR
\NC data      \NC string  \NC data to execute\NC\NR
\stoptabulate

\subsubsection{close_lua nodes}

Valid fields: \showfields{whatsit,close_lua}

\starttabulate[|lT|l|p|]
\NC \bf field     \NC \bf type \NC  \bf explanation \NC\NR
\NC attr      \NC node    \NC  \NC\NR
\NC reg    \NC number  \NC \LUA\ state id number\NC\NR
\stoptabulate

\subsubsection{pdf_colorstack  nodes}

Valid fields: \showfields{whatsit,pdf_colorstack}

\starttabulate[|lT|l|p|]
\NC \bf field     \NC \bf type \NC  \bf explanation \NC\NR
\NC attr      \NC node    \NC  \NC\NR
\NC stack    \NC number  \NC colorstack id number\NC\NR
\NC cmd      \NC number  \NC command to execute\NC\NR
\NC data     \NC string  \NC data\NC\NR
\stoptabulate

\subsubsection{pdf_setmatrix nodes}

Valid fields: \showfields{whatsit,pdf_setmatrix}

\starttabulate[|lT|l|p|]
\NC \bf field     \NC \bf type \NC  \bf explanation \NC\NR
\NC attr      \NC node    \NC  \NC\NR
\NC data     \NC string  \NC data\NC\NR
\stoptabulate

\subsubsection{pdf_save nodes}

Valid fields: \showfields{whatsit,pdf_save}

\starttabulate[|lT|l|p|]
\NC \bf field     \NC \bf type \NC  \bf explanation \NC\NR
\NC attr      \NC node    \NC  \NC\NR
\stoptabulate

\subsubsection{pdf_restore nodes}

Valid fields: \showfields{whatsit,pdf_restore}

\starttabulate[|lT|l|p|]
\NC \bf field     \NC \bf type \NC  \bf explanation \NC\NR
\NC attr      \NC node    \NC  \NC\NR
\stoptabulate

\subsubsection{user_defined nodes}

Valid fields: \showfields{whatsit,user_defined}

\starttabulate[|lT|l|p|]
\NC \bf field     \NC \bf type \NC  \bf explanation \NC\NR
\NC attr      \NC node    \NC  \NC\NR
\NC user_id  \NC number  \NC id number\NC\NR
\NC type     \NC number  \NC type of the value\NC\NR
\NC value    \NC number  \NC \NC\NR
\NC          \NC string  \NC \NC\NR
\NC          \NC node   \NC \NC\NR
\NC          \NC table \NC \NC\NR
\stoptabulate

% \section{User-defined whatsits}

\chapter{Modifications}

Besides the expected changes caused by new functionality, there are a
number of not|-|so|-|expected changes. These are sometimes a side|-|effect
of a new (conflicting) feature, or, more often than not, a change
necessary to clean up the internal interfaces.

\section{Changes from \TEX\ 3.141592}

\startitemize

\item There is no pool file, all strings are embedded during compilation.

\item \type {plus 1 fillll} does not generate an error. The extra \quote{l} is
simply typeset.

\item The \tex{endlinechar} can be either added (values 0 or more), or not
(negative values). If it is added, the character is always decimal 13 a/k/a
\type{^^M} a/k/a carriage return (This change may be temporary).

\stopitemize

\section{Changes from \ETEX\ 2.2}

\startitemize

\item The \ETEX\ functionality is always present and enabled
   (but see below about \TEXXET), so the prepended asterisk or
   \type{-etex} switch for \INITEX\ is not needed.

\item \TEXXET\ is not present, so the primitives

\starttyping
\TeXXeTstate
\beginR
\beginL
\endR
\endL
\stoptyping

are missing

\stopitemize

\section{Changes from \PDFTEX\ 1.40}

\startitemize

\item The (experimental) support for snap nodes has been removed, because 
it much more natural to build this functionality on top of node
processing and attributes. The associated primitives that are now gone
are: \tex{pdfsnaprefpoint}, \tex{pdfsnapy}, and \tex{pdfsnapycomp}.

\item A number of \quote{utility functions} is removed:

\startcolumns[n=3]
\starttyping
\pdfelapsedtime
\pdfescapehex
\pdfescapename
\pdfescapestring
\pdffiledump
\pdffilemoddate
\pdffilesize
\pdflastmatch
\pdfmatch
\pdfmdfivesum
\pdfresettimer
\pdfshellescape
\pdfstrcmp
\pdfunescapehex
\stoptyping
\stopcolumns

\item A few other experimental primitives are also provided without the
      extra  \luatex {pdf} prefix, so they can also be called like this:

\startcolumns[n=3]
\starttyping
\primitive
\ifprimitive
\ifabsnum
\ifabsdim
\stoptyping
\stopcolumns

\item The definitions for new didot and new cicero are patched.

\item The \tex{pdfprimitive} is bugfixed.

\item The \tex{pdftexversion} is set to 200.

\stopitemize

\section{Changes from \ALEPH\ RC4}

\startitemize

\item The input translations from \ALEPH\ are not implemented, the
   related primitives are not available

\startcolumns[n=2]
\starttyping
\DefaultInputMode
\noDefaultInputMode
\noInputMode
\InputMode
\DefaultOutputMode
\noDefaultOutputMode
\noOutputMode
\OutputMode
\DefaultInputTranslation
\noDefaultInputTranslation
\noInputTranslation
\InputTranslation
\DefaultOutputTranslation
\noDefaultOutputTranslation
\noOutputTranslation
\OutputTranslation
\stoptyping
\stopcolumns

\item A small series of bounds checking fixes to \tex{ocp} and
   \tex{ocplist} has been added to prevent the system from crashing
   due to array indexes running out of bounds.

\item The \tex{hoffset} bug when \tex{pagedir TRT}  is fixed,
removing the need for an explicit fix to \tex{hoffset}

\item A bug causing \tex{fam} to fail for family numbers above
    15 is fixed.

\item Some bits of \ALEPH\ assumed \type{0} and \type{null} were identical.
This resulted for instance in a bug that sometimes caused an eternal
loop when trying to \tex{show} a box.

\item A fair amount of minor bugs are fixed as well, most of these
related to \tex{tracingcommands} output.

\item The number of possible fonts, ocps and ocplists is
smaller than their maximum \ALEPH\ value (around 5000 fonts and
30000 ocps / ocplists).

\item The internal function \type{scan_dir()} has been renamed to
\type{scan_direction()} to prevent a naming clash.

\item The \type{^^} notation can come in five and six item repetitions also, to
insert characters that do not fit in the BMP.

\stopitemize

\section{Changes from standard \WEBC}

\startitemize

\item There is no mltex

\item There is no enctex

\item The following commandline switches are silently ignored, even
in non|-|\LUA\ mode:

\starttyping
-8bit
-translate-file=TCXNAME
-mltex
-enc
-etex
\stoptyping

\item \tex{openout} whatsits are not written to the log file.

\item Some of the so|-|called web2c extensions are hard to set up
  in non|-|\KPSE\ mode because texmf.cnf is not read: \type{shell-escape}
  is off (but that is not a problem because of \LUA's
  \lua{os.execute}), and the paranoia checks on \type{openin} and
  \type{openout} do not happen (however, it is easy for a \LUA\ script
  to do this itself by overloading \lua{io.open}).


\stopitemize

\chapter{Implementation notes}

\section{Primitives overlap}

The primitives

\starttabulate[|l|l|]
\NC \tex{pdfpagewidth} \NC \tex{pagewidth}  \NC \NR
\NC \tex{pdfpageheight}\NC \tex{pageheight} \NC \NR
\NC \tex{fontcharwd}   \NC \tex{charwd}     \NC \NR
\NC \tex{fontcharht}   \NC \tex{charht}     \NC \NR
\NC \tex{fontchardp}   \NC \tex{chardp}     \NC \NR
\NC \tex{fontcharic}   \NC \tex{charic}     \NC \NR
\stoptabulate

are all aliases of each other.

\section{Memory allocation}

The single internal memory heap that traditional \TEX\ used for tokens
and nodes is split into two wholly separate arrays. Each of those can
grow dynamically as needed.

The \type{texmf.cnf} settings related to main memory are no longer
used (these are: \type{main_memory}, \type{mem_bot},
\type{extra_mem_top} and \type{extra_mem_bot}). \quote{Out of main memory} errors can still occur,
but the limiting factor is now solely the amount of RAM in your
system.

Also, the memory (de)allocation routines for nodes are completely
rewritten. The relevant code now lives in the C file \type{luanode.c},
and now normally uses a dozen or so avail lists instead of a
doubly|-|linked model. At this moment, speed is still a little
suboptimal because separate helper structures are maintained for
debugging checks.

Because of the split in two arrays and the resulting differences in
the data structures, some of the pascal web macros have been split.
For instance, there are now \type{vlink} and \type{vinfo} as well as
\type{link} and \type{info}. All access to the variable memory array
is now hidden behind a macro called \type{vmem}.

The implementation of the growth of two arrays (via reallocation)
introduces a potential pitfall: the memory arrays should never be used
as the left hand side of a statement that can modify the array in
question.

The input line buffer and pool size are now also reallocated when
needed, and the \type{texmf.cnf} settings \type{buf_size} and
\type{pool_size} are silently ignored.

\section{Sparse arrays}

The \tex{mathcode}, \tex{delcode}, \tex{catcode},
\tex{sfcode}, \tex{lccode} and \tex{uccode} tables are now
sparse arrays that are implemented in~C. They are no longer part of
the \TEX\ \quote{equivalence table} and because each had 1.1 million
entries with a few memory words each, this makes a major difference
in memory usage.

These assignments do not yet show up when using the etex tracing
routines \tex{tracingassigns} and \tex{tracingrestores} (code
simply not written yet)

A side|-|effect of the current implementation is that \tex{global} is
now more expensive in terms of processing than non|-|global assignments.

See \type{mathcodes.c} and \type{textcodes.c} if you are interested in
the gory details.

Also, the glyph ids within a font are now managed by means
of a sparse array and glyph ids can go up to index $2^{21}-1$.

\section{Simple single-character csnames}

Single|-|character commands are no longer treated aspecially in the
internals, they are stored in the hash just like the multiletter
csnames.

The code that displays control sequences explicitly checks if
the length is one when it has to decide whether or not to add a
trailing space.

\section{Compressed format}

The format is passed through zlib, allowing it to shrink to roughly
half of the size it would have had in uncompressed form. This takes a
bit more CPU cycles but much less disk I/O, so it should still be
faster.

\section{Binary file reading}

All of the internal code is changed in such a way that if one of the
\type{read_xxx_file} callbacks is not set, then the file is read by
a C function using basically the same convention as the callback: a
single read into a buffer big enough to hold the entire file
contents. While this uses more memory than the previous code (that
mostly used \type{getc} calls), it can be quite a bit faster
(depending on your I/O subsystem).

\chapter{Known bugs and limitations}

The bugs below are going to be fixed eventually.

The top ones will be fixed soon, but in the later items either the
actual problem is hard to find, or the code that causes the bug is
going to be replaced by a new subsystem soon anyway.

\startitemize

\item Not all of \ALEPH's direction commands are handled properly in
\PDF\ mode yet: this affects all the top|-|bottom and bottom|-|top writing
directions. And also, the \tex{textdir} command is broken.

\item There is a problem with jobnames on 64-bit architectures: 
unless \type{--jobname} is given explicitly on the command line, the
\tex{jobname} will be empt, resulting in output files without names
(only extensions).


\item The \TFM\ loader sometimes stores a full path instead of only a
name, this prevents \VF\ loading from working properly.

\item The Lua setting of \type{false_boundarychar=65536} inside 
of a returned font metrics table results in an error.

\item Hyphenation can only deal with the Base Multilingual Plane (BMP)

\item \luatex{tex.print()} and \luatex{tex.sprint()} do not work if
\tex{directlua} is used in an \OTP\ file (in the output of an
\type{expression} rule).

\item Handling of attributes in math mode is not complete. The data
structures in math mode are quite different from those in text mode,
so this will take some extra effort to implement correctly.

\item When used inside  \type{\directlua}, \type{pdf.print()} should create a
literal node instead of flushin immedately.

\item At the moment, only characters in plane~0 and plane~1 can be
assigned catcode~13 (i.e. turned into active characters). This is a
temporary measure to reduce the memory requirements of \LUATEX.

\item Letter spacing (\tex{letterspacefont}) is currently non|-|functional
due to massive changes in the virtual font handling. This
functionality may actually be removed completely in the future,
because it is straightforward to set up letterspacing using the \LUA\
\quote{define_font} interface.

\stopitemize

\chapter{TODO}

On top of the \quote{normal} extensions that are planned, there are some
more specific small feature requests .

\startitemize

\item Implement the \TEX\ primitive \tex{dimension}, cf.\ \tex{number}

\item Change the \LUA\ table type{tex.dimen} to accept and return float
values instead of strings

\item Do something about \tex{withoutpt} and/or a new register type \tex{real}?

\item Create callback for the automatic creation of missing
characters in fonts

\item Implement the \TEX\ primitive \tex{htdp}?

\item Do boxes with dual baselines.

\item A way to (re?)calculate the width of a  \tex{vbox}, taking only
the natural width of the included items into account.

\item Make the number of the output box configurable.

\item Complete the attributes in math and switch the nodes to a double|-|linked list.

\item Implement csname lookups for \type{tex.box} access.

\item Integrate the various \PDFTEX\ extended font codes for hz en protruding into
the font table

\item Type1C support.

\item Support font reencoding of 8-bit fonts via char index.
\stopitemize

\stoptext
