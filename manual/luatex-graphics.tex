% language=uk

\environment luatex-style

\startcomponent luatex-graphics

\startchapter[reference=graphics,title={The graphic libraries}]

\section{The \type {img} library}

\topicindex {images}
\topicindex {images+library}
\topicindex {graphics}

The \type {img} library can be used as an alternative to \orm {pdfximage} and
\orm {pdfrefximage}, and the associated \quote {satellite} commands like \type
{\pdfximagebbox}. Image objects can also be used within virtual fonts via the
\type {image} command listed in~\in {section} [virtualfonts].

\subsection{\type {new}}

\startfunctioncall
<image> var = img.new()
<image> var = img.new(<table> image_spec)
\stopfunctioncall

This function creates a userdata object of type \quote {image}. The \type
{image_spec} argument is optional. If it is given, it must be a table, and that
table must contain a \type {filename} key. A number of other keys can also be
useful, these are explained below.

You can either say

\starttyping
a = img.new()
\stoptyping

followed by

\starttyping
a.filename = "foo.png"
\stoptyping

or you can put the file name (and some or all of the other keys) into a table
directly, like so:

\starttyping
a = img.new({filename='foo.pdf', page=1})
\stoptyping

The generated \type {<image>} userdata object allows access to a set of
user|-|specified values as well as a set of values that are normally filled in
and updated automatically by \LUATEX\ itself. Some of those are derived from the
actual image file, others are updated to reflect the \PDF\ output status of the
object.

There is one required user-specified field: the file name (\type {filename}). It
can optionally be augmented by the requested image dimensions (\type {width},
\type {depth}, \type {height}), user|-|specified image attributes (\type {attr}),
the requested \PDF\ page identifier (\type {page}), the requested boundingbox
(\type {pagebox}) for \PDF\ inclusion, the requested color space object (\type
{colorspace}).

The function \type {img.new} does not access the actual image file, it just
creates the \type {<image>} userdata object and initializes some memory
structures. The \type {<image>} object and its internal structures are
automatically garbage collected.

Once the image is scanned, all the values in the \type {<image>} except \type
{width}, \type {height} and \type {depth}, become frozen, and you cannot change
them any more.

You can use \type {pdf.setignoreunknownimages(1)} (or at the \TEX\ end the \lpr
{pdfvariable} \type {ignoreunknownimages}) to get around a quit when no known
image type is found (based on name or preamble). Beware: this will not catch
invalid images and we cannot guarantee side effects. A zero dimension image is
still included when requested. No special flags are set. A proper workflow will
not rely in such a catch but make sure that images are valid.

\subsection{\type {keys}}

\startfunctioncall
<table> keys = img.keys()
\stopfunctioncall

This function returns a list of all the possible \type {image_spec} keys, both
user-supplied and automatic ones.

\starttabulate[|l|l|p|]
\DB field name             \BC type     \BC description \NC \NR
\TB
\NC \type{attr}            \NC string   \NC the image attributes for \LUATEX \NC \NR
\NC \type{bbox}            \NC table    \NC table with 4 boundingbox dimensions \type
                                            {llx}, \type {lly}, \type {urx} and \type
                                            {ury} overruling the \type {pagebox} entry \NC \NR
\NC \type{colordepth}      \NC number   \NC the number of bits used by the color space \NC \NR
\NC \type{colorspace}      \NC number   \NC the color space object number \NC \NR
\NC \type{depth}           \NC number   \NC the image depth for \LUATEX \NC \NR
\NC \type{filename}        \NC string   \NC the image file name \NC \NR
\NC \type{filepath}        \NC string   \NC the full (expanded) file name of the image\NC \NR
\NC \type{height}          \NC number   \NC the image height for \LUATEX \NC \NR
\NC \type{imagetype}       \NC string   \NC one of \type {pdf}, \type {png}, \type {jpg},
                                            \type {jp2} or \type {jbig2} \NC \NR
\NC \type{index}           \NC number   \NC the \PDF\ image name suffix \NC \NR
\NC \type{objnum}          \NC number   \NC the \PDF\ image object number \NC \NR
\NC \type{page}            \NC number   \NC the identifier for the requested image page \NC \NR
\NC \type{pagebox}         \NC string   \NC the requested bounding box, one of
                                            \type {none}, \type {media}, \type {crop},
                                            \type {bleed}, \type {trim}, \type {art} \NC \NR
\NC \type{pages}           \NC number   \NC the total number of available pages \NC \NR
\NC \type{rotation}        \NC number   \NC the image rotation from included \PDF\ file,
                                            in multiples of 90~deg. \NC \NR
\NC \type{stream}          \NC string   \NC the raw stream data for an \type {/Xobject}
                                            \type {/Form} object\NC \NR
\NC \type{transform}       \NC number   \NC the image transform, integer number 0..7 \NC \NR
\NC \type{orientation}     \NC number   \NC the (jpeg) image orientation, integer number 1..8
                                            (0 for unset) \NC \NR
\NC \type{width}           \NC number   \NC the image width for \LUATEX \NC \NR
\NC \type{xres}            \NC number   \NC the horizontal natural image resolution
                                            (in \DPI) \NC \NR
\NC \type{xsize}           \NC number   \NC the natural image width \NC \NR
\NC \type{yres}            \NC number   \NC the vertical natural image resolution
                                            (in \DPI) \NC \NR
\NC \type{ysize}           \NC number   \NC the natural image height \NC \NR
\NC \type{visiblefilename} \NC string   \NC when set, this name will find its way in the
                                            \PDF\ file as \type {PTEX} specification; when
                                            an empty string is assigned nothing is written
                                            to file; otherwise the natural filename is
                                            taken \NC \NR
\NC \type{userpassword}   \NC string   \NC  the userpassword needed for opening a \PDF\ file \NC \NR
\NC \type{ownerpassword}  \NC string   \NC  the ownerpassword needed for opening a \PDF\ file \NC \NR
\LL
\stoptabulate

A running (undefined) dimension in \type {width}, \type {height}, or \type
{depth} is represented as \type {nil} in \LUA, so if you want to load an image at
its \quote {natural} size, you do not have to specify any of those three fields.

The \type {stream} parameter allows to fabricate an \type {/XObject} \type
{/Form} object from a string giving the stream contents, e.g., for a filled
rectangle:

\startfunctioncall
a.stream = "0 0 20 10 re f"
\stopfunctioncall

When writing the image, an \type {/Xobject} \type {/Form} object is created, like
with embedded \PDF\ file writing. The object is written out only once. The \type
{stream} key requires that also the \type {bbox} table is given. The \type
{stream} key conflicts with the \type {filename} key. The \type {transform} key
works as usual also with \type {stream}.

The \type {bbox} key needs a table with four boundingbox values, e.g.:

\startfunctioncall
a.bbox = { "30bp", 0, "225bp", "200bp" }
\stopfunctioncall

This replaces and overrules any given \type {pagebox} value; with given \type
{bbox} the box dimensions coming with an embedded \PDF\ file are ignored. The
\type {xsize} and \type {ysize} dimensions are set accordingly, when the image is
scaled. The \type {bbox} parameter is ignored for non-\PDF\ images.

The \type {transform} allows to mirror and rotate the image in steps of 90~deg.
The default value~$0$ gives an unmirrored, unrotated image. Values $1-3$ give
counterclockwise rotation by $90$, $180$, or $270$~degrees, whereas with values
$4-7$ the image is first mirrored and then rotated counterclockwise by $90$,
$180$, or $270$~degrees. The \type {transform} operation gives the same visual
result as if you would externally preprocess the image by a graphics tool and
then use it by \LUATEX. If a \PDF\ file to be embedded already contains a \type
{/Rotate} specification, the rotation result is the combination of the \type
{/Rotate} rotation followed by the \type {transform} operation.

\subsection{\type {scan}}

\startfunctioncall
<image> var = img.scan(<image> var)
<image> var = img.scan(<table> image_spec)
\stopfunctioncall

When you say \type {img.scan(a)} for a new image, the file is scanned, and
variables such as \type {xsize}, \type {ysize}, image \type {type}, number of
\type {pages}, and the resolution are extracted. Each of the \type {width}, \type
{height}, \type {depth} fields are set up according to the image dimensions, if
they were not given an explicit value already. An image file will never be
scanned more than once for a given image variable. With all subsequent \type
{img.scan(a)} calls only the dimensions are again set up (if they have been
changed by the user in the meantime).

For ease of use, you can do right-away a

\starttyping
<image> a = img.scan { filename = "foo.png" }
\stoptyping

without a prior \type {img.new}.

Nothing is written yet at this point, so you can do \type {a=img.scan}, retrieve
the available info like image width and height, and then throw away \type {a}
again by saying \type {a=nil}. In that case no image object will be reserved in
the PDF, and the used memory will be cleaned up automatically.

\subsection{\type {copy}}

\startfunctioncall
<image> var = img.copy(<image> var)
<image> var = img.copy(<table> image_spec)
\stopfunctioncall

If you say \type {a = b}, then both variables point to the same \type {<image>}
object. if you want to write out an image with different sizes, you can do
\type {b = img.copy(a)}.

Afterwards, \type {a} and \type {b} still reference the same actual image
dictionary, but the dimensions for \type {b} can now be changed from their
initial values that were just copies from \type {a}.

\subsection{\type {write}}

\startfunctioncall
<image> var = img.write(<image> var)
<image> var = img.write(<table> image_spec)
\stopfunctioncall

By \type {img.write(a)} a \PDF\ object number is allocated, and a whatsit node of
subtype \type {pdf_refximage} is generated and put into the output list. By this
the image \type {a} is placed into the page stream, and the image file is written
out into an image stream object after the shipping of the current page is
finished.

Again you can do a terse call like

\starttyping
img.write { filename = "foo.png" }
\stoptyping

The \type {<image>} variable is returned in case you want it for later
processing.

\subsection{\type {immediatewrite}}

\topicindex {images+immediate}

\startfunctioncall
<image> var = img.immediatewrite(<image> var)
<image> var = img.immediatewrite(<table> image_spec)
\stopfunctioncall

By \type {img.immediatewrite(a)} a \PDF\ object number is allocated, and the
image file for image \type {a} is written out immediately into the \PDF\ file as
an image stream object (like with \prm {immediate}\orm {pdfximage}). The object
number of the image stream dictionary is then available by the \type {objnum}
key. No \type {pdf_refximage} whatsit node is generated. You will need an
\type {img.write(a)} or \type {img.node(a)} call to let the image appear on the
page, or reference it by another trick; else you will have a dangling image
object in the \PDF\ file.

Also here you can do a terse call like

\starttyping
a = img.immediatewrite { filename = "foo.png" }
\stoptyping

The \type {<image>} variable is returned and you will most likely need it.

\subsection{\type {node}}

\startfunctioncall
<node> n = img.node(<image> var)
<node> n = img.node(<table> image_spec)
\stopfunctioncall

This function allocates a \PDF\ object number and returns a whatsit node of
subtype \type {pdf_refximage}, filled with the image parameters \type {width},
\type {height}, \type {depth}, and \type {objnum}. Also here you can do a terse
call like:

\starttyping
n = img.node ({ filename = "foo.png" })
\stoptyping

This example outputs an image:

\starttyping
node.write(img.node{filename="foo.png"})
\stoptyping

\subsection{\type {types}}

\topicindex {images+types}

\startfunctioncall
<table> types = img.types()
\stopfunctioncall

This function returns a list with the supported image file type names, currently
these are \type {pdf}, \type {png}, \type {jpg}, \type {jp2} (JPEG~2000), and
\type {jbig2}.

\subsection{\type {boxes}}

\startfunctioncall
<table> boxes = img.boxes()
\stopfunctioncall

This function returns a list with the supported \PDF\ page box names, currently
these are \type {media}, \type {crop}, \type {bleed}, \type {trim}, and \type
{art}, all in lowercase.

\section{The \type {mplib} library}

\topicindex {\METAPOST}
\topicindex {\METAPOST+mplib}
\topicindex {images+mplib}
\topicindex {images+\METAPOST}

The \MP\ library interface registers itself in the table \type {mplib}. It is
based on \MPLIB\ version \ctxlua {context(mplib.version())}.

\subsection{\type {new}}

To create a new \METAPOST\ instance, call

\startfunctioncall
<mpinstance> mp = mplib.new({...})
\stopfunctioncall

This creates the \type {mp} instance object. The argument hash can have a number
of different fields, as follows:

\starttabulate[|l|l|pl|pl|]
\DB name               \BC type     \BC description              \BC default           \NC \NR
\TB
\NC \type{error_line}  \NC number   \NC error line width         \NC 79                \NC \NR
\NC \type{print_line}  \NC number   \NC line length in ps output \NC 100               \NC \NR
\NC \type{random_seed} \NC number   \NC the initial random seed  \NC variable          \NC \NR
\NC \type{math_mode}   \NC string   \NC the number system to use:
                                        \type {double},
                                        \type {scaled},
                                        \type {binary} or
                                        \type {decimal}          \NC \type {scaled}    \NC \NR
\NC \type{interaction} \NC string   \NC the interaction mode:
                                        \type {batch},
                                        \type {nonstop},
                                        \type {scroll} or
                                        \type {errorstop}        \NC \type {errorstop} \NC \NR
\NC \type{job_name}    \NC string   \NC \type {--jobname}        \NC \type {mpout}     \NC \NR
\NC \type{find_file}   \NC function \NC a function to find files \NC only local files  \NC \NR
\LL
\stoptabulate

The \type {find_file} function should be of this form:

\starttyping
<string> found = finder (<string> name, <string> mode, <string> type)
\stoptyping

with:

\starttabulate[|l|p|]
\DB name        \BC the requested file \NC \NR
\TB
\NC \type{mode} \NC the file mode: \type {r} or \type {w} \NC \NR
\NC \type{type} \NC the kind of file, one of: \type {mp}, \type {tfm}, \type {map},
                    \type {pfb}, \type {enc} \NC \NR
\LL
\stoptabulate

Return either the full path name of the found file, or \type {nil} if the file
cannot be found.

Note that the new version of \MPLIB\ no longer uses binary mem files, so the way
to preload a set of macros is simply to start off with an \type {input} command
in the first \type {mp:execute()} call.

The \PDF\ file is kept open after its properties are determined. After inclusion,
which happens when the page that references the image is flushed, the file is
closed. This means that when you have thousands of images on one page, your
operating system might decide to abort the run. When you include more than one
page from a \PDF\ file you can set the \type {keepopen} flag when you allocate an
image object, or pass the \type {keepopen} directive when you refer to the image
with \lpr {useimageresource}. This only makes sense when you embed many pages.
An \prm {immediate} applied to \lpr {saveimageresource} will also force a
close after inclusion.

\starttyping
\immediate\useimageresource{foo.pdf}%
          \saveimageresource         \lastsavedimageresourceindex % closed
          \useimageresource{foo.pdf}%
          \saveimageresource         \lastsavedimageresourceindex % kept open
          \useimageresource{foo.pdf}%
          \saveimageresource keepopen\lastsavedimageresourceindex % kept open

\directlua{img.write(img.scan{ file = "foo.pdf" })}                  % closed
\directlua{img.write(img.scan{ file = "foo.pdf", keepopen = true })} % kept open
\stoptyping

\subsection{\type {mp:statistics}}

You can request statistics with:

\startfunctioncall
<table> stats = mp:statistics()
\stopfunctioncall

This function returns the vital statistics for an \MPLIB\ instance. There are
four fields, giving the maximum number of used items in each of four allocated
object classes:

\starttabulate[|l|l|p|]
\DB field  \BC type \BC explanation \NC \NR
\TB
\NC \type{main_memory} \NC number \NC memory size \NC \NR
\NC \type{hash_size}   \NC number \NC hash size\NC \NR
\NC \type{param_size}  \NC number \NC simultaneous macro parameters\NC \NR
\NC \type{max_in_open} \NC number \NC input file nesting levels\NC \NR
\LL
\stoptabulate

Note that in the new version of \MPLIB, this is informational only. The objects
are all allocated dynamically, so there is no chance of running out of space
unless the available system memory is exhausted.

\subsection{\type {mp:execute}}

You can ask the \METAPOST\ interpreter to run a chunk of code by calling

\startfunctioncall
<table> rettable = mp:execute('metapost language chunk')
\stopfunctioncall

for various bits of \METAPOST\ language input. Be sure to check the \type
{rettable.status} (see below) because when a fatal \METAPOST\ error occurs the
\MPLIB\ instance will become unusable thereafter.

Generally speaking, it is best to keep your chunks small, but beware that all
chunks have to obey proper syntax, like each of them is a small file. For
instance, you cannot split a single statement over multiple chunks.

In contrast with the normal stand alone \type {mpost} command, there is
\notabene {no} implied \quote{input} at the start of the first chunk.

\subsection{\type {mp:finish}}

\startfunctioncall
<table> rettable = mp:finish()
\stopfunctioncall

If for some reason you want to stop using an \MPLIB\ instance while processing is
not yet actually done, you can call \type {mp:finish}. Eventually, used memory
will be freed and open files will be closed by the \LUA\ garbage collector, but
an explicit \type {mp:finish} is the only way to capture the final part of the
output streams.

\subsection{Result table}

The return value of \type {mp:execute} and \type {mp:finish} is a table with a
few possible keys (only \type {status} is always guaranteed to be present).

\starttabulate[|l|l|p|]
\DB field  \BC type \BC explanation \NC \NR
\TB
\NC \type{log}    \NC string \NC output to the \quote {log} stream \NC \NR
\NC \type{term}   \NC string \NC output to the \quote {term} stream \NC \NR
\NC \type{error}  \NC string \NC output to the \quote {error} stream
                                 (only used for \quote {out of memory}) \NC \NR
\NC \type{status} \NC number \NC the return value:
                                 \type {0} = good,
                                 \type {1} = warning,
                                 \type {2} = errors,
                                 \type {3} = fatal error \NC \NR
\NC \type{fig}    \NC table  \NC an array of generated figures (if any) \NC \NR
\LL
\stoptabulate

When \type {status} equals~3, you should stop using this \MPLIB\ instance
immediately, it is no longer capable of processing input.

If it is present, each of the entries in the \type {fig} array is a userdata
representing a figure object, and each of those has a number of object methods
you can call:

\starttabulate[|l|l|p|]
\DB field  \BC type \BC explanation \NC \NR
\TB
\NC \type{boundingbox}  \NC function \NC returns the bounding box, as an array of 4
                                         values \NC \NR
\NC \type{postscript}   \NC function \NC returns a string that is the ps output of the
                                         \type {fig}. this function accepts two optional
                                         integer arguments for specifying the values of
                                         \type {prologues} (first argument) and \type
                                         {procset} (second argument) \NC \NR
\NC \type{svg}          \NC function \NC returns a string that is the svg output of the
                                         \type {fig}. This function accepts an optional
                                         integer argument for specifying the value of
                                         \type {prologues} \NC \NR
\NC \type{objects}      \NC function \NC returns the actual array of graphic objects in
                                         this \type {fig} \NC \NR
\NC \type{copy_objects} \NC function \NC returns a deep copy of the array of graphic
                                         objects in this \type {fig} \NC \NR
\NC \type{filename}     \NC function \NC the filename this \type {fig}'s \POSTSCRIPT\
                                         output would have written to in stand alone
                                         mode \NC \NR
\NC \type{width}        \NC function \NC the \type {fontcharwd} value \NC \NR
\NC \type{height}       \NC function \NC the \type {fontcharht} value \NC \NR
\NC \type{depth}        \NC function \NC the \type {fontchardp} value \NC \NR
\NC \type{italcorr}     \NC function \NC the \type {fontcharit} value \NC \NR
\NC \type{charcode}     \NC function \NC the (rounded) \type {charcode} value \NC \NR
\LL
\stoptabulate

Note: you can call \type {fig:objects()} only once for any one \type {fig}
object!

When the boundingbox represents a \quote {negated rectangle}, i.e.\ when the
first set of coordinates is larger than the second set, the picture is empty.

Graphical objects come in various types that each has a different list of
accessible values. The types are: \type {fill}, \type {outline}, \type {text},
\type {start_clip}, \type {stop_clip}, \type {start_bounds}, \type {stop_bounds},
\type {special}.

There is a helper function (\type {mplib.fields(obj)}) to get the list of
accessible values for a particular object, but you can just as easily use the
tables given below.

All graphical objects have a field \type {type} that gives the object type as a
string value; it is not explicit mentioned in the following tables. In the
following, \type {number}s are \POSTSCRIPT\ points represented as a floating
point number, unless stated otherwise. Field values that are of type \type
{table} are explained in the next section.

\subsubsection{fill}

\starttabulate[|l|l|p|]
\DB field  \BC type \BC explanation \NC \NR
\TB
\NC \type{path}       \NC table  \NC the list of knots \NC \NR
\NC \type{htap}       \NC table  \NC the list of knots for the reversed trajectory \NC \NR
\NC \type{pen}        \NC table  \NC knots of the pen \NC \NR
\NC \type{color}      \NC table  \NC the object's color \NC \NR
\NC \type{linejoin}   \NC number \NC line join style (bare number)\NC \NR
\NC \type{miterlimit} \NC number \NC miterlimit\NC \NR
\NC \type{prescript}  \NC string \NC the prescript text \NC \NR
\NC \type{postscript} \NC string \NC the postscript text \NC \NR
\LL
\stoptabulate

The entries \type {htap} and \type {pen} are optional.

There is helper function (\type {mplib.pen_info(obj)}) that returns a table
containing a bunch of vital characteristics of the used pen (all values are
floats):

\starttabulate[|l|l|p|]
\DB field  \BC type \BC explanation \NC \NR
\TB
\NC \type{width} \NC number \NC width of the pen \NC \NR
\NC \type{sx}    \NC number \NC $x$ scale        \NC \NR
\NC \type{rx}    \NC number \NC $xy$ multiplier  \NC \NR
\NC \type{ry}    \NC number \NC $yx$ multiplier  \NC \NR
\NC \type{sy}    \NC number \NC $y$ scale        \NC \NR
\NC \type{tx}    \NC number \NC $x$ offset       \NC \NR
\NC \type{ty}    \NC number \NC $y$ offset       \NC \NR
\LL
\stoptabulate

\subsubsection{outline}

\starttabulate[|l|l|p|]
\DB field  \BC type \BC explanation \NC \NR
\TB
\NC \type{path}       \NC table  \NC the list of knots \NC \NR
\NC \type{pen}        \NC table  \NC knots of the pen \NC \NR
\NC \type{color}      \NC table  \NC the object's color \NC \NR
\NC \type{linejoin}   \NC number \NC line join style (bare number) \NC \NR
\NC \type{miterlimit} \NC number \NC miterlimit \NC \NR
\NC \type{linecap}    \NC number \NC line cap style (bare number) \NC \NR
\NC \type{dash}       \NC table  \NC representation of a dash list \NC \NR
\NC \type{prescript}  \NC string \NC the prescript text \NC \NR
\NC \type{postscript} \NC string \NC the postscript text \NC \NR
\LL
\stoptabulate

The entry \type {dash} is optional.

\subsubsection{text}

\starttabulate[|l|l|p|]
\DB field  \BC type \BC explanation \NC \NR
\TB
\NC \type{text}       \NC string \NC the text \NC \NR
\NC \type{font}       \NC string \NC font tfm name \NC \NR
\NC \type{dsize}      \NC number \NC font size \NC \NR
\NC \type{color}      \NC table  \NC the object's color \NC \NR
\NC \type{width}      \NC number \NC \NC \NR
\NC \type{height}     \NC number \NC \NC \NR
\NC \type{depth}      \NC number \NC \NC \NR
\NC \type{transform}  \NC table  \NC a text transformation \NC \NR
\NC \type{prescript}  \NC string \NC the prescript text \NC \NR
\NC \type{postscript} \NC string \NC the postscript text \NC \NR
\LL
\stoptabulate

\subsubsection{special}

\starttabulate[|l|l|p|]
\DB field  \BC type \BC explanation \NC \NR
\TB
\NC \type{prescript} \NC string \NC special text \NC \NR
\LL
\stoptabulate

\subsubsection{start_bounds, start_clip}

\starttabulate[|l|l|p|]
\DB field  \BC type \BC explanation \NC \NR
\TB
\NC \type{path} \NC table \NC the list of knots \NC \NR
\LL
\stoptabulate

\subsubsection{stop_bounds, stop_clip}

Here are no fields available.

\subsection{Subsidiary table formats}

\subsubsection{Paths and pens}

Paths and pens (that are really just a special type of paths as far as \MPLIB\ is
concerned) are represented by an array where each entry is a table that
represents a knot.

\starttabulate[|l|l|p|]
\DB field  \BC type \BC explanation \NC \NR
\TB
\NC \type{left_type}  \NC string \NC when present: endpoint, but usually absent \NC \NR
\NC \type{right_type} \NC string \NC like \type {left_type} \NC \NR
\NC \type{x_coord}    \NC number \NC X coordinate of this knot \NC \NR
\NC \type{y_coord}    \NC number \NC Y coordinate of this knot \NC \NR
\NC \type{left_x}     \NC number \NC X coordinate of the precontrol point of this knot \NC \NR
\NC \type{left_y}     \NC number \NC Y coordinate of the precontrol point of this knot \NC \NR
\NC \type{right_x}    \NC number \NC X coordinate of the postcontrol point of this knot \NC \NR
\NC \type{right_y}    \NC number \NC Y coordinate of the postcontrol point of this knot \NC \NR
\LL
\stoptabulate

There is one special case: pens that are (possibly transformed) ellipses have an
extra string-valued key \type {type} with value \type {elliptical} besides the
array part containing the knot list.

\subsubsection{Colors}

A color is an integer array with 0, 1, 3 or 4 values:

\starttabulate[|l|l|p|]
\DB field  \BC type \BC explanation \NC \NR
\TB
\NC \type{0} \NC marking only \NC no values                                                     \NC \NR
\NC \type{1} \NC greyscale    \NC one value in the range $(0,1)$, \quote {black} is $0$         \NC \NR
\NC \type{3} \NC \RGB         \NC three values in the range $(0,1)$, \quote {black} is $0,0,0$  \NC \NR
\NC \type{4} \NC \CMYK        \NC four values in the range $(0,1)$, \quote {black} is $0,0,0,1$ \NC \NR
\LL
\stoptabulate

If the color model of the internal object was \type {uninitialized}, then it was
initialized to the values representing \quote {black} in the colorspace \type
{defaultcolormodel} that was in effect at the time of the \type {shipout}.

\subsubsection{Transforms}

Each transform is a six|-|item array.

\starttabulate[|l|l|p|]
\DB index  \BC type \BC explanation \NC \NR
\TB
\NC \type{1} \NC number \NC represents x  \NC \NR
\NC \type{2} \NC number \NC represents y  \NC \NR
\NC \type{3} \NC number \NC represents xx \NC \NR
\NC \type{4} \NC number \NC represents yx \NC \NR
\NC \type{5} \NC number \NC represents xy \NC \NR
\NC \type{6} \NC number \NC represents yy \NC \NR
\LL
\stoptabulate

Note that the translation (index 1 and 2) comes first. This differs from the
ordering in \POSTSCRIPT, where the translation comes last.

\subsubsection{Dashes}

Each \type {dash} is two-item hash, using the same model as \POSTSCRIPT\ for the
representation of the dashlist. \type {dashes} is an array of \quote {on} and
\quote {off}, values, and \type {offset} is the phase of the pattern.

\starttabulate[|l|l|p|]
\DB field  \BC type \BC explanation \NC \NR
\TB
\NC \type{dashes} \NC hash   \NC an array of on-off numbers \NC \NR
\NC \type{offset} \NC number \NC the starting offset value  \NC \NR
\LL
\stoptabulate

\subsection{Character size information}

These functions find the size of a glyph in a defined font. The \type {fontname}
is the same name as the argument to \type {infont}; the \type {char} is a glyph
id in the range 0 to 255; the returned \type {w} is in AFM units.

\subsubsection{\type {mp:char_width}}

\startfunctioncall
<number> w = mp:char_width(<string> fontname, <number> char)
\stopfunctioncall

\subsubsection{\type {mp:char_height}}

\startfunctioncall
<number> w = mp:char_height(<string> fontname, <number> char)
\stopfunctioncall

\subsubsection{\type {mp:char_depth}}

\startfunctioncall
<number> w = mp:char_depth(<string> fontname, <number> char)
\stopfunctioncall

\stopchapter
