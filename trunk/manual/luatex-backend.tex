% language=uk

\environment luatex-style

\startcomponent luatex-backend

\startchapter[reference=backend,title={The backend libraries}]

\section{The \type {pdf} library}

\topicindex{backend}
\topicindex{\PDF}

This library contains variables and functions that are related to the \PDF\
backend. You can find more details about the expected values to setters in \in
{section} [backendprimitives].

\subsection{\type {mapfile}, \type {mapline}}

\topicindex{map files}
\topicindex{files+map}

\startfunctioncall
pdf.mapfile(<string> map file)
pdf.mapline(<string> map line)
\stopfunctioncall

These two functions can be used to replace primitives \orm {pdfmapfile} and
\orm {pdfmapline} inherited from \PDFTEX. They expect a string as only
parameter and have no return value. The first character in a map line can be
\type {-}, \type {+} or \type {=} which means as much as remove, add or replace
this line.

\subsection {\type {[set|get][catalog|info|names|trailer]}}

\topicindex{\PDF+trailer}
\topicindex{\PDF+catalog}
\topicindex{\PDF+info}

These functions complement the corresponding \PDF\ backend token lists dealing
with metadata. The value types are strings and they are written out to the \PDF\
file directly after the token registers.

\subsection {\type {[set|get][pageattributes|pageresources|pagesattributes]}}

\topicindex{\PDF+page attributes}
\topicindex{\PDF+page resources}

These functions complement the corresponding \PDF\ backend token lists dealing
with page resources. The variables have no interaction with the corresponding \PDF\
backend token register. They are written out to the \PDF\ file directly after the
token registers.

\subsection{\type {[set|get][xformattributes|xformresources]}}

\topicindex{\PDF+xform attributes}
\topicindex{\PDF+xform resources}

These functions complement the corresponding \PDF\ backend token lists dealing
with reuseable boxes and images. The variables have no interaction with the
corresponding \PDF\ backend token register. They are written out to the \PDF\
file directly after the token registers.

\subsection{\type {getversion} and \type {[set|get]minorversion}}

\topicindex{\PDF+version}

The version is frozen in the binary but you can set the minor version. What minor
version you set depends on what \PDF\ features you use. This is out of control
of \LUATEX.

\subsection{\type {getcreationdate}}

\topicindex{\PDF+date}

This function returns a string with the date in the format that ends up in the
\PDF\ file, in this case it's: {\tttf \cldcontext{pdf.getcreationdate()}}.

\subsection{\type {[set|get]inclusionerrorlevel}, \type {[set|get]ignoreunknownimages}}

\topicindex{\PDF+options}

These variable control how error in included image are treated. They are modeled
after the \PDFTEX\ equivalents.

\subsection{\type {[set|get]suppressoptionalinfo}}

\topicindex{\PDF+options}

This bitset determines what kind of info gets flushes. By default we flush all.

\subsection{\type {[set|get]trailerid}}

\topicindex{\PDF+trailer}

You can set your own trailer id. This has to be a valid array string
with checksums.

\subsection{\type {[set|get]compresslevel}}

\topicindex{\PDF+compression}

These two functions set the level of compression. The minimum value is~0,
the maximum is~9.

\subsection{\type {[set|get]objcompresslevel}}

\topicindex{\PDF+compression}

These two functions set the level of compression. The minimum value is~0,
the maximum is~9.

\subsection{\type {[set|get]gentounicode}}

\topicindex{\PDF+unicode}

This flag enables tounicode generation (like in \PDFTEX).

\subsection{\type {[set|get]omitcidset}}

\topicindex{\PDF+options}

This flag disables inclusion of a so called CIDSet which can be handy when aiming
at some of the many \PDF\ substandards.

\subsection{\type {[set|get]decimaldigits}}

\topicindex{\PDF+precision}

These two functions set the accuracy of floats written to the \PDF file. You can
set any value but the backend will not go below~3 and above~6.

\subsection{\type {[set|get]pkresolution}}

\topicindex{\PDF+resolution}

These setter takes two arguments: the resolution and an optional zero or one that
indicates if this is a fixed one. The getter returns these two values.

\subsection{\type {getlast[obj|link|annot]} and \type {getretval}}

\topicindex{\PDF+objects}
\topicindex{\PDF+annotations}

These status variables are similar to the ones traditionally used in the backend
interface at the \TEX\ end.

\subsection{\type {maxobjnum} and \type {objtype}, \type {fontname}, \type {fontobjnum},
\type {fontsize}, \type {xformname}}.

These (and some other) introspective helpers were moved from the the \type {tex}
namespace to the \type {pdf} namespace but kept their original names. They are
mostly used when you construct \PDF\ objects yourself and need for instance
information about a (to be) embedded font.

\subsection{\type {[set|get]origin}}

\topicindex{\PDF+positioning}

This one is used to set the horizonal and|/|or vertical offset, a traditional
backend property.

\starttyping
pdf.setorigin() -- sets both to 0pt
pdf.setorigin(tex.sp("1in")) -- sets both to 1in
pdf.setorigin(tex.sp("1in"),tex.sp("1in"))
\stoptyping

The counterpart of this function returns two values.

\subsection{\type {[set|get]imageresolution}}

\topicindex{\PDF+resolution}

These two functions relate to the imageresolution that is used when the image
itself doesn't provide a non|-|zero x or y resolution.

\subsection {\type {[set|get][link|dest|thread|xform]margin}}

\topicindex{\PDF+margins}

These functions can be used to set and retrieve the margins that are added to the
natural bounding boxes of the respective objects.

\subsection {\type {get[pos|hpos|vpos]}}

\topicindex{\PDF+positions}

These function get current location on the output page, measured from its lower
left corner. The values return scaled points as units.

\starttyping
local h, v = pdf.getpos()
\stoptyping

\subsection {\type {[has|get]matrix}}

\topicindex{\PDF+matrix}

The current matrix transformation is available via the \type {getmatrix} command,
which returns 6 values: \type {sx}, \type {rx}, \type {ry}, \type {sy}, \type
{tx}, and \type {ty}. The \type {hasmatrix} function returns \type {true} when a
matrix is applied.

\starttyping
if pdf.hasmatrix() then
    local sx, rx, ry, sy, tx, ty = pdf.getmatrix()
    -- do something useful or not
end
\stoptyping

\subsection {\type {print}}

\topicindex{\PDF+print to}

You can print string to the \PDF\ document from within a within a \lpr {latelua}
call. This function is not to be used inside \prm {directlua} unless you know
{\it exactly} what you are doing.

\startfunctioncall
pdf.print(<string> s)
pdf.print(<string> type, <string> s)
\stopfunctioncall

The optional parameter can be used to mimic the behavior of \PDF\ literals: the
\type {type} is \type {direct} or \type {page}.

\subsection {\type {immediateobj}}

\topicindex{\PDF+objects}

This function creates a \PDF\ object and immediately writes it to the \PDF\ file.
It is modelled after \PDFTEX's \prm {immediate} \orm {pdfobj} primitives. All
function variants return the object number of the newly generated object.

\startfunctioncall
<number> n =
    pdf.immediateobj(<string> objtext)
<number> n =
    pdf.immediateobj("file", <string> filename)
<number> n =
    pdf.immediateobj("stream", <string> streamtext, <string> attrtext)
<number> n =
    pdf.immediateobj("streamfile", <string> filename, <string> attrtext)
\stopfunctioncall

The first version puts the \type {objtext} raw into an object. Only the object
wrapper is automatically generated, but any internal structure (like \type {<<
>>} dictionary markers) needs to provided by the user. The second version with
keyword \type {file} as first argument puts the contents of the file with name
\type {filename} raw into the object. The third version with keyword \type
{stream} creates a stream object and puts the \type {streamtext} raw into the
stream. The stream length is automatically calculated. The optional \type
{attrtext} goes into the dictionary of that object. The fourth version with
keyword \type {streamfile} does the same as the third one, it just reads the
stream data raw from a file.

An optional first argument can be given to make the function use a previously
reserved \PDF\ object.

\startfunctioncall
<number> n =
    pdf.immediateobj(<integer> n, <string> objtext)
<number> n =
    pdf.immediateobj(<integer> n, "file", <string> filename)
<number> n =
    pdf.immediateobj(<integer> n, "stream", <string> streamtext, <string> attrtext)
<number> n =
    pdf.immediateobj(<integer> n, "streamfile", <string> filename, <string> attrtext)
\stopfunctioncall

\subsection{\type{obj}}

\topicindex{\PDF+objects}

This function creates a \PDF\ object, which is written to the \PDF\ file only
when referenced, e.g., by \type {refobj()}.

All function variants return the object number of the newly generated object, and
there are two separate calling modes. The first mode is modelled after \PDFTEX's
\orm {pdfobj} primitive.

\startfunctioncall
<number> n =
    pdf.obj(<string> objtext)
<number> n =
    pdf.obj("file", <string> filename)
<number> n =
    pdf.obj("stream", <string> streamtext, <string> attrtext)
<number> n =
    pdf.obj("streamfile", <string> filename, <string> attrtext)
\stopfunctioncall

An optional first argument can be given to make the function use a previously
reserved \PDF\ object.

\startfunctioncall
<number> n =
    pdf.obj(<integer> n, <string> objtext)
<number> n =
    pdf.obj(<integer> n, "file", <string> filename)
<number> n =
    pdf.obj(<integer> n, "stream", <string> streamtext, <string> attrtext)
<number> n =
    pdf.obj(<integer> n, "streamfile", <string> filename, <string> attrtext)
\stopfunctioncall

The second mode accepts a single argument table with key--value pairs.

\startfunctioncall
<number> n = pdf.obj {
    type           = <string>,
    immmediate     = <boolean>,
    objnum         = <number>,
    attr           = <string>,
    compresslevel  = <number>,
    objcompression = <boolean>,
    file           = <string>,
    string         = <string>
}
\stopfunctioncall

The \type {type} field can have the values \type {raw} and \type {stream}, this
field is required, the others are optional (within constraints).

Note: this mode makes \type{obj} look more flexible than it actually is: the
constraints from the separate parameter version still apply, so for example you
can't have both \type {string} and \type {file} at the same time.

\subsection {\type {refobj}}

\topicindex{\PDF+objects}

This function, the \LUA\ version of the \orm {pdfrefobj} primitive, references an
object by its object number, so that the object will be written out.

\startfunctioncall
pdf.refobj(<integer> n)
\stopfunctioncall

This function works in both the \prm {directlua} and \lpr {latelua} environment.
Inside \prm {directlua} a new whatsit node \quote {pdf_refobj} is created, which
will be marked for flushing during page output and the object is then written
directly after the page, when also the resources objects are written out. Inside
\lpr {latelua} the object will be marked for flushing.

This function has no return values.

\subsection {\type {reserveobj}}

\topicindex{\PDF+objects}

This function creates an empty \PDF\ object and returns its number.

\startfunctioncall
<number> n = pdf.reserveobj()
<number> n = pdf.reserveobj("annot")
\stopfunctioncall

\subsection {\type {registerannot}}

This function adds an object number to the \type {/Annots} array for the current
page without doing anything else. This function can only be used from within
\lpr {latelua}.

\startfunctioncall
pdf.registerannot (<number> objnum)
\stopfunctioncall

\subsection {\type {newcolorstack}}

\topicindex{\PDF+color stack}

This function allocates a new color stack and returns it's id. The arguments
are the same as for the similar backend extension primitive.

\startfunctioncall
pdf.newcolorstack("0 g","page",true) -- page|direct|origin
\stopfunctioncall

\subsection {\type {setfontattributes}}

\topicindex{\PDF+fonts}

This function will force some additional code into the font resource. It can for
instance be used to add a custom \type {ToUnicode} vector to a bitmap file.

\startfunctioncall
pdf.setfontattributes(<number> font id, <string> pdf code)
\stopfunctioncall

\section {The \type {pdfscanner} library}

\topicindex{\PDF+scanner}

The \type {pdfscanner} library allows interpretation of \PDF\ content streams and
\type {/ToUnicode} (cmap) streams. You can get those streams from the \type
{epdf} library, as explained in an earlier section. There is only a single
top|-|level function in this library:

\startfunctioncall
pdfscanner.scan (<Object> stream, <table> operatortable, <table> info)
\stopfunctioncall

The first argument, \type {stream}, should be either a \PDF\ stream object, or a
\PDF\ array of \PDF\ stream objects (those options comprise the possible return
values of \type {<Page>:getContents()} and \type {<Object>:getStream()} in the
\type {epdf} library).

The second argument, \type {operatortable}, should be a \LUA\ table where the
keys are \PDF\ operator name strings and the values are \LUA\ functions (defined
by you) that are used to process those operators. The functions are called
whenever the scanner finds one of these \PDF\ operators in the content stream(s).
The functions are called with two arguments: the \type {scanner} object itself,
and the \type {info} table that was passed are the third argument to \type
{pdfscanner.scan}.

Internally, \type {pdfscanner.scan} loops over the \PDF\ operators in the
stream(s), collecting operands on an internal stack until it finds a \PDF\
operator. If that \PDF\ operator's name exists in \type {operatortable}, then the
associated function is executed. After the function has run (or when there is no
function to execute) the internal operand stack is cleared in preparation for the
next operator, and processing continues.

The \type {scanner} argument to the processing functions is needed because it
offers various methods to get the actual operands from the internal operand
stack.

A simple example of processing a \PDF's document stream could look like this:

\starttyping
function Do (scanner, info)
   local val       = scanner:pop()
   local name      = val[2] -- val[1] == 'name'
   local resources = info.resources
   local xobject   = resources:lookup("XObject"):getDict():lookup(name)
   print (info.space ..'Use XObject '.. name)
   if xobject and xobject:isStream() then
      local dict = xobject:getStream():getDict()
      if dict then
        local name = dict:lookup("Subtype")
        if name:getName() == "Form" then
          local newinfo =  {
            space = info.space .. "  " ,
            resources = dict:lookup("Resources"):getDict()
          }
          pdfscanner.scan(xobject, operatortable, newinfo)
        end
      end
   end
end

operatortable = { Do = Do }

doc      = epdf.open(arg[1])
pagenum  = 1

while pagenum <= doc:getNumPages() do
   local page = doc:getCatalog():getPage(pagenum)
   local info = {
     space     = "  " ,
     resources = page:getResourceDict()
   }
   print('Page ' .. pagenum)
   pdfscanner.scan(page:getContents(), operatortable, info)
   pagenum = pagenum + 1
end
\stoptyping

This example iterates over all the actual content in the \PDF, and prints out the
found \type {XObject} names. While the code demonstrates quite some of the \type
{epdf} functions, let's focus on the type \type {pdfscanner} specific code
instead.

From the bottom up, the following line runs the scanner with the \PDF\ page's
top-level content.

\starttyping
   pdfscanner.scan(page:getContents(), operatortable, info)
\stoptyping

The third argument, \type {info}, contains two entries: \type {space} is used to
indent the printed output, and \type {resources} is needed so that embedded \type
{XForms} can find their own content.

The second argument, \type {operatortable} defines a processing function for a
single \PDF\ operator, \type {Do}.

The function \type {Do} prints the name of the current \type {XObject}, and then
starts a new scanner for that object's content stream, under the condition that
the \type {XObject} is in fact a \type {/Form}. That nested scanner is called
with new \type {info} argument with an updated \type {space} value so that the
indentation of the output nicely nests, and with an new \type {resources} field
to help the next iteration down to properly process any other, embedded \type
{XObject}s.

Of course, this is not a very useful example in practise, but for the purpose of
demonstrating \type {pdfscanner}, it is just long enough. It makes use of only
one \type {scanner} method: \type {scanner:pop()}. That function pops the top
operand of the internal stack, and returns a \LUA\ table where the object at index
one is a string representing the type of the operand, and object two is its
value.

The list of possible operand types and associated \LUA\ value types is:

\starttabulate[|l|c|]
\DB types           \BC type      \NC \NR
\TB
\NC \type{integer}  \NC <number>  \NC \NR
\NC \type{real}     \NC <number>  \NC \NR
\NC \type{boolean}  \NC <boolean> \NC \NR
\NC \type{name}     \NC <string>  \NC \NR
\NC \type{operator} \NC <string>  \NC \NR
\NC \type{string}   \NC <string>  \NC \NR
\NC \type{array}    \NC <table>   \NC \NR
\NC \type{dict}     \NC <table>   \NC \NR
\LL
\stoptabulate

In case of \type {integer} or \type {real}, the value is always a \LUA\ (floating
point) number. In case of \type {name}, the leading slash is always stripped.

In case of \type {string}, please bear in mind that \PDF\ actually supports
different types of strings (with different encodings) in different parts of the
\PDF\ document, so may need to reencode some of the results; \type {pdfscanner}
always outputs the byte stream without reencoding anything. \type {pdfscanner}
does not differentiate between literal strings and hexadecimal strings (the
hexadecimal values are decoded), and it treats the stream data for inline images
as a string that is the single operand for \type {EI}.

In case of \type {array}, the table content is a list of \type {pop} return
values and in case of \type {dict}, the table keys are \PDF\ name strings and the
values are \type {pop} return values.

There are few more methods defined that you can ask \type {scanner}:

\starttabulate[|l|p|]
\DB method           \BC explanation \NC \NR
\TB
\NC \type{pop}       \NC see above \NC \NR
\NC \type{popNumber} \NC return only the value of a \type {real} or \type {integer} \NC \NR
\NC \type{popName}   \NC return only the value of a \type {name} \NC \NR
\NC \type{popString} \NC return only the value of a \type {string} \NC \NR
\NC \type{popArray}  \NC return only the value of a \type {array} \NC \NR
\NC \type{popDict}   \NC return only the value of a \type {dict} \NC \NR
\NC \type{popBool}   \NC return only the value of a \type {boolean} \NC \NR
\NC \type{done}      \NC abort further processing of this \type {scan()} call \NC \NR
\LL
\stoptabulate

The \type {popXXX} are convenience functions, and come in handy when you know the
type of the operands beforehand (which you usually do, in \PDF). For example, the
\type {Do} function could have used \type {local name = scanner:popName()}
instead, because the single operand to the \type {Do} operator is always a \PDF\
name object.

The \type {done} function allows you to abort processing of a stream once you
have learned everything you want to learn. This comes in handy while parsing
\type {/ToUnicode}, because there usually is trailing garbage that you are not
interested in. Without \type {done}, processing only end at the end of the
stream, possibly wasting \CPU\ cycles.

\section{The \type {epdf} library}

\topicindex{\PDF+analyze}
\topicindex{\PDF+\type{epdf}}

The \type {epdf} library provides \LUA\ bindings to many \PDF\ access functions
that are defined by the poppler \PDF\ viewer library (written in C$+{}+$ by
Kristian H\o gsberg, based on xpdf by Derek Noonburg). Within \LUATEX\ xpdf
functionality is being used since long time to embed \PDF\ files. The \type
{epdf} library allows to scrutinize an external \PDF\ file. It gives access to
its document structure: catalog, cross|-|reference table, individual pages,
objects, annotations, info, and metadata. The \type {epdf} library only provides
read|-|only access. At some point we might decide to limit the interface to a
reasonable subset.

For a start, a \PDF\ file is opened by \type {epdf.open()} with file name, e.g.:

\starttyping
doc = epdf.open("foo.pdf")
\stoptyping

This normally returns a \type {PDFDoc} userdata variable; but if the file could
not be opened successfully, instead of a fatal error just the value \type {nil} is
returned.

All \LUA\ functions in the \type {epdf} library are named after the poppler
functions listed in the poppler header files for the various classes, e.g., files
\type {PDFDoc.h}, \type {Dict.h}, and \type {Array.h}. These files can be found
in the poppler subdirectory within the \LUATEX\ sources. Which functions are
already implemented in the \type {epdf} library can be found in the \LUATEX\
source file \type {lepdflib.cc}. For using the \type {epdf} library, knowledge of
the \PDF\ file architecture is indispensable.

There are many different userdata types defined by the \type {epdf} library,
currently these are \type {AnnotBorderStyle}, \type {AnnotBorder}, \type
{Annots}, \type {Annot}, \type {Array}, \type {Attribute}, \type {Catalog}, \type
{Dict}, \type {EmbFile}, \type {GString}, \type {LinkDest}, \type {Links}, \type
{Link}, \type {ObjectStream}, \type {Object}, \type {PDFDoc}, \type
{PDFRectangle}, \type {Page}, \type {Ref}, \type {Stream}, \type {StructElement},
\type {StructTreeRoot} \type {TextSpan}, \type {XRefEntry} and \type {XRef}.

All these userdata names and the \LUA\ access functions closely resemble the
classes naming from the poppler header files, including the choice of mixed upper
and lower case letters. The \LUA\ function calls use object|-|oriented syntax,
e.g., the following calls return the \type {Page} object for page~1:

\starttyping
pageref = doc:getCatalog():getPageRef(1)
pageobj = doc:getXRef():fetch(pageref.num, pageref.gen)
\stoptyping

But writing such chained calls is risky, as an intermediate function may return
\type {nil} on error. Therefore between function calls there should be \LUA\ type
checks (e.g., against \type {nil}) done. If a non|-|object item is requested (for
instance a \type {Dict} item by calling \type {page:getPieceInfo()}, cf.~\type
{Page.h}) but not available, the \LUA\ functions return \type {nil} (without
error). If a function should return an \type {Object}, but it's not existing, a
\type {Null} object is returned instead, also without error. This is in|-|line
with poppler behavior.

All library objects have a \type {__gc} metamethod for garbage collection. The
\type {__tostring} metamethod gives the type name for each object.

These are the object constructors:

\startfunctioncall
<PDFDoc>       = epdf.open(<string> PDF filename)
<PDFDoc>       = epdf.openMemStream(<string> PDF code)
\stopfunctioncall

The functions \type {StructElement_Type}, \type {Attribute_Type} and \type
{AttributeOwner_Type} return a hash table \type .

\type {Annot} methods:

\startfunctioncall
<boolean>     = <Annot>:isOK()
<boolean>     = <Annot>:match(<Ref>)
<Object>      = <Annot>:getAppearance() -- gone
<AnnotBorder> = <Annot>:getBorder()     -- gone
\stopfunctioncall

\type {AnnotBorderStyle} methods (gone):

\startfunctioncall
<number> = <AnnotBorderStyle>:getWidth() -- gone
\stopfunctioncall

\type {Annots} methods:

\startfunctioncall
<integer> = <Annots>:getNumAnnots()
<Annot>   = <Annots>:getAnnot(<integer>)
\stopfunctioncall

\type {Array} methods:

\startfunctioncall
<integer> = <Array>:getLength()
<Object>  = <Array>:get(<integer>)
<Object>  = <Array>:getNF(<integer>)
<string>  = <Array>:getString(<integer>)
\stopfunctioncall

\type {Attribute} methods:

\startfunctioncall
<boolean>  = <Attribute>:isOk()
<integer>  = <Attribute>:getType()
<integer>  = <Attribute>:getOwner()
<string>   = <Attribute>:getTypeName()
<string>   = <Attribute>:getOwnerName()
<Object>   = <Attribute>:getValue()
<Object>   = <Attribute>:getDefaultValue
<string>   = <Attribute>:getName()
<integer>  = <Attribute>:getRevision()
<boolean>  = <Attribute>:isHidden()
<string>   = <Attribute>:getFormattedValue()
\stopfunctioncall

\type {Catalog} methods:

\startfunctioncall
<boolean>  = <Catalog>:isOK()
<integer>  = <Catalog>:getNumPages()
<Page>     = <Catalog>:getPage(<integer>)
<Ref>      = <Catalog>:getPageRef(<integer>)
<string>   = <Catalog>:getBaseURI()
<string>   = <Catalog>:readMetadata()
<Object>   = <Catalog>:getStructTreeRoot()
<integer>  = <Catalog>:findPage(<integer> object number, <integer> object generation)
<LinkDest> = <Catalog>:findDest(<string> name)
<Object>   = <Catalog>:getDests()
<integer>  = <Catalog>:numEmbeddedFiles()
<EmbFile>  = <Catalog>:embeddedFile(<integer>)
<integer>  = <Catalog>:numJS()
<string>   = <Catalog>:getJS(<integer>)
<Object>   = <Catalog>:getOutline()
<Object>   = <Catalog>:getAcroForm()
\stopfunctioncall

\type {EmbFile} methods:

\startfunctioncall
<integer>  = <EmbFile>:size()
<string>   = <EmbFile>:modDate()
<string>   = <EmbFile>:createDate()
<string>   = <EmbFile>:checksum()
<string>   = <EmbFile>:mimeType()
<boolean>  = <EmbFile>:isOk()
<string>   = <EmbFile>:name()         -- not (yet) implemented
<string>   = <EmbFile>:description()  -- not (yet) implemented
<Object>   = <EmbFile>:streamObject() -- not (yet) implemented
<boolean>  = <EmbFile>:save()         -- will go
\stopfunctioncall

\type {FileSpec} methods:

\startfunctioncall
<boolean> = <FileSpec>:isOk()
<string>  = <FileSpec>:getFileName()
<string>  = <FileSpec>:getFileNameForPlatform()
<string>  = <FileSpec>:getDescription()
<EmbFile> = <FileSpec>:getEmbeddedFile()
\stopfunctioncall

\type {Dict} methods:

\startfunctioncall
<integer> = <Dict>:getLength()
<boolean> = <Dict>:is(<string>)
<Object>  = <Dict>:lookup(<string>)
<Object>  = <Dict>:lookupNF(<string>)
<integer> = <Dict>:lookupInt(<string>, <string>)
<string>  = <Dict>:getKey(<integer>)
<Object>  = <Dict>:getVal(<integer>)
<Object>  = <Dict>:getValNF(<integer>)
<boolean> = <Dict>:hasKey(<string>)
\stopfunctioncall

% \type {Link} methods:

% \startfunctioncall
% <boolean>  = <Link>:isOK()
% <boolean>  = <Link>:inRect(<number>, <number>)
% \stopfunctioncall

\type {LinkDest} methods:

\startfunctioncall
<boolean>  = <LinkDest>:isOK()
<integer>  = <LinkDest>:getKind()
<string>   = <LinkDest>:getKindName()
<boolean>  = <LinkDest>:isPageRef()
<integer>  = <LinkDest>:getPageNum()
<Ref>      = <LinkDest>:getPageRef()
<number>   = <LinkDest>:getLeft()
<number>   = <LinkDest>:getBottom()
<number>   = <LinkDest>:getRight()
<number>   = <LinkDest>:getTop()
<number>   = <LinkDest>:getZoom()
<boolean>  = <LinkDest>:getChangeLeft()
<boolean>  = <LinkDest>:getChangeTop()
<boolean>  = <LinkDest>:getChangeZoom()
\stopfunctioncall

\type {Links} methods:

% <Link>     = <Links>:getLink(<integer>)

\startfunctioncall
<integer>  = <Links>:getNumLinks()
\stopfunctioncall

\type {Object} methods:

\startfunctioncall
<Object>  = <Object>:fetch(<XRef>)
<integer> = <Object>:getType()
<string>  = <Object>:getTypeName()
<boolean> = <Object>:isBool()
<boolean> = <Object>:isInt()
<boolean> = <Object>:isReal()
<boolean> = <Object>:isNum()
<boolean> = <Object>:isString()
<boolean> = <Object>:isName()
<boolean> = <Object>:isNull()
<boolean> = <Object>:isArray()
<boolean> = <Object>:isDict()
<boolean> = <Object>:isStream()
<boolean> = <Object>:isRef()
<boolean> = <Object>:isCmd()
<boolean> = <Object>:isError()
<boolean> = <Object>:isEOF()
<boolean> = <Object>:isNone()
<boolean> = <Object>:getBool()
<integer> = <Object>:getInt()
<number>  = <Object>:getReal()
<number>  = <Object>:getNum()
<string>  = <Object>:getString()
<string>  = <Object>:getName()
<Array>   = <Object>:getArray()
<Dict>    = <Object>:getDict()
<Stream>  = <Object>:getStream()
<Ref>     = <Object>:getRef()
<integer> = <Object>:getRefNum()
<integer> = <Object>:getRefGen()
<string>  = <Object>:getCmd()
<integer> = <Object>:arrayGetLength()
<Object>  = <Object>:arrayGet(<integer>)
<Object>  = <Object>:arrayGetNF(<integer>)
<integer> = <Object>:dictGetLength(<integer>)
<Object>  = <Object>:dictLookup(<string>)
<Object>  = <Object>:dictLookupNF(<string>)
<string>  = <Object>:dictgetKey(<integer>)
<Object>  = <Object>:dictgetVal(<integer>)
<Object>  = <Object>:dictgetValNF(<integer>)
<boolean> = <Object>:streamIs(<string>)
          = <Object>:streamReset()
<integer> = <Object>:streamGetChar()
<integer> = <Object>:streamLookChar()
<integer> = <Object>:streamGetPos()
          = <Object>:streamSetPos(<integer>)
<Dict>    = <Object>:streamGetDict()
\stopfunctioncall

\type {Page} methods:

\startfunctioncall
<boolean>      = <Page>:isOk()
<integer>      = <Page>:getNum()
<PDFRectangle> = <Page>:getMediaBox()
<PDFRectangle> = <Page>:getCropBox()
<boolean>      = <Page>:isCropped()
<number>       = <Page>:getMediaWidth()
<number>       = <Page>:getMediaHeight()
<number>       = <Page>:getCropWidth()
<number>       = <Page>:getCropHeight()
<PDFRectangle> = <Page>:getBleedBox()
<PDFRectangle> = <Page>:getTrimBox()
<PDFRectangle> = <Page>:getArtBox()
<integer>      = <Page>:getRotate()
<string>       = <Page>:getLastModified()
<Dict>         = <Page>:getBoxColorInfo()
<Dict>         = <Page>:getGroup()
<Stream>       = <Page>:getMetadata()
<Dict>         = <Page>:getPieceInfo()
<Dict>         = <Page>:getSeparationInfo()
<Dict>         = <Page>:getResourceDict()
<Object>       = <Page>:getAnnots()
<Links>        = <Page>:getLinks(<Catalog>)
<Object>       = <Page>:getContents()
\stopfunctioncall

\type {PDFDoc} methods:

\startfunctioncall
<boolean>  = <PDFDoc>:isOk()
<integer>  = <PDFDoc>:getErrorCode()
<string>   = <PDFDoc>:getErrorCodeName()
<string>   = <PDFDoc>:getFileName()
<XRef>     = <PDFDoc>:getXRef()
<Catalog>  = <PDFDoc>:getCatalog()
<number>   = <PDFDoc>:getPageMediaWidth()
<number>   = <PDFDoc>:getPageMediaHeight()
<number>   = <PDFDoc>:getPageCropWidth()
<number>   = <PDFDoc>:getPageCropHeight()
<integer>  = <PDFDoc>:getNumPages()
<string>   = <PDFDoc>:readMetadata()
<Object>   = <PDFDoc>:getStructTreeRoot()
<integer>  = <PDFDoc>:findPage(<integer> object number,
                <integer> object generation)
<Links>    = <PDFDoc>:getLinks(<integer>)
<LinkDest> = <PDFDoc>:findDest(<string>)
<boolean>  = <PDFDoc>:isEncrypted()
<boolean>  = <PDFDoc>:okToPrint()
<boolean>  = <PDFDoc>:okToChange()
<boolean>  = <PDFDoc>:okToCopy()
<boolean>  = <PDFDoc>:okToAddNotes()
<boolean>  = <PDFDoc>:isLinearized()
<Object>   = <PDFDoc>:getDocInfo()
<Object>   = <PDFDoc>:getDocInfoNF()
<integer>  = <PDFDoc>:getPDFMajorVersion()
<integer>  = <PDFDoc>:getPDFMinorVersion()
\stopfunctioncall

\type {PDFRectangle} methods:

\startfunctioncall
<boolean>  = <PDFRectangle>:isValid() -- setindex/newindex will go
\stopfunctioncall

%\type {Ref} methods:
%
%\startfunctioncall
%\stopfunctioncall

\type {Stream} methods:

\startfunctioncall
<integer>  = <Stream>:getKind()
<string>   = <Stream>:getKindName()
           = <Stream>:reset()
           = <Stream>:close()
<integer>  = <Stream>:getChar()
<integer>  = <Stream>:lookChar()
<integer>  = <Stream>:getRawChar()
<integer>  = <Stream>:getUnfilteredChar()
           = <Stream>:unfilteredReset()
<integer>  = <Stream>:getPos()
<boolean>  = <Stream>:isBinary()
<Stream>   = <Stream>:getUndecodedStream()
<Dict>     = <Stream>:getDict()
\stopfunctioncall

\type {StructElement} methods:

\startfunctioncall
<string>         = <StructElement>:getTypeName()
<integer>        = <StructElement>:getType()
<boolean>        = <StructElement>:isOk()
<boolean>        = <StructElement>:isBlock()
<boolean>        = <StructElement>:isInline()
<boolean>        = <StructElement>:isGrouping()
<boolean>        = <StructElement>:isContent()
<boolean>        = <StructElement>:isObjectRef()
<integer>        = <StructElement>:getMCID()
<Ref>            = <StructElement>:getObjectRef()
<Ref>            = <StructElement>:getParentRef()
<boolean>        = <StructElement>:hasPageRef()
<Ref>            = <StructElement>:getPageRef()
<StructTreeRoot> = <StructElement>:getStructTreeRoot()
<string>         = <StructElement>:getID()
<string>         = <StructElement>:getLanguage()
<integer>        = <StructElement>:getRevision()
<string>         = <StructElement>:getTitle()
<string>         = <StructElement>:getExpandedAbbr()
<integer>        = <StructElement>:getNumChildren()
<StructElement>  = <StructElement>:getChild()
<integer>        = <StructElement>:getNumAttributes()
<Attribute>      = <StructElement>:getAttribute(<integer>)
<string>         = <StructElement>:appendAttribute(<Attribute>)      -- will go
<Attribute>      = <StructElement>:findAttribute(<Attribute::Type>,
                        boolean,Attribute::Owner)
<string>         = <StructElement>:getAltText()
<string>         = <StructElement>:getActualText()
<string>         = <StructElement>:getText(<boolean>)
<table>          = <StructElement>:getTextSpans()
\stopfunctioncall

\type {StructTreeRoot} methods:

\startfunctioncall
<StructElement> = <StructTreeRoot>:findParentElement
<PDFDoc>        = <StructTreeRoot>:getDoc
<Dict>          = <StructTreeRoot>:getRoleMap
<Dict>          = <StructTreeRoot>:getClassMap
<integer>       = <StructTreeRoot>:getNumChildren
<StructElement> = <StructTreeRoot>:getChild
<StructElement> = <StructTreeRoot>:findParentElement
\stopfunctioncall

\type {TextSpan} han only one method:

\startfunctioncall
<string> = <TestSpan>:getText()
\stopfunctioncall

\type {XRef} methods:

\startfunctioncall
<boolean>  = <XRef>:isOk()
<integer>  = <XRef>:getErrorCode()
<boolean>  = <XRef>:isEncrypted()
<boolean>  = <XRef>:okToPrint()
<boolean>  = <XRef>:okToPrintHighRes()
<boolean>  = <XRef>:okToChange()
<boolean>  = <XRef>:okToCopy()
<boolean>  = <XRef>:okToAddNotes()
<boolean>  = <XRef>:okToFillForm()
<boolean>  = <XRef>:okToAccessibility()
<boolean>  = <XRef>:okToAssemble()
<Object>   = <XRef>:getCatalog()
<Object>   = <XRef>:fetch(<integer> object number,
                <integer> object generation)
<Object>   = <XRef>:getDocInfo()
<Object>   = <XRef>:getDocInfoNF()
<integer>  = <XRef>:getNumObjects()
<integer>  = <XRef>:getRootNum()
<integer>  = <XRef>:getRootGen()
<integer>  = <XRef>:getSize()
<Object>   = <XRef>:getTrailerDict()
\stopfunctioncall

\topicindex{\PDF+memory streams}

There is an experimental function \type {epdf.openMemStream} that takes three
arguments:

\starttabulate
\DB value           \BC explanation      \NC \NR
\TB
\NC \type {stream}  \NC this is a (in low level \LUA\ speak) light userdata
                        object, i.e.\ a pointer to a sequence of bytes \NC \NR
\NC \type {length}  \NC this is the length of the stream in bytes \NC \NR
\NC \type {name}    \NC this is a unique identifier that is used for hashing the
                        stream, so that multiple doesn't use more memory \NC \NR
\LL
\stoptabulate

Instead of a light userdata stream you can also pass a \LUA\ string, in which
case the given length is (at most) the string length.

The function returns a \type{epdf} object and a string. The string can be used in
the \type {img} library instead of a filename. You need to prevent garbage
collection of the object when you use it as image (for instance by storing it
somewhere).

Both the memory stream and it's use in the image library is experimental and can
change. In case you wonder where this can be used: when you use the swiglib
library for \type {graphicmagick}, it can return such a userdata object. This
permits conversion in memory and passing the result directly to the backend. This
might save some runtime in one|-|pass workflows. This feature is currently not
meant for production and we might come up with a better implementation.

\stopchapter

\stopcomponent
